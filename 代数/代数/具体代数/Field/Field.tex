\documentclass[UTF8]{ctexart}[a4paper,12pt]
\usepackage[thmmarks]{ntheorem}
\usepackage{amsmath}
\usepackage{amsfonts,amssymb} 
\usepackage{thmtools}
\usepackage[hmargin=2.5cm,vmargin=2.5cm]{geometry}
\usepackage{tikz-cd,tikz}
\usepackage{graphicx,float}
\usepackage{fancyhdr}
\usepackage{fourier-orns}


%声明环境
\newtheorem{example}{例}[section]              
\newtheorem{algorithm}{算法}[subsection]
\newtheorem{theorem}{定理}[subsection]            
\newtheorem{definition}{定义}[section]
\newtheorem{axiom}{公理}[section]
\newtheorem{property}{性质}[section]
\newtheorem{proposition}{命题}[subsection]
\newtheorem{lemma}[theorem]{引理}
\newtheorem{corollary}[theorem]{推论}
{
    \theoremheaderfont{\sffamily}
    \newtheorem*{remark}{注解} 
}
\newtheorem{condition}{条件}
\newtheorem{conclusion}{结论}[section]
\newtheorem{assumption}{假设}
{
\theoremstyle{nonumberplain}
\theoremheaderfont{\bfseries}
\theorembodyfont{\normalfont}
\theoremsymbol{\mbox{$\Box$}}
\newtheorem{proof}{证明}
}
%定义命令
\def\N{\mathbb{N}}
\def\Z{\mathbb{Z}}
\def\Q{\mathbb{Q}}
\def\R{\mathbb{R}}
\def\C{\mathbb{C}}
\def\S{\mathbb{S}}
\def\D{\mathbb{D}}
\def\H{\mathbb{H}}
\def\F{\mathbb{F}}
\newcommand{\Gal}{\mathrm{Gal}}
%外测度
\def\outmQ{m_*(Q)}

%页眉设计
\renewcommand 
\headrule{
\hrulefill
\raisebox{-2.1pt}
{\quad{\FourierOrns M T S N}\quad}
\hrulefill}
\pagestyle{fancy}

%超链接红色
\usepackage[colorlinks,linkcolor=red]{hyperref}

\usepackage{enumerate}


\title{Field}
\author{颜成子游}
\begin{document}
\maketitle
\tableofcontents
\newpage
经过了群,环,模,线性代数的学习,我们开始研讨代数性质最丰富的结构——域。

\section{域的基本概念}
域的概念在学习环的时候就已经有介绍过。我们再重复一遍
\begin{definition}
    交换的除环是域。即在域中,除零外的其他元素都有逆元,并且他们之间的乘法运算封闭。

\end{definition}
对于域,有如下比较明显的事实:
\begin{theorem}
    \quad

    1.设$R$是交换幺环,$M$是$R$的极大理想,那么$R/M$是一个域。

    2.有限的整环是域。

    3.$R$是无零因子环,并且只有有限个理想(包括左,右,双边理想)。
    则$R$是除环。(可以用这个定理说明域中除的性质。)
\end{theorem}
\section{域的代数闭包}
前面我们讨论了多项式$f(x) \in F[x]$在$F$中的分裂域。我们证明了它一定存在,并且在保$F$同构的意义下唯一。
自同构群的阶有限,$|Aut(E)| \leq [E:F]$.并且$[E:F] |n!$。

我们考虑下列三个问题:
\begin{enumerate}
    \item 取$\{f_k\}_{k=1}^n$,是否存在域扩张$E$使得每个$f_k$都分裂?
    \item 取$S$为多项式集合,并且为无限集合。是否存在域扩张$E$使得每个$f\in S$都分裂?
    \item $\forall f \in F[x]$,是否存在一个域扩张$E$使得$f$分裂?
\end{enumerate}

第一个问题,只需要取$g$为$n$个多项式的积即可。

我们先讨论第三个问题,这引出了“代数闭包”的概念:
\subsection{代数闭包的概念,等价形式}
\begin{definition}
    称$E$是$F$的代数闭包,如果$E$是$F$的代数扩张,并且任意$F[x]$中的多项式
    都在$E$中分裂。
\end{definition}

代数闭包的名字让人联想到代数闭域。值得庆幸的是,代数闭包确实是代数闭域。
\begin{definition}
    称$E$是$F$的代数闭包,如果$E$是$F$的代数扩张,并且$E$代数封闭。
\end{definition}

我们验证这两个定义等价:
\begin{proposition}
    两个定义等价。
\end{proposition}
\begin{proof}
    定义2推导定义1:
    
    显然$F$上多项式都是$E$上的多项式。由于$E$封闭,从而只要一步步去掉一次因式,
    就有$F$分裂。

    定义1推导定义2:

    如果有$g \in E[x]$在$E$上没有根,取其分裂域$K$,根据代数扩张传递性,则$K$是$F$的代数扩张。
    取$\alpha \in K$,则$f_\alpha \in F[x]$是$\alpha$最小多项式。
    从而$f_\alpha$分裂于$E$,于是$\alpha \in E$。矛盾!从而任意多项式都有根。
\end{proof}

因此我们意识到,如果$K/E/F$,$E$是代数闭包,那么$K=E$。这某种意义上说明$E$是“最大的代数扩张”。
我们还可以从另外一个意义上说明这件事:
\begin{theorem}[代数闭包是最大的代数扩张]
    设$E$是$F$的代数闭包,那么任取$K$是$F$的代数扩张,存在嵌入映射:
    $$
    i: K \to E ,\quad i|_F=id_F
    $$
    
    如果$K/E/F$,$K$也是代数扩张,那么$K=E$。
\end{theorem}
\begin{proof}
    考虑集合$S=\{(L,i_L)|L \subset K\}$。
    $i_L$表示从$L$到$E$存在嵌入映射,为代数扩张。

    显然集合$S$不是空集。

    定义偏序关系:
    $<$,若$L_1 \subset L_2$且
    $i_{L_2}|_{L_1}=i_{L_1}$.

    那么$S$中的全序集合有上界。因为把这个全序集合中的所有$L$取并,定义$i$,$i$作用在并集上,
    每个元素的像就是在这个全序集合中的像。(映射是保持子集不变的)。

    根据Zorn引理,$S$也有最大值。即有一个$K'$满足:
    $$
    \forall (L,i_L) \in S,L\subset K',i_{K'}|_L=i_L
    $$

    下面证明$K'=K$。

    我们假设$\alpha \in K\setminus K'$,那么$\alpha$在
    $K'$中有最小多项式$f_\alpha$。从而$K'(\alpha)$满足:
    $$
    j:K'(\alpha) \to E
    $$
    (这是容易验证的)
    这就与$K'$的最大性矛盾了。于是$\alpha$是不存在的。
    
    于是$K'=K$
\end{proof}
    由这种“最大”性,我们不难想到,代数闭包一定是唯一的:
   \begin{theorem}[代数闭包唯一性]
       任何一个域$K$的代数闭包,在同构意义下(保持$F$)
       一定是唯一的。
   \end{theorem}
   \begin{proof}
       取$K$的代数闭包$F$,$\overline{F}$。我们有:
       $$
       i_1:F \to \overline{F}
       $$
       $$
       i_2:\overline{F} \to F
       $$
       
       于是:
       $$
       i:F \to F,i=i_2 \circ i_1
       $$
       是一个嵌入映射。

       假如这个嵌入映射不是满射,
       那么有:$\alpha \in F\setminus i(F)$.
       $i(F)$包含$K$,从而$\alpha$是$i(F)$上的代数元。取最小
       多项式$f_\alpha \in i(F)[x]$,定义:
       $$
       i^{-1}:i(F) \to F
       $$
       则$i^{-1}(f_\alpha)$在$F$中必须分裂(代数闭包)。从而
       $i$映射回去后,$f_\alpha$也分裂。
       于是$\alpha \in i(F)$矛盾!

       于是有$F$与$\overline{F}$同构。
   \end{proof}
   \subsection{代数闭包的存在性}
    接下来要解决的问题是,代数闭包一定存在。这一点是比较困难的,需要巧妙地构造:
    \begin{theorem}
        任意一个域$K$的代数闭包一定存在。
    \end{theorem}
    \begin{proof}
        我们定义多项式环:
        $$
        K[\dots,x_f,x_g,\dots]
        $$
        其中$x_f$的指标集是$K$中所有不可约多项式。

        考虑由$\{f(x_f)\}$生成的理想$I$。我们证明$1 \notin I$。

        事实上,如果$1 \in I$,那么$1$被$f(x_f)$有限生成。

        我们取$f(x)$(这是有限个多项式)的分裂域,那么带入$x_f$为$f$的
        根,在这个分裂域上,$1=0$,显然这是矛盾的。于是$1 \notin I$。

        根据Zorn引理,存在一个极大理想$m$:$I \subset m$。

        考虑$K_0=K[\dots,x_f,\dots]/m$。那么$K_0$是由$F \cup \{x_f+m\}$生成。

        我们证明这是一个代数扩张:由
        $$
        f(x_f+m)=f(x_f)+m=0
        $$
        从而生成元都是代数元,于是$K_0$是一个代数扩张。并且在这个代数扩张中,
        每个$K$上的不可约多项式都有一个根:$x_f+m$。从而我们一直把这个操作进行下去:
        $$
        K \to K_0 \to K_1 \to K_2 \dots
        $$

        取$F=\bigcup_{n \geq 0} K_n$,$F$中的元素都是代数元,并且对于每个$f$不可约,都能把所有的根
        给逐步包进去。从而$F$是代数闭包。
    \end{proof}
    \section{域的正规扩张}
    这是一种介于“分裂域”和代数闭包之间的域扩张。
    \begin{definition}
        $E/F$是代数扩张,称$E/F$是\textbf{正规扩张},若$\forall f(x) \in F[x]$
        ,若$f$不可约且在$E$中有根,则$f$在$E$中分裂。
    \end{definition}
    \begin{definition}
        $F$是域,
        $S \subset F[x] \setminus F$,
        若$E/F$满足:

        1.$\forall f \in S$,$f$在$E$中分裂。

        2.若$E'$有:$E/E'/F$且$E'$也满足:$\forall f \in S$,$f$在$E'$中分裂。
        那么$E=E'$

        则称$E$是$S$在$F$上的分裂域。
    \end{definition}
    同构意义下,$S$的分裂域是使其中所有多项式都分裂的
    最小代数扩张。若$S$有限,则回到之前的问题。

    介绍$S$的分裂域的原因来源于下面这个让人感到愉快的定理:
    \begin{theorem}
        $E/F$正规等价于
        存在$S \subset F[x]\setminus F$,
        $E/F$是$S$的分裂域扩张。
    \end{theorem}
    \begin{corollary}
        若$E/F$是正规扩张,则$\forall E/K/F$,$E/K$也是正规扩张。

    \end{corollary}
    \begin{proof}
        $E/F$是正规扩张,则可以找到$S$使得$E/F$是$S$的分裂域扩张
        
        记$E'$是$S$在$K$上的分裂扩张。由于$S$在$E$上分裂,$K \subset E$得:
        $$
        E'\subset E
        $$
        但$E/F$是$S$的分裂域扩张,于是:
        $$
        E \subset E'
        $$
        于是$E=E'$
        
        于是$E/K$分裂域扩张。
    \end{proof}
    \begin{corollary}
        $E/F$是有限扩张,则:
        
        则$E/F$正规$\Longleftrightarrow$$\exists f(x) \in F[x]$,$E/F$是$f$的分裂域扩张。
    \end{corollary}
    \begin{definition}
        取$K/F$扩张,称$K$的一个扩张$E$为$F$关于$K/F$的正规闭包
        ,若:
        
        1.$E/F$正规。

        2.$\forall M,E/M/K,M/F$正规,则$M=E$。 
    \end{definition}
    \begin{corollary}
        若$K/F$有限,则$K$有一个有限次的正规闭包$E/K$。
    \end{corollary}
    \begin{proof}
        若$K/F$有限,则有:
        $$
        K=F(\alpha_1,\dots,\alpha_n)
        $$
        记$f_i\in F[x]$为$\alpha_i$的最小多项式,$g=f_1\dots f_n$
        $E/F$是$g$的分裂扩张。则$E/K/F$,$E/F$正规扩张。

        下证$E$最小。

        设$\exists E/M/K/F$满足$M/F$正规。

        由于$\alpha_1,\dots,\alpha_n \in K\subset M$,$g$在$M$有根,从而分裂(不可约因子都有根)。

        从而$E \subset M$。

    \end{proof}
    \begin{lemma}
        记$F$为域,$f(x)\in F[x]$不可约,$E/F$为$f$的分裂域扩张。

        记$f$的根为$\alpha_1,\dots,\alpha_n$。则:
        $$
        \forall \alpha_i,\alpha_j \exists\varphi:E \to E
        $$
        为同构:
        $$
        \varphi|_F=\text{id}_F,  \varphi(\alpha_i)=\alpha_j
        $$
    \end{lemma}
    什么时候正规扩张的中间域也能给出一个正规扩张呢?
    \begin{theorem}
        $E/F$正规且有限,记$K:E/K/F$,以下叙述等价:
        \begin{enumerate}
            \item $K/F$正规。
            \item $\forall \sigma:E \to E,\sigma|_F=id_F$,则$\sigma(K)=K$。
            \item $\forall \sigma:E \to E,\sigma|_F=id_F$,则$\sigma(K)=K$。
        \end{enumerate}
    \end{theorem}
    \begin{proof}
        $1 \Rightarrow 2$:
        $2 \Rightarrow 3$:
        $3 \Rightarrow 1$:
    \end{proof}
    \section{域的可分扩张}
    \subsection{可分扩张的基本概念}
    \begin{definition}
        $F$是一个域,$f(x) \in F[x]$不可约,记$K/F$是$f(x)$的分裂域扩张。
        称$f(x)$可分,若$f$在$K[x]$中可分解:
        $$
        f(x)=(x-\alpha_1)\dots(x-\alpha_n)
        $$
        满足$\alpha_j$各不相同。

        $g\in F[x]$可分当且仅当其不可约因子都可分。
    \end{definition}
    \begin{definition}
        $F$域,$E/F$为代数扩张。称$\alpha \in E$为可分元,若$\alpha$在$F$中的
        最小多项式可分。
    \end{definition}
    \begin{definition}
        $E/F$是可分扩张,若$\forall \alpha \in E$,$\alpha$是$F$中的可分元。
    \end{definition}
    注:并不要求$f_\alpha$在$E$中可分。只要求其在自身的分裂域扩张中,
    根互不相同。

    \begin{proposition}
        可分扩张的中间域,与两头都可分:
        
        $E/F$可分,$E/K/F$,从而$E/K,K/F$均可分。
    \end{proposition}
    \begin{proof}
        显然$K/F$是可分扩张。我们只证明$E/K$是可分扩张。

        任取$\alpha \in E$,考虑$f_\alpha \in F[x]$是$\alpha$在$F$上的最小多项式。
        $g_\alpha \in K[x]$是$\alpha$在$K$上的最小多项式。

        我们有:在$K[x]$中,$g_\alpha |f_\alpha$。

        记$f_\alpha$的分裂域为$L$,从而$f_\alpha$没有重根。于是$g_\alpha$在$L$上也没有重根,
        于是在其自己的分裂域也没有重根,于是$\alpha$在$K$上可分。于是$E/K$可分。
    \end{proof}

    给出一个引理:
    \begin{lemma}
        记$i:F \to F'$是域的同构,$f(x) \in F[x]$可分,则$i(f)\in F'[x]$也可分。
    \end{lemma}
    \begin{proof}
        同构的域诱导的分裂域扩张也是同构的。从而只要一个没有重根,则另一个也没有重根。


    \end{proof}
    \begin{theorem}
        $F(\alpha)$是单代数扩张,且$[F(\alpha):F]=n$。记$f_\alpha$是$\alpha$的最小多项式
        ,设$\varphi:F \to E$域同态,则有:
        
        1.若$\alpha$可分,且$\varphi(f_\alpha) \in E[x]$在$E$中分裂,则
        有且仅有$n$个:
        $$
        \tilde{\varphi}:F(\alpha) \to E,\tilde{\varphi}|_F=\varphi
        $$

        2.否则,这样的$\tilde{\varphi}$的个数小于$n$。
    \end{theorem}
    \begin{proof}
        
    \end{proof}
    \begin{theorem}\label{thm:413}
        $K/F$域扩张,$[K:F]=n$。$\forall \alpha \in K$,记
        $f_\alpha \in F[x]$是$\alpha$上的最小多项式。记$\varphi:F \to E$:
        
        1.若$\forall \alpha \in K$,$\alpha$可分,且$\varphi(f_\alpha)$在$E$
        上分裂,则 有且仅有$n$个:
        $$
        \tilde{\varphi}:K \to E,\tilde{\varphi}|_F=\varphi
        $$
        
        2.否则,这样的$\tilde{\varphi}:K\to E$的个数小于$n$
    \end{theorem}
    \begin{figure}[h]
        \centering
        \begin{tikzcd}
            K && E \\
            \\
            F
            \arrow["{id_F}", from=3-1, to=1-1]
            \arrow["\varphi"', from=3-1, to=1-3]
            \arrow["{\tilde{\varphi}}", from=1-1, to=1-3]
        \end{tikzcd}
\end{figure}
    \begin{proof}
        由于是有限扩张,从而只扩张了有限个代数元:
        $$
        K=F(\alpha_1,\alpha_2,\dots,\alpha_n)
        $$
        $$
        [K:F]=[k:F(\alpha_1,\dots,\alpha_{n-1})]:\dots:[F(\alpha_1):F]=n
        $$
        从而我们注意到,每一步扩张中,$\alpha_k$在$F$可分,那么在更大的域上也可分。并且其最小多项式也分裂。
        于是每一步都有几个选择。乘起来就是$n$。若有满足条件的,那么第一步扩张就不满足。从而小于$n$。
    \end{proof}
    \begin{corollary}
        单代数扩张是可分扩张,当且仅当$\alpha$是可分元。
    \end{corollary}
    \begin{proof}
        记$f_\alpha \in F[x]$是$\alpha$的最小多项式,扩张次数$n$。

        记$E/F$是$f_\alpha$的分裂域扩张,从而$E/F$正规。

        任取$\beta \in E$,$\beta$在$F[x]$的最小多项式$f_\beta$分裂。
    
        记$i:F \to E$。

        若$\alpha$可分,那么就存在$n$个:$\tilde{i}:F(\alpha)\to E$。

        由于$\forall \beta \in F(\alpha)$,有$\beta_1=\beta,\beta_2,\dots,\beta_n$使得:
        $$
        F(\beta_1,\dots,\beta_n)=F(\alpha)
        $$
        第一次扩张中,必须有$[F(\beta):F]$个映射。从而$\beta$可分。
    \end{proof}
    \begin{corollary}
        设$L=F(\alpha_1,\dots,\alpha_n)$是有限扩张。$L/F$可分等价于
        $\alpha_1,\dots,\alpha_n$在$F$中可分。
    \end{corollary}
    
    相比于可分不可约多项式,不可分的不可约多项式反而要少很多。接下来我们做这样的研究:
    \begin{definition}
        如同$\C[x]$的形式微商,我们也定义一般多项式的形式微商。不再赘述定义。
    \end{definition}
    \begin{proposition}
        $F$是域,$f(x)\in F[x]$,deg$f>0$。$E/F$是分裂域扩张,下面叙述等价:
    
        1.$f$有重根。

        2.存在$\alpha  \in E$,$f(\alpha)=Df(\alpha)=0$。

        3.$\exists g(x)\in F[x]$,有deg $g>0$,满足$g|f$,且$g|Df$。
    \end{proposition}
    \begin{proof}
        1 $\Rightarrow$ 2: 设出重根,求导即可得到答案。

        2 $\Rightarrow$ 3: $g$设为$\alpha$在$F$中的最小多项式。

        3 $\Rightarrow$ 1:
         
        $g|f$意味着$g$在$E$中分裂。设$\alpha$是$g$的一个根:
        $$
        g=(x-\alpha)g_1(x) ,f=(x-\alpha)f_1(x)
        $$
        求导即可得到完整的证明。
    \end{proof}
    我们可以得出下面的结论:
    \begin{theorem}
        $F$是一个域,$f(x)\in F[x]$不可约。那么:

        $f(x)$不可分 $\Leftrightarrow$ $\exists p$为素数,使得$Ch(F)=p$,
        且存在$a_k$($k=0,1,\dots,n$),使:
        $$
        f(x)=\sum_{k=0}^n  a_kx^{kp}
        $$ 
    \end{theorem}
    \begin{proof}
        若$f$不可分,那么存在$\alpha$使得:$f(\alpha)=Df(\alpha)=0$。
        由于$f$不可约,那么$f$是$\alpha$在$F$上的最小多项式。

        由于求导后多项式的次数必然降低,而$Df(\alpha)=0$,于是我们得到$f$
        求导后必然是零多项式。

        于是我们有$n_k*a_k=0$对于每个$k$都成立。从而有质数$p$作为域的特征。
        并且$n_k=pk$.

        若$f$有如下的形式且不可约,
        那么在$f$的分裂域上的$f$的根$\alpha$都是重根。则
        $f$不可分。
    \end{proof}
    \begin{example}
        记$F=\F_p(t)$是$\F_p[t]$的分式域。证明$x^p-t \in \F_p(t)[x]$不可分。
    \end{example}
    \begin{proof}
    \end{proof}
    \begin{corollary}
        若$Ch(F)=0$,则:

        1.任何$f(x) \in F[x]\setminus F$,$f$可分。

        2.$E/F$代数扩张,则$E/F$可分。
    \end{corollary}
    \begin{definition}
      完备域:若$F$上的所有多项式都可分。  
    \end{definition}
    
    设$F$是域,$F[x]$中的不可约多项式:
    $$
    f(x)=\sum_{k=0}^n a_kx^k
    $$
    设$p$为素数,考虑$d=\gcd(k|a_k\neq 0)$的素数分解:
    $$
    d=p^m d_1,p\nmid d_1
    $$
    则有:$\exists g(x) \in F[x]$,
    $$
    f(x)=g(x^{p^m})
    $$
    我们给出以下命题:
    \begin{proposition}
        若$Ch(F)=p$,则$g(x)$不可约,且可分。
    \end{proposition}
    \begin{proof}
        不可约显然(因为$f$是不可约的)。

        若$g$不可分,则有:
        $$
        g=\sum_{k=0}^n  a_kx^{kp}
        $$
        $$
        f=\sum_{k=0}^n a_kx^{kp^{m+1}}
        $$
        从而$p^{m+1}|d$矛盾!
    \end{proof}
    \begin{proposition}
        设$Ch(F)=p$,
        记$E/F$为$f$的分裂域扩张,则$f$在$E$中所有根的重数都为$p^m$
    \end{proposition}
    \subsection{Frobenius同态}
    \begin{definition}
        $F$是域,$Ch(F)=p$为素数。定义:
        $$
        Fr:F \to F   \quad Fr(a)=a^p
        $$
    \end{definition}
    \begin{proposition}
        $F$是域,$Ch(F)=p$,则
        $$
        Fix(Fr)=\{a \in F|a^p=a\}
        $$
        是$F$的素域$\F_p$
    \end{proposition}
    \begin{proof}
        因为方程$x^p-x=0$的根只有$p$个,并且我们找到了$p$个:
        $$
        0,1,\dots,p-1
        $$
    \end{proof}
    \begin{proposition}
        $F$是特征为$p$的域。若$F$是其素域的代数扩张,那么Frobenius同态是自同构。
    \end{proposition}
    在证明这个命题前,我们给出两个引理:
    \begin{lemma}
        $E/F$是有限扩张,则$End_{F}(E)=Aut_F(E)$。
    \end{lemma}
    \begin{proof}
        比次数。
    \end{proof}
    \begin{lemma}
        $E/F$是代数扩张,则$End_F(E)=Aut_F(E)$。
    \end{lemma}
    \begin{proof}
        给出$\varphi:E \to E, \varphi|_F=id_F$.任给$\alpha \in E$,
        考虑$\alpha$的最小多项式$f_\alpha \in F[x]$。

        记$R=\{\alpha_1,\alpha_2,\dots,\alpha_k\}$是$E$中$f_\alpha$
        所有根。

        由于$\varphi(f_\alpha)=f_\alpha$,从而$\varphi(R)\subset R$.

        但是$R$是有限的,因此单射$\varphi$也是同构:
        $\exists \alpha_i,\varphi(\alpha_i)=\alpha$。

        因此$\varphi$是满射,得到了同构。
    \end{proof}
    
    从而命题的证明是显然的。

    于是我们有推论:
    \begin{corollary}
        $F$是域,$Ch(F)=p$是素数,$F$是其素域代数扩张。则$F$是完备的。
    \end{corollary}
    \begin{proof}
        不妨设$F/\F_p$。考虑$Fr:F \to F$同构。

        任取
        $$
        f(x)=\sum_{k=0}^n a_kx^{kp}
        $$
        只需说明$f$可约。但由于$Fr$同构,于是就有:$b_k^p=a_k$。

        于是$f(x)=(b_0+b_1x+\dots+b_nx^n)^p$
        从而$f$可分。
    \end{proof}
    
    若扩张不是代数的,我们不能说$F$一定不完备。但是如果$Fr$不同构,则一定不完备:
    因为$x^p-a$不可约,不可分。
    \begin{corollary}
        $F$是特征为$p$的域。$E/F$是代数扩张。则$F$完备可以导出$E$扩张。
    \end{corollary}
    \begin{proof}
        
    \end{proof}
    \section{Galois理论}
    \subsection{域的自同构及其不变子域}
    \begin{definition}
        $E/F$是扩张,$E$上的$F$自同构的全体构成一个群,被称为$E/F$的
        Galois群,记为:$\Gal(E/F)$。
    \end{definition}
    
    由于代数扩张的域自同态全部为自同构,从而:
    $$
    End_F(E)=\Gal(E/F)
    $$
    \begin{definition}
        取$\varphi \in \Gal(E/F)$,$E/F$是扩张。记:
        $$
        \mathcal{K}(\varphi):=\{a\in E|\varphi(a)=a\}
        $$
        为$\varphi$的不变子域。
    \end{definition}
    \begin{proposition}
        $\varphi$的不变子域是$E/F$的中间域。
    \end{proposition}
    \begin{proof}
        $F \subset  \mathcal{K}(\varphi)$。容易验证$\mathcal{K}(\varphi)$
        是一个域。
    \end{proof}
    

    类似地,我们可以考虑$S\subset \Gal(E/F)$,定义$S$的不变子域为:
    \begin{definition}
        $S\subset \Gal(E/F)$,则:
        $$
        \mathcal{K}(S):\{a \in E|\forall \varphi \in S,\varphi(a)=a\}
        $$
        定义为$S$的不变子域。
    \end{definition}
    \begin{proposition}
        $\mathcal{K}(S)=\bigcap_{\varphi \in S}\mathcal{K}(\varphi)$
    \end{proposition}
    
    根据域的相交也是域的性质,我们不难得出,$\mathcal{K}(S)$也是$E/F$的中间域。
    
    并且$S_1\subset S_2 \Longrightarrow \mathcal{S_2}\subset \mathcal{S_1}$

    从而,$\Gal(E/F)$中的子集似乎就能对应$E/F$中的一个中间域。

    \begin{definition}
        $K$是$E/F$的中间域。定义:
        $$
        \Gamma(K)=\{\varphi \in \Gal(E/F)|\varphi|_K=id_K\}
        $$
        
    \end{definition}
    
    必须验证$\Gamma(K) <\Gal(E/F)$。但是验证的工作我们略去。
    \begin{proposition}
        $\Gamma(K) <\Gal(E/F)$。
    \end{proposition}
    \begin{proposition}
        设$E/F$扩张,$K_1,K_2$是$E/F$的中间域。则:
        $$
        K_2 \subset K_1 \Longrightarrow \Gamma(K_1) <\Gamma(K_2)
        $$
    \end{proposition}
    但是反过来是不成立的。
    \begin{example}
        记$\F_p(t)$是$\F_p$上多项式的分式域,考虑$x^p-t$。
        假设$\F_p(t)(\alpha)/\F_p(t)$是$x^p-t$的分裂域扩张,$\alpha^p=t$。

        此时$\Gal(E/F)$是一个平凡群
    \end{example}

    现在对于集合$\Gal(E/F)$和$\{E/F\}$的中间域之间建立了两个映射:
    $$
    \mathcal{K},\quad \Gamma
    $$
    一个很自然的问题是,这两个映射到底有什么关系?
    \begin{proposition}
        $S$是$\Gal(E/F)$的非空子集,$K$是$E/F$的中间域。
        \begin{enumerate}
            \item $S \subset \Gamma(\mathcal{K}(S))$
            \item $K \subset \mathcal{K}(\Gamma(K))$
            \item $\mathcal{K}(S) = \mathcal{K}(\Gamma(\mathcal{K}(S)))$
            \item $\Gamma(K) = \Gamma(\mathcal{K}(\Gamma(K)))$
        \end{enumerate}
    \end{proposition}
    证明略。

    对于$\Gal(E/F)$中的非空子集$S$,考虑其生成的子群$<S>$。我们验证两个集合的不变子域是
    相同的:
    $$
    \mathcal{K}(S)=\mathcal{K}(<S>)
    $$
    \subsection{Galois扩张}
    \begin{definition}
        设$E/F$扩张,称$E/F$是一个\textbf{Galois扩张},如果$E/F$正规可分。
    \end{definition}
    \begin{example}
        \quad

        \begin{enumerate}
            \item $\Q[\sqrt[3]{2},\omega]/Q$是正规的可分扩张,是Galois扩张。
            \item $\Q$的代数闭包是正规可分的扩张。
            \item $\F_p(t)(\alpha)/F_p(t)(p\neq 2)$
            是$x^2-t$的分裂域。则是Galois扩张。
        \end{enumerate}
    \end{example}
    \begin{example}
        \quad

        \begin{enumerate}
            \item $\Q(\sqrt[3](2))/Q$不正规,但可分。
            \item $\F_p(t)(\alpha)/F_p(t)$
            是$x^p-t$的分裂域。正规但不可分。
        \end{enumerate}
    \end{example}

    Galois扩张继承了两种扩张的性质:
    \begin{theorem}
        $E/F$有限,以下叙述等价:
        \begin{enumerate}
            \item $E/F$Galois扩张。
            \item $|\Gal(E/F)|=[E:F]$
            \item 有Galois子群$G$
            使得$\mathcal{K}(G)=F$
            \item $\mathcal{K}(\Gal(E/F))=F$
        \end{enumerate}
    \end{theorem}
    \begin{proof}
        $1 \to 2$:若$E/F$是Galois扩张

        那么任取$\alpha \in E$,$f_\alpha  \in F[x]$分裂且没有重根。
        
        我们要寻找满足如下交换图的$\varphi$:
        \begin{figure}[h]
            \centering
            \begin{tikzcd}
                E && E \\
                \\
                F
                \arrow["{id_F}", from=3-1, to=1-1]
                \arrow["{id_F}"', from=3-1, to=1-3]
                \arrow["\varphi", from=1-1, to=1-3]
            \end{tikzcd}
        \end{figure}

        根据定理\ref{thm:413},此时每个$\alpha$都可分,$f_\alpha$分裂,于是有$n$个这样的
        $\varphi$。

        $2\to 1$:定理\ref{thm:413}有逆命题。

        $3 \to 4, 4 \to 3$:
        $$
        F= \mathcal{K}(G) \supset \mathcal{K}(\Gal(E/F)) \supset F
        $$
        
        $2 \to 4$:

        记$K=\mathcal{K}(\Gal(E/F))$是中间域。于是:
        $$
        [E:F]=|\Gal(E/F)|=|Gal(E/K)|=[E:K]
        $$
        从而$K=F$.
        
        在上述等式中,需要注意,Galois扩张具有对中间域的传递性。
        即$E/K/F$,则$E/K$也是Galois扩张。另一方面,由于$K=\mathcal{K}(\Gal(E/F))$,从而
        任何保持$F$的同构也保持$K$的同构。于是$|\Gal(E/F)|=|Gal(E/K)$。根据12等价,上述
        等式就连起来了。

        $4 \to 2$:

        反证法,假设:$[E:F]=n$,$|\Gal(E/F)|=s+1$,$s+1<n$:
        $$
        a_1,a_2,a_3,\dots,a_n 
        $$
        是基。
        $$
        id=\varphi_0,\varphi_1,\dots,\varphi_s
        $$
        是Galois群。
        于是我们有方程组:
        $$
        \sum_{j=1}^n\varphi_k(a_j) x_j=0,k=0,1,2,\dots,s
        $$
        这个方程有非零解。取其中零元素最多的解$(1,b_2,\dots,b_n)$:我们断言,$b_2,b_3\dots,b_n$是$F$中
        的元素:

        事实上,容易验证$\varphi(1,b_2,\dots,b_n)$也是方程组的解。取$\varphi$使得$\varphi(b_2)\neq b_2$
        
        于是,$(0,b_2-\varphi(b_2),\dots,\varphi(b_n)-b_n)$也是一组解,从而矛盾于零元素最多。

        但是,若每个$b_j$都属于$F$,就与$a_1,\dots,a_n$为基矛盾!

    从而最初的假设错误,从而证明结果。
        
    \end{proof}
    \begin{lemma}[Artin引理]
        $E$是一个域,且$G<Aut(E)$,$|G|<\infty$。记$F$是$G$的不变子域,那么
        $[E:F]<\infty$,且$[E:F]<|G|$。
    \end{lemma}
    利用定义证明:
    \begin{proof}
        和上述的定理证明思路一样。
    \end{proof}
    \begin{corollary}
        $G<Aut(E)$且为有限群,$F=\mathcal{K})(G)$,$\mathcal{B}$是$E$中的一组向量,那么下面叙述等价:
        \begin{enumerate}
            \item $a_1,a_2,\dots,a_m$ $F$-线性无关。
            \item 向量组:${\vec{\varphi}(a_i)}$ $E$线性无关。
            \item 向量组:${\vec{\varphi}(a_i)}$ $F$线性无关
        \end{enumerate}
    \end{corollary}
    \begin{corollary}
        $E/F$是有限扩张,$H<\Gal(E/F)$,则$/\mathcal{K}(H)$是Galois扩张。且:
        $$
        H=(\Gamma\circ \mathcal{K})(H)
        $$
    \end{corollary}
    \begin{corollary}
        $E/F$是有限扩张,则:
        $$
        \Gamma \circ \mathcal{K}
        $$
        从Galois群的子群到Galois群的子群是一个恒等映射。
    \end{corollary}
    \begin{corollary}
        $E/F$是有限扩张,$H_1$,$H_2$是Galois群的子群。如果:
        $$
        \mathcal{K}(H)_1\supset \mathcal{K}(H)_2
        $$
        则:$H_1<H_2$。
    \end{corollary}

    \subsection{Galois基本定理}
    接触了以上概念后,我们将给出Galois理论中最基本的定理:
    \begin{theorem}
        设$E/F$有限且是Galois扩张。那么有:
        \begin{enumerate}
            \item $\mathcal{K}$和$\Gamma$都是双射。并且互为逆运算。
            \item 任意$H<\Gal(E/F)$,则有:
            $$
            [\Gal(E/F):H]=[\mathcal{K}(H):F]
            $$
            \item $H \vartriangleleft \Gal(E/F)  \Leftrightarrow 
            \mathcal{K}(H)/F$是正规扩张,从而是Galois扩张。
            \item $H \vartriangleleft \Gal(E/F)$,那么:
            $$
            \frac{\Gal(E/F)}{H}\backsimeq Gal(\mathcal{K}(H)/F)
            $$
        \end{enumerate}
    \end{theorem}
    定理给出了Galois群与中间域之间密切的关系。在Galois理论中起到了密切的作用。
    \begin{proof}
        1.结合之前的定理,已经显然。

        2.记$K=\mathcal{K}(H)$,$H=Gal(E/K)$.
        $$
        |\Gal(E/F)|=[E:F]=[E:K]\times [K:F]=|Gal(E/K)|[K:F]=
        |H|[K:F]
        $$
        从而:$[\Gal(E/F):H]=[K:F]$.

        3.假设$H$是正规子群,则$\forall \varphi \in Gal(E/F)$:
        $$
        \varphi H \varphi^{-1}=H
        $$

        记$K=\mathcal{K}(H)$,$\phi\in H=Gal(E/K)$.
        
        从而:$\varphi \phi \varphi^{-1} \in Gal(E/\varphi(K))$。

        于是$Gal(E/\varphi(K))=\varphi H\varphi^{-1}=H$,
        从而$\varphi(K)=K$。$K$是正规扩张。

        假设$K=\mathcal{K}(H)/F$是正规扩张,
    \end{proof}
    $\forall g(x)\in \Q[x]$,$\exists f(x)\in Q[x]$:
    $$
    f(g(\alpha))=\alpha
    $$
\end{document}