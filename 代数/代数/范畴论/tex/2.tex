\ifx\allfiles\undefined

	% 如果有这一部分另外的package,在这里加上
	% 没有的话不需要
	
	\begin{document}
\else
\fi
\chapter{幺半范畴}
\section{基础定义}
\begin{definition}[幺半范畴]{}
	幺半范畴意指一组资料($\mathcal{V},\otimes,a,1,\iota$),其中:
	
	1.$\mathcal{V}$是一个范畴。

	2.$\otimes:\mathcal{V}\times \mathcal{V} \to \mathcal{V}$是一个二元函子,其在对象和态射集上定义的映射分别记为:$(X,Y)\mapsto X\otimes Y$和$(f,g)\mapsto f\otimes g$.

	3.$a$是函子范畴$\rm{Fct}(\mathcal{V}\times\mathcal{V}\times \mathcal{V},\mathcal{V})$中的同构:
	\begin{align}
		a:((\cdot \otimes \cdot)\otimes \cdot) \cong (\cdot \otimes (\cdot \otimes \cdot))
	\end{align}
	使得对于所有对象$X,Y,Z,W$,下图:
	\[\centering
	\begin{tikzcd}
	& {((X\otimes Y)\otimes Z)\otimes W} \\
	{(X\otimes(Y\otimes Z))\otimes W} && {(X\otimes Y)\otimes (Z\otimes W)} \\
	{X\otimes((Y\otimes Z)\otimes W)} && {X\otimes(Y\otimes(Z\otimes W))}
	\arrow["{a(X,Y,Z)\otimes \mathrm{id}_W}"{description}, from=1-2, to=2-1]
	\arrow["{a(X\otimes Y,Z,W)}"{description}, from=1-2, to=2-3]
	\arrow["{a(X,Y\otimes Z,W)}", from=2-1, to=3-1]
	\arrow["{a(X,Y,Z\otimes W)}"', from=2-3, to=3-3]
	\arrow["{\mathrm{id}_X\otimes a(Y,Z,W)}", from=3-1, to=3-3]
\end{tikzcd}\]
   交换。

   4.对象$1$称为幺元,相应的函子$1\otimes -$和$-\otimes 1$给出范畴$\mathcal{V}$到自身的等价。

   5.$\iota:1\otimes 1\to 1$是同构。
\end{definition}
 \ifx\allfiles\undefined
	
	% 如果有这一部分的参考文献的话,在这里加上
	% 没有的话不需要
	% 因此各个部分的参考文献可以分开放置
	% 也可以统一放在主文件末尾。
	
	%  bibfile.bib是放置参考文献的文件,可以用zotero导出。
	% \bibliography{bibfile}
	
	end{document}
	\else
	\fi