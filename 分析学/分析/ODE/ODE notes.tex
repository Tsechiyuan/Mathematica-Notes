\documentclass[UTF8]{ctexart}[a4paper,12pt]
\usepackage[thmmarks]{ntheorem}
\usepackage{amsmath}
\usepackage{amsfonts,amssymb} 
\usepackage{thmtools}
\usepackage[hmargin=2.5cm,vmargin=2.5cm]{geometry}
\usepackage{tikz-cd,tikz}
\usepackage{graphicx,float}
\usepackage{fancyhdr}
\usepackage{fourier-orns}


%声明环境
\newtheorem{example}{例}[section]              
\newtheorem{algorithm}{算法}[subsection]
\newtheorem{theorem}{定理}[section]            
\newtheorem{definition}{定义}[section]
\newtheorem{axiom}{公理}[section]
\newtheorem{property}{性质}[section]
\newtheorem{proposition}{命题}[section]
\newtheorem{lemma}[theorem]{引理}
\newtheorem{corollary}[theorem]{推论}
{
    \theoremheaderfont{\sffamily}
    \newtheorem*{remark}{注解} 
}
\newtheorem{condition}{条件}
\newtheorem{conclusion}{结论}[section]
\newtheorem{assumption}{假设}
{
\theoremstyle{nonumberplain}
\theoremheaderfont{\bfseries}
\theorembodyfont{\normalfont}
\theoremsymbol{\mbox{$\Box$}}
\newtheorem{proof}{证明}
}
%定义命令
\def\N{\mathbb{N}}
\def\Z{\mathbb{Z}}
\def\Q{\mathbb{Q}}
\def\R{\mathbb{R}}
\def\C{\mathbb{C}}
\def\S{\mathbb{S}}
\def\D{\mathbb{D}}
\def\H{\mathbb{H}}
%外测度
\def\outmQ{m_*(Q)}

%页眉设计
\renewcommand 
\headrule{
\hrulefill
\raisebox{-2.1pt}
{\quad{\FourierOrns M T S N}\quad}
\hrulefill}
\pagestyle{fancy}

%超链接红色
\usepackage[colorlinks,linkcolor=red]{hyperref}

\usepackage{enumerate}


\title{ODE Stduy Notes}
\author{颜成子游}
\begin{document}
\maketitle
\tableofcontents
\newpage
\section{引论}
这一章将给出这本书会涉及到的基础概念;以及几种最简单的微分方程的解法。最后讨论了复值函数的微分方程和它们的复值解,以及一些线性微分方程的最简单的知识。

\subsection{一阶微分方程式}

\textbf{微分方程}是指这样的方程:其中未知的是单变量或者多变量的(一个或者几个)函数,且方程中不仅含有函数本身,还含有它们的导函数;如果
未知数是多变量函数,则称为\textbf{偏微分方程};否则,则称为\textbf{常微分方程}。ODE即"Ordinary Differential Equation"。

首先讨论一阶的情况,此时,方程可以写为:
$$
F(t,x,\dot{x})=0
$$

由此,我们首先应该考虑$F$的定义域$B$。这里的定义域指的是三个变量$t,x.\dot{x}$所构成的空间。当$r_1<t<r_2$时,解的解:$\varphi(t)$必须满足以上的
定义域。

如果分离变量,由隐函数定理:

\begin{equation}\label{equ:basicequ}
    F(t,x,\dot{x})=0\Rightarrow \dot{x}=f(t,x)
\end{equation}

我们假定$f(t,x)$的定义域$\Gamma$是开集,并且$f$和$\displaystyle{\frac{\partial f}{\partial x}}$是连续的。这将有利于之后的研究。

从方程\eqref{basicequ}解出来的函数$\varphi(t)=x$所对应的曲线称为\textbf{积分曲线}。

\subsection{解的存在性和唯一性}
显然微分方程的解可以有无数多个的情况。因此我们应当讨论如何描述出微分方程所有的解的集合。我们不加证明给出\textbf{存在和唯一性定理}:

\begin{theorem}\label{thm:solexi}
    给定微分方程
    \begin{equation} \label{eq:spesol}
    \dot{x}=f(x,t) 
    \end{equation}
    假设函数$f(x,t)$在变量$(t,x)$平面$P$的某一个开集$\Gamma$上有定义,并且在整个开集$\Gamma$上函数$f$本身及其偏导数$\displaystyle{\frac{\partial f}{\partial x}}$
    都是$t,x$的连续函数;那么,我们断定:
    \begin{enumerate}[1]
        \item 对于集合$\Gamma$的任意一个点$(t_0,x_0)$,方程\eqref{eq:spesol}都存在解$x=\varphi(t)$,它满足:
        \begin{equation}
            \varphi(t_0)=x_0
        \end{equation}
        \item 如果方程有两个解,只要它们在某一个值$t=t_0$处是一样的,也就是如果满足
        $$
        \psi(t_0)=\chi(t_0)  
        $$
        那么对于两者都有定义的变量$t$值,它们是恒等的。
    \end{enumerate}
\end{theorem}

证明留在很久后给出。

定理\ref{thm:solexi}的几何含义:通过集合$\Gamma$的每一点$(t_0,x_0)$,有且只有一条满足方程积分曲线经过方程\eqref{basicequ}。

在实际问题中,我们必须考虑到$t$的定义区间。定理只表明在定义区间的相同地方,不同解的值会是相同的。但是并没有说明定义区间必须完全相同。

记$\psi(t)$和$\chi(x)$的定义区间分别为:$r_1<t<r_2$,$s_1<t<s_2$,并且前者完全包含了后者,那么我们说$\psi(t)$是$\chi(x)$的延拓。

对于同一个初值,其拥有最大定义区间的解称为\textbf{不可延拓}的解。容易看出来,如果使用图来描述,那么不可延拓的解是集合$\Gamma$里面所能包含的最长曲线。

任何积分曲线$x=\varphi(x)$在它的每一个点的切线可以写为:
$$
x=x_0+f(t_0,x_0)(t-t_0)
$$

这是微分方程\eqref{equ:basicequ}的几何解释。
\subsection{一些初等的求积分方法}
这些方法在最后的理解上都不难。我们最需要关注的是解的存在区域(a<t<b)此类。
一种可能出现的情况是,几个解可以用一个式字表示,但是由于他们的存在范围是分开不连通的,因而不是一个解。
\subsubsection{$\dot{x}=f(t),t\in (a,b)$}
无慈悲的讲,这个方程小学生都会。唯一要注意的是,解写为:
$$
x=\int_{t_0}^t f(\tau)d\tau+c
$$
$t_0 \in (a,b)$是最合适的。此时可以完美解决定义域的问题。
\subsubsection{$\dot{x}=g(x),x \in(a,b)$}
不考虑定义域的话,这个方程将是简单的。这里我们假定$g(x)$拥有连续的导函数,以便于使用唯一性定理。

此时直接积分:
$$
t-t_0=\int_{x_0}^x \frac{d\xi}{g(\xi)}
$$
这要求$g(x)\neq 0$这是自然的。因为如果有$x_0$:$g(x_0)=0$。那么$x =x_0$是一个解。根据存在唯一性,此时$x=x_0$是唯一解。同时,$g(x)\neq 0$也表明$g(x)$不变号。于是上面的方程可以给出反函数。

因此该方程的初值决定了$g(x)$的正负性。在不同的正负区间,解的曲线是不同的。

\subsubsection{$\dot{x}=f(t)g(x),a<t<b,c<x<d$}
直接分离变量积分即可。注意我们要求$f$连续,$g$连续可导。同样要求$g \neq 0$.
这样同样可以给出反函数。
\subsubsection{齐次方程:$\dot{y}=h(\frac{y}{t})$}
换元$x=\frac{y}{t}$。即可得到一个变量可分离的方程。此时$x$的范围为$h$所规定的范围。
\begin{example}
    求解:
    $$
    \dot{x}=\frac{x^2-1}{2}
    $$   
\end{example}
\begin{proof}
    我们把$x$的存在区间分为$(-\infty,-1),(-1,1),(1,+\infty)$。容易知道$-1,1$时解为$x=-1,x=1$。
    
    此时$\frac{x^2-1}{2}$符号固定。可以直接分离参量求解。我们得到:
    $$
    x=\frac{1+ce^{t}}{1-ce^t}
    $$

    $c$的取值决定了解的图像。若$c>0$,其确定了谅解,一个定义在$-\infty<t<-\ln c$上,一个定义在$-\ln c<t<+\infty$上。取值分别为$x>1$和$x <-1$。

    若$c>0$,其确定了一个解。$x$取值为$-1<x<1$。
\end{proof}
最后,我们说明如果$\dot{x}=f(x,t)$中的$f(x,t)$不满足在区域$\Gamma$内连续,对$x$偏导连续,那么唯一性可能不存在。
\begin{example}
    $$
    \dot{x}=3x^{\frac{2}{3}}
    $$
\end{example}
\begin{proof}
    这个方程的在初值$(0,0)$处可以有两个解:
    $$
    x=0
    $$
    $$
    x=t^3
    $$
\end{proof}
\subsection{存在性与唯一性定理的叙述}
我们在本节展开叙述存在与唯一性定理的一般情况。证明不会给出。
\begin{theorem}[存在性与唯一性定理]
    考虑常微分方程组
    $$
    \dot{x}_i=f_i(t,x_1,x_2,\dots,x_n),i=1,2,\dots,n
    $$
    设$f_i$有一个公共的区域$\Gamma$作为定义域。这是一个开集。

    若$f_i$与$\frac{\partial f_i}{\partial x_j}$都连续,那么对于$\Gamma$中的每个初值$(t_0,\vec{x}_0)$,都存在唯一的积分曲线满足其是微分方程的解。
    
\end{theorem}

我们首先注意到,若$G\subseteq \mathbf{R}$为开集,则$f^{-1}(G)$为$F_{\sigma}$集
事实上,由于$R$中的开集$G$是可数个构成区间的并,故不妨设$G$是一个开区间$(a,b)$.而我们知道
\[ \{x|f_{k}(x)\geq a+\varepsilon\}  \]
是闭集,由于$f_{k}(x)$的连续性,从而
\[ \{x|f(x)>a \}=\bigcup_{n=1}^{\infty}\bigcup_{m=1}^{\infty}\bigcap_{k=m}^{\infty}\left\{x|f_{k}(x)\geq a+\frac{1}{n}\right\} \]
是$F_{\sigma}$集.同理$\{x|f(x)<b\}$也是$F_{\sigma}$集合.而他们的交集
\[ f^{-1}((a,b))=\{x|f(x)>a \}\bigcap\{x|f(x)<b\} \]
也是$F_{\sigma}$集.为了证明$f(x)$的连续点是稠密的$G_{\delta}$集,我们只要证$f(x)$的不连续点是没有内点的$F_{\sigma}$集.记$D(f)$为$f(x)$的不连续点集合,任意的$a\in D(f)$,存在$p,q\in \mathbf{Q}$使得
\[ p<f(a)<q\]
\[ \Leftrightarrow\qquad a\in f^{-1}((p,q)) \]
\[ a\notin f^{-1}(R-(p,q)) \]
而存在点列$\{a_{n}\}$使得
\[ \lim_{n\to\infty}a_{n}=a,\qquad f(a_{n})\notin(p,q) \]
\[ \Leftrightarrow\qquad a_{n}\notin f^{-1}(p,q) \]
\[ \Leftrightarrow\qquad a_{n}\in f^{-1}(R-(p,q)), \lim_{n\to\infty}a_{n}=a \]
于是
\[ a\in \overline{f^{-1}(R-(p,q))}-f^{-1}(R-(p,q)) \]
所以
\[ D(f)=\bigcup_{p<q,\text{$p,q$为有理数}}[\overline{f^{-1}(R-(p,q))}-f^{-1}(R-(p,q))] \]
而$f^{-1}(R-(p,q))$为$G_{\delta}$集,故
\[ \overline{f^{-1}(R-(p,q))}-f^{-1}(R-(p,q)) \]
为$F_{\sigma}$集.并且$\overline{f^{-1}(R-(p,q))}-f^{-1}(R-(p,q))=\partial f^{-1}(R-(p,q))$是没内点的.它是可数个无内点的闭集的并,所以$D(f)$也是可数个无内点的闭集的并。由Baire定理,$D(f)$是无内点的$F_{\sigma}$集.所以$R^{n}\setminus D(f)$是稠密集合.

 

从这个结论可以看出,导函数的连续点也是稠密的,只要注意到

\[ f'(x)=\lim_{n\to\infty}\frac{f\left(x+\frac{1}{n}\right)-f(x)}{\frac{1}{n}} \]

就好.
\end{document}