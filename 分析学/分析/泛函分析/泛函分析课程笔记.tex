\documentclass[UTF8]{ctexart}[a4paper,12pt]
\usepackage[thmmarks]{ntheorem}
\usepackage{amsmath}
\usepackage{amsfonts,amssymb} 
\usepackage{thmtools}
\usepackage[hmargin=2.5cm,vmargin=2.5cm]{geometry}
\usepackage{tikz-cd,tikz}
\usepackage{graphicx,float}
\usepackage{fancyhdr}
\usepackage{fourier-orns}
\usepackage{quiver}

%声明环境
\theorembodyfont{\rmfamily}
\newtheorem{example}{例}[section]              
\newtheorem{algorithm}{算法}[subsection]
\newtheorem{theorem}{定理}[section]            
\newtheorem{definition}{定义}[section]
\newtheorem{axiom}{公理}[section]
\newtheorem{property}{性质}[section]
\newtheorem{proposition}{命题}[section]
\newtheorem{lemma}[theorem]{引理}
\newtheorem{question}{问题}[section]
\newtheorem{corollary}[theorem]{推论}
{
    \theoremheaderfont{\sffamily}
    \newtheorem*{remark}{注解} 
}
\newtheorem{condition}{条件}
\newtheorem{conclusion}{结论}[section]
\newtheorem{assumption}{假设}
{
\theoremstyle{nonumberplain}
\theoremheaderfont{\bfseries}
\theorembodyfont{\normalfont}
\theoremsymbol{\mbox{$\Box$}}
\newtheorem{proof}{证明}
}
%定义命令
\def\N{\mathbb{N}}
\def\Z{\mathbb{Z}}
\def\Q{\mathbb{Q}}
\def\R{\mathbb{R}}
\def\C{\mathbb{C}}
\def\S{\mathbb{S}}
\def\D{\mathbb{D}}
\def\H{\mathbb{H}}
\def\F{\mathbb{F}}


%页眉设计
\renewcommand 
\headrule{
\hrulefill
\raisebox{-2.1pt}
{\quad{\FourierOrns M T S N}\quad}
\hrulefill}
\pagestyle{fancy}

%超链接红色
\usepackage[colorlinks,linkcolor=red]{hyperref}

\usepackage{enumerate}


\title{泛函分析课程笔记·2023春}
\author{整理者:颜成子游/南郭子綦}
\begin{document}
\maketitle
\tableofcontents
\section{课程简介与参考书}
参考书:

1.刘炳初《泛函分析》

2.武大。刘培德 《泛函分析基础》

3.哈工大-武大 许全华,马涛,尹智《泛函分析讲义》

4.清华 步尚全 《泛函分析习题集》

5.英文:J.Conway GTM /W Rudin
\section{度量空间}
\subsection{基本概念}
\begin{definition}[度量空间(距离空间)]
    天天用,已经不想写了
\end{definition}
\begin{example}
    $\R^n,\C^n$都是距离空间。距离定义为$d(x,y)=(\displaystyle \sum_{i=1}^N|x_i-y_i|^2)^{1/2}$.因为有三角不等式;
    $$
    |\sum_{k=1}^N a_kb_k|^2 \leq (\sum_{k=1}^N a_k^2)(\sum_{k=1}^N b_k^2)
    $$
\end{example}
\begin{example}[描述性的”一致收敛“距离化]
    $[a,b]$上连续函数的全体$C[a,b]$:$d(f,g)=\sup_{t \in [a,b]}|f(t)-g(t)|$, $\forall f,g \in C[a,b]$.
\end{example}
\begin{example}
    集合$S=\{\{a_n\}_{n \in \N},a_n \in\R\}$。设$x=(x_k)_{k=1}^\infty$,$y=(y_k)_{k=1}^\infty \in S$.其距离可定义为:
    $$
    d(x,y)=\sum_{k=1}^\infty \frac{1}{2^k}\frac{|x_k-y_k|}{1+|x_k-y_k|}
    $$

    这是对数列列逐点收敛的刻画。因为会被$\dfrac{1}{2^k}$给控制住。
\end{example}
\begin{example}
    $S[a,b]$记为几乎处处有限的可测函数全体。定义:
    $$
    d(f,g)=\int_a^b \frac{|f(t)-g(t)|}{1+|f(t)-g(t)|}dt
    $$
    
    若$f_n \to g$,$d(f_n,g) \to 0$.这是依测度收敛的距离化。即$\forall \delta >0$,$m(\{t \in [a,b]:|f_n(t)-g(t)| \geq \delta\}) \to 0$。
\end{example}
当然可以用距离定义收敛:
\begin{definition}
    $\{x_n\}\subset X$,$x_0 \in X$。称$\{x_n\}$以$x_0$为极限,若$\{x_n\}$收敛到$x_0$,$x_n \stackrel{d}{\longrightarrow} x_0$,若$d(x_n,x_0)\to 0$.
\end{definition}
\begin{proposition}
    若$x_n \to x_0$,$y_n \to y_0$,则$d(x_n,y_n)\to d(x_0,y_0)$。这说明$d:X \times X \to \R$是二元连续的。
\end{proposition}
\begin{definition}
    设$X_1$和$X_2$是两个度量空间,$f:X_1 \to X_2$。若$d_1(x,y)=d_2(f(x),f(y)),\forall x,y \in X_1$,则称其为$X_1$到$X_2$的等距嵌入。若$f$还是满射,则$(X_1,d_1)$和$(X_2,d_2)$等距。
\end{definition}
\begin{proposition}
    等距嵌入一定是连续嵌入。等距嵌入一定是单射,因而被视为嵌入。
\end{proposition}

等距的条件是比较强的。我们给出弱一点的条件:
\begin{definition}
    $f:X_1 \to X_2$是满射。且存在$A,B \geq 0$,$Ad_1(x,y)\leq d_2(f(x),f(y)) \leq Bd_1(x,y),\forall x,y \in X_1$。
\end{definition}

\subsection{度量空间中的点集和可分性}
用$\overline{B}(x_0,r)$表示闭球,$B(x_0,r)$表示开球。值得注意的是,下面的命题是错误的!
\begin{proposition}[错误的命题]
    $\overline{B(x_0,r)}=\overline{B}(x_0,r)$
\end{proposition}
边界取两次后就保持不变,聚点可以一直严格减少且不为空集。

\begin{definition}
    $(X,d)$是度量空间。若$X$中存在一个可列的稠密子集,则称$X$是可分的。
\end{definition}
\begin{example}
    $C[a,b]$,$d(f,g)=\sup_{t \in [a,b]}|f(t)-g(t)$是一个可分集合。只需要使用有理系数的多项式来做即可。
\end{example}
\begin{example}
    $L^\infty=\{\{a_n\}\}$且$\{a_n\}$有界。距离定义为$d_\infty(x,y)=\sup_{n \in \N}|x_n-y_n|$。则该空间不可分。原因是子集$A=\{\{a_n\}\}$,$a_n=0$或者$1$。该集合不可数且每个点之间的距离为$1$.
\end{example}
\begin{proposition}
    $(X,d)$不可分等价于$\exists \delta>0$,$\exists$不可数集$A \subset X$,使得$\forall x \neq y \in A$,$d(x,y) \geq \delta$。
\end{proposition}
\begin{proof}
    充分性显然。关键是必要性。
\end{proof}
\subsection{距离空间的完备性与压缩映像原理}
略
\section                                                                                                       {赋范线性空间}
\subsection{基本概念}
\begin{definition}
    $X$是线性空间(实或复),映射$\|\cdot\|:X \to \R$满足:
    \begin{enumerate}
        \item $\| x\|\geq 0$且只在$x=0$时取等。
        \item $\|\alpha \cdot x\|=|\alpha|\cdot \|x\|$
        \item $\|x+y\| \geq \|x\|+\|y\|$。
    \end{enumerate}
    称为$\|\cdot \|$为$X$上的一个范数。$(X,\|\cdot\|)$称为赋范空间。
\end{definition}
显然范数诱导度量:
$$
\forall x,y \in X, d_{\|\cdot\|}(x,y):=\|x-y\|
$$
从而我们有一系列度量空间的概念:收敛,柯西列,完备等等。

\begin{definition}
    完备的赋范空间$(X,\| \cdot\|)$被称为Banach空间。
\end{definition}
\begin{proposition}
    $X$是赋范空间,则:

    (1)范数是连续函数,即:
    $$
    x_n \to 0 \Rightarrow \|x_n\| \to \|x\|
    $$

    (2)线性运算都是多元连续的。
\end{proposition}
\begin{proof}
    只需要调教三角不等式即可。并不困难。
\end{proof}
\begin{proposition}
    $(X,\|\cdot\|)$赋范空间,若$X$完备且级数$\sum_{k=1}^\infty \|x_k\|$收敛,则$\sum_{k=1}^\infty x_k$收敛。

    反之,若上述收敛性质对所有收敛$\sum_{k=1}^\infty \|x_k\|$成立,则$(X,\|\cdot\|)$是Banach空间。
\end{proposition}
\begin{proof}
    证明的要点是取子列和三角不等式。构造收敛的子列+本身是Cauthy列即可说明收敛。而子列的收敛则来源于对应范数列的收敛。
\end{proof}
\begin{definition}
    凸集是指线性空间$X$这样的一个子集$A$,对于$\forall x,y \in A $,和$\forall \alpha \in (0,1)$,都有$\alpha x+(1-\alpha)y \in A$。

    凸包是指$X$这样的一个子集:为$X$钟包含$A$的最小凸集。由于凸集的交显然是凸集,则可以写为:
    $$
    \mathrm{Co}(A)=\bigcap_{A\subset F \subset X}F
    $$
    也可以从内到外写凸集:
    $$
    \mathrm{Co}(A)=\{z \in X|\exists n \in \N,\alpha_k \geq 0,1\leq k \leq n,\sum_{k=1}^n \alpha_k=1,z=\sum_{k=1}^n \alpha_k x_k\}    
    $$
\end{definition}
\begin{example}
    $l^\infty$是有界数列的全体:
    $$
    \|(\xi_k)\|_\infty=\sup_{k \in \N}|\xi_k|
    $$
    
    容易证明这是一个范数,并且完备,因而是Banach空间。

    设$l^p=\{(\xi_k)|\displaystyle \sum_{k=1}^\infty |\xi_k|^p<\infty\}$。定义范数为:
    $$
    \|(\xi_k)\|_p=(\sum_{k=1}^\infty|\xi_k|^p)^{1/p}
    $$
    这也是完备的度量。我们先给出逐点收敛的极限:
    $$
    |\xi_k^{(n)-\xi_k^{(m)}}| \leq \|x_n-x_m\|_p
    $$
    因此根据逐点收敛的Cauthy定理,自然有:
    $$
    \lim_{n \to \infty}\xi_k^{(n)}=\xi_k^{(0)}
    $$
    从而我们构造数列:$x_0=(\xi_k^{(0)})$。要说明:
    $$
    x_0 \in l^p\quad \|x_n-x_0\| \to 0
    $$

    $\forall \epsilon >0$,$\exists N$,$\forall n.m>N,\|x_n-x_m\|_p <\epsilon$。
\end{example}
\begin{example}[变差空间]
    空间$V[a,b]$是$[a,b]$上有界变差的函数的空间。即:
    $$
    \sup\{\sum_{k=1}^n |f(x_k)-f(x_{k-1})||\forall n \in \N,\forall a=x_0<x_1<\dots<x_n=b\}
    $$是有限的。

    定义该空间上的范数:
    $$
    \|f\|=V_a^b(f)=\sup\{\sum_{k=1}^n |f(x_k)-f(x_{k-1})||\forall n \in \N,\forall a=x_0<x_1<\dots<x_n=b\}
    $$
    \end{example}

    定义有界变差数列为:
    $$
    (a_n):\sum_{k=1}^\infty |a_{k+1}-a_k| <+\infty
    $$
    \begin{proposition}
        给定各项非零的数列$(x_n)$,定义:
        $$
        A=\{(a_n)|\sum_{n=1}^\infty a_nx_n \quad \mathrm{convergence}\}
        $$
        若数列$(y_n)$满足$\sum_{n=1}^\infty y_nb_n$对于任意的$(b_n) \in A$都收敛,则存在一个有界变差数列$(\upsilon_n)$满足$y_n=\upsilon_n x_n,\forall n$。反之亦然。
    \end{proposition}
    \begin{proposition}
        给定非零数列$(x_n),(y_n)$。定义$A=\{(a_n)|a_nx_n,a_ny_n \quad \mathrm{convergence}\}$。则$(z_n)$满足对$\forall b_n \in A$有$z_nb_n$收敛当且仅当存在有界变差数列$(v_n^{(1)})$和$(v_n^{(2)})$使得:
        $$
        z_n=v_n^{(1)}x_n+v_n^{(2)}y_n
        $$
    \end{proposition}
    \subsection{$L^p[0,1]$空间}
    几乎处处相等的函数将被视为同一元。

    \subsection{赋范空间进一步的性质}
    \begin{definition}[子空间]
        $(X,\|\cdot \|)$是赋范空间,子空间$X_1 \subset  X$。沿用$\|\cdot \|$作为范数,称$(X_1,\| \cdot \|)$为$(X,\|\cdot \|)$的子空间。若$X$是闭集,称为闭子空间。
    \end{definition}
    \begin{example}
        设$l^{\infty}$是有界数列全体。$\|x\|=\sup_{n \in \N}|x_n|$。考虑其子空间$c$为所有收敛数列的全体。其为$l^{\infty}$的闭集,因为数列一致收敛意味着极限列也是收敛的。

        考虑$c_0$为收敛于$0$的数列全体。同样定义范数为$\|x\|=\sup_{n \in \N}|x_n|$。则$c_0$是$l^{\infty}(c)$的闭子空间。从而$c_0 \subset c \subset l^{\infty}$

        由于$l^{\infty}$是完备空间,并且完备性是闭遗传的,因此$c$和$c_0$也是Banach空间。

        继续考虑更小的子空间$c_{00}$为有限项非零的数列全体,范数仍定义:$\|x\|=\sup_{n \in \N}|x_n|$。同样其$c_{00}$是子空间。但其不是闭子集。
    \end{example}

    相同于度量空间的完备化,赋范空间定义距离$d(x,y)$为$|x-y|$。沿用度量空间的完备化$\tilde{X}$:
    \begin{definition}
        设$X$中任意柯西列$\overline{x},\overline{y}$。可以定义自然的加法和数乘运算。定义其范数为:
        $$
        \| \overline{x}\|=\lim_{n \to \infty}\|x_n\|
        $$
        此时$(\tilde{X},\|\|)$是一个Banach空间。
    \end{definition}
    完备化后,$X$等距同构于$\tilde{X}$的一个稠子集。

    \textbf{思考问题}:回忆刚才的例子,试证明:$c$不等距同构于$c_0$的任何子空间。关键在于$c$内有个特殊的元$(1)$(恒为1),其与$c$中其它元产生的关系无法嵌入到$c_0$里面。

    在赋范空间的意义下,我们默认两个空间的同构代表$X$首先与$Y$线性同构,以及存在$T:X \to Y$作为线性双射满足:
    $$
    \exists A,B >0, s.t. A\|x\| \leq \|T(x)\| \leq B\|x\|, \forall  x \in X
    $$

    \begin{theorem}[Mazur]
       if $V$ and $W$ are normed spaces over $\R$ and the mapping:
       $$
       f : V → W 
       $$
       is a surjective isometry, then $f$ is affine. If $f(0)=0$,$f$ is linear isomorphic mapping.
    \end{theorem}

    我们目前有两种空间关系:同构和等距同构。我们接着引入商空间:
    \begin{definition}[赋范线性空间的商空间]
        $X$是线性空间,$M$是$X$的线性子空间。定义$X/M$是线性空间的商。我们在上面定义范数:
        $$
        \|\tilde{x}\|=\inf_{ y \in \tilde{x}}\|y\|=d(x,M)=\inf_{y \in M}\|x-y\|
        $$
    \end{definition}
     上述定义自然的不能再自然了。
     \begin{theorem}
        $X$是Banach空间,$M$是$X$的闭子空间,则$X/M$是Banach空间。
     \end{theorem}
     \begin{proof}
        设$\{\tilde{x_n}\}$是一个$X/M$中的Cauthy列。那么存在子列:
        $$
        \|\widetilde{x_{n_{k+1}}}-\widetilde{x_{n_k}}\| \leq 1/2^k
        $$
        根据商范数的定义,则存在:
        $$
        y_k \in \widetilde{x_{n_{k+1}}}-\widetilde{x_{n_k}}=\widetilde{x_{n_{k+1}}}-\widetilde{x_{n_k}} \quad \|y_k\| <1/2^k
        $$

        $\sum y_k$是收敛的(其有限项是容易计算的)。从而任取$x_{n_1} \in \tilde{x_{n_1}}$,有:
        $$
        x_{n_1}+y_1+\dots+
        $$
        收敛于$x$。

        其实$\tilde{x_{n_k}}$收敛于$\tilde{x}$。记$S_k=x_{n_1}+y_1+\dots+y_k$,由于$x_{n_1} \in \tilde{x_{n_1}}$,$y_k \in \tilde{x_{n_{k+1}}}-\tilde{x_{n_k}}$,:
        $$
        \|\tilde{x}-\tilde{x_{n_{k+1}}}\| \leq \|x-S_k\| \to 0
        $$
     \end{proof}
     \subsection{赋范空间的基}
     我们不介绍Hamel基,其只是理论意义上存在。由于很多Hamel基都是不可数的,因而难以进行加减运算。

     我们介绍Schauder基:

     设$X$是赋范线性空间,$\{x_n\}$是$X$的可数子集。称其为$X$的Schauder基。若$\forall x \in X$,$\exists ! (\alpha_n)\subset \R$使得:
     $$
     x =\sum_{n=1}^\infty \alpha_n x_n=\lim_{N \to \infty}\sum_{n=1}^N \alpha_n x_n
     $$

     显然若$X$有Schauder基,则其可分。但可分的赋范线性空间空间$X$不一定有Schauder基。可见Mazur和Enflo的故事。
     \subsection{习题讲解}
    \begin{proposition}
        $(X,d)$是度量空间不可分当且仅当存在不可数的子集$\Gamma$使得存在$\delta>0$,满足$\forall x\neq y \in \Gamma, d(x,y)\geq \delta$
    \end{proposition}
  \begin{proof}
    若$X$可分,$\{x_n\}$是稠密,则$X=\bigcup_{n=1}^\infty B(x_n,\delta/3) \supset X$。从而至少有一个$B(x_{n_0},\delta/3)$与$\Gamma$相交。$\Gamma$不可数,则$\exists x \neq y$满足$x,y \in B(x_{n_0},\delta/3)$,矛盾!

    若$X$不可分。对于$n$是正整数,$A \subset X$称为$1/n$分离集,若$\forall x \neq y \in A$,有:
    $$
    d(x,y)\geq 1/n
    $$
    设$\mathcal{F}_n$是所有$1/n$分离集的全体。使用包含关系作为偏序。这显然是一个满足Zorn引理的集族,故其存在极大元$A_n$。

    设$B=\bigcup_{n=1}^\infty A_n$。我们断言$B$是不可数的集合。若不是,任取$y \notin B$,根据Zorn引理的极大元断言,则对于$\forall n$,$d(y,A_n)<1/n$。从而$B$是$X$的一个稠子集。由于$X$是不可分的,从而$B$一定是不可数的。

    从而存在$A_n$是不可数集。这个$A_n$就是我们想要找的集合。
  \end{proof}
  \begin{proposition}
    任何线性空间都存在范数(内积)
  \end{proposition}
  \begin{proof}
    Hamel基。
  \end{proof}
  \begin{proposition}
    $[a,b]$上多项式空间上不存在完备范数。
  \end{proposition}
  \begin{proof}
    问题实际上是集合$c_{00}=\{(\alpha_n)_{n=1}^\infty|\exists N ,\forall n>N ,\alpha_n=0\}$

    若其是Banach空间。考虑$E_n=\mathrm{Span}\{e_k\}_{k=1}^n$。则$c_{00}=\bigcup_{n=1}^\infty E_n$。

    根据完备性,存在$\overline{E_{n_0}}=E_{n_0}$有内点。若$B(x_0,r_0)\subset E_{n_0}$。平移,$B(\theta,r_0)\subset E_{n_0}$即$c_{00}=E_{n_0}$。
  \end{proof}
  \begin{proposition}[Mazur-Ulam]
    $X,Y$是Banach空间.$T:X \to Y$满等距,$\|T(x)-T(y)\|=\|x-y\|$。则$T$是仿射的。若$T(0)=0$,则$T$是线性的。
  \end{proposition}
  \begin{proof}
    不妨设$T(0)=0$。如果能证明$T$保中点,就能根据二分性和连续性说明其是仿射。
  \end{proof}
  \begin{remark}
    不能去掉满的条件。设$T:\R \to l^{(2)}_\infty, x\mapsto (x,\sin x)$。则:
    $$
    \|(x,\sin x)-(y,\sin y)\|_{\infty}=\max(|x-y|,|\sin x,\sin y|)=\|x-y\|
    $$
    $T$是不满的映射,但是保距。

    设$S:l^{(2)}_\infty \to R, (x,y)\mapsto x$。$S$是线性,有界的。$S \circ T=\mathrm{id}_{\R}$.因此$T$虽然不线性,但是有线性的左逆。
  \end{remark}
  下面的定理自然可以推出Mazur-Ulam定理。
  \begin{theorem}[Figiel定理]
    $X,Y$是Banach空间,$T:X \to Y$是等距映射。$T(0)=0$。则存在$S:\overline{\mathrm{Span}T(X)} \to X$线性且$\|S\|=1$满足:
    $$
    ST=\mathrm{id}_X
    $$
  \end{theorem}

  对Mazur-Ulam定理的思考还可以对等距进行操作。等距的映射实际上是比较强的.
  \begin{remark}
    $X,Y$是Banach空间,$T:X \to Y$是满射且Lipschitz等价。即$\exists A,B >0$:
    $$
    A\|x-y\| \leq \|Tx-Ty\| \leq B\|x-y\|, \forall x,y \in X
    $$
    问是否有$X$与$Y$同构?
  \end{remark}
  这个问题目前仍然是open的。在不可分的Banach空间中,已经找到了反例。而对于可分的Banach空间,甚至进展停留在固定$X$或者$Y$都无法得到很好的解决。例如,当$X=c_0$时问题被证明是成立的。以及$X=l_1$,$Y=X^*$时,也被证明成立了。但是其他情况完全是open的。
\section{有界线性算子}
\subsection{有界线性算子}
设$X,Y$是Banach空间,$T:X \to Y$满足线性,连续,则称$T$是$X$到$Y$的有界线性算子。

为什么这个概念的名字里有"bounded"却没有"continuous"?

\begin{proposition}
    设$T:X \to Y$只是线性的。若$T$在零点连续,则$T$在所有点都是连续的。
\end{proposition}
\begin{proposition}[Lipschitz]
    若$T$是连续线性的算子,则存在$C>0$,使得:
    $$
    \|T(x)\|_Y \leq C\|x\|_X, \forall x \in X
    $$
\end{proposition}
\begin{proof}
    若不存在这样的$C$,则存在$\{x_n\}$满足:$\|T(x_n)\| >n \|x_n\|$。于是:
    $$
    \|T(\frac{1}{\sqrt{n}}\frac{x_n}{\|x_n\|}) \|>\sqrt{n}
    $$
    左边的表达式趋近于$0$,但是右边却表明极限不可能为$0$。从而这与连续性矛盾。
\end{proof}
     我们当然关注上述命题中$C$的最优选择。即我们关心:
     $$
     \inf \{C \geq 0|\|T(x)\| \leq C\|x\|, \forall x \in X\}=\sup_{\|x\|=1}\|T(x)\|=\sup_{\|x\| \leq 1}\|T(x)\|
     $$

     上述量被定义为算子$T$的\textbf{(算子)范数}。用$(B(X,Y),\|\cdot\|)$表示所有有界线性算子的空间。

\begin{example}
    考虑$B(\C^N,\C^N)=M_{N \times N}(\C)$。$T$是正规矩阵,则$\|T\|$是$T$的特征值的绝对值中的最大值。
\end{example}
\begin{proposition}
    结合算子范数,$B(X,Y)$成为赋范空间。
    
    设$X$是赋范空间。则$Y$是Banach空间等价于$B(X,Y)$是Banach空间。
\end{proposition}
\begin{proof}
    命题第一句话是比较显然的。我们直接考虑第二句话。

    设$B(X,Y)$是Banach空间。$(y_n)\subset Y$是柯西列。(?)(注记:老师在这里卡住了,并且表示后面会讲)

    设$Y$是完备的。设$(T_n)$是$B(X,Y)$中的柯西列。即:
    $$
    \|T_n(x)-T_m(x)\|=\|(T_n-T_m)(x)\| \leq \|T_n-T_m\| \cdot \|x\|
    $$
    从而$\{T_n(x)\}$也是柯西列。由于$Y$完备可知:$\lim T_n(x):=T_0(x)$.

    由定义,$T_0(x)$线性。

    任意$\epsilon>0,\exists N$,$\forall n,m >N$,有:
    $$
    \|T_n-T_m\|<\epsilon
    $$
    翻译为:
    $$
    \|T_n(x)-T_m(x)\| \leq \epsilon \|x\|
    $$
    令$m \to \infty$,则:
    $$
    \|T_n(x)-T_0(x)\| \leq \epsilon \|x\|
    $$
    于是$T_n-T_0 \in B(X,Y)$且$\|T_n-T_0\| <\epsilon$。
\end{proof}

设$K=\R$或者是$\C$,则称$B(X,K)$是共轭空间,是所有有界线性泛函的全体。记为$X^*$。

\textbf{remark}:还讲了逐点收敛(强收敛)。笔者此处未作记录。
\subsection{Banach-Steinhaus定理}
\textbf{remark}:关于定理的引入没有做记录。

\begin{theorem}
    设$\{T_\alpha\}$是Banach空间$X$上到赋范空间$X_1$的有界线性算子族。如果有$\sup_\alpha \|T_\alpha(x)\| <\infty, \forall x \in X$,则$\sup_{\alpha \in I}\|T_\alpha\| < \infty$.
\end{theorem}
\begin{proof}
    证明主要是利用了Baire纲定理。

    设$p(x)=\sup_{\alpha \in I}\|T_\alpha x\|$。记$M_k=\{x \in X:p(x)\leq k\}$是一个闭集。则:
    $$
    X=\bigcup_{k=1}^\infty M_k
    $$
    由于$X$完备,则至少有$M_{k_0}$存在内点。

    未记录完所有的证明。
\end{proof}
我们介绍Banach-Steinhaus定理的应用——Fourier级数的发散性。

设$C_{2\pi}$是以$2\pi$为周期的连续函数。这显然可以只考虑$C[-\pi,\pi]$且$f(-\pi)=f(\pi)$。
定义范数为$\|x\|=\max_{t\in R}|x(t)|$。则$x(t)$可以被展开为Fourier级数:
\begin{align}
    x(t) \sim a_0/2+\sum_{k=1}^\infty (a_k\cos kt+b_k \sin kt)
\end{align}
前$n+1$项部分可以写为:
\begin{align}
    a_0/2+ \sum_{k=1}^n (a_k\cos kt+b_k \sin kt)=\int_{-\pi}^\pi x(s)K_n(s,t)ds
\end{align}
其中$K_n(s,t)=\dfrac{\sin(n+1/2)(s-t)}{2\pi \sin 1/2(s-t)}$是核函数。

若Banach空间$X$有Schauder基,则$X$的任意稠子空间也有Schauder基。
\subsection{开映射定理与闭图像定理}
\begin{definition}[逆算子]
    $T \in B(X,Y)$,若存在$S \in B(Y,X)$满足:$S \circ T=\mathrm{id}_X$,$T \circ S=\mathrm{id}_Y$,则称$T$可逆,$T^{-1}=S$。
\end{definition}
\begin{theorem}
    $X$是Banach空间,$T\in B(X)$。若$\|T\|<1$,则$I-T$是可逆的且:
    \begin{align}
        \|(I-T)^{-1}\| \leq \dfrac{1}{1-\|T\|}
    \end{align}
\end{theorem}
\begin{proof}
    本质是等比数列求和。略。
\end{proof}
\begin{corollary}
    $X$是\rm{Banach}空间,$(x_n)$是\rm{Schaduer}基。对于$X$的任意稠子空间$X_0$,存在$(z_n)\subset X_0$使得$(z_n)$是$X$的\rm{Schauder}基。
\end{corollary}
\begin{proof}
    
\end{proof}
\begin{theorem}[扰动]
    $X$是\rm{Banach}空间,$(x_n)$是\rm{Schaduer}基。相应的系数泛函记为$f_n \in X^*$.对于一列$(z_n)$,只要满足:
    \begin{align}
        \sum_{n=1}^\infty \|f_n\|\|z_n-x_n\|<1
    \end{align}
    则$(z_n)$是\rm{Schaduer}基。
\end{theorem}
\begin{proof}
    构造$T=\sum_{n=0}^\infty f_n(x)(x_n-z_n)$。显然$I-T$可逆。于是:
    \begin{align}
        y=(I-T)((I-T)^{-1}y)=\sum_{n=0}^\infty f_n((I-T)^{-1})(y)z_n \forall y \in X
    \end{align}
    唯一性是显然的,可以用$I-T$保证。
\end{proof}
不是所有的可分Banach空间都有Schaduer基。但下面这个问题是耐人寻味的:
\begin{proposition}
    $X$是可分的Banach空间,则存在稠密的子空间$X_0$且拥有Schaduer基。
\end{proposition}
这个问题的答案是肯定的。证明暂时不给出。
\subsection{闭图像定理}
$X,Y$是赋范空间。$T:X \to Y$是线性算子。在乘积空间$X \times Y$中,记$G(T)=\{(x,Tx)\in X \times Y:x \in X\}$。乘积空间的拓扑是显然诱导的。

\begin{definition}
    若$G(T)$是闭集,则称$T$是闭算子。
\end{definition}
\begin{theorem}[闭图像定理]\label{thm:closegraph}
    设$T:X \to Y$是线性的闭算子。则$T\in B(X,Y)$。
\end{theorem}
\begin{proof}
    $X,Y$是Banach空间,则$X\times Y$也是Banach空间。由于$G(T)$是$X\times Y$的闭集。也是Banach空间

    定义投射$p:G(T) \to X$。则$p$是单,满,线性的。$p \in B(G(T),X)$且$\|p\| \leq 1$。

    根据Banach逆算子定理,有$p^{-1}\in B(X,G(T))$。即存在$c>0$使得:
    \begin{align}
        \|Tx\| \leq C\|x\|
    \end{align}
    于是$T \in B(X,Y)$。
\end{proof}
\section{Hahn-Banach定理}
$1<p <\infty$,考虑空间$l_p$。取$x=(x_n)_{n=1}^\infty \in l_p$。设$\|x\|$的范数为$1$。根据Hahn-Banach定理得知,$f \in l_p^*$满足$\|f\|=1$且$f(x)=1$是存在的。但是如何构造具体形式呢?

即我们需要构造:
\begin{align}
    f(y)=f(y_1,y_2,y_3,\dots)
\end{align}
注意到我们只需要考虑其Schauder基$e_n$。于是我们定义:
\begin{align}
    f(e_n)=\|x_n\|^{p-1}\mathrm{sgn}(x_n)
\end{align}
则$f(x)=\sum_{n=1}^\infty \|x_n\|^p=1$。对于一般的$y$:
\begin{align}
    f(y)=f(\sum_{n=1}^\infty y_ne_n)=\sum_{n=1}^\infty y_n\|x_n\|^{p-1}\mathrm{sgn}(x_n)
\end{align}
这是表示定理能说明的事实。因此这里就不具体做了。

事实上,注意到$f^q$有一些独特的性质。我们可以猜测$l_p^*$和$l_q$的神秘关系:$l_p^* \cong l_q$。

\begin{theorem}[共轭空间表示定理:$c_0^*\cong l_1$]
    $c_0^*\cong l_1$。
\end{theorem}
\begin{proof}
    $c_0^*\cong l_1$。考虑$f \in c_0^*$,则存在唯一的$y \in l_1$使得$f(x)=\sum_{n=0}^\infty x_ny_n$且$\|f\|_{c_0^*}=\|y\|_{l_1}$.

    这是因为,$f((x_n))=\sum_{n=0}^\infty x_n f(e_n)$。于是定义$y_n=f(e_n)$。我们接下来要说明范数之间的关系。

    令$z_n=(\mathrm{sgn}(f(e_1)),\dots,\mathrm{sgn}(f(e_n)),0,0,\dots)$。于是$z_n \in c_0$且$\|z_n\| \leq 1$。$f(z_n)=\sum_{k=0}^\infty |f(e_k)| \leq \|f\|_{c_0^*}$.于是$y$的范数:
    \begin{align}
        \|y\|=\sum_{k=0}^\infty |y_k| \leq \|f\|_{c_0^*}
    \end{align} 

    另外,显然有:
    \begin{align}
        |f((x_n))|=|\sum_{n=0}^\infty x_ny_n| \leq \sup_{n}|x_n|\sum_{n=0}^\infty |y_n|=\|x\|\|y\|
    \end{align}
    从而$\|f\|_{c_0^*} \leq \|y\|$。

    另一方面,对于任意$y \in l_1$,定义$f(x)=\sum_{n=0}^\infty x_ny_n$,$\forall x \in c_0$。则$f \in c_0^*$且$\|f\|_{c_0^*}=\|y\|$。
\end{proof}
    
\subsection{表示定理}
\subsubsection*{有界变差函数}
设区间$[a,b]$上的有界变差空间为$\mathrm{BV}[a,b]$。定义上面的范数为:
\begin{align*}
    \|\omega\|=\mathrm{Var}(\omega)+|\omega(a)|
\end{align*}
\begin{theorem}
    有界变差空间在上述范数下构成Banach空间。
\end{theorem}
\begin{proof}
    
\end{proof}
设$x \in C[a,b]$,$P$是分划$a=t_0<t_1 \dots <t_n=b$,定义:
\begin{align*}
    S(x,\omega,P)=\sum_{i=1}^n x(t_{i-1})[\omega(t_i)-\omega(t_{i-1})]
\end{align*}
我们拓宽黎曼积分的定义:Riemann-Stieltjes积分:
\begin{align*}
    \int_a^b x(t)d\omega(t)=\lim_{\delta(P)\to 0}S(x,\omega,P)
\end{align*}
关于$P$-Cauthy网收敛。

我们考量该积分与$x$和$\omega$范数的关系:
\begin{proposition}
    $|\int_a^b xd\omega|\leq \|x\| \cdot \|\omega\|_{\mathrm{bv}}$
\end{proposition}
\begin{theorem}[Riese]
    设$f \in C[a,b]^*$。存在唯一的$\omega \in \mathrm{BV}(a,b)$满足:$\omega(a)=0$,$\omega$在$(a,b)$右连续,即$\omega(t^{\perp})=\omega(t),\forall t \in (a,b)$,使得:
    $$
    f(x)=\int_a^b x(t)d\omega(t), x\in C[a,b]
    $$
    此时$\|f\|=\|\omega\|_{\mathrm{bv}}$。
\end{theorem}
记$\mathrm{BV}[a,b]$中所有$\omega(a)=0$,$\omega$右连续的函数全体为$\mathrm{BV}_0[a,b]$.则我们有表示定理:
\begin{corollary}
    $C[a,b]^* \cong \mathrm{BV}_0[a,b]$。
\end{corollary}
\begin{proof}
    考虑$C[a,b] \subset L^\infty[a,b]$.$C[a,b]$是闭子空间。

    对于$f \in C[a,b]^*$,根据$H-B$保范延拓:

    存在$F \in L^\infty[a,b]^*$,$F|_{C[a,b]}=f$。

    考虑$x \in C[a,b]$,$P$为分划。令:
    $$
    z(x,P)=x(a)x_{t_1}+\sum_{i=2}^n x(t_{i-1})[x_{t_i}-x_{t_{i-1}}]
    $$
    则$F(z(x,p))$是部分和$S(x,\alpha,P)$。显然$\|z(x,P)-x\|_{\infty}\to 0,\delta(P)\to 0$。则$\alpha$的存在保证。

    唯一性:对$\alpha$进行修正,使得$\omega(t)=0,t=a;\alpha(t^+),a<t<b;\alpha(b),t=b$。我们断言$\omega$与$\alpha$积分值相同。即:
    \begin{align*}
        \int_a^b xd\alpha =\int_a^b xd \omega
    \end{align*}

    由于$\alpha$的不连续点至多可列,可以推的上述结论。

    我们现在说明修正得到$\omega$是唯一的。即:
    \begin{align*}
        \int_a^b xd\beta =\int_a^b xd \omega \Rightarrow \beta=\omega
    \end{align*}
    若$t_0$是两者的连续点,则也是$t \mapsto \mathrm{Var}_{[a,t]}(\alpha)$和$t \mapsto \mathrm{Var}_{[a,t]}(\beta)$的连续点。
\end{proof}

\section{第10周}
\subsection{Tuesday}
\begin{theorem}
    设$A,B,C$是Banach空间,且:
    \begin{align*}
        0 \to A \to B \to C \to 0
    \end{align*}
    是Banach空间范畴下的正合列。则诱导的列:
    \begin{align*}
        0 \to C^* \to B^* \to A^* \to 0
    \end{align*}
    也是正合的列。
\end{theorem}
\begin{question}
一个$n$点的离散距离空间可以等距嵌入到$l_\infty^{n-1}$中吗?
\end{question}
\begin{question}
一个可分距离空间总可以嵌入到一个可分的Banach空间吗?
\end{question}
\subsection{Thursday}
对于空间$X$,定义弱拓扑:$(X,w)$:
\begin{definition}
    取遍$\forall x_0 \in X$,$\forall \epsilon>0$,$\forall n \in \N$,$\forall f_k \in X^*,1\leq k \leq n$。集合:
    $$
    \{x \in X||f_k(x)-f_k(x_0)|\leq \epsilon,1 \leq k \leq n\}
    $$
    取$x_0=0$,称为弱拓扑在原点的邻域基。
\end{definition}
注:取$\{x \in X|\|x\| \leq 1\}$是弱(拓扑)开集当且仅当$X$是有限维。

\begin{proposition}
    $(B_X,w)$是紧空间等价于$X$是自反的。$B_X=\{\|x\| \leq 1\}$。
\end{proposition}
\begin{definition}[弱星拓扑]
    在$X^*$上,定义拓扑基如下:

    取遍$f_0 \in X^*$,$\epsilon>0$,$n \in \N$,取遍$x_k \in X,1 \leq k \leq n$。取遍所有的集合:
    $$
    \{f \in X^*||f(x_k)-f_0(x_k)|\leq \epsilon,1\ \leq k \leq n\}
    $$
\end{definition}
\begin{theorem}
    $(B_{X^*},w^*)$是紧空间。
\end{theorem}
\begin{question}
    对于Banach空间$Y$,是否存在$X$满足$X^*=Y$?
\end{question}
\begin{question}[Ball covering Property:球覆盖性质]
    设$X$是赋范空间,$S_X=\{x \in X|\|x\|=1\}$。$B(x_0,r)=\{x|\|x-x_0\|<r\}$。称$X$有球覆盖性质,若$X$的单位球面$S_X$可以被写为$X$中的至多可列个球的并集覆盖。即:
    $$
    S_X \subset \bigcup_{n=1}^\infty B(x_n,r_n)
    $$
    且$0 \notin B(x_n,r_n), \forall n$。
\end{question}

\begin{lemma}
    设$\Omega$是紧Hausdorff空间,$C(\Omega)$是连续函数空间,则$C(\Omega)$有BCP例,当且仅当$\Omega$有至多可数的$\pi$基(弱拓扑基)。
\end{lemma}
\begin{example}
    可分空间必定BCP;$l_\infty$有BCP;$L^\infty [0,1]$不满足BCP。
\end{example}

\section{第十一周}
\subsection{Tuesday}
\begin{theorem}[Schur性质]
    对于空间$l^1$,序列按范数收敛等价于按照弱拓扑收敛。即$x_n \stackrel{\|\cdot\|}{\rightarrow} x_0$当且仅当$f(x_n) \to f(x_0)$,$\forall f \in l^1(\cong l^\infty)$
\end{theorem}
\begin{proof}
    只需要说明右边可推得左边。设$x_n \stackrel{w}{\rightarrow} x_0$且$x_n \stackrel{\|\cdot\|}{\rightarrow}$不成立。我们不妨设$x_0=0$。

    根据Banach-Steinhaus定理,$\|x_n\| \leq M$。由于按照范数不收敛,不妨设存在$\delta>0$:
    \begin{align*}
        0 <\delta \leq \|x_n\| \leq M
    \end{align*}
    做归一化$y_n=x_n/\|x_n\|$。则$y_n \stackrel{w}{\rightarrow} 0$(因为范数有下界。)。

    考虑赋值泛函$e_m:y \mapsto y(m)$。$e_m \in l_1^* \cong l^\infty$。

    $e_m$是连续线性泛函,所以$y_n \stackrel{w}{\rightarrow} 0$表明$e_m(y_n)=y_n(m)\to 0$,并且$\|y_n\|_{l^1}=1$。接下来我们构造$f \in l_1^*$使得:
    \begin{align*}
    |f(y_{n_k})|>1/4,\forall n \in \N
    \end{align*}

    这样的构造可以根据前若干项的绝对值和逐渐趋近0,从而小于$1/4$,中间项足够多时大于$3/4$,从而保证这$3/4$的若干项可以全部取绝对值。
\end{proof}
\subsubsection*{紧算子}
\begin{definition}
    设$T \in B(X,Y)$称为紧算子,若任意有界集合$M \subset X$,有$\overline{T(M)}$是$Y$中的紧集。等价的,即$\overline{T(B_x)}$是紧集。
\end{definition}
\begin{example}
    无限维空间$X$的恒等映射不是紧算子。因为单位球面不是紧集。
\end{example}
\begin{theorem}
    紧算子把弱收敛列映射为按范收敛列。
\end{theorem}
\begin{proof}
    不妨设$x_n \stackrel{w}{\rightarrow} x_0$但是$Tx_n \stackrel{\|\cdot\|}{\rightarrow} Tx_0$不成立。

    于是存在$\|Tx_{n_k}-Tx_0\| \geq \epsilon$。但是$T$是紧算子,从而$\{Tx_{n_k}\}$是紧集,有聚点,从而有$\|y_0-Tx_0\| \geq \epsilon$。

    但是$\forall f \in Y^*$,$f(Tx_0)=(f\circ T)(x_0)=f(y_0)$。根据Hahn-Banach定理,$Tx_0=y_0$矛盾!
\end{proof}
\begin{theorem}
    $T_n,T_0 \in B(X,Y)$。若$T_n$按照算子范数收敛到$T$,则$T_n$紧得到$T$紧算子。
\end{theorem}
\begin{proof}
    对角线法。
\end{proof}

考虑一类特殊的算子,其秩1:
\begin{align*}
    T(y)=f(y)x,\forall y \in X; \Rightarrow T=x\otimes f,f \in X^*
\end{align*}

若秩为$N$:
\begin{align*}
    T(y)=\sum_{k=1}^N f_k(y)x_k,\forall y\in X;\Rightarrow T=\sum_{k=1}^n x_k \otimes f_k,f_k \in X^*
\end{align*}

把所有紧算子的集合记为$K(X,Y)$,$F(X,Y)$记为有限秩的算子全体,则:
\begin{align*}
    \overline{F(X,Y)}\subset K(X,Y)
\end{align*}
\subsection{Thursday}
\begin{definition}
    称$X$具有逼近性质(AP),若任给紧子集$K \subset X$,$\forall \epsilon>0$,都存在有限秩的算子$F$使得:
    \begin{align*}
        \|T(x)-x\| <\epsilon,\forall x \in K
    \end{align*}
\end{definition}
\begin{theorem}[Grothendieck]
    $X$具有AP性质当且仅当$K(X)=\overline{F(X)}$
\end{theorem}
\begin{proposition}
    AP性质等价于:任取$x_n \in X$,$f_n \in X^*$,且$\sum_{n=1}^\infty \|x_n\|\cdot \|f_n\|<\infty$。若$T=\sum_{n=1}^\infty x_n \otimes f_n=0$,则$\sum_{n=1}^\infty f_n(x_n)=0$。
\end{proposition}

\begin{definition}[bounded approximation property]
    $X$满足有界逼近性质,若$\exists M>0$,使得$\forall \text{compact} K \subset X$,$\forall \epsilon>0$,存在$T \in F(X)$且$\|T\|<M$使得:
    \begin{align*}
        \|T(x)-x\| <\epsilon,\forall x \in K
    \end{align*}
\end{definition}

\begin{theorem}
    对于$T \in B(X,X_1)$,$T$紧与$T^*$紧等价。
\end{theorem}
\begin{proof}
    若$T$紧。
\end{proof}
\section{第12周}
\subsection{Thursday}

\begin{question}[公开问题:不变子空间]
    $l_2$空间。对于任何$T \in B(L_2)$,是否存在$l_2$的一个无穷维子空间$H_0$使得$T(H_0) \subset H_0$。

    将问题特殊化:设$T$是紧算子,上述命题是否正确?答案是否定的。
\end{question}
\begin{question}
    $B(l_2)$有球覆盖性质。
\end{question}
\begin{proposition}
    当$M$是闭子空间,$H=M \oplus M^{\bot}$。
\end{proposition}
\section{第13周}
\subsection{Tuesday}
\begin{theorem}
    对于内积空间$H$,存在$M \subset H$是完备的凸集,则对于$\forall x \in H$,存在唯一的$x_0 \in M$使得:
    \begin{align*}
        \|x-x_0\|=d(x,M)=(\inf_{y \in M}\|x-y\|)
    \end{align*}
\end{theorem}
\begin{proof}
    设$x \notin M$。$\alpha=d(x,M)$。由$\inf$的定义,存在$\{x_n\} \subset M$满足:$\|x_n-x\| \to \alpha$。下证:$\{x_n\}$是Cauthy列。

    由于$M$是凸集,则$(x_m+x_n)/2 \in M$。则$\|x-(x_m+x_n)/2\|\geq \alpha$。
    \begin{align*}
        \|x_m-x_n\|^2=2\|x_m-x\|^2+2\|x-x_n\|^2-4\|x-(x_m+x_n)/2\|^2\leq 2\|x_m-x\|^2+x\|x_n-x\|^2-4\alpha \to 0
    \end{align*}
    从而是Cauthy列。于是根据完备性知$\{x_n\}$有极限。
\end{proof}

\begin{theorem}[正交分解]
    设$H$是Hilbert空间,$M$是闭子空间。$\forall x \in H$,$\exists ! x_0 \in M$,$y \in M^{\bot}$使得$x=x_0+y$即:
    \begin{align*}
        H=M\oplus M^{\bot}
    \end{align*}
\end{theorem}
\begin{proof}
    $M$是闭子空间,$\forall x \in H$,存在唯一的$x_0 \in M$使得$\|x-x_0\|=d(x,M)=\alpha$。

    $\forall z \in M$,$z \neq 0$。$\forall \lambda \in \C$,$x_0+\lambda z\in M$。并且
    \begin{align*}
        (x-x_0-\lambda z)^2
    \end{align*}
\end{proof}
\begin{definition}
    对于Banach空间$X$,$E \subset X$是闭的子空间。若有在闭子空间$F$使$\forall x \in X$,存在!的$x_0 \in E$,$y \in F$使得$x=x_0+y$,即$X=E\oplus F$。此时称$E$是可补的,$F$是$E$的一个补空间。
\end{definition}
\begin{example}
    $c_0 \subset l_\infty$是不可补的。
\end{example}
\begin{theorem}
    $X$是Banach空间,$E$是可补的闭子空间。定义线性映射$P:X \to X$,$P(x)=x_0 \in E$。则$P \in B(X)$。
\end{theorem}
\begin{lemma}
    $\N$中存在不可数个子列记为$\{S_\alpha\}_{\alpha_\Gamma}$且两两几乎不交。
\end{lemma}
\begin{lemma}
    有界线性算子$P \in B(l_\infty)$:$P|_{c_0}=0$,即$P(x)=0,\forall x \in c_0$。则存在$\N$的子列$A$使得$P|_{l_\infty(A)}=0$。即$P(y)=0,\forall y \in l_\infty$且$y(k)=0,\forall k \notin A$。
\end{lemma}
\begin{proof}
    反证,假设不成立。由于$\N$上存在不可数个子列$\{A_\alpha\}$且两两几乎不交。$\forall i \in I$,$\exists x_i \in l_\infty(A_i)$且$P(x_i)=0$。显然$x_i \notin c_0$。

    不妨设$\|x_i\|=1$,$\forall i \in I$。定义$\forall n \in N$:
    \begin{align*}
        I_n=\{i \in I:(p(x_i))(n)\neq 0\}
    \end{align*}
    则有$I=\bigcup_{n=1}^\infty I_n$。则有$I_n$是不可数的,

    再进行分解:
    \begin{align*}
        I_{n,k}=\{i \in I:|P(x_i)(n)|\geq 1/k\}
    \end{align*}
    则
    \begin{align*}
        I_n=\bigcup_{k=1}^\infty I_{n,k}
    \end{align*}
    则有$I_{n,k}$不可列。其任意有限子集$J$:
    \begin{align*}
        \text{令}y=\sum_{j\in J}\mathrm{sgn}[P(x_j)(n)]x_j
    \end{align*}
    于是
    \begin{align*}
        P(y)(n)=\sum_{j \in J}|P(x_j)(n)|\geq 1/k |J|
    \end{align*}
    注意到$J$有限,则存在$f$:$y=f+z$且$\{i \in \N:f(i) \neq 0\}$是有限集且$\|z\|_{\infty}\leq 1$。

    于是$P(y)=P(z)\leq \|P\|$。但是$\|P(y)\|\geq |P(y)(n)| \geq |J|/k$。注意到$J$的个数任意大,则产生了矛盾!
\end{proof}
\begin{corollary}
    $c_0$在$l_\infty$中不可补。
\end{corollary}
\begin{proof}
    若可补,则存在$l_\infty=c_0+d$。有界线性算子$P_d|_{c_0}=0$.于是存在$\N$的子列$A$使得$P_d|_{l_\infty(A)}=0$。注意到$P_d+P_{c_0}=\mathrm{id}_{l_\infty}$。则$P_{c_0}|_{l_\infty(A)}=\mathrm{id}$。然而$l_\infty(A)$并未包含在$c_0$中,矛盾!
\end{proof}
\subsection{Thursday}
对于Hilbert空间,使用Zorn引理可以得到其一定有极大的正交系$A_\alpha$。并且$A_\alpha$张成的子空间的闭包是整个Hilbert空间。

考虑$\{e_i\}$是一列极大标准正交基。对于$\forall x \in H$,有:
\begin{align*}
    \|x-\sum_{n=1}^N (x,e_n)e_n\|=d(x,\mathrm{Span}(e_n)_{n=1}^N)
\end{align*}
注意到:
\begin{align*}
    \lim_{N \to \infty}\|x-\sum_{n=1}^N (x,e_n)e_n\|=0
\end{align*}
\begin{example}
    在空间$l^2$中,$e_n$直接可取为Schaduer基。
\end{example}
\begin{example}
    $L^2[0,2\pi]$中,取:
    \begin{align*}
        e_n(t)=\frac{1}{\sqrt{2\pi}}e^{int},n\text{的取值是}0,\pm 1,\dots,\pm n,\dots
    \end{align*}
    显然:
    \begin{align*}
        \int_0^{2\pi}e^{i(n-m)t}dt=\frac{1}{i(n-m)}e^{i(n-m)t}|_0^{2\pi}=0, \text{若}n \neq m
    \end{align*}
    并且$\|e_n\|=1$。从而这是一个正交系。
\end{example}
\section{拓扑线性空间}
\subsection{基本性质}
\subsubsection*{定义}
\begin{definition}
    拓扑线性空间的局部基是指$0$点附近的邻域族$\beta$,使得$0$点的任意邻域$U$,都存在$V \in \beta$使得$V \subset U$。
\end{definition}
\subsubsection*{分离性}
\begin{theorem}
    设$0 \in W$是邻域,则存在邻域$U$使得$U+U \subset W$且$U=-U$。
\end{theorem}
\begin{proof}
    考虑$0+0=0$.于是有:
    \begin{align*}
        V_1+V_2 \subset W
    \end{align*}
    记$U=V_1\cap V_2 \cap (-V_1)\cap (-V_2)$。
\end{proof}
\begin{proposition}
    $K,C$是$X$的子集且$K$紧,$C$是闭集。若$K\cap C$是空集,则存在$0$的邻域$V$:
    \begin{align*}
        (K+V)\cap (C+V)=\emptyset
    \end{align*}
\end{proposition}
\begin{proof}
    考虑单点$x \in K$。显然有$V_x$:
    \begin{align*}
        (x+V_x+V_x+V_x)\cap C=\emptyset \Rightarrow (x+V_x+V_x)\cap (C+V_x)=\emptyset
    \end{align*}
    $K$是紧集,自然有有限覆盖:
    \begin{align*}
        K=\bigcup_{i=1}^n (x_i+V_{x_i})
    \end{align*}
    并记$V=\bigcap_{i=1}^n V_{x_i}$。则$K+V \subset \bigcup_{i=1}^n(x_i+V_{x_i}+V) \subset \bigcup_{i=1}^n (x_i+V_{x_i}+V_{x_i})$

    于是$K+V$与$C+V$不交。
\end{proof}
\begin{corollary}
    拓扑线性空间是Hausdorff空间。
\end{corollary}
\begin{proposition}
    设$\beta$是局部基,则$\forall U \in \beta$,存在$V \in \beta$满足:
    \begin{align*}
        \bar{V}\subset U
    \end{align*}
\end{proposition}
\begin{proof}
    $U^c$是闭集。则存在$V:(U^c+V)\cap V=\emptyset$。
    \begin{align*}
        (U^c+V)\cap \bar{V}=\emptyset \Rightarrow (U^c) \cap \bar{V}=\emptyset, \bar{V}\subset U
    \end{align*}
\end{proof}
\subsubsection*{平衡集和有界集}
\begin{definition}
    $B$是平衡集,若$\alpha B\subset B$对于$|\alpha|\leq 1$成立。
\end{definition}
\begin{theorem}
    每个$0$点的邻域包含一个包含$0$的平衡邻域。
\end{theorem}
\end{document}
