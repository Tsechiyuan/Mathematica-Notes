\documentclass[UTF8]{ctexart}[a4paper,10pt]
\usepackage[thmmarks]{ntheorem}
\usepackage{amsmath}
\usepackage{amsfonts,amssymb} 
\usepackage{thmtools}
\usepackage[hmargin=2.5cm,vmargin=2.5cm]{geometry}
\usepackage{tikz-cd,tikz}
\usepackage{graphicx,float}
\usepackage{fancyhdr}
\usepackage{fourier-orns}
\usepackage{quiver}

%声明环境
\newtheorem{example}{例}[section]              
\newtheorem{algorithm}{算法}[subsection]
\newtheorem{theorem}{定理}[section]            
\newtheorem{definition}{定义}[section]
\newtheorem{axiom}{公理}[section]
\newtheorem{property}{性质}[section]
\newtheorem{proposition}{命题}[section]
\newtheorem{lemma}[theorem]{引理}
\newtheorem{corollary}[theorem]{推论}
{
    \theoremheaderfont{\sffamily}
    \newtheorem*{remark}{注解} 
}
\newtheorem{condition}{条件}
\newtheorem{conclusion}{结论}[section]
\newtheorem{assumption}{假设}
{
\theoremstyle{nonumberplain}
\theoremheaderfont{\bfseries}
\theorembodyfont{\normalfont}
\theoremsymbol{\mbox{$\Box$}}
\newtheorem{proof}{证明}
}
%定义命令
\def\N{\mathbb{N}}
\def\Z{\mathbb{Z}}
\def\Q{\mathbb{Q}}
\def\R{\mathbb{R}}
\def\C{\mathbb{C}}
\def\S{\mathbb{S}}
\def\D{\mathbb{D}}
\def\H{\mathbb{H}}
%外测度
\def\outmQ{m_*(Q)}

%页眉设计
\renewcommand 
\headrule{
\hrulefill
\raisebox{-2.1pt}
{\quad{\FourierOrns M T S N}\quad}
\hrulefill}
\pagestyle{fancy}

%超链接红色
\usepackage[colorlinks,linkcolor=red]{hyperref}

\usepackage{enumerate}


\title{Graph}
\author{颜成子游}
\begin{document}
\maketitle
\tableofcontents
\section{图的基本概念}
\subsection{图与子图}
我们要研究图论,那么首先就要给出图的基本概念。以及一些图的附属概念。
\begin{definition}
    图(Graph)是这样一个二元组,记作:$G=(V(G),E(G)$。其中:
    \begin{enumerate}
        \item $V(G)={v_1,v_2,\dots,v_n}$。这个集合被称为图的顶点集合,$v_i$被称为$G$的顶点。
        \item $E(G)=\{(v_i,v_j)|i,j\leq n\}$。这个集合被称为图的边集合。其中$i,j$并不一定遍历所有的$n$.
    \end{enumerate}
    如果对于任意的$(i,j)$,若$(v_i,v_j)\in E(G)$,那么就有$(v_j,v_i)$就在$E(G)$中,那么称$G$是无向图。反之称之为有向图。
\end{definition}

边到$E\times E$的映射被称为关联函数。

对于有向图,如果把所有的有序对都补充上他的转置,那么形成的无向图称为原图的underlying graph(基础图)。

\begin{definition}
    若$G$的$V(G),E(G)$是有限集合,则称之为有限图。$|V|$被称为$G$的阶,记为$\nu(G)$。$|E|$被称为$G$的边数,记为$\epsilon(G)$。此时$G$被称为$(\nu,\epsilon)$图。
\end{definition}
    某种意义上有序对是可以重复的。但是为了研究的方便,我们认为所有的有序对是不重复的。这样的图我们称之为“简单图”。对于有限简单图:
    $$
    0\leq \epsilon(G) \leq \frac{1}{2}\nu(G)(\nu(G)-1)
    $$

    今后未加说明的情况下,所指的图是顶点集和边集有限的无向简单图。
\begin{example}[几类图的例子]
    \quad

    1.完全图:$|E|=\dfrac{1}{2}\nu(G)(\nu(G)-1)$的简单图。

    2.空图:没有边的图。

    3.二部图(偶图):$V(G)=V_1 \cup V_2$,$V_1 \cap V_2= \emptyset$。且$G$的每个边的端点都分别属于$V_1,V_2$。

    4.完全二部图:如果$V_1$中的顶点和$V_2$中的顶点都相邻,则称为完全二部图。显然$G$的边数为$mn$,若$V_1$个数$m$,$V_2$个数$n$。

    
\end{example}

作为一个类似代数结构的东西,图也可以定义子图:
\begin{definition}
    对于$G=(V,E)$,称$H=(V_1,E_1)$是$G$的子图,若$H$也是图且$V_1 \subset V$,$E_1 \subset E$。
\end{definition}
\begin{definition}[生成子图]
    若$H \subset G$,$V(H)=V(G)$,则称$H$是$G$的生成子图或支撑子图。
\end{definition}
\begin{definition}[导出子图]
    某种意义上的“生成子图”。由$E$或者$V$的子集所生成的子图。
\end{definition}
\begin{definition}
    补图:$G$的补图$H$是这样的图,其顶点是$V(G)$,其边集是$V(G)$能生成所有边集中$E(G)$的子集。
\end{definition}
显然,图与补图的并是完全图。

\begin{definition}
    图的并,交:很容易理解的定义。

    图的差:之去掉边。

    图的环和:并中去掉公共边。
\end{definition}
\begin{definition}
    对于不相交的图,我们定义和:$G_1+G_2=(V_1 \cup V_2,E1\cup E_2 \cup E_3)$.其中$E_3=\{uv|u \in V_1,v \in V_2\}$

    笛卡尔积:顶点集合是顶点集的笛卡尔积,而顶点之间有连线当且仅当其有一个分量相同,另一个分量在原图中相邻。
\end{definition}
\begin{definition}[图的顶点度]
    
    
\end{definition}
\subsection{图的连通度}
\end{document}