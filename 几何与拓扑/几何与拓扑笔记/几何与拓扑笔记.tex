\def\allfiles{}
\documentclass[lang=cn,newtx,10pt,scheme=chinese]{elegantbook}

\title{几何与拓扑笔记}
\subtitle{主体:微分几何与代数拓扑}

\author{夏目凉}
\institute{陈省身数学研究所 Chern Institute of Mathematica}
\date{\today}
\version{1.0}
\bioinfo{Bio}{Information}

\extrainfo{\textcolor{red}{黑格尔在某个地方说过,一切伟大的世界历史事变和人物,可以说都出现两次。他忘记补充一点:第一次是作为悲剧出现,第二次是作为笑剧出现。}
}
\setcounter{tocdepth}{3}

\logo{logo-blue.png}
\cover{cover.jpg}
%记号命名
\newcommand\til[1]{\tilde{#1}}
\def\g{\mathfrak{g}}
\newcommand{\II}{\mathrm{II}}
\newcommand{\U}{\mathcal{U}}
\newcommand{\id}{\mathrm{id}}
\newcommand{\Hom}{\mathrm{Hom}}
\newcommand{\Ext}{\mathrm{Ext}}
\newcommand{\Tor}{\mathrm{Tor}}
\newcommand{\N}{\mathbb{N}}
\newcommand{\OO}{\mathcal{O}}
\newcommand{\Z}{\mathbb{Z}}
\newcommand{\Q}{\mathbb{Q}}
\newcommand{\F}{\mathbb{F}}
\newcommand{\R}{\mathbb{R}}
\newcommand{\PP}{\mathbb{P}}
\newcommand{\C}{\mathbb{C}}
\newcommand{\HH}{\mathbb{H}}
\newcommand{\RP}{\mathbb{RP}}
\newcommand{\dd}{\mathrm{d}}
\newcommand{\HdR}[1]{H_{\mathrm{dR}}^#1}
\newcommand{\ii}{\sqrt{-1}}
\newcommand{\la}{\langle}
\newcommand{\ra}{\rangle}
\newcommand{\pa}[3][]{
	\frac{\partial^{#1} #2}{\partial {#3}^{#1}}
	}
\newcommand{\bpa}{\bar{\partial}}
\newcommand{\npa}{\partial}

% 本文档命令
\usepackage{array}
\newcommand{\ccr}[1]{\makecell{{\color{#1}\rule{1cm}{1cm}}}}
%调用图表包
\usepackage{quiver}

%极限、余极限与滤过余极限的实现
\makeatletter
\newcommand{\Colim@}[2]{
  \vtop{\m@th\ialign{##\cr
    \hfil$#1\operator@font lim$\hfil\cr
    \noalign{\nointerlineskip\kern1.5\ex@}#2\cr
    \noalign{\nointerlineskip\kern-\ex@}\cr}}%
}
\newcommand{\Colim}{%
  \mathop{\mathpalette\Colim@{\rightarrowfill@\scriptscriptstyle}}\nmlimits@
}
\makeatother

\makeatletter
\newcommand{\Lim@}[2]{%
  \vtop{\m@th\ialign{##\cr
    \hfil$#1\operator@font lim$\hfil\cr
    \noalign{\nointerlineskip\kern1.5\ex@}#2\cr
    \noalign{\nointerlineskip\kern-\ex@}\cr}}%
}
\newcommand{\Lim}{%
  \mathop{\mathpalette\Lim@{\leftarrowfill@\scriptscriptstyle}}\nmlimits@
}
\makeatother


\makeatletter
\newcommand{\colim@}[2]{%
  \vtop{\m@th\ialign{##\cr
    \hfil$#1\operator@font oli~$\hfil \cr
    \noalign{\nointerlineskip\kern1.5\ex@}#2\cr
    \noalign{\nointerlineskip\kern-\ex@}\cr}}%
}
\newcommand{\colim}{%
  \mathop{\mathrm{c}\mathpalette\colim@{\rightarrowfill@\scriptscriptstyle}\mathrm{\!\!m}}\nmlimits@
}
\makeatother

\makeatletter
\newcommand{\cone@}[1]{%
  \vtop{\m@th\ialign{##\cr
    \hfil$#1\operator@font cone$\hfil\cr
    \noalign{\nointerlineskip\kern1.5\ex@}\cr
    \noalign{\nointerlineskip\kern-\ex@}\cr}}%
}
\newcommand{\cone}{%
  \mathop{\mathpalette\cone@{\scriptscriptstyle}}\nmlimits@
}
\makeatother



% 修改标题页的橙色带
\definecolor{customcolor}{RGB}{32,178,170}
\colorlet{coverlinecolor}{customcolor}
\usepackage{cprotect}

\addbibresource[location=local]{reference.bib} % 参考文献,不要删除

%编译顺序
%\includeonly{title,1,2,3}

\begin{document}

\maketitle

\frontmatter

\tableofcontents

\mainmatter
\include{part/Riemann Geometry/Riemann geometry.tex}
\ifx\allfiles\undefined

	% 如果有这一部分另外的package,在这里加上
	% 没有的话不需要
	
	\begin{document}
\else
\fi

\part{李群}
\chapter{2023.02.16}
\section{一些定义}
\begin{definition}
\textbf{拓扑群}$(G, \cdot )$:$G$是群且是一个拓扑空间,满足$\cdot:G \times G \to G$和$()^{-1}:G \to G$都是连续的。

\textbf{李群}:$(G,\cdot)$:$G$是拓扑群,且本身是一个微分流形,满足$\cdot: G \times G \to G$和$()^{-1}:G \to G$是光滑的。
\end{definition}

为什么没有拓扑群专门的课程呢?
\begin{proposition}[Hilbert第五问题]
    拓扑群+局部欧氏能否成为李群呢?

    叙述如下:

    任意局部欧的拓扑群是李群且微分结构是唯一实解析的。
\end{proposition}
该问题在1950年代已经被证明了。从而拓扑群方向基本没有人研究了。
\begin{proposition}
    任意连通的微分流形是道路连通的。
\end{proposition}
\begin{proof}
    
\end{proof}
\begin{definition}[李子群]
    设$H$是李群$G$的子群。若$H$是$G$的浸入子流形,则$H$是$G$的李子群。
\end{definition}
回忆:什么是浸入子流形?

\begin{proposition}[Yamabe]
    李群的道路连通的子群是李子群。
\end{proposition}
但是李群的连通子群就不一定是李子群了。
\begin{example}[李子群但不是嵌入李子群]
    考虑$T^2$作为李群(显然是李群)。考虑子流形:
    $$
    H^n=\{(e^{it},e^{int})|n \in \N,t \in \R\}
    $$
    $$
    H^a=\{(e^{it},e^{iat})|a \in \R/\Q,a>0,t \in \R\}
    $$
    
    $H^n$是嵌入的李子群,其同胚于$S^1$。但$H^a$是非嵌入的李子群,只是浸入李子群。($\overline{H^a}=T^2$.)
\end{example}
\section{李群与矩阵李群}
一般线性群:
$$
\mathrm{GL}(n,\C)=\{A \in \C^{n\times n}||A|\neq 0\}
$$
$$
\mathrm{GL}(n,\R)=\{A \in \R^{n\times n}||A|\neq 0\}
$$
其中$\mathrm{GL}(n,\C)$是复李群(乘法是全纯的),实李群。另外一个是实李群。
\begin{definition}[矩阵李群]
    $\mathrm{GL}(n,\C)$的闭子群$G$称为矩阵李群。换句话说,若$\mathrm{GL}(n,\C)$满足若序列$\{A_m\}\subset G$,则$\lim_{n \to \infty}A_m =A \in M_n(\C)$,有$A \in G$或者不可逆。
\end{definition}
注意:

1.李群不一定是矩阵李群。但紧李群一定是矩阵李群。(Peter-Weyl)

2.矩阵李群是$\mathrm{GL}(n,\C)$的闭子群,而不是$M_n(\C)$的“闭子群”。

3.任何闭子群都是嵌入李子群(Cartan)。

\begin{example}
    $\mathrm{GL}(n,\R)$是矩阵李群,但不是$M_n(\C)$的闭子集。

    $\mathrm{GL}(n,\Q)$是群但并非矩阵李群。其不是闭集。
\end{example}
\begin{example}[矩阵群的例子]
    A.一般线性群。$\mathrm{GL}(n,\R)$,$\mathrm{GL}(n,\C)$。

    B.特殊线性群。$\mathrm{SL}(n,\C)=\{A \in \mathrm{GL}(n,\C)||A|=1\}$,$\mathrm{SL}(n,\R)=\{A \in \mathrm{GL}(n,\R)||A|=1\}$。

    C.正交群:$\mathrm{O}(n,\R)=\{A \in \mathrm{GL}(n,\R),A^T A=I_n\}$.$\mathrm{O}(n,\C)=\{A \in \mathrm{GL}(n,\C)|A^T A=I_n\}$

    D.特殊正交群:$\mathrm{SO}(n,\R)=\{A \in \mathrm{O}(n,\R)||A|=1\}$。

    E.酉群:$A \in M_n(\C)$,使得$A*A =I_n$。其中$A*=(\overline{A})^T$。这样的矩阵称为酉矩阵。酉群:$U(n)=\{A \in \mathrm{GL}(n,\C):A^*A=I_n\}$。酉矩阵变换保证酉内积的不变。\textbf{注意}:$U(n)$是实李群而非复李群。特殊辛群:$SU(n)=\{A \in U(n)|\mathrm{det}A=1\}$

    F.辛群:$S_p(n)$。$\mathrm{GL}(n,\mathbb{H})=\{A\in \mathbb{H}^{n \times n}:\mathrm{det} A\neq 0\}$是实李群。注意,由于四元数一些神秘的性质,虽然行列式,迹是良定义的,但是有$\mathrm{det}(AB)\neq \mathrm{det}(BA)$,$\mathrm{Tr}(AB)\neq \mathrm{Tr}(BA)$。
 
    $S_p(n)=\{A \in \mathrm{GL}(n,\mathbb{H})|A*A=I\}$。称为辛群,不是复李群。

    $S_p(1)\cong SU(2) \cong S^3$。

    G.实辛群。$\R^{2n}$的反对称双线性型。$w(x,y)=\sum_{j=1}^n(x_jy_{n+j}-x_{n+j}y_j)$。

    $$
    S_p(n,\R)=\{A \in M_{2n}(\R)|:w(Ax,Ay)=w(x,y),\forall x,y \in \R^{2n}\}=\{A \in \mathrm{SL}(2n,\R):\Omega A^T \Omega=A^{-1}\}, \Omega=\begin{bmatrix}
        0&I\\-I&0
    \end{bmatrix}
    $$
\end{example}
\section{球面上的李群结构}
$S^n$有李群结构等价于$n=1$或者$n=3$。并且$S^1 \cong SO(2,\R)$,$S^3 \cong \mathrm{SU}(2) \cong S_p(1)$.

\begin{proposition}
    李群上的切丛是平凡的。即对于$n$维李群,有$TG \cong G \times \R^n$。
\end{proposition}
\begin{proof}
    $$
    G\times T_e G \to TG:(g,v)\mapsto (g,(L_g)_* v)
    $$
    $L_g$是左平移作用,是一个微分同胚。详细证明可以见梅加强。
\end{proof}

对于$S^n$,若$S^n$的切丛是平凡的,则$n=1$,$n=3$,$n=7$。因而想要关注$S^n$是否为李群,只需要考虑这三个。

\chapter{2023.02.23}
\section{李群的局部性质}
\subsection{单位元邻域生成连通子群}
\begin{proposition}
    
连通李群$G$可以由$e$处任意邻域生成。
\end{proposition}
\begin{lemma}
    设$H$是李群$G$的开子群,则$H$是$G$的闭子群。
\end{lemma}
\begin{proof}
    这一点在拓扑群的考量中就可实现。固定$g \in G$,$L_g:G \to G$是一个微分同胚。故任意左陪集$gH$是开集。考虑$G \times H \to G$群作用,则$G=\bigcup_{g \in G}gH$且是不交并。从而$H$的余集是开集,$H$是闭子群。
\end{proof}
因此开子集是闭子群。是即开又闭的集合。
\begin{lemma}
    设$U$是$e$处的开邻域,则$U$的生成子群$H$是开集。
\end{lemma}
\begin{proof}
    根据定义,群$H$包含所有的乘积$x_1^{\epsilon_1}\dots x_n^{\epsilon_n}$。故$H=\bigcup_{x \in U}xU$是开集。
\end{proof}
两个引理立马就得到了命题。
\begin{proposition}
    设$G$是李群,$G_0$是$G$的单位连通分支,则$G_0$是$G$的子群。
\end{proposition}
\begin{proof}
    对给定的$x \in G_0$,其中$x$可以是任意指定的。由于$e \in G_0 \cap x^{-1}G_0$,从而$G_0=x^{-1}G_0$。于是$\forall y\in G_0$,$x^{-1}y \in G_0$.$G_0$是子群。
\end{proof}
\begin{corollary}
    $\forall x_1,x_2 \in G$,则$x_1G_0 \cap x_2G_0$要么是空集,要么是$x_1G_0=x_2G_0$
\end{corollary}
\begin{proof}
    如果相交非空,则两者都同胚于$G_0$。
\end{proof}
\begin{corollary}
    $G_0$是$G$的正规子群。
\end{corollary}
\begin{proof}
    考虑群作用$G\times G \to G:(g,h)\mapsto ghg^{-1}$。于是$G$成为了若干共轭类的并。

    由于$e \in gG_0g^{-1} \cap eG_0 e^{-1}$,因此轨道$gG_0g^{-1}=G_0$。于是$G_0$是正规子群。
\end{proof}
正规子群意味着可以做商。那么:
$$
1 \to G_0 \to G \to G/G_0 \to G
$$
是正合列。

\subsection{单位元的切空间具有Lie代数结构}
\begin{definition}[李代数]
    $V$是有限维$k$向量空间。$[,]$是$k$双线性:$[,]V \times V \to V$,满足反对称和Jacobbi恒等式:
    $$
    [X,[Y,Z]]+[Y,[Z,X]]+[Z,[X,Y]]=0
    $$

\end{definition}
\begin{example}
    $\mathrm{GL}(n,\C)$是李代数。$[A,B]=AB-BA$。此时这是Lie代数。
\end{example}
\begin{theorem}[Ado定理]
    有限维李代数是$(\mathrm{GL}(n,\C),[,])$的李子代数。
\end{theorem}

我们自然关注李群的李代数问题。
\begin{definition}[左不变向量场]
    李群$G$上的向量场$X$称为左不变的向量场,如果$L_g(Xh)=X(gh)$对于任意的$g,h$都成立。其中$L_g$是左平移作用。
\end{definition}
\begin{proposition}
    设$g$是$G$上左不变向量场的集合,$T_eG$是$G$在$e$处的切空间,则$I:g \to T_eG$,$X \to X(e)$是向量空间的同构。
\end{proposition}
\begin{proof}
    根据左不变的定义,左不变向量场由$T_eG$中的元素确定。即$X(g)=L_g(X(e))$,从而$I$是双射。

    根据向量场的运算$(X+Y)(e)=X(e)+Y(e)$,$(kX)(e)=k(X(e))$知道$I$是同构。
\end{proof}
由于$g$是有限维的李代数,其中$[X,Y]=XY-YX$,则$I$诱导了$T_e G$上的李括号结构:
$$
[X(e),Y(e)]:=[XY-YX](e)
$$
\begin{definition}
    $(T_e G,[,])$称为李群$G$的李代数。
\end{definition}

我们借此计算一下一些李群的李代数。
\begin{example}
    $\mathrm{GL}(n,\R)$的李代数为$\mathrm{GL}(n,\R)$。李括号为$XY-YX$.

    $\mathrm{GL}(n,\R)$在$I$处的切空间为$\mathrm{GL}(n,\R)$。左不变向量场由$\tilde{X}(I)=X \in G$决定。
\end{example}
\begin{example}
    $\mathrm{SL}(n,\R)$的李代数。$A \in \mathrm{SL}(n,\R)$,则$|A|=1$。

    $\mathrm{det}(I+\epsilon X)=1+\epsilon tr(X)+o(\epsilon^2)=1$,则
\end{example}
\begin{example}
    
\end{example}
\subsection{Hall-Wilt恒等式}
令$[x,y]=x^{-1}y^{-1}xy$是群上的交换子。考虑伴随作用$x^y=y^{-1}xy$.则下面有恒等式:
$$
[[x,y^{-1}],z]^y[[y,z^{-1}],x]^z[[z.x^{-1}],y]^x=1
$$
\begin{example}
    
\end{example}
\section{李群与李代数的关系——指数映射}
1.矩阵(复数元)的指数映射。
$$
e^X:=\sum_{n=0}^\infty \frac{X^m}{m!}, \quad X \in \mathrm{GL}(n,\C)
$$

\begin{proposition}
     对于任何$X \in \mathrm{GL}(n,\C)$,上述级数收敛。    
\end{proposition}
\begin{proof}
    分析方法:

    考虑$\mathrm{GL}(n,\C)$上的范数是所有元素的平方和的平方根。若$\lim X_n \to X$,则有$\lim |X_n -X|=0$。

    从而转化为:
    $$
    \|\sum_{m=0}^\infty \frac{X^m}{m!}\| \leq \sum_{m=0}^\infty \frac{\|X\|^m}{m!}=e^{\|X\|}
    $$
    收敛。(代数范数)

    代数方法:考虑可对角化矩阵
    $$
    X=CDC^{-1}
    $$
    若$D$对角,则显然收敛。若$X$幂零,则显然也收敛。一般的情况而言,由于任意的$X$可唯一分解为$X=S+N$,且$SN=NS$。从而$e^X=e^{N+S}=e^N e^S$。故收敛。
\end{proof}
$(R,+) \to (\mathrm{GL}(n,\C))$是李群同态。

\begin{lemma}
    
    设$\mathrm{Sym}_n(\R)$是$n$阶实对称矩阵,$\mathrm{Sym}_n^+(\R)$是正定矩阵。则$\mathrm{Sym}_n(\R) \to \mathrm{Sym}_n^+(\R)$是微分同胚。
\end{lemma}

\begin{proposition}
    对于$A \times $
\end{proposition}

\section{矩阵李群的性质与李代数}
\chapter{2023.03.02}
\section{李群的指数映射}
\begin{definition}[李群同态]
    设$H,G$是李群。若$\varphi:H \to G$是光滑的群同态,则称$\varphi$是李群同态。
\end{definition}
\begin{definition}[李代数同态]
    设$g,h$是李代数,线性映射$\varphi:h \to g$称为李代数同态,若$\varphi[x,y]_h =[\varphi(x),\varphi(y)]_g$.
\end{definition}
\begin{definition}[单参数变换群]
    李群同态$\varphi:(\R,+) \to G$称为单参数变换群。
\end{definition}
\begin{proposition}\label{pro:eee}
    设$G$是李群且李代数是$g$。对于任意给定的$X \in g$,存在唯一的单参数变换子群$\varphi_x:\R \to G$满足:
    $$
    \dfrac{d}{dt}|_{t=0}\varphi_x(t)=X(e)
    $$
\end{proposition}
\begin{proof}[积分曲线+ODE解的完备性]
    只需证明任意给定的$X \in g$,存在完备的积分曲线$\varphi_X$使得:$\varphi_X(t+s)=\varphi_X(t)\varphi_X(s)$。

    对于任意的$X \in g$,存在$\epsilon>0$使得在$(-\epsilon,\epsilon)$,$X$的积分曲线$\varphi_X(t)$存在,且满足$\varphi_X(0)=e$,$\dfrac{d}{dt}|_{t=0}\varphi_x(t)=X(e)$.这是可以做到的,因为是局部的性质。

    我们验证这是同态。设$\varphi_1(t)=\varphi_X(s+t),\varphi_2(t)=\varphi_X(s)\varphi_X(t)$。则$\dfrac{d}{dt}|_{t=0}\varphi_1(t)=X(\varphi_X(s))$,$\dfrac{d}{dt}|_{t=0}\varphi_2(t)=\dfrac{d}{dt}|_{t=0}L_{\varphi_X(s)}\varphi_X(t)=(L_{\varphi_X(s)})_*\dfrac{d}{dt}|_{t=0}\varphi_x(t)=(L_{\varphi_X(S)})_* X(e)=X(\varphi_X(s))$。

    根据ODE解的存在唯一性,则$\varphi_1(t)=\varphi_2(t)$。从而这是同态。

    再证明完备性。作曲线$\varphi_x^{\R}(t):=\varphi_X(\epsilon/2)\varphi_X(t-\epsilon/2)$。则根据同态性,$\varphi_X^{\R}$与$\varphi_X(t)$在$(-\epsilon/2,\epsilon)$是重合的。由此我们把区间延拓到了$(-\epsilon,3/2\epsilon)$.类推可以延拓$(-\epsilon,+\infty)$。同理可以延拓到$(-\infty,\epsilon)$。因此这是完备的曲线。
\end{proof}
\begin{theorem}
    给定李代数的同态$\phi:g \to h$,若$G$是单连通的,则存在唯一的李群同态$\Phi:G \to H$满足$\Phi_{*e}=\phi$:
    \begin{tikzcd}
	{\Phi:G} && H \\
	\\
	{\phi:g} && h
	\arrow[from=1-1, to=1-3]
	\arrow[from=3-1, to=3-3]
	\arrow[from=3-1, to=1-1]
	\arrow[from=3-3, to=1-3]
     \end{tikzcd}
\end{theorem}
\begin{proof}[命题\ref{pro:eee}李代数同态的提升]
    对任意的$X$是$g$里的元素,$\phi_X:\R \to g$,$t \mapsto tX$是李代数的同态。这里$\R$的李代数结构为$[x,y]=0$.

    由于$\R$单连通,$\exists$唯一的李群同态$\varphi_x:\R \to G$使得$(\varphi_X)_{*e}=\phi_X$。即为所求的单参数变换群。
\end{proof}
\begin{definition}[李群的指数映射]
    设$G$是李群,李代数为$g$。考虑映射$\mathrm{exp}:g \to G$,$X \mapsto \varphi_X(1)$称为$G$的指数映射。
\end{definition}
\begin{remark}
    指数映射一般非满射。\textbf{$G$是紧李群,则$\exp$是满射。这一点暂时不证明。}
\end{remark}
\begin{example}
    考虑$\R$是李群,则$g=\R$。我们计算指数映射。对于给定的$a \in \R$,其单参数变换群为$\varphi_a(t)=ta$。从而指数映射$\exp(a)=a$。
\end{example}
\begin{example}
    设$G=S^1$。$g=\R$。给定$a \in \R$,则$\varphi_a(t)=e^{2\pi i at}$,则$\exp(a)=e^{2\pi i a}$
\end{example}
\begin{example}
    $G=\mathrm{GL}(n,\C)$。任取$A$是可逆矩阵,$\varphi_A(t)=e^{tA}$于是$\exp(A)=e^A$.
\end{example}
我们自然的给出指数映射的性质。
\section{指数映射的性质}
\begin{proposition}
    存在$(g,+)$的单位元邻域$U(0)$以及$(G,\cdot)$的单位元邻域$V(e)$使得$\exp:U(0) \to V(e)$是微分同胚。且满足$\exp_{*0}=\mathrm{id}$,$T_0g\cong g$.从而$T_eG =g$。
\end{proposition}
\begin{proof}
    首先证明$\exp$是光滑映射。考虑$G \times g$上由向量场$(X,0)$诱导的流。
    $$
    \Phi:\R \times G \times g \to G \times g, (t,g,X)\mapsto (g \exp{tX},X)
    $$
    这是光滑映射。设$G \times g \to G$是自然投影,从而也使光滑的。因此$\exp=P\circ \Phi(0,e,X)$也是光滑的。

    由定义$\exp_{*0}(X)=\dfrac{d}{dt}|_{t=0} \exp{tX}=X(e)$,得$\exp_{*e}=\mathrm{id}$。

    从而根据反函数定理可知,存在两个邻域使其为微分同胚。
\end{proof}
\begin{remark}
    该性质可以定义$e$处得一个局部坐标系:
    $$
    \phi:V(e) \to \R^n \quad \exp(t_1x_1+\dots+t_nx_n)\mapsto (t_1,\dots,t_n)
    $$
    其中$X_i$是$g=T_e G$的一组基。
\end{remark}
\begin{proposition}\label{pro:extension}
    设$n \geq 1$,$X_1,\dots,X_n$是$g$里面的元素。当$\|t\|$充分小的时候,有:
    \begin{align}
        \exp(tX_1)\exp(tX_2)\dots \exp(tX_n)=\exp(t\sum_{1\leq i \leq n}X_i+\dfrac{t^2}{2}\sum_{1\leq i \leq j\leq n}[x_i,x_j]+o(t^3))
    \end{align}
\end{proposition}
先给出一个引理:
\begin{lemma}
    设$f$是$G$上的光滑函数,当$\|t\|$充分小的时候,有:
    \begin{align}
        f( \exp(tX_1)\exp(tX_2)\dots \exp(tX_n))=f(e)+t\sum_{i}X_if(e)+\frac{t^2}{2}(\sum_i X_i^2f(e)+2\sum{i<j}X_iX_jf(e))+o(t^3)
    \end{align}
\end{lemma}
\begin{proof}
    对于$\forall f \in C^\infty(G)$,$X \in g$有:
    \begin{align}
        (Xf)(a)=X(a)f=(L_a)_*X(e)(f)=X(e)((L_a)^*f)=\dfrac{d}{dt}|_{t=0}f(a\exp{tX})
    \end{align}
    于是对于任意的$t \in \R$,有:
    \begin{align}
        (Xf)(a\exp tX)=\frac{d}{ds}f(\exp(t+s)X)=\frac{d}{ds}|_{s=t}f(a\exp sX)
    \end{align}
    对于$X_1,\dots,X_k \in \g$,有:
    \begin{align}
        (X_1X_2f)(a)=\frac{d}{dt_1}|_{t_1=0}(X_2f)(a\exp t_1X_1)=\frac{d}{dt_1}\frac{d}{dt_2}f(a\exp(t_1X_1)\exp(t_2X_2))
    \end{align}
    以此可以类推,从而可知$(X_1X_2\dots X_k f)a$的情况。取$a=e$可得上述引理。
\end{proof}
\begin{proof}[命题\ref{pro:extension}]
    由于足够小的邻域内$\exp$是同胚,因此构造其逆映射$\log$。这里我们要求$\|t\|$足够的小。由于$\exp(0)=e$,则$\log(e)=0$。且对于任意的$X \in \g$,有:
    \begin{align}
        Xf(e)=\frac{d}{dt}|_{t=0}f(\exp tX)=\frac{d}{dt}|_{t=0} tX=X
    \end{align}
    对于任意$n>1$,$X^n f(e)=\dfrac{d}{dt^n}|_{t=0}(tX)=0$.

    注意到$\sum X_i^2 +\sum 2X_iX_j= (X_1+\dots+X_n)^2+\sum_{i<j}[X_i,X_j]$。
    
    对$\exp(tX_1)\dots \exp(tX_n)$作用$\log$。只要$t$足够小,那么就有右边式子结论。
\end{proof}
\begin{proposition}
    设$G$是李群,李代数$\g$。$H$是$G$的闭子群,则$\mathfrak{h}:=\{X \in \g|\exp tX\in H,\forall t \in \R\}$是$\g$的子代数。
\end{proposition}
\begin{proof}
    首先我们说明$\mathfrak{h}$是子空间。由定义$\forall X \in \mathfrak{h},s \in \R$有$sX \in \mathfrak{h}$。

    由上述命题可知:
    \begin{align}
        \exp(t/n X)\exp(t/n Y)=\exp(t/n(X+Y)+t^2/2n^2[X,Y]+o(1/n^3))
    \end{align}
    上式$n$次方,即可得到:
    \begin{align}
        (\exp(t/n X)\exp(t/n Y))^n=\exp(t(X+Y)+t^2/2n[X,Y]+o(1/n^2))
    \end{align}
    对$n$取极限$n \to \infty$,则右式自然有为$\exp(t(X+Y))$为左式的极限。而$H$是闭子群,从而极限也属于$H$。这就说明$X+Y \in \mathfrak{h}$.

    再证明$[X,Y]\in \mathfrak{h}$。根据上式的估计:
    \begin{align}
        (\exp(-t/n X)\exp(-t/n Y)\exp(t/n X)\exp(t/n Y))^{n^2}=\exp(t^2[X,Y]+o(1/n))
    \end{align}
    同样给极限,从而$[X,Y] \in \mathfrak{h}$.
\end{proof}
我们可以看到,在进行指数映射的计算性质前,我们常常会使用关于乘积的估计。

\begin{proposition}
    设$\|\cdot\|$是$\g$上的范数,$\{X_i\}$是$\g$中的序列满足:
    \begin{enumerate}
        \item $X_i \to 0, i \to \infty$.
        \item $\exp X_i \in H,\forall i$.
        \item $\lim_{i \to \infty}\dfrac{X_i}{\|X_i\|}=X \in g$
    \end{enumerate}
    则$X \in \mathfrak{h}$如上面性质的定义。
\end{proposition}
\begin{proof}
    给定$t\neq 0$,取$n_i:=\max\{n \in \Z,n\leq \dfrac{t}{\|X_i\|}\}$
    \begin{align}
        \exp tX=\exp(\lim_{i \to \infty}\dfrac{tX_i}{\|X_i\|})=\lim_{i \to \infty}\exp(n_iX_i)=\lim_{i \to \infty}(\exp X_i)^{n_i}\in H
    \end{align}
    于是根据上述性质得到$X \in \mathfrak{h}$。
\end{proof}
\begin{proposition}
    $\mathfrak{h},H$的定义如上。$(\mathfrak{h},+)$存在单位元邻域$U(0)$和$(H,\cdot)$的单位元邻域$V(e)$使得:
    \begin{align}
        \exp_{G}|_{U(0)}:U(0)\to V(e)
    \end{align}
    是微分同胚。
\end{proposition}
\begin{proof}
    设$\mathfrak{h}'$是$\g$的子空间,使得$g=\mathfrak{h}\oplus \mathfrak{h}'
    $。令$\Phi:\g \to G$,使得$\Phi(X+Y)=\exp_G X\exp_G Y$。显然$\Phi_{*0}(X+Y)=X+Y$,则$\Phi$是局部的微分同胚。

    注意到$\exp|_h=\Phi|_h$,只需要证明$\Phi$将$\mathfrak{h}$中的单位邻域微分同胚的映射到$H$中的单位邻域。

    假设对于$U(0)\subset \mathfrak{h}$,$\Phi$都无法将其微分同胚的映射到$V(e)\subset H$。即需依赖$h'$中分量$Y_i \neq 0$的元素,才能通过$\Phi$得到$V(e)$。
    \begin{align}
        \exp X_i \exp Y_i \in H \Rightarrow \exp(Y_i)\in H
    \end{align}
    又因为$Y_i \to 0$,所以$Y_i/\|Y_i\|$的极限$Y \in \mathfrak{h}\cap \mathfrak{h'}$。这产生了矛盾,因为$\| Y\|=1$。

\end{proof}

最后我们给出闭子群定理。
\begin{theorem}[Cartan]
    李群$G$的闭子群$H$是$G$的嵌入李子群。
\end{theorem}
\begin{proof}
    对$G$的单位元邻域$U(e)$,存在$(g,+)$的单位元邻域$V(0)$使得:$\log:U(e) \to V(0)$是微分同胚。

    根据上述命题,自然有:$\log(U(e)\cap H)=V(0)\cap \mathfrak{h}$。于是$U(e)$的坐标使得$H$中$e$的邻域是嵌入子流形。

    对于$\forall g \in G$,由于$L_g$是微分同胚,故给定$h \in H$:
    \begin{align}
        U(h)\to U(e) \to V(0)
    \end{align}
    是$h$邻域$U(h):=(L_h)(U(e))$的坐标。

    这意味着每个点$h \in H$,都有邻域$U(h)$使得$U(h)\cap H \to L_h^{-1}(U(h)\cap H) \to \log(L_h^{-1}(U(h)\cap H))=V(0)\cap \mathfrak{h}$。于是这意味着$H$是嵌入子流形。
\end{proof}
\chapter{2023.03.09}
\section{闭子群定理应用}
\begin{proposition}
    $\varphi: G \to H$是李群同态,则$\ker \varphi$是嵌入(正规)李子群。
\end{proposition}

\begin{proof}
$\varphi$是李群同态,则$\ker \varphi=\varphi^{-1}(e)$是闭子群。根据闭子群定理,$\ker \varphi$是嵌入李子群。

法2:常秩定理。引理:李群同态$\varphi: G \to H$是常秩映射。这是因为$\mathrm{rank}\varphi_{*e}=\mathrm{rank}\varphi_{*g}$。

引理证明:$\phi \circ L_g=L_{\varphi(g)}\circ \varphi$得到$\varphi_{*g}\circ(L_g)_{*e}=(L_{\phi(g)})_{*e}\phi_{*e}$。由于$L_g$是微分同胚,所以$\forall g \in G$,有$\mathrm{rank}\phi_{*g}=\mathrm{rank}\phi_{*e}$。

\end{proof}
\begin{theorem}[秩的整体性定理(Global rank theorem)]
    $\phi$是$M$到$N$的光滑映射,则$\phi$是常秩的。则有:

    (1)$\phi$是单射意味着$\phi$是浸入。

    (2)$\phi$是满射意味着$\phi$是淹没。

    (3)$\phi$是双射意味着$\phi$是微分同胚。
\end{theorem}
\begin{theorem}
    两个李群的连续同态是李群同态。
\end{theorem}
\begin{proof}
    设$\varphi:G \to H$是连续同态。则$\Gamma_\varphi:=\{(g,\varphi(g))|g \in G\}$是$G \times H$的闭子群。

    由闭子群定理知$\Gamma_\varphi$是李子群。

    故$P:\Gamma_\varphi \to G \times H \to G, \quad (g,\varphi(g)) \mapsto (g,\varphi(g))\mapsto g$是一个光滑映射,并且作为抽象群的同构,并且是李群的同态。

    现在只用说明$\Gamma_\varphi$到$G$的映射$P$的逆是光滑的。由于$P$是常秩的映射,从而根据Global rank theorem知这是微分同胚。则$\varphi:P_2 \circ P^{-1}$是李群同态。
\end{proof}
\begin{proposition}
    任何拓扑群都有唯一的光滑结构使之成为李群。但群上的拓扑结构不一定是唯一的。
\end{proposition}
我们考虑闭子群定理的逆命题:
\begin{proposition}
    设$G$是李群,$H$是$G$的嵌入李子群,则$H$是闭子群。
\end{proposition}
\begin{proof}
设$H$是嵌入李子群,对于$\forall g \in G$,存在邻域$U(g)$使得$U(g) \cap H=U(g)\cap \overline{H}$。(嵌入李子群的局部性质)。

令$g=e$,则$U(e)\cap H=U(e)\cap \overline{H}$。下证$\overline{H}\subset H$。

对于$ h\in \overline{H}$,$hU(e)\cap H \neq \emptyset$.取$h'\in hU(e)\cap H$,则$h'h \in U(e)$,又$h \in \overline{H}$,存在序列$\{h_n\} \subset H$使得$h_n \to h$。于是$\{h_n^{-1}h'\}$收敛于$h^{-1}h'$。于是$h{-1}h' \in U(e)\cap \overline{H}=U(e)\cap H$。故$h \in H$
\end{proof}
\section{李群同态和李代数同态}
\begin{proposition}
    设$\varphi: H \to G$是李群同态,则$\varphi_{*e} h \cong T_eH \to g \cong T_e G$是李代数同态。
\end{proposition}
\begin{lemma}
    $f:M \to N$的光滑映射。若$M$上的向量场$X_1,X_2$与$N$上的向量场$Y_1,Y_2$是$f$相关的(即$f_{*m}X_i(m)=Y_i(f(m))$),则$[X_1,X_2]$和$[Y_1,Y_2]$是$f$相关的。
\end{lemma}
\begin{proof}[命题3.4]
    只需要证明由$v \in T_eH$所诱导的左不变向量场$X$与$\varphi_{*e}(v) \in T_e G$诱导的左不变向量场$Y$是$\varphi$相关的。
    $$
    \varphi_{*g}(X(g))=\varphi_{*g}(Lg)_{*e}v=(L_{\varphi(g)})_* \circ \varphi_{*e}v=Y(\varphi(g))
    $$
\end{proof}
\begin{proposition}
    设$\phi$是李群同态$H \to G$。则图标可换:
\[\begin{tikzcd}
	H && G \\
	\\
	h && g
	\arrow["\phi", from=1-1, to=1-3]
	\arrow["{\mathrm{exp}}", from=3-1, to=1-1]
	\arrow["{\mathrm{exp}}"', from=3-3, to=1-3]
	\arrow["{\phi_*}"', from=3-1, to=3-3]
\end{tikzcd}\]
\end{proposition}
\begin{proof}
    考虑$\psi(t)=\phi(\mathrm{exp}tX)$。由于$\phi$是李群同态,则$\psi$是单参数变换群$R \to G$使得$\dfrac{d}{dt}|_{t=0} \psi(t)=\phi_{*e}\circ \mathrm{exp}_{*0}(X(e))$。从而:
    $$
    \mathrm{exp}_G(t\phi_{*e}(X))=\mathrm{exp}^{(H)}_{*0}(X(e))
    $$
    根据单参数变换群的唯一性。

    令$t=1$,就得到$\mathrm{exp}_G(\phi_*(X))=\phi(\mathrm{exp}_H(X))$。
\end{proof}
\section{李子群与李子代数}
\begin{proposition}
    $H$是$G$的李子群,则$\mathrm{Lie}(H):=h$是$g$的李子代数。
\end{proposition}
\begin{proof}
    设$i :H \to G$是包含映射。则$i_{*e}:h \to g$的李代数单同态。从而$i_{*(e)}h \cong h \subset g$是李子代数。
\end{proof}
\begin{proposition}
    设$G$是李群且$h$是$\mathrm{Lie}G=g$的子代数。则存在唯一的连通李群$H\subset G$使得:$\mathrm{Lie}H =h$。
\end{proposition}
\begin{proof}
    设$X_1,\dots,X_k$是$h \subset g$的基底,由于$X_i$是左不变的且$X_i(e)$的值确定了$X_i$,并且$\{X_i(e)\}$是线性无关的,则$\{X_i(g)\}$对于任意的$g$都是线性无关的。

    故$D_g=\mathrm{Span}\{X_1(g),\dots,X_k(g)\}$是$G$上的$k$维-分布。

    由于$[X_i,X_j] \in h=\mathrm{Span}\{X_1,\dots,X_k\}$。根据Frobenius定理,存在唯一的$D_g$的极大连通积分子流形$H \subset G$。

    下证$H$具有群结构。由于$X_i$左不变,$(L_h)_*(S_g)=S_g$。

    故$L_h H=H,\forall h \in H$。

    最后证明唯一性。设$K$亦是$V_g$的连通积分子流形。则$K\subset H$。由于$T_e K=T_eH$。由反函数定理,存在$U(e)\subset K,V(e)\subset H$使得$U(e)$和$V(e)$是微分同胚。由$H,K$群乘法相同且$H,K$连通,则$H$与$K$相同。
\end{proof}
\section{李的基本定理}
\begin{definition}
    设$G,H$是李群,$U(e) \subset G$,$V(e) \subset H$是邻域。

    (1)$f:U(e)\subset G \to V(e) \subset H$若满足$\forall g_1,g_2 \in U(e)$,使得$g_1g_2\in U(e)$且$f(g_1g_2)=f(g_1)f(g_2)$。则称$f$是$G,H$是局部同态。

    (2)若$f$还是(局部)微分同胚,称$f$是局部同构。 
\end{definition}
\begin{theorem}[李的第一基本定理]
    设$G$和$H$是局部同构的李群,则$\mathrm{Lie}G:=g$,$h$是同构的李代数。$(f: G \to H)$。
\end{theorem}
\begin{proof}
    局部同构能得到$f_{*e}$是双射且为李代数的同态。
\end{proof}
\begin{theorem}[李的第二定理]
    $g,h$是$G,H$的李代数。$\varphi:g \to h$同构推到出$G,H$的局部同构。
\end{theorem}
\begin{proof}
    令$a=\mathrm{Graph}(\rho)=\{(x,\rho(x))|x\in g\}$。
    $$
    [(x_1,\rho(x_1)),(x_2,\rho(x_2))]=([x_1,x_2],[\rho(x_1),\rho(x_2)])=([x_1,x_2],\rho[x_1,x_2])
    $$
    从而$a$是$g \oplus h$上的子代数。存在唯一连通的李子群$A \subset G\times H$使得$\mathrm{Lie}(A)=a$。

    设$i: A \to G \times H$是包含映射,则$\varphi:A \to G\times H \to G$是李群同态且$\varphi_{*e}$是$\mathrm{id}$。

    根据反函数定理及$\varphi$是同态,则$\varphi:A \to G$是局部同构。

    同理$\psi: A \to G\times H \to H$是李群同态,由于$\psi_{*e}:(x,\rho(x))\mapsto \rho(x)$是李代数同构,从而$\psi:A \to H$是局部同构。

    令$w(e)=\varphi^{-1}(l)\cap \psi^{-1}(v)$.则$\psi \circ \varphi^{-1}:\varphi(w(e)) \to \psi(w(e))$是局部同构。
    \end{proof}
    \begin{theorem}[李的第三定理]
        设$g$是有限维李代数,则存在唯一的单连通李群$\tilde{G}$使得$\mathrm{Lie}(\tilde{G})=g$。

        从而李代数和单连通有着一一对应的关系。
    \end{theorem}
    \begin{proof}
        根据Ado引理,$g$是$\mathrm{gl}(n,\C)$的子代数。则存在唯一连通的李子群$G \subset \mathrm{gl}(n,\C)$使得$\mathrm{Lie}(G)=g$。


    \end{proof}
    \chapter{2023.03.16}
    \section{覆盖群及其应用}
    \begin{definition}
        $G$是连通李群,$G$的覆叠空间$\tilde{G}$且$\tilde{G} \to G$是李群同态,则称$\tilde{G}$是$G$的一个覆盖群。
    \end{definition}
    \begin{proposition}\label{pro:cover}
        连通李群$G$的覆叠空间$\tilde{G}$自然蕴含李群结构且使得$\tilde{\pi}:\tilde{G}\to G$是李群同态。
    \end{proposition}
    \begin{lemma}
        设$\pi:X \to M$是连通流形上的覆盖。$Z$是连通流形且满足对于任何光滑映射$\alpha,\pi$,有$\alpha_* (\pi(Z))\subset \pi_*(\pi_1(X))$且$\alpha(z_0)=m_0$。对于$\forall x_0 \in$ 
    \end{lemma}
    \begin{proof}[命题\ref{pro:cover}]
        我们说明$\tilde{G}$有群结构。考虑图表:\begin{tikzcd}
	&& {\tilde{G}} \\
	\\
	{\tilde{G}\times\tilde{G}} && G
	\arrow["\pi", from=1-3, to=3-3]
	\arrow["{\tilde{\alpha}}", from=3-1, to=1-3]
	\arrow["\alpha"', from=3-1, to=3-3]
\end{tikzcd}

    其中$\alpha(\tilde{g_1},\tilde{g_2})=\pi(\tilde{g_1})\pi(\tilde{g_2})^{-1}$。由$\alpha$定义得到:
    \begin{align}
        \alpha_*(\pi_1(\tilde{G}\times \tilde{G}))\subset \pi_*(\pi_1(\tilde{G}))
    \end{align}
    任取$\tilde{e}\in \pi^{-1}(e)$,则存在唯一的$\tilde{\alpha}:\tilde{G} \times \tilde{G} \to \tilde{G}$使得其为提升且$\tilde{\alpha}(\tilde{e},\tilde{e})=\tilde{e}$。

    我们定义$\tilde{G}$中元素的逆元。对于任意的$\til{g},\til{g_1},\til{g_2}$,定义$\til{g}$的逆元为$\til{\alpha}(\til{e},\til{g})$,$\til{g_1}\til{g_2}=\til{\alpha}(\til{g_1},\til{g_2}^{-1})$.
\end{proof}
\begin{example}
    $\mathrm{Sp}(1) \times \mathrm{Sp}(1)$是$\mathrm{SO}(4,\R)$的覆盖群。

    对于$a,b \in \mathbb{H}$,考虑$T_{ab}:\mathbb{H} \to \mathbb{H} \cong \R^4,v \mapsto avb$.可以验证$T_{a,b}\in \mathrm{SO}(4,\R)$。
\end{example}
\begin{example}

\end{example}
\begin{theorem}
    $G,H$连通子群,$\Phi: G \to H$是李群同态,则$\Phi$是李群覆盖等价于$\Phi_{*e}:g \to h$是李代数同构。
\end{theorem}
\begin{theorem}
    $G,H$是李群,$G$是单连通的。若$\varphi:g  \to h$是李代数同态,则存在唯一的李群同态$\Phi:G \to H$满足$\Phi_{*e}=\varphi$。
\end{theorem}
\begin{proof}[证法2:BCH公式]
    \textbf{BCH公式是指:}
    
    设$G$是李群,设$X,Y\in g$,$\|X\|$和$\|Y\|$足够小,则定义:
    \begin{align}
        X *Y:=\mathrm{log}(\mathrm{exp}X \mathrm{exp} Y)=X+Y=\sum_{m\geq 2}P_m(X,Y)
    \end{align}
    其中$P_m(X,Y)$是由$m-1$层$X,Y$李括号的线性求和:
    \begin{align}
        P_2(X,Y)=1/2[X,Y] \quad P_3=1/12([X,[X,Y]]-[Y,[X,Y]])\quad P_4=1/24 [X,[Y,[Y,X]]]
    \end{align}
     
    接下里阐述证明:

    首先构造局部的同态:令$\Phi:=\mathrm{exp}\varphi \mathrm{log}:U \to H$,$U$是满足BCH公式的足够小邻域。

    对于$A,B \in U$,令$X=\log A,Y=\log B$,则:
    \begin{align}
        \Phi(AB)=\Phi(\mathrm{exp})
    \end{align}
\end{proof}
\begin{definition}[李群中心]
    $Z(G)$定义为$G$的交换李子群。$Z(g)=\{Z \in g|[Z,X]=0,\forall X \in g\}$是李代数的中心。
\end{definition}
\begin{theorem}
    设$G,\til{G}$是连通李群:
    \begin{enumerate}
        \item 若$\Phi:\til{G}\to G$是李群覆叠,则$\ker \Phi$是$Z(G)$的离散子群。
        \item 若$\Gamma$是$Z(G)$的离散子群,则$G/\Gamma$是李群且$\Phi$是李群覆盖。
    \end{enumerate}
\end{theorem}
\begin{proof}
    令$\Gamma$是$\ker \Phi$。由于$\Phi$是局部微分同胚,则存在$U(e)$是$\til{G}$的开子集,满足$U(e)\cap \Gamma=\{e\}$。对于$\forall r\in \Gamma$,$rU\cap \Gamma=\{r\}$。则$\Gamma$是离散子群。

    对$\forall g \in \til{G}$,$\gamma \in \Gamma$
\end{proof}
\begin{corollary}
    $G$是连通李群,$\mathrm{Lie}G=g$,则$G \cong \til{G}/\Gamma$,$\Gamma$是$Z(G)$中的离散子群。


\end{corollary}
\begin{example}
    设$G$是$\mathrm{Sl}(n,\R)$的覆盖群。则其不是矩阵李群。换言之,不存在单的李群同态:$\varphi:G \to \mathrm{GL}(n,\R)$。 

    为了说明这一事实,我们采取反证法。即假设存在$\varphi$。考虑图表:
    
\end{example}
\chapter{2023.03.23}
\section{李群的基本群求法}
以$\mathrm{SL}(n,\R)$为例。考虑$n \geq 3$的情况。

第一步使用极分解。即$\mathrm{SL}(n,\R)=\cong \mathrm{SO}(n,\R)\times \R^m$。从而:
\begin{align}
    \pi_1(\mathrm{SL}(n,\R))=\pi_1(\mathrm{SO}(n,\R))
\end{align}
在已知同伦群的作用下构造可迁群作用。(轨道唯一,纤维从):
\begin{align}
    \mathrm{SO}(n+1,\R)\times S^n \to S^n: A \times (e_1,\dots,e_{n+1})^T=A(e_1,\dots,e_{n+1})^T
\end{align}
考虑稳定化子:$\mathrm{Stab}(e_1,0,0,\dots,0)=\begin{pmatrix}
    1& \quad \\ \quad &\mathrm{SO}(n,\R)
\end{pmatrix}$
从而$S^n\cong \mathrm{SO}(n+1,\R)/\mathrm{SO}(n,\R)$得到正合列:$\mathrm{SO}(n) \to \mathrm{SO}(n+1) \to S^n$。从而诱导长正合列:
\begin{align}
    \to \pi_{i+1}(C) \to \pi_i(A)\to \pi_1(B) \to \pi_1(C) \to \pi_{i-1}(C) \to
\end{align}
已知$\pi_i(S^n)=0,i=1,2,\dots,i-1$.上式带入$n=3$,有:
\begin{align}
    0=\pi_2(S^3) \to \pi_1(\mathrm{SO}(3)) \to \pi_1(\mathrm{SO}(4)) \to \pi_1(S^3)=0
\end{align}
从而$\pi_1(\mathrm{SO}(3))\cong \pi_1(\mathrm{SO}(4))$。同理$\pi_1(\mathrm{SO}(3)) \cong \pi_1(\mathrm{SO}(n))$。

\section{李代数的复化和实形式}
复李代数$\C$向量空间,有李括号$[,]$。

\begin{proposition}
    一个复李代数$(\mathfrak{g},[,])$可以看为实李代数$(\mathfrak{g},[,],I)$。$I$是$R$线性变换.且$I^2=-\mathrm{id}$。且$[Iu,v]=[u,Iv]=I[u,v]$。
\end{proposition}
\begin{proof}
    先给定$(\mathfrak{g},[,])$作为复李代数。定义$I:\mathfrak{g}^{\R}\to \mathfrak{g}^{\R}$为$X \to iX$。

    给定$(\mathfrak{g}^{\R},[,])$,定义数乘:
    $$
    \C \times \mathfrak{g}^{\R} \to \mathfrak{g}^{\R},(a+bi)(u):=au+bIu
    $$
\end{proof}
李代数的复化。设$\mathfrak{g}$是实李代数,$\mathfrak{g}_{\C}=\mathfrak{g}\oplus i \mathfrak{g}$是向量空间的复化。定义:
\begin{align}
    [u+iv,x+iy]=[u,x]-[v,y]+i[u,y]+i[v,x]
\end{align}
则称$(\mathfrak{g}_{\C},[,])$称为$(\mathfrak{g},[,])$的复化。

\begin{proposition}
    $\mathfrak{g}_{\C}\cong ({\mathfrak{g}_{\C}}{\R},I)$。其中$I(u+vi)=-v+ui \in {\mathfrak{g}_{\C}}^{\R}$.
\end{proposition}
\begin{definition}[实形式]
    设$h$是实李代数$\mathfrak{h}_{\C}\cong \mathfrak{g}$,则是实形式。
\end{definition}
\begin{proposition}
    实形式等价于$\tau$是$\mathfrak{g} \to \mathfrak{g}$的共轭线性对合自同构。
\end{proposition}
\begin{remark}
    \begin{enumerate}
        \item 并非所有复李代数都有实形式。若实形式存在,则$(\g,I)\cong (\g,-I)$。
        \item 实形式不一定唯一。比如$\mathrm{SL}(n,\C)$:
        \begin{align}
            \tau(x)=\overline{x} \Rightarrow \g^{\tau}=\mathrm{SL}(n,\R)
        \end{align}
        分裂实形式。
        \begin{align}
            \tau(x)=-x^* \Rightarrow \g^\tau=\mathrm{SU}(n)
        \end{align}
        紧实形式。
    \end{enumerate}
\end{remark}
\section{复流形}
复流形是具有复结构的实流形。即$M$上有开覆盖$\{U_\alpha\}$,其中$\{U_\alpha\}$与$\C$中的开子集微分同胚。使得$U_\alpha$具有复坐标$(z_1,\dots,z_n)$,满足任意坐标变换的转移函数全体光滑。

然而复结构的流形很难做实际的验证。我们考虑$(M,J)$,$J$是一个$(1,1)$型张量,$J :TM \to TM$满足$J^2=-\mathrm{id}$。$J$称为近复结构。

当$M$是偶数维,$J_x:T_x M \to T_x M$,$J_x^2=-\mathrm{id}$意味着$\mathrm{det}(J_x)^2=(-1)^n>0$

近复结构是复结构等价$\forall X,Y \in \Gamma(TM)$,
\begin{align}
    [JX,JY]-J[JX,Y]-J[X,JY]-[X,Y]=0
\end{align}
\begin{definition}[复李群]
    复流形且是个群。群乘法和逆都是全纯的。
\end{definition}
\begin{theorem}
    一个连通李群$G$是复李群等价于$G$的李代数是复的李代数。
\end{theorem}
\begin{proof}
    $(G,J)$
\end{proof}

\section{泛包络代数}
对于域$\F$的李代数$\mathfrak{g}$,存在唯一的结合代数$U(\mathfrak{g})$以及双线性映射$i:\mathfrak{g}\to U(\mathfrak{g})$使得:
\begin{enumerate}
    \item $i[X,Y]=i(X)i(Y)-i(Y)i(X)$
    \item $U(g)$由$i(g)$生成。
    \item 
\end{enumerate}
\chapter{2023.3.30}
\section{$U(g)$}

定义$T(g)$为:
\begin{align}
    T(g)=\bigoplus_{k=0}^\infty g^{\oplus k}
\end{align}
其中加法是形式的加法,乘法我们只考虑基:单纯做张量积即可。于是上述结构其实是一个分次环。

定义$U(g)=T(g)/(e_i \otimes e_j-e_j \otimes e_i-[e_i,e_j])$。

\begin{theorem}[PBW]
    设$\{a_1,\dots,a_m\}$是$\g$的基底,则
\end{theorem}
\begin{remark}
    \begin{enumerate}
        \item PBW基形式上与$k[x_1,\dots,x_n]$保持一致。但是不交换。
        \item $U(g)$上有滤子$F_0 \subset F_1 \subset F_2 \dots$,其中:
        $$
        F_k=\mathrm{Span}_{\F}(a_1^{k_1}\dots a_n^{k_n},k_1+\dots+k_n \leq k)
        $$
        得到交换的$k$代数:
        $$
        F_0 \oplus F_1/F_0 \oplus \dots \cong S(g)\cong k[x_1,\dots,x_n] 
        $$
        其中$S(g)$是对称的张量集。
        \item 考虑单射$i:\g \to U(\g)$,则$i(a_1),\dots,i(a_n)$是线性无关的。
    \end{enumerate}
\end{remark}
\begin{definition}[$U(\g)$的上乘法]
    定义$\Delta:U(\g)\to U(\g)\otimes U(\g)$。我们给出基底的定义即可:
    设$e_i$是$\g$的基底,由:
    $$
    \Delta(e_i)=e_i \otimes 1+1 \otimes e_i
    $$
    并且直接同态的定义乘法:
    $$
    \Delta(e_i\otimes e_j)=\Delta(e_i)\Delta(e_j)
    $$

\end{definition}
\begin{definition}
    对于$r \in U(\g)$,若$\Delta(r)=r \otimes 1+1 \otimes r$,则称其为本原元(primitive)。若$\Delta(r)=r \otimes r$,则称其为类群元(grouplike)。 
\end{definition}
\begin{proposition}
    \begin{enumerate}
        \item 若$r,s$本原,则$[r,s]$是本原的。即本原元构成李代数。
        \item 若$r,s$是类群元,则$rs$是类群元。从而所有的类群元构成一个$\g$的形式李群。
        \item 若$r$是本原的,则$\exp r$是$\hat{U(\g)}$(定义为PBW基生成的形式幂级数)的类群元。
        \item 若$r$是类群元,且常数项是$1$,则$\log r$是$\hat{U(\g)}$的本原元。
    \end{enumerate}
\end{proposition}
\begin{theorem}
    设$\g$是特征为$0$域上的李代数,则$\g$是$U(\g)$中所有的本原元。
\end{theorem}
\begin{proof}
    先设$\g$是交换的基底。此时$U(\g)=k[x_1,\dots,x_n]$。考虑:
    $$
    \Delta(f)=1\otimes f+f \otimes 1 \Leftrightarrow f(x)+f(y)=f(x+y),\forall x,y \in k^n
    $$
    对于$f^{(k)}$中的$k$次齐次多项式:
    $$
    2^k f^{(k)}(x)=f^{(k)}(2x)=2f^{(k)}(x) \Rightarrow (2^k-2)f^{(k)}=0
    $$
    于是$\mathrm{deg}(f)=1$。

    接着考虑非交换。$U(\g):F_0 \subset F_1 \subset F_2$。接着考虑:
    $$
    \mathrm{Span}_k\{a_1^{k_1}\dots a_n^{k_n}:k_1\dots k_n \leq k\}
    $$
\end{proof}
\begin{example}
    反例:$\g$是一维的,$\mathrm{char}k=p>0$.基底$X \in \g$,则$X^p$也是本原元。
\end{example}
\begin{theorem}[BCH公式]
    $\exp X \exp Y=\exp(X+Y+1/2[X,Y]+[X,[X,Y]],\dots)$
\end{theorem}
\begin{proof}
    不妨假设$X,Y$线性无关。
\end{proof}
\begin{remark}
    BCH对特征为$p$的李代数不成立。
\end{remark}
\section{代数群和李群}
\begin{definition}
    代数群:$G$.$k$上的仿射态射簇(多项式零点集)且是个群。满足群乘法是态射(多项式函数)
\end{definition}
\begin{proposition}
    当$k=\R$,任何$k$上的代数群都是李群且是$\mathrm{Gl}(n,\C)$的闭子群。即矩阵李群。则非矩阵李群不是代数群。
\end{proposition}
代数群$G$的李代数是$k[G]$上的满足$\delta \circ L_g=L_g \circ \delta$的导子$\delta$。李括号受到$\mathrm{Char}k$的影响。若$k$的特征是$0$,则$[\delta_1,\delta_2]=\delta_1\circ \delta_2-\delta_2 \circ \delta_1$。
\section{李群李代数的表示}
$V$是复的向量空间。$\mathrm{GL}(V)$是$V$上的线性同构构成的集合,自然根据维数有:$\mathrm{GL}(V)=\cong \mathrm{GL}(n,\C)$。$\mathrm{gl}(V)$是$V$上线性变换。$\mathrm{gl}(V)\cong \mathrm{gl}(n,\C)$。

\begin{definition}
    设$G$是李群,光滑的李群同态:$\rho:G \to \mathrm{GL}(V)$称为$G$的复表示。

    设$\g$是李代数,则李代数同态:$\g \to \mathrm{gl}(V)$称为$\g$的表示。
\end{definition}
\begin{remark}
    \begin{enumerate}
        \item 李群表示等价于线性群作用
        \item 李代数表示等价于$\g$模。
        \item 对于任何的李代数表示$\pi:\g \to \mathrm{gl}(V)$都存在唯一的$U(\g)$上的同态是的交换图成立(泛性质):\begin{tikzcd}
            {U(\mathfrak{g})} \\
            \\
            {\mathfrak{g}} && {\mathrm{gl}(V)}
            \arrow[from=3-1, to=1-1]
            \arrow["\pi"', from=3-1, to=3-3]
            \arrow["{\tilde{\pi}}", from=1-1, to=3-3]
        \end{tikzcd}
    \end{enumerate}
\end{remark}
\begin{definition}[伴随表示]
    李代数:$\mathfrak{g} \to \mathrm{gl}(\g):X \to \mathrm{ad}_X \in \mathrm{gl}(\g)$称为$\g$的伴随表示。$\mathrm{ad}_X(Y)=[X,Y]$。
\end{definition}
Jacobbi恒等式:

\chapter{2023.04.06}
\subsection{李代数的表示和李代数模}
\begin{proposition}
    李代数$\g$的自同构群是$\mathrm{GL}(\g)$的嵌入李子群,李代数为$\mathrm{Der}(\g)$。其中$\mathrm{Der}(g)$是导子李代数。
\end{proposition}
\begin{proof}
    这是因为$\mathrm{Aut}(\g)$是由方程$A[x,y]=[Ax,Ay]$定义的。因此$\mathrm{Aut}(g)$是$\mathrm{GL}(\g)$的闭子群,从而是嵌入李子群。

    考虑$\mathrm{Aut}(\g)$的李代数:
    \begin{align}
        \mathrm{Lie}(\mathrm{Aut}(\g))=\{D \in \mathrm{gl}(\g)|e^{tD}\in \mathrm{Aut}(\g),\forall t\}
     \end{align}
    即$e^{tD}[X,Y]=[e^{tD}X,e^{tD}Y]$
    
    下面证明导子满足上述要求。

    考虑$\g$值函数$y_1(t)=e^{tD}[X,Y],y_2=[e^{tD}X,e^{tD}Y]$.当$t=0$,$y_1=y_2$。为了证明$y_1=y_2$,验证发现$y_1'=y_2'$。根据ODE解的存在唯一性,$y_1=y_2$。
    \end{proof}
    
\begin{proposition}
    设$G$是连通李群,则:(a)$\ker \mathrm{Ad}=Z(G)$ (b)$\mathrm{Int}(\g)= \mathrm{Im}(\mathrm{Ad}) \cong G/Z(G)$。
\end{proposition}

\begin{definition}
    \begin{enumerate}
        \item 如果表示是单射,则称为忠实表示。
        \item $V$的子空间$W$满足$g \cdot W=\{\pi(g)w, \forall w \in W\} \subset W$,$\forall g \in G$。
        \item 若表示$(G,\pi,V)$的不变子空间只有$0$,$V$,则称表示$\pi$是不可约的。
    \end{enumerate}
\end{definition}
\begin{example}[非矩阵李群]
    设$G=\R \times \R \times S^1$。定义乘法为$(x_1,y_1,u_1)(x_2,y_2,u_2)=(x_1+x_2,y_1+y_2,e^{ix_1y_2}u_1u_2)$.

    对于$G$的任意表示(有限维),$\pi_G:G \to \mathrm{GL}(V)$,$\pi_G$都不是忠实的。

    考虑海森堡群$$
    H=\{\begin{pmatrix}
        1&a&b\\0&1&c\\0&0&1
    \end{pmatrix}|a,b,c\in \R\}$$
    以及李群同态:$\Phi:H \to G$.$G \to \mathrm{GL}(V)$。

    其中
    $$
    \begin{pmatrix}
        1&a&b\\0&1&c\\0&0&1
    \end{pmatrix} \mapsto (a,c,e^{ib}) \mapsto \pi_G(a,c,e^{ib})
    $$

    $\varphi$的核是$\ker \Phi=\{\begin{pmatrix}
        1&0&2n \pi\\0&1&0\\0&0&1
    \end{pmatrix}|n \in  \Z\}$。是$Z(H)$的离散正规子群。

    于是$H$是$G$的覆盖群。李代数相同。为
    $$
    \mathfrak{h}=\{\begin{pmatrix}
        0&a&b\\0&0&c\\0&0&0
    \end{pmatrix}|a,b,c \in \R\}
    $$
    基底为$A=\begin{pmatrix}
        0&1&0\\0&0&0\\0&0&0
    \end{pmatrix},B=\begin{pmatrix}
        0&0&1\\0&0&0\\0&0&0
    \end{pmatrix},C=\begin{pmatrix}
        0&0&0\\0&0&1\\0&0&0
    \end{pmatrix}$.且$[A,C]=B,[A,B]=[C,B]=0$。
\end{example}
\begin{proposition}
    设$\pi$是$H$的表示。若$\ker \Phi \subset \ker \pi$,则$Z(H)\subset \ker \pi$.
\end{proposition}
若上述性质成立,假设存在忠实表示$\pi_G$,则$\ker \pi_H=\ker \Phi \subset Z(H) \subset \pi_H$矛盾!从而前面我们的$G$没有忠实表示。

\begin{proof}
    考虑两个引理:$\pi(B)$是幂零矩阵。$X$是非零幂零矩阵,则$e^{tX}=I$等价于$t=0$。

    我们说明根据两个引理可以得到上述命题。

    由于$e^{tB}=\begin{pmatrix}
        1&0&t\\0&1&0\\0&0&1
    \end{pmatrix}$,$e^{kn\pi(B)}=\pi(e^{knB})=I(\ker \Phi \subset \ker \pi)$。对所有的$n \in \Z$成立,则由引理1,2知$\pi(B)=0$。因此对于$t \in \R$,$\pi(e^{tB})=e^{t\pi(B)}=I(Z(H)\subset \ker \pi)$。

    对于两个引理的证明,我们放在下面。
\end{proof}
\begin{lemma}
    $\pi(B)$是幂零矩阵。
\end{lemma}
\begin{proof}
    
\end{proof}
\begin{lemma}
    $X$是非零幂零矩阵,则$e^{tX}=I$等价于$t=0$。
\end{lemma}
\begin{proof}
    由于$X$幂零,$e^{tX}$是关于$t$的多项式,因此存在$P_{jk}(t)$使得$(e^{tX})
    _{jk}=P_{jk}(t)$.

    假设$\exists t \neq 0$,使得$e^{tX}=I$。则$e^{ntX}=(e^{tX})^n=I$。从而$P_{jk}(nt)=\delta_{jk}$。于是$e^{tX}\equiv I$.这说明$e^{tX}$不显含$t$。对$e^{tX}$求导,则$Xe^{tX}=X=0$。因此$X=0$与题设矛盾!
\end{proof}
\chapter{2023.04.13}
\subsection{不变内积的存在性}
\begin{theorem}
    交换李群的表示都是1维的。
\end{theorem}
\begin{proof}
    设表示为$(G,\pi,V)$。对于$\forall g \in G$,$\pi(g): V \to V$是$G$可换的。则$\pi(g)=\lambda \mathrm{id}$。从而$V$的任何子空间一定是不变子空间,因此$V$是1维的。
\end{proof}
Haar测度:紧李群存在左右不变的积分(等价于测度)

即$\forall f \in C^{\infty}(G)$:
$$
\int_G f(g)\omega =\int_G f(hg)\omega= \int_G f(gh)\omega =\int_G f(g^{-1})\omega
$$
其中$\omega$是体积形式,即$\int 1\omega=1$。

第一步,在一般的李群上定义左不变形式。在$e \in G$,取定$\omega_e \in \wedge^n T_e^* G$,$n =\mathrm{dim}G$。在$G$上定义体积形式:
\begin{align*}
    \omega \in \Omega^n(G) \Leftrightarrow (L_h^*\omega)(g)=\omega(hg), \forall g,h \in G
\end{align*}
从而
\begin{align*}
    \int_G f(g)\omega(g)\text{是左不变的,即}\int_G f(hg)\omega(hg)=\int_G f(g)\omega(g)
\end{align*}
我们对$\omega$做正规化,即定义:
\begin{align*}
    \int_G 1\omega=1
\end{align*}
我们称正规的左不变测度为左Haar测度。

第二步,我们说明紧李群上模函数恒为$1$,这等价于左不变测度是右不变的。

由于对于$\forall g \in G$,$R_g^* \omega$仍然是左不变的,左不变$n$形式是$1$维向量空间。故存在$\Delta:G \to R_{>0}$(称为模函数)使得$\omega=\Delta(g)R_g^*(\omega)$。

下证:紧李群左Haar测度是右不变的等价于$\Delta \equiv 1$。

思路:证明$\Delta$是李群(反)同态。

考虑$\omega(hg_1g_2)=\Delta(g_1g_2)(R_{g_1g_2}^*)h=\Delta(g_1g_2)R_{g_1}^*(R_{g_2}^*\omega)h$。

又$R_{g_2}^*\omega$是左不变的,则$(R_{g_2}^*)(hg_1)=\Delta(g_1)(R_{g_1}^*)(R_{g_2}^*\omega)h$。

于是$\omega(hg_1g_2)=\Delta(g_2)R_{g_2}^*(hg_1)=\Delta(g_2)\Delta(g_1)R_{g_1}^*(R_{g_2}^*\omega)h$.

因此可以看出来$\Delta$是反同态。但是$R$是交换的,从而这也是同态。

由于$(R_{+},\times)$紧子群只有$\{1\}$,因此$\Delta(g)\equiv 1$.因此紧李群是右不变的。

\begin{theorem}
    紧李群表示$G \to \mathrm{GL}(V)$表示空间上有不变内积。
\end{theorem}
\begin{proof}
    取$V$的一个内积$\langle,\rangle$。在$V$上定义新的内积:\begin{align*}
        \langle v,u \rangle:=\int_G \langle g\cdot v,g\cdot u\rangle dg, g\text{是Haar不变测度}
    \end{align*}
    根据积分的左右不变性:
    $$
    \langle hv,hu\rangle_G =\langle u,v\rangle
    $$
\end{proof}
\begin{corollary}
    紧李群李代数$\g$上存在Ad不变的内积。
\end{corollary}
\begin{theorem}
    紧李群的表示完全可约。
\end{theorem}
\subsection{一个例子:$\mathrm{SU}(2)$}
首先考虑一个例子。这个例子本身很重要.
\begin{example}[$\mathrm{SU}(2)$的表示]
    \textbf{1.李代数方法:}

    目标给出了$\mathrm{SU}(2)$的不可约表示的分类和构造。
    $\mathrm{SU}(2)=\{M \in \mathrm{GL}(2,\C):M^*M=I,\mathrm{det}M=1\}\cong S^3$单连通。
\end{example}
考虑$\mathrm{SU}(2)$的

\chapter{2023.04.20}
令$W_{kl}=\{v \in V:D(\theta_1,\theta_2)v=e^{i(k\theta_1+l\theta_2)},k,l \in \Z\}$权为$(k,l)$的权空间。则$V=\oplus_{k,l \in \Z} W_{kl}$。

$\mathrm{SU}(3)$的李代数的复化为$\mathrm{sl}(3,\C)$。考察$\mathrm{sl}(3,\C)$基底在$W_{kl}$的作用。

李代数$\mathrm{sl}(3,\C)$的基底:
\begin{align*}
    H_1=\begin{pmatrix}
        1&0&0\\0&-1&0\\0&0&0
    \end{pmatrix},H_2=\begin{pmatrix}
        0&0&0\\0&1&0\\0&0&-1
    \end{pmatrix}
\end{align*}

\begin{remark}
    当$k,l$固定的时候,$\lambda:=k\theta_1+l\theta_2$可以看为线性函数:$\mathfrak{h} \to \C$,$\begin{pmatrix}
        \theta_1&0&0\\0&\theta_2&0\\0&0&-\theta_1\theta_2
    \end{pmatrix}\mapsto k\theta_1+l\theta_2$

    即$\lambda\in \mathfrak{h}^*$。

    此时,权空间$W_{kl}$记为:
    $$
    W_\lambda=\{v:H(\theta_1,\theta_2)v=\lambda(H(\theta_1,\theta_2))v,\text{所有}H(\theta_1,\theta_2)\in \mathfrak{h}\}
    $$
\end{remark}
\begin{example}[标准表示]
    设$\C^3=\mathrm{span}\{e_1,e_2,e_3\}$。
\end{example}
\begin{example}[$\mathrm{Sym}^2\C^3$]
    考虑$e_1^2=e_1\otimes e_1$
\end{example}
\begin{example}[伴随表示]
    设$H(\theta_1,\theta_2)$如上。设$\theta_3=\theta_1+\theta_2$。

    (a)$i=j$
\end{example}
\chapter{李群的表示}
\begin{definition}[基本权]
    1.设$\{H_i\}$为$(h_i,\langle \rangle)$的标准正交基,定义:
    \begin{align*}
        \lambda_i(H_j)=\delta_{ij},\lambda_i \in h^*
    \end{align*}
    $\lambda_i$称为基本权,$\lambda$是整支配权 等价于$\lambda=\sum k_i\lambda_i,k_i \in \N$.

    2.素根系的基本权:

    (a)当选定$h$和素根系$\Phi$,基本权$\lambda_i:=\dfrac{2\langle \lambda_i,\alpha_j\rangle}{\langle \alpha_j,\alpha_j\rangle}=\delta_{ij},\forall \alpha_i \in \Phi$
\end{definition}
\begin{example}
    $\mathrm{SU}(2)=\mathrm{sl}(2,\C)$。素根系$\Phi=\{L_1\}$。

    Cartan矩阵$A=(2)$。则$\alpha_1=2\lambda_1$,$\lambda_1=1/2\alpha_1$。
\end{example}
\textbf{基本权空间构造}
Type $A_n$:$\mathrm{SU}(n+1)$与$\mathrm{sl}(n+1,\C)$。

基本权:$\lambda_i=L_1+L_2+\dots+L_i$,$1 \leq i \leq n$.权系:$\Gamma_w=\{\sum c_i\lambda_i|c_i \in \Z\}=\{\sum k_i L_i|k_i \in \Z\}$。

整支配权:$\Gamma_w^d=\{\sum c_i \lambda_i|c_i \in \N\}=\{\sum k_i L_i|k_1 \geq k_2 \dots k_n\}$。

设$R_n$是$\mathrm{SU}(n+1)$的基本表示。
\section{Schur正交化定理}
\begin{theorem}
    设$(\pi_1,V_1)$和$(\pi_2,V_2)$是紧李群$G$的不可约表示。$\langle,\rangle_i$是$V_i$上的$G$不变内积,$i=1,2$。则有:
    \begin{align*}
        \int_G \langle \pi_1(x)u_1,v_1\rangle_1\langle \pi_2(x)u_2,v_2\rangle_2 dx=0
    \end{align*}
\end{theorem}
\begin{proof}
    设$l:V_2 \to V_1$是线性映射,定义新的线性映射:$L_2$
    \begin{align*}
        L_2:V_2 \to V_1,\quad L=\int_G \pi_1(x)\circ l \circ \pi_2(x^{-1})dx
    \end{align*}
    下面验证$L$是$G$可换的。这等价于:
    \begin{align*}
        \forall y\in G,\pi_1(y)\circ L \circ \pi_2(y^{-1})=L
    \end{align*}
    对于$v_2 \in V_2$,有:
    \begin{align*}
        \pi_1(y) \circ L \circ \pi_2(y^{-1})v=\pi_1(y)\int_G \pi_1(x) \circ l \circ \pi_2((yx)^{-1})v_2 dx=\int_G \pi_1(x) \circ l\circ \pi_2(x^{-1})v_2dx=Lv_2
    \end{align*}
    由于$\pi_1,\pi_2$是不等价的,根据Schur引理,这说明$L=0$。故$\langle Lv_2,v_1\rangle_1=0$。

    接下来我们令$l:V_2 \to V_1, \omega_2 \mapsto \langle \omega_2,u_2\rangle_2 u_1$。对于$\forall \omega_2 \in V_2$:
    \begin{align*}
        0=\langle Lv_2,v_1\rangle_1=\int_G \langle \pi_1(x)\circ l\circ \pi_2(x^{-1})v_2,v_1 \rangle_1 dx \text{带入}l,\text{根据内积不变得到结果。} vr
    \end{align*}
\end{proof}
\begin{definition}
    对于任意给定$v,L \in V^*$,$\phi:G \to \C$.$\phi(g)=L(\pi(g)v)$,称为$G$的矩阵系数。
\end{definition}
\begin{remark}
    当$(\pi,V,\langle,\rangle_G)$是酉表示(eg.$G$是紧李群):
    \begin{align*}
        \phi(g):=\langle \pi(g)v,u\rangle_{G'},\text{给定}u,v \in V
    \end{align*}
\end{remark}
\begin{theorem}
    $\phi$是矩阵系数当且仅当$\mathrm{Span}(R_g^*\phi:g \in G)$是有限维向量空间,其中$R_g: G \to G \forall g,h  \mapsto hg$。
\end{theorem}
\begin{theorem}[Schur正交化定理]
    $G$是紧李群。$\pi_1,\pi_2$是不等价的表示。设$\phi_1,\phi_2$分别是对应的矩阵系数,则有:
    \begin{align*}
        \int_G \phi_1(g)\phi_2(g)dg=0
    \end{align*}
\end{theorem}
\begin{corollary}
    \begin{align*}
    \int_G \chi_1(g)\overline{\chi_2(g)}dg=0
    \end{align*}
\end{corollary}
下面两个判别法可以判定是否有等价表示:
\begin{theorem}
    1.$(\pi_1,V_1)$不可约等价于:
    \begin{align*}
        \int_G |\chi_{\pi_1}(g)|^2dg=1
    \end{align*}
    
    2.两个表示等价等价于$\chi_{\pi_1}=\chi_{\pi_2}$。
\end{theorem}
\begin{remark}
    设$\pi=\bigoplus_{i=1}^n \pi_i$是紧李群$G$上的表示且$\tau$是$G$的不可约表示,则:
    \begin{align*}
        \int_G \chi_{\tau}(g)\chi_{\pi}(g)dg
    \end{align*}
    是与$\tau$等价的不可约表示的个数,即在$\pi$中的重数。
\end{remark}
\subsection*{类函数}
\begin{definition}
    类函数定义为$\phi: G \to \C$使得$\phi(ghg^{-1})=\phi(h)$,$\forall g,h \in G$.
\end{definition}
因而特征标是连续的类函数。我们用特征标来对表示进行分类。
\begin{example}[$T^n=(S^1)^n$]
    $T^n=(S^1)^2$的不可约分类。

    注意到有用的事实:$T^n$是交换李群,所以其不可约表示只可能是$1$维。考虑不可约表示族:
    \begin{align*}
        (e^{i\theta_1},\dots,e^{i\theta_n})\cdot v=e^{i(\sum_{k=1}^n m_i\theta_i)}\cdot v,m_i \in \Z
    \end{align*}
    我们证明$T^n$的不可约表示与$\Z^n$中的格点有一一对应。

    首先说明对于不同的对$n$元整数对,上面的表示都是不等价的。

    注意到这是1维表示,则特征标是明显的。因此特征标显然不同。

    由于$L^2((S^1)^n)$的基恰为:
    \begin{align*}
        \{e^{i(m_1\theta_1+\dots+m_n\theta_n)}:m_1,\dots,m_n \in \Z\}
    \end{align*}
   故不存在不可约表示的特征标$\chi$与上面所有的$\chi_{\pi}$都正交,则不存在其他不可约表示。
\end{example}
\begin{example}[$\mathrm{SU}(2)$]
    $\mathrm{Sym}^n \C^2$。

    基底$\{e_1^k,e_2^{n-k}\}=\mathrm{Span}\{z_1^kz+2^{n-k}:0\leq k\leq n\}$.

    表示可以由标准表示诱导的线性作用:
    \begin{align*}
        (g:p)(v):=p(g^{-1}v)
    \end{align*}

    \begin{proof}
        先证明不可约推导不等价。

        $\forall g \in \mathrm{SU}(2)$,特征根$e^{i\theta},e^{-i\theta}$,因此$g \sim \begin{pmatrix}
            e^{i\theta}&0\\0&e^{-i\theta}
        \end{pmatrix}:=t(\theta)$
       对于$\forall 0\leq k \leq n$,令$p_k:=z_1^{k}z_2^{n-k}$。由定义:$\pi_n(t(\theta))p_k$,所以算得$\chi_{\pi_n}(g)=\chi_{\pi_n}(t(\theta))=\sum_{k=0}^n e^{i(2k-n)\theta}=\dfrac{\sin (n+1) \theta}{\sin \theta}$。所以计算有:
     \begin{align*}
        \int_{\mathrm{SU}(2)}|\chi_{\pi_n}(g)|^2dg=\frac{1}{2\pi^2}\int_0^\pi \int_0^\pi \int_0^{2\pi} \dfrac{\sin (n+1) \theta}{\sin \theta}^2\sin^2 \varphi \sin \psi d\psi d\varphi d\theta=1
     \end{align*}
    
    下证$(\pi_n,V_n)$是所有不可约表示。等价于说明$\{\chi_{\pi_n}:n \in \N\}$在所有$\mathrm{SU}(2)$的连续类函数中稠密($L^2$-范数下).这等价于说明$\{t(\theta)\}\cong S^1$中的类函数。由于$t(\theta)\sim t(-\theta)$,则$S^1$上的所有偶函数。

    根据Fourier分析,$\{\cos(n\theta)\}$是$S^1$上的偶函数空间上的稠密子集,且$\chi_{\pi_n}(t(\theta))-\chi_{\pi_{n-2}}(t(\theta))=2\cos(n\theta)$。则$\{\chi_{\pi_n}\}$是$\mathrm{SU}(2)$类函数空间上的稠密子集。于是是所有的不可约表示。
    \end{proof}
\end{example}
\subsection*{Peter-Weyl定理}
\begin{theorem}
    分析:矩阵系数空间是$(C(G),\|\cdot\|_\infty)$的稠密子集。

    代数:紧李群是矩阵李群。
\end{theorem}
\begin{corollary}
    特征标空间是类函数空间的稠密子集。
\end{corollary}
\begin{proof}
    代数到分析(Stone-Weiseestrass定理)

    设$X$是紧拓扑空间,$C(X)$是连续复值函数构成的代数。若$A \subset C(X)$是子代数。满足1.$A$可分类,即$\forall x_1 \neq x_2$,$\exists f:f(x_1)\neq f(x_2)$。2.$A$中有常函数。3.若$f \in A$,则$\overline{f}\in A$。则$A$是$C(X)$的稠密子集。

    我们验证矩阵系数构成子代数。略。

    由于存在单同态$i:G \to \mathrm{GL}(n,\C)$的闭子群。对于$\forall g\in G$,$g \mapsto g_{ij}$,$g \mapsto overline{g_{ij}}$,$g \mapsto 1$都是矩阵系数。因此1,2,3成立,这意蕴着矩阵系数空间是稠密的。

    分析到代数:一个引理:
    \begin{lemma}
        设$G$是紧李群,对于$\forall g \neq e$,存在不可约表示$(\pi,v)$使得$\pi(g)$不是$\mathrm{id}$。
    \end{lemma}
    事实上,取$f \in C(G)$,使得$f(e)=0,f(g)=1$。则存在$\pi$以及矩阵系数$\phi$使得$\|f-\phi\|_{\infty}<\epsilon$.于是$\phi(e)\neq \phi(g)$推的$\pi(g)\neq \pi(e)=\mathrm{id}$。

    下面构造$G$的忠实表示。

    对于$\forall g \in G^0$(单位连通分支),$g_1 \neq e$。于是存在表示$\pi(g_1)\neq \mathrm{id}$从而$G^0$不是$\ker \pi_1$的子集。于是$\ker \pi_1$的维数小于$G$的维数。

    若$\ker$的维度不是$0$,则取$g_2 \neq e$使其在$(\ker \pi_1)^0$中。则表示的直和$\pi_1\oplus \pi_2$的$\ker$维度进一步降低。

    最终把$\ker$降为$0$维。设$\ker=\{c_1,\dots,c_n\}$是有限集合。取$\{\varphi_i\}$:$\varphi_i(c_i)\neq \mathrm{id}$。于是进一步减少$ker$的个数。
    \end{proof}
    \section{紧李群的分类}
    先考虑紧李代数。紧李代数的分类为:
    \begin{align*}
        \g=Z(\g)\oplus S_1 \oplus S_2 \oplus \dots S_m
    \end{align*}
    $S_i$是单李代数。

    但是以$\g$为李代数的李群$G$不一定是紧李群。问题出在交换的部分。

    然而$[G,G]:=\{ghg^{-1}h:g,h \in G\}$是以$\g$的交换子$[\g:\g]=S_1\oplus S_m$为李代数的连通紧半单李群。

    从而由Dynkin分类得到紧李代数的分类,然后得到紧李代数对应的连通李群$G$的分类:$G=\tilde{G}/\Gamma$。$\Gamma$是$Z(\tilde{G})$离散的正规子群。

    从而在得到连通紧半单李群的分类$[G:G]$。而连通紧李群的分类为$[G \times G] \times T^n$
    \begin{align*}
        S_1 \otimes S_2 \dots \otimes S_m/\Gamma \times T^n
    \end{align*}
    其中$\Gamma$是$Z(S_1\times \dots \times S_m)$的有限子群。

    从而Peter-Weyl定理对连通成立。如果不连通,则对每个连通分支:$G/G_0$是有限群。从而$\forall g \notin G_0$,存在表示$\rho$使得$\rho(g)$不是$\mathrm{id}$。

\ifx\allfiles\undefined
	
	% 如果有这一部分的参考文献的话,在这里加上
	% 没有的话不需要
	% 因此各个部分的参考文献可以分开放置
	% 也可以统一放在主文件末尾。
	
	%  bibfile.bib是放置参考文献的文件,可以用zotero导出。
	% \bibliography{bibfile}
	
	\end{document}
	\else
	\fi
\ifx\allfiles\undefined

	% 如果有这一部分另外的package,在这里加上
	% 没有的话不需要
	
	\begin{document}
\else
\fi
\part{黎曼曲面与复几何}
\chapter{流形上的非退化光滑函数}
\section{Morse函数}
我们先用一个引理说明在非临界点$M$的平凡性质。
\begin{lemma}[非临界点]
	设$M^a=\{p\in M|f(p)\leq a\}$。若$a$不是临界值,则$M^a$是带边的光滑流形。
\end{lemma}

引理的证明留作练习。主要使用到隐函数定理以及带边流形的定义。

\begin{definition}[非退化点]
	考虑$M$上的函数$f$.若在$f$的临界点$p$处存在一个局部坐标$(U;x^i)$使得矩阵:
	\begin{align}\label{Hess}
		(\frac{\partial^2 f}{\partial x^i\partial x^j}(p))
	\end{align}
	非奇异,则称$p$是一个非退化点。
\end{definition}
这里需要注意的是,$p$的非退化性显然与局部坐标$x^i$无关。因此$p$的非退化性是$f$内蕴的性质。

如果$p$是$f$的临界点,我们就可以定义在$T_pM$上的双线性函数$f_{**}$。若$v,w \in T_pM$,用$\tilde{v}$和$\tilde{w}$表示在$p$处值为$v,w$的向量场。定义:
\begin{align}
	f_{**}(v,w)=\tilde{v}_p(\tilde{w}f)
\end{align}

我们断言:
\begin{lemma}
	$f_**$是对称的良定双线性函数。
\end{lemma}
\begin{proof}
	考虑:
	\begin{align}
		\tilde{v}_p(\tilde{w}f)-\tilde{w}_p(\tilde{v}f)=[\tilde{v},\tilde{w}]_p(f)=0
	\end{align}
	最后一个等号成立,是因为$p$是$f$的临界点。

	从而$f_{**}$是对称的。因此,$\tilde{v}_p(\tilde{w}f)$与$\tilde{v}$的选取无关,$\tilde{w}_p(\tilde{v}f)$与$\tilde{w}$的选取无关.

	于是$f_{**}$与$\tilde{v},\tilde{w}$的选取都无关,因而是良定的双线性函数。
\end{proof}
\begin{definition}[Hessian,指数,零化度]
    \quad \quad 称$f_{**}$为函数$f$在$p$处的Hessian双线性函数。
	
	而$f_{**}$的指数定义为满足$f_**$限制在上为负定双线性函数的子空间$V$的最大维数。$f_{**}$的零化度定义为$f_{**}$的零空间$W$的维数,即子空间$W=\{v \in V|f_{**}(v,w)=0,\forall w\in V\}$的维数。
\end{definition}

可以验算$f_{**}$在坐标$(U;x^i)$下给出的就是矩阵\ref{Hess}.显然,$f$在$p$处非退化等价于$f_{**}$的零化度是$0$。$f_{**}$的指数也称为$f$在$p$处的指数。

\begin{lemma}[Morse引理]\label{Morse-lemma}
	设$p$是$f$的非退化点,则存在一个$p$处的局部坐标$(U;y^i)$满足$y^i(p)=0,\forall i$且:
	\begin{align}
		f(q)=f(p)-(y^1)^2-\dots-(y^\lambda)^2+(y^{\lambda+1})^2+\dots+(y^n)^2
	\end{align}
	在整个$U$上都成立.其中$\lambda$是$f$在$p$处的指数。
\end{lemma}
\begin{proof}
	我们首先说明如果$f$拥有这样的表达式,则指数为$\lambda$.

	对于坐标$(z^i)$,若:
	\begin{align*}
		f(q)=f(p)-(z^1(q))^2-\dots-(z^\lambda(q))^2+\dots+(z^n(q))^2
	\end{align*}
	则容易求出$f_{**}$在该坐标下的矩阵为$\mathrm{diag}(-2,\dots,-2,2,\dots,2)$.其中$-2$一共有$\lambda$个。

	因此存在一个$\lambda$维的子空间使得$f_{**}$是负定的,存在一个$n-\lambda$维的子空间$V$使得$f_{**}$是正定的。如果$p$处的指数大于$\lambda$,则对应的子空间与$V$相交不为空。但这是不可能的,因此$\lambda$是$f$在$p$处的指数。

	接下来我们说明$(y^i)$坐标存在。不妨设$p$是$\R^n$的原点且$f(p)=f(0)=0$.从而有:
	\begin{align}
		f(x_1,\dots,x_n)=\int_0^1\frac{\dd f(tx_1,\dots,x_n)}{\dd   t}\dd t=\int_0^1 \sum_{i=1}^n \pa{f}{x_i}(tx_1,\dots,tx_n)x_i\dd t
	\end{align}
	令$g_j=\int_0^1\pa{f}{x_i}(tx_1,\dots,tx_n)\dd t$,则$f=\sum_j  x_jg_j$在$0$处的一个邻域上成立。

	因为$0$是$f$的临界点,从而$\pa{f}{x)i}(0)=0$。这意味着$g_j(0)=\pa{f}{x_i}(0)$.因而对$g_j$作上述$f$同样的分解:
	\begin{align*}
		f(x_1,\dots,x_n)=\sum_{i,j} x_ix_jh_{ij}(x_1,\dots,x_n)
	\end{align*}
	
	不妨假设$h_{ij}$关于$i,j$对称。通过计算,不难验证矩阵$(h_{ij}(0))$等于:
	\begin{align*}
		(\frac{1}{2}\frac{\partial^2f}{\partial x^i\partial x^j}(0))
	\end{align*}
	因此$h_{ij}(0)$是非奇异的矩阵。仿照模仿有理标准型的构造,可以证明存在一组坐标$(y^i)$使得$f$呈现为引理中的形式。具体的构造办法详见Milnor原书。(附录)
\end{proof}

Morse引理的好处在于我们可以用指数唯一确定$f$在$p$处的一个标准形式。根据这个引理,可以得知$f$在非退化点$p$的一个邻域内只有$p$一个临界点。
\begin{corollary}
	非退化临界点是离散的。特别的,紧致流形$M$上的非退化临界点只有有限个。
\end{corollary}

\begin{definition}[Morse函数]
	若$f\in \mathcal{O}(M)$且只有非退化的临界点,则称该函数为Morse函数。
\end{definition}
Morse函数的好处是显而易见的。然而存在性则是一个问题。本节我们剩下的内容为下面的定理。
\begin{theorem}
	任何流形$M$上都存在一个可微的函数$f$,满足不存在退化临界点,且$M^a$对于任何$a \in \R$都是紧致的。
\end{theorem}

根据Whitney嵌入定理,任何流形都可以嵌入到维数足够高的欧氏空间。因而我们考虑$M$是$\R^n$的$k$维嵌入子流形(之后统称为“子流形”)。

定义$N \subset M\times \R^n$为:
\begin{align*}
	N=\{(q,v):q \in M,v\in T_q\R^n,v\perp M\}
\end{align*}
即$N$是$M$在$\R^n$中的法丛。不难验证$N$是$n$维的流形,且光滑的嵌入进$\R^{2n}$中。定义$E:N \to \R^n$为映射$(q,v)\mapsto q+v$.

\begin{definition}[焦点]
	称$e \in \R^n$是$(M,q)$重数为$\mu$的焦点,若$e=E(q,v),(q,v)\in N$且$E$在$(q,v)$处的Jacobian矩阵有零化度$\mu$.
\end{definition}

根据Sard定理,两个微分流形之间的可微映射的临界点是很有限的——临界值只有$0$测度。显然焦点是$E$的临界值,因而:
\begin{corollary}\label{0focus}
	对于几乎所有的$x \in \R^n$,$x$都不是$M$的焦点。
\end{corollary}

现在固定$p \in \R^n$.定义函数$f:M \to \R$为:
\begin{align}
	L_p=f:q\mapsto\|q-p\|^2
\end{align}
在坐标$(U;u^1,\dots,u^k)$下,$f$的表达式为:
\begin{align*}
	f(u^1,\dots,u^k)=\|\vec{x}(u^1,\dots,x^k)-\vec{p}\|^2=\vec{x}\vec{x}-2\vec{x}\vec{p}+\vec{p}\vec{p}
\end{align*}

因此可以计算:
\begin{align*}
	\pa{f}{u^i}=2\pa{\vec{x}}{u^i}\cdot(\vec{x}-\vec{p})
\end{align*}

因此$q$是$f$的临界点,当且仅当$q-p$垂直与$M$垂直。

考虑$f$的二阶导数。我们有:
\begin{align*}
	\frac{\partial^2 f}{\partial u^i\partial u^j}=2\pa{\vec{x}}{u^i}\pa{\vec{x}}{u^j}+\pa{\vec{x}}{u^i}u^j\cdot(\vec{x}-\vec{p})
\end{align*}

从而有
\begin{lemma}
	$q$是$f=L_p$的退化临界点等价于$p$是$(M,q)$的焦点。根据推论\ref{0focus},总存在这样的$L_p$使得该函数不存在退化的临界点。另外,若$q$是$L_p$的退化临界点,则该点的零化度是$p$的重数。
\end{lemma}

\section{临界值处的伦形}
\section{Morse不等式}


\chapter{Morse理论的应用——测地线变分}
\section{道路的能量积分}
\section{指标定理}
\section{道路空间的伦型}

\ifx\allfiles\undefined

	% 如果有这一部分另外的package,在这里加上
	% 没有的话不需要
	\newcommand{\id}{\mathrm{id}}
\newcommand{\Hom}{\mathrm{Hom}}
\newcommand{\N}{\mathbb{N}}
\newcommand{\Z}{\mathbb{Z}}
\newcommand{\Q}{\mathbb{Q}}
\newcommand{\R}{\mathbb{R}}
\newcommand{\C}{\mathbb{C}}
\newcommand{\HH}{\mathbb{H}}
\newcommand{\RP}{\mathbb{RP}}
	\begin{document}
\else
\fi
\chapter{Tor函子和Ext函子}
本章的目的是介绍Tor函子和Ext函子的诸多性质。他们是同调代数初等应用中的常客。
\section{Abel群的Tor函子}
我们首先观察一个经典的PID上的模——Abel群的Tor函子。其实,Tor函子的名字就来源于其对Abel群的研究。

\begin{example}{}
    对于Abel群$B$而言,$\mathrm{Tor}_0^{\Z}(\Z/p,B)=B/pB$,$\mathrm{Tor}_1^\Z(\Z/p,B)={}_pB=\{b \in B:pB=0\}$.对于$n\geq 2$,$\mathrm{Tor}_2^\Z(\Z/p,B)=0$.

    上述结果可以这么看。取$\Z/p$的投射解消
    \begin{align}
        0 \to \Z \stackrel{p}{\rightarrow}\Z \to \Z/p \to 0
    \end{align}
    从而我们计算的是:
    \begin{align}
        0 \to B \stackrel{p}{\rightarrow} B \to 0
    \end{align}
    的同调群。
\end{example}
 特殊情况下,Tor函子表现出$1$阶挠子群,高阶为$0$的特点。实际上,我们有下面的命题:
 \begin{proposition}{}
    对于两个Abel群$A$,$B$,我们有:
    
    (a)$\mathrm{Tor}_1^\Z(A,B)$是一个挠群。

    (b)$\mathrm{Tor}_n^\Z(A,B)$在$n \geq 2$的情况下为$0$.
 \end{proposition}
 \begin{proof}
    证明依赖Tor函子与滤过余极限交换性。$A$是其有限生成子群的滤过余极限,所以$\mathrm{Tor}_n(A,B)$是$\mathrm{Tor}_n(A_\alpha,B)$的滤过余极限。

    Abel群的余极限总是他们直和的商子群。所以我们只需要证明对于有限生成子群上述命题成立即可。

    设$A=\Z^m \oplus \Z/p_1 \oplus \Z/p_2 \dots \Z/p_r$。因为$\Z^m$是投射的,所以只用考虑:
    \begin{align}
        \mathrm{Tor}_n(A,B)=\mathrm{Tor}_n(\Z/p_1,B)\oplus \mathrm{Tor}_n(\Z/p_2,B) \oplus \dots \mathrm{Tor}_n(\Z_r,B)
    \end{align}
    于是根据之前的例子我们知道结论成立。
 \end{proof}
 \begin{proposition}{}
    $\mathrm{Tor}_1^\Z(\Q/\Z,B)$是$B$的挠子群。
 \end{proposition}
 \begin{proof}
    可以想见,$\Z/p$提取出$B$中挠性为$p$的元素。$\Q/\Z$是其有限子群的滤过极限,并且每个有限子群都同构于某个$\Z/p$($p$不一定是素数。)
    \begin{align}
        \mathrm{Tor}_*^\Z(\Q/\Z,B)\cong \Colim \mathrm{Tor}_1^\Z(\Z/p,B)\cong \Colim({}_pB)=\cup_p\{b\in B:pb=0\}
    \end{align}

 \end{proof}
 \begin{proposition}{}
    如果$A$是一个无挠交换群,则$\mathrm{Tor}_n(A,B)$对于$n \neq 0$和Abel群$B$总是$0$。
 \end{proposition}
 \begin{proof}
    $A$是有限生成子群的滤过余极限。然而$A$无挠意味着这些有限生成子群都是自由群。用Tor保滤过余极限即可。
 \end{proof}
 如果$R$是交换环,则张量积有典范的同构,因此$\mathrm{Tor}_*(A,B)\cong \mathrm{Tor}_*(B,A)$.

 \begin{corollary}{}
    $\mathrm{Tor}_1^\Z(A,-)=0$等价于$A$无挠等价于$\mathrm{Tor}_1^\Z(-,A)=0$.
 \end{corollary}
 但是Tor函子并非对于所有环都有这么好的性质。比如下面的例子就说明在$R=\Z/m$的情况下可能失败:
 \begin{example}{}
   设$R=\Z/m$,$A=\Z/d$。其中$d|m$。从而$A$是$R$模。

   我们考虑$A$周期性的自由解消:
   \begin{align}
      \dots \to \Z/m \to Z/m \to \Z/m \to \Z/d
   \end{align}
   其中从$\Z/m$到$\Z/d$的映射是商映射,而$\Z/m$各自之间交替出现$d$和$m/d$。所以对于任何一个$\Z/m$模$B$,我们都有:
   \begin{align}
      \mathrm{Tor}_n^{\Z/m}(\Z/d,B)=\begin{cases}
      B/dB,n=0\\ \{b\in B:db=0\}/(m/d)B,n \text{是奇数}\\ \{b \in B:(m/d)b=0\}/dB,n \text{是偶数且}>0
      \end{cases}
   \end{align}
 \end{example}
 然而我们可以尝试对下面特殊的情况进行一些讨论。
 \begin{example}{}
   设$r$是$R$的一个左非零除子。即${}_rR=\{s \in R|rs=0\}$是$0$。对于每个$R$模$B$,记${}_rB=\{b \in B:rb=0\}$。用$R/rR$代替上述$\Z/p\Z$,用相同的计算办法可以算的:
   \begin{align}
      \mathrm{Tor}_0(R/rR,B)=B/rB;\quad \mathrm{Tor}_1^R(R/rR,B)={}_r B; \quad \mathrm{Tor}_n^R(R/rR,B)=0, n\geq 0
   \end{align}
 \end{example}
 \begin{proposition}{}
   若${}_r R\neq 0$,我们只能得到一个并非投射的解消:
   \begin{align}
      0 \to {}_r R \to R \stackrel{r}{\rightarrow} R \to R/rR \to 0
   \end{align}
   然而第二章我们介绍了dimension shelfting办法\ref{dim-Shifting}。所以我们对于$n \geq 3$,存在:
   \begin{align}
      \mathrm{Tor}_n^R(R/rR,B) \cong \mathrm{Tor}_{n-2}^R({}_r R,B)
   \end{align}

   其次,还有正合列:
   \begin{align}
      0 \to \mathrm{Tor}_2^R(R/rR,B) \to {}_rR \otimes B \to {}_rB \to \mathrm{Tor}_1^R(R/rR,B) \to 0
   \end{align}
   因为$\mathrm{Tor}_2^R(R/rR,B)$是$0 \to {}_rR\otimes B \to R\otimes B=B$的核。而该映射的像就在${}_r B$中,所以上述正合列中第一个和第二个已经确实成立。

   考虑$\mathrm{Tor}_1(R/rR,B)$。根据导引长正合列:
   \begin{align}
      0 \to \mathrm{Tor}_1(R/rR,B) \to rR\otimes B \to B \to B/rB
   \end{align}
   为了定义${}_r B \to \mathrm{Tor}_1(R/rR,B)$.我们定义${}_r B \to rR\otimes B$.即$b \mapsto r \otimes b$。则该映射实际上打进$\mathrm{Tor}_1(R/rR,B)$.

   若$\sum (rr_i)\otimes b_i \in \mathrm{Tor}_1(R/rR,B)$且在$B$中像为$\sum r(1\otimes r_ib_i)=0$,则${}_r B$中$\sum r_ib_i$的像是$\sum (rr_i)\otimes b_i$。于是我们定义了满射。

   最后需要说明${}_r B$处的正合。若$r \otimes b=0$,则存在$r_i$和$b_i$使得$rr_i=0$,$b=\sum r_ib_i$.
 \end{proposition}
 \begin{proposition}{}
   设$R$是交换整环,分式域$F$。则$\mathrm{Tor}_1^R(F/R,B)$是$B$的挠子群:$\{b \in B:(\exists r\neq 0)rb=0\}$
 \end{proposition}
 \begin{proposition}{}
   $\mathrm{Tor}_1^R(R/I,R/J) \cong \dfrac{I\cap J}{IJ}$对于任何右理想$I$和左理想$J$都成立。特别的,对于双边理想$I$:
   \begin{align}
      \mathrm{Tor}_1(R/I,R/I)\cong I/I^2
   \end{align}
 \end{proposition}
 \begin{proof}
   % https://q.uiver.app/#q=WzAsMTYsWzAsMSwiMCJdLFsxLDEsIklKIl0sWzIsMSwiSSJdLFszLDEsIklcXG90aW1lcyBSL0oiXSxbNCwxLCIwIl0sWzEsMiwiSiJdLFsyLDIsIlIiXSxbMywyLCJSXFxvdGltZXMgUi9KIl0sWzAsMiwiMCJdLFs0LDIsIjAiXSxbMSwwLCIwIl0sWzEsMywiSi8oSUopIl0sWzIsMCwiMCJdLFsyLDMsIlIvSSJdLFszLDAsIlxca2VyIGkiXSxbMywzLCJSL0kgXFxvdGltZXMgUi9KIl0sWzAsMV0sWzEsMl0sWzIsM10sWzMsNF0sWzEsNV0sWzIsNl0sWzMsNywiaVxcb3RpbWVzXFxtYXRocm17aWR9Il0sWzgsNV0sWzUsNl0sWzYsN10sWzcsOV0sWzEwLDFdLFs1LDExXSxbMTIsMl0sWzYsMTNdLFsxNCwzXSxbNywxNV0sWzEwLDEyXSxbMTIsMTRdLFsxNCwxMSwiIiwxLHsic3R5bGUiOnsiYm9keSI6eyJuYW1lIjoiZGFzaGVkIn19fV0sWzExLDEzXSxbMTMsMTVdXQ==
\[\begin{tikzcd}
	& 0 & 0 & {\ker i} \\
	0 & IJ & I & {I\otimes R/J} & 0 \\
	0 & J & R & {R\otimes R/J} & 0 \\
	& {J/(IJ)} & {R/I} & {R/I \otimes R/J}
	\arrow[from=2-1, to=2-2]
	\arrow[from=2-2, to=2-3]
	\arrow[from=2-3, to=2-4]
	\arrow[from=2-4, to=2-5]
	\arrow[from=2-2, to=3-2]
	\arrow[from=2-3, to=3-3]
	\arrow["{i\otimes\mathrm{id}}", from=2-4, to=3-4]
	\arrow[from=3-1, to=3-2]
	\arrow[from=3-2, to=3-3]
	\arrow[from=3-3, to=3-4]
	\arrow[from=3-4, to=3-5]
	\arrow[from=1-2, to=2-2]
	\arrow[from=3-2, to=4-2]
	\arrow[from=1-3, to=2-3]
	\arrow[from=3-3, to=4-3]
	\arrow[from=1-4, to=2-4]
	\arrow[from=3-4, to=4-4]
	\arrow[from=1-2, to=1-3]
	\arrow[from=1-3, to=1-4]
	\arrow[dashed, from=1-4, to=4-2]
	\arrow[from=4-2, to=4-3]
	\arrow[from=4-3, to=4-4]
\end{tikzcd}\]
上图是蛇形引理\ref{snake}.验证$I/(IJ)$和$I \otimes R/J$有典范同构可以得出第一行正合。第二行则典范正合。

 最右边的列是计算$\mathrm{Tor}_1(R/I,R/J)$的定义式。感觉Dimesion Shifting,$\ker i$是$\mathrm{Tor}_1(R/I,R/J)$。根据snake引理,$\ker i$是$J/(IJ)  \to R/I$的核:$\dfrac{I\cap J}{IJ}$。
 \end{proof}
\section{Tor函子与平坦性}
我们在这一节着重研究Tor函子的ayclic对象——平坦对象。
\begin{definition}[平坦模]{flat-module}
   称一个左$R$模是平坦模,若函子$\otimes_R B$是正合函子。同样,对于右$R$模,也可以定义类似的平坦性。
\end{definition}
如果$A$是投射的,则$\mathrm{Tor}_n(A,B)=0$。不难说明$A$此时是平坦的。因为投射模一定是平坦模。然而平坦模不一定是投射模。例如$\Q$作为交换群而言是平坦的,但不是投射的。(为什么?)
\begin{theorem}{}
   若$S$是$R$中的乘法封闭集,则$S^{-1}R$是一个平坦模。
\end{theorem}
这个定理当然很交换代数,不过影响不大,我们可以尝试证明:
\begin{proof}
   构造一个滤过范畴$I$。对象是$S$中的元素,态射$\Hom_I(s_1,s_2)=\{s \in S:s_1s=s_2\}$。定义函子$F:I \to R$。$F(s)=R$,$F(s_1 \to s_2)$则定义为$R$上该态射自然给出的右乘法。

  
我们断言$F$的余极限$\Colim F(s) \cong S^{-1}R$。从而因为$S^{-1}R$是平坦模的滤过余极限,所以其是平坦的。

   下面计算$\Colim F$。首先定义$F(s) \to S^{-1}R$的映射为$r \mapsto r/s$.这样交换图显然成立:
   % https://q.uiver.app/#q=WzAsMyxbMCwwLCJGKHNfMSk9UjpyIl0sWzEsMCwiRihzXzIpPVI6cnMiXSxbMCwxLCJTXnstMX1SOnIvc18xPXJzLyhzXzFzKT1ycy9zXzIiXSxbMCwxLCJzIl0sWzAsMl0sWzEsMl1d
\[\begin{tikzcd}
	{F(s_1)=R:r} & {F(s_2)=R:rs} \\
	{S^{-1}R:r/s_1=rs/(s_1s)=rs/s_2}
	\arrow["s", from=1-1, to=1-2]
	\arrow[from=1-1, to=2-1]
	\arrow[from=1-2, to=2-1]
\end{tikzcd}\]

如果存在一个新的$B$使得余极限中关系成立,我们直接定义$S^{-1}R$中的元素$r/s$到$B$的态射为$F(s)=R$中$r$在$B$中的像即可。这是唯一的定义方式!
\end{proof}
\begin{proposition}[Tor和平坦]
   下面三个命题等价:

   (1)$B$是平坦模。

   (2)$\mathrm{Tor}_n^R(A,B)=0,\forall n\neq 0$

   (3)$\mathrm{Tor}_1^R(A,B)=0$
\end{proposition}
\begin{corollary}
   若$0 \to A \to B \to C \to 0$是正合列且$B,C$是平坦模,则$A$平坦。
\end{corollary}
\begin{proposition}
   设$R$是主理想整环,则$B$平坦等价于$B$无挠。
\end{proposition}
对于上述命题,我们给出一个反例。首先平坦显然无挠。但是无挠不一定平坦。设$k$是域且$R=k[x,y]$。$R$是经典的非主理想整环。设$I=(x,y)R$。考虑$k=R/I$有投射解消:
\begin{align}
   0 \to R \to R^2 \to R \to k
\end{align}
其中第一个$R$到$R^2$为$[-y,x]$.而$R^2$到$R$为$(x,y)$.从而$\mathrm{Tor}_1^R(I,k)\cong \mathrm{Tor}_2^R(k,k)\cong k$。于是$I$不是平坦模。

我们深入的研究一下平坦模。
\begin{definition}[Pontrjagin对偶]{Pontrjagi}
   左模$B$的Pontrjagin对偶模$B^*$是一个右模:
   \begin{align}
      B^*:=\Hom_{\mathrm{Ab}}(B,\Q/\Z); (fr)(b)=f(rb)
   \end{align}
\end{definition}
\begin{proposition}{}
   下面的命题等价。

   (1)$B$平坦。

   (2)$B^*$内射。

   (3)$I\otimes_R B\cong IB=\{x_1b_1+\dots+x_nb_n\in B:x_i\in I,b_i\in B\}$对于任何右理想$I$都成立。

   (4)$\mathrm{Tor}_I^R(R/I,B)=0$对于任何右理想$I$都成立。
\end{proposition}
\begin{proof}
   (3)和(4)的等价性来源于正合列:
   \begin{align}
      0 \to \mathrm{Tor}_1(R/I,B) \to I\otimes B \to B \to B/IB \to 0
   \end{align}
   现在考虑$A'$是$A$的子模。考虑:
   \begin{align}
      \Hom(A,B^*) \to \Hom(A',B^*)
   \end{align}
   $B^*$等价于说上述映射是满射。根据伴随关系,我们有:
   \begin{align}
      \Hom(A\otimes B,\Q/\Z) \to \Hom(A'\otimes B,\Q/Z)
   \end{align}是满射。即$(A\otimes B)^* \to (A'\otimes B)^*$是满射。

   用下面的\textbf{引理},可以知道此时$A' \otimes B \to A\otimes B$是单射,所以$B$是平坦模。同理也可以反推回去。所以(1)(2)等价。另外带入$A'=I,A=R$,可以推出$I\otimes B \to R\otimes B$是单射。于是$I\otimes B\cong IB$且根据Baer判别法,这是可逆的。所以(1)(3)等价。
\end{proof}
我们描述一个引理。
\begin{lemma}{}
   $f:A' \to A$是单射等价于$f^*:A^* \to A'^*$是满射。
\end{lemma}
\begin{proof}
   因为$\Q/\Z$是内射的$\Z$模,所以保正合。
\end{proof}
\begin{proposition}[Pontrjagin对偶与正合]{}
   $A \to B \to C$是正合的当且仅当对偶$C^* \to B^* \to A^*$是正合的。
\end{proposition}
\begin{proof}
   因为$\Q/\Z$是内射模,所以$\Hom(-,\Q/\Z)$是正合函子,因此$C^* \to B^* \to A^*$是正合的。

   如果$C^* \to B^* \to A^*$正合,则$A \to B \to C$首先复形。若$b \in B$且在$C$中的像为$0$,我们证明$b$在$A$的像中。若不然,则$b+\mathrm{im}A$是$B/\mathrm{im}A$中的非$0$元。我们定义$g:B/\mathrm{im}A \to \Q/\Z$使得$g(b+\mathrm{im}A)\neq 0$。则$g$也给出了$B^*$中的非$0$元且在$A^*$中的像为$0$。

   所以可以给出一个$f \in C^*$。剩下的就是显然了。
\end{proof}
这个证明写的比较模糊。

我们邀请读者回忆有限展示的概念。然后不加证明的给出有限展示与生成元的选取无关.

\begin{proposition}{}
   若$\varphi:F \to M$是满射且$F$是有限生成的,$M$是有限展示的,则$\ker \varphi$是有限生成的。
\end{proposition}
HINT:用蛇形引理。

仍然用$A^*$表示$A$的Pontrjagin对偶,则存在一个自然的映射$\sigma:A^* \otimes_R M \to \Hom_R(M,A)^*$
\begin{align}
   \sigma(f\otimes m)=h \mapsto f(h(m))
\end{align}
其中$f\in A^*,m \in M,h \in \Hom(M,A)$.我们的问题是,什么时候$\sigma$是一个同构?
\begin{theorem}{}
   对于任何有限展示的$M$,$\sigma$都是一个同构。
\end{theorem}
\begin{proof}
   若$M=R$,则自然有$\sigma$是同构。根据可加性,$M=\R^n$的时候也是如此。所以有:
   % https://q.uiver.app/#q=WzAsOCxbMCwwLCJBXipcXG90aW1lcyBSXm0iXSxbMCwxLCJcXEhvbShSXm0sQSleKiJdLFsxLDAsIkFeKlxcb3RpbWVzIFJebiJdLFsyLDAsIkFeKlxcb3RpbWVzIE0iXSxbMywwLCIwIl0sWzMsMSwiMCJdLFsyLDEsIlxcSG9tKE0sQSleKiJdLFsxLDEsIlxcSG9tKFJebixBKV4qIl0sWzAsMV0sWzAsMl0sWzIsM10sWzMsNF0sWzYsNV0sWzcsNl0sWzEsN10sWzIsN10sWzMsNl1d
\[\begin{tikzcd}
	{A^*\otimes R^m} & {A^*\otimes R^n} & {A^*\otimes M} & 0 \\
	{\Hom(R^m,A)^*} & {\Hom(R^n,A)^*} & {\Hom(M,A)^*} & 0
	\arrow[from=1-1, to=2-1]
	\arrow[from=1-1, to=1-2]
	\arrow[from=1-2, to=1-3]
	\arrow[from=1-3, to=1-4]
	\arrow[from=2-3, to=2-4]
	\arrow[from=2-2, to=2-3]
	\arrow[from=2-1, to=2-2]
	\arrow[from=1-2, to=2-2]
	\arrow[from=1-3, to=2-3]
\end{tikzcd}\]
    因为$\otimes$是右正合的,$\Hom$是左正合的,所以图中两个行正合.根据5引理\ref{5lemma}可知$\sigma$是同构。
\end{proof}
\begin{theorem}{}
   每个有限展示的平坦模是投射模。
\end{theorem}
\begin{proof}
   我们证明$\Hom(M,-)$是正合的。设$B\to C$是满射,则$C^* \to B^*$是单射。若$M$是平坦的,则:
   % https://q.uiver.app/#q=WzAsNCxbMCwwLCJDXipcXG90aW1lc19SIE0iXSxbMSwwLCJCXipcXG90aW1lcyBNIl0sWzAsMSwiXFxIb20oTSxDKV4qIl0sWzEsMSwiXFxIb20oTSxCKV4qIl0sWzAsMV0sWzAsMiwiXFxzaWdtYSJdLFsxLDMsIlxcc2lnbWEiLDJdLFsyLDNdXQ==
\[\begin{tikzcd}
	{C^*\otimes_R M} & {B^*\otimes M} \\
	{\Hom(M,C)^*} & {\Hom(M,B)^*}
	\arrow[from=1-1, to=1-2]
	\arrow["\sigma", from=1-1, to=2-1]
	\arrow["\sigma"', from=1-2, to=2-2]
	\arrow[from=2-1, to=2-2]
\end{tikzcd}\]
   给出了$\Hom(M,B)\to Hom(M,C)$的满射。所以$M$是投射模。
  \end{proof}  
   下面的引理来源于dimension shifting.
   \begin{lemma}[平坦解消引理]{}
      群$\mathrm{Tor}_*(A,B)$可以用平坦模进行计算。
   \end{lemma}
\begin{proposition}[Tor的平坦基变换]
    设$R \to T$是环同态,使得$T$成为了$R$模。从而对于所有的$R$模$A$,所有的$T$模$C$和所有的$n$:
    \begin{align}
      \mathrm{Tor}_n^R(A,C)\cong \mathrm{Tor}_n^T(A \otimes_R T,C)
    \end{align}
\end{proposition}
\begin{proof}
   选择$R$模的投射解消$P \to A$,则$\mathrm{Tor}_*^R(A,C)$是$P \otimes_R C$的同调。

   因为$T$是平坦的$R$模,所以$P_n\otimes T$是投射的$T$模且$P\otimes T \to A \otimes T$是$T$模的投射解消。所以$\mathrm{Tor}_n^T(A \otimes T,C)$是复形$(P\otimes_R T)\otimes_T C \cong P\otimes_R C$的同调。
\end{proof}
\begin{corollary}{}
   若$R$是交换环,$T$是平坦的$R$代数,则对于所有的$R$模$A,B$和所有的$n$:
   \begin{align}
      T\otimes_R \mathrm{Tor}_n^R(A,B)\cong \mathrm{Tor}_n^T(A\otimes_R T,T\otimes_R B)
   \end{align}
\end{corollary}
\begin{proof}
   设$C=T\otimes_R B$.根据上面的命题,我们只需要证明$\mathrm{Tor}_*^R(A,T\otimes B)=T\otimes \mathrm{Tor}_*^R(A,B)$.因为$T\otimes_R$是正合函子,所以$T\otimes \mathrm{Tor}_*^R(A,B)$是$T\otimes_R (P\otimes _R B)$的同调,从而为$\mathrm{Tor}_*^R(A,T\otimes B)$.
\end{proof}
为了使得$\mathrm{Tor}$给出模结构,我们必须假设$R$是交换环。原因是下面的引理:

\begin{lemma}{}
   设$\mu:A \to A$是左乘一个中心元$r$。则诱导的$\mu_*:\mathrm{Tor}_n^R(A,B)\to \mathrm{Tor}_n^R(A,B)$也是左乘$r$.
\end{lemma}
\begin{proof}
   选择$A$的投射解消$P \to A$。左乘$r$是一个$R$模的链复形映射$\tilde{\mu}:P \to P$.(因为$r$是一个中心元)。从而$\tilde{mu}\otimes B$是$P\otimes B$的$r$左乘。作为商群$\mathrm{Tor}$也是如此。
\end{proof}
\begin{corollary}{}
   若$A$是一个$R/r$模,则对于每个$R$模$B$,$R$模$\mathrm{Tor}_*^R(A,B)$也是$R/r$模。换句话说,$rR$乘在该模得$0$.
\end{corollary}
\begin{corollary}[Tor的局部化]{}
   若$R$是一个交换环且$A,B$都是$R$模。下面的命题对于所有$n$都成立:
   \begin{enumerate}
      \item $\mathrm{Tor}_n^R(A,B)=0$
      \item 对于$R$的任意素理想$p$,$\mathrm{Tor}_n^{R_p}(A_p,B_p)=0$
      \item 对于$R$的任意极大理想$m$,$\mathrm{Tor}_n^{R_m}(A_m,B_m)=0$.
   \end{enumerate}
\end{corollary}
\begin{proof}
   对于$R$模而言,$M=0$等价于任意素理想$p$,$M_p=0$等价于任意极大理想$m$,$M_m=0=0$.设$M=\mathrm{Tor}(A,B)$:
   \begin{align}
      M_p=R_p \otimes_R M=\mathrm{Tor}_n^{R_p}(A_p,B_p)
   \end{align}
\end{proof}
\section{性质较好的环的Ext函子}
讨论了Ext后,我们讨论Ext函子的性质。首先我们计算一些性质很好的环的Ext函子。

\begin{lemma}{}
   $\mathrm{Ext}_\Z^n(A,B)=0$,$\forall n \geq 2$和所有的交换群$A,B$.
\end{lemma}
\begin{proof}
   把$B$嵌入到一个内射的交换群$I^0$.其商群$I^1$是可除的,因而是内射的,所以我们给出了$B$的内射解消$0 \to B \to I^0 \to I^1 \to 0$.

   所以$\mathrm{Ext}^*(A,B)$可以计算为:
   \begin{align}
      0 \to \Hom(A,I^0) \to \Hom(A,I^1) \to 0
   \end{align}
   的上同调。
\end{proof}
因此我们只需要考虑$n=1$的情况。
\begin{example}{}
   $\mathrm{Ext}_{\Z}^0(\Z/p,B)={}_p B$.$\mathrm{Ext}_\Z^1(\Z/p,B)=B/pB$.

   可以使用$0 \to \Z \to \Z \to \Z/p$作为$\Z/p$的投射解消计算。
\end{example}

因为$\Z$是投射模,所以$\Ext^1(\Z,B)=0$对于任何$B$总是成立。我们可以依据这个结果和上述结果,在$A$是有限生成的Abel群时计算$\Ext(A,B)$:
\begin{align}
   A\cong \Z^m \oplus \Z/p  \Rightarrow \Ext(A,B)=\Ext(\Z/p,B)
\end{align}
然而无限生成的情况因为余极限不交换,要复杂得多。
\begin{example}[$B=\Z$]{}
   设$A$是一个挠群,用$A^*$表示Pontrjagin对偶。$\Z$有经典的内射解消:$0 \to \Z \to \Q \to \Q/\Z \to 0$。用这个解消计算$\Ext^*(A,\Z)$:
   \begin{align}
      0 \to \Hom(A,\Q) \to \Hom(A,\Q/\Z) \to 0 
   \end{align}
   从而$\Ext_\Z^0(A,\Z)=\Hom(A,\Z)=0$,$\Ext_\Z^1(A,\Z)=A^*$。

   为了对这个例子有更深的印象,注意到$\Z_{p^\infty}$是$\Z/p^n$的余极限(并).于是可以计算:
   \begin{align}
      \Ext_\Z^1(\Z_{p^\infty},\Z)=(\Z_{p^\infty})^*
   \end{align}
   这个群是$p$-adic整数的无挠群,$\hat{\Z}_p=\Lim (\Z/p^n)$。

   再考虑一个例子:$A=\Z[1/p],B=\Z$.此时:
   \begin{align}
      0 \to \Q=\Hom(\Z[1/p],\Q) \to \Hom(\Z[1/p],\Q/\Z) \to 0
   \end{align}
   $\Ext^0$比较容易,我们考虑$\Ext^1$.此时给定$f \in \Hom(\Z[1/p],\Q/\Z)$,筛出掉$\Hom(\Z[1/p],\Q)$的元素,本质上留存的是一个$p$-adic数。并且若两个$p$-adic数只差一个整数,与他们给出的$f$是一致的。因此$\Ext^1(\Z[1/p],\Z)=\Z_{p^\infty}$。

   这说明$\Ext$对于平坦模而言也不是vanish的。
\end{example}
\begin{example}[$R=\Z/m$,$B=\Z/p$]{}
   $\Z/p$在这种情况下有无穷的周期内射解消:
   \begin{align}
      0 \to \Z/p \xrightarrow{\iota} \Z/m \xrightarrow{p}  \Z/m \xrightarrow{m/p} \Z/m \xrightarrow{p} \dots 
   \end{align}

   于是$\Ext_{\Z/m}^n(A,\Z/p)$可以计算为:
   \begin{align}
      0 \to \Hom(A,\Z/m) \to \Hom(A,\Z/m) \to \Hom(A,\Z/m) \dots
   \end{align}
   的上同调。

   比如,若$p^2|m$,则$\Ext_{\Z/m}^n(\Z/p,\Z/p)=\Z/p$
\end{example}
\begin{proposition}{}
   对于所有的$n$和$R$:
   \begin{enumerate}
      \item $\Ext_R^n(\bigoplus_\alpha A,B)\cong \prod_{\alpha}\Ext_R^n(A_\alpha,B)$
      \item $\Ext_R^n(A,\prod_\beta B) \cong \prod_\beta \Ext_R^n(A,B_\beta)$
   \end{enumerate}
\end{proposition}
\begin{proof}
   设$P_\alpha$是$A_\alpha$的投射解消。于是$\oplus P_\alpha$是$\oplus A_\alpha$的投射解消。同理,$Q_\beta$是$B_\beta$的内射解消,则$\prod Q_\beta$是$\prod B_\beta$的内射解消。

   根据$\Hom$的性质,再加上:
   \begin{align}
      H^*(\prod C_\gamma)\cong \prod H^*(C_\gamma)
   \end{align}
   可得结果。
\end{proof}
\begin{lemma}{}
   设$R$是交换环,则$\Hom_R(A,B)$和$\Ext^*(A,B)$都是$R$模。若$\mu,\tau$分别是$r$的左乘($A,B$),则诱导的$\mu^*$和$\tau^*$也是左乘。
\end{lemma}
可以看到,这是Tor函子的相似版本,可用于给出Ext与局部化交换的性质。
\begin{proof}
   给$P \to A$投射解消.左乘$r$给出了$\tilde{mu}:P \to P$作为链复形映射。映射$\Hom(\tilde{mu},B)$是$\Hom(P,B)$上链复形,是左乘$r$.

   因此商群$\Ext^n(A,B)$被$\mu^*$作用也是$r$左乘。
\end{proof}
\begin{corollary}{}
   设$R$是交换环,$A$是$R/r$模。则对于$R$模$B$,$\Ext^*_R(A,B)$是$R/r$模。
\end{corollary}
接下来的引理,定理我们不写证明,读者可自查Weibel原书。

考虑$S^{-1}\Hom_R(A,B)$.其到$\Hom_{S^{-1}R}(S^{-1}A,S^{-1}B)$有一个自然的态射$\Phi$。但这个态射一般不是同构。
\begin{lemma}{}
   如果$A$是有限展示的$R$模,则对于每个中心可乘集合$S$,$\Phi$是同构。
\end{lemma}
不难想象证明用到的是5引理\ref{5lemma}。

\begin{proposition}{}
   设$A$是交换Noether环上的有限生成模.则$\Phi$也诱导了Ext的同构:
   \begin{align}
      \Phi:S^{-1}\Ext_R^n(A,B) \cong \Ext_{S^{-1}R}^n(S^{-1}A,S^{-1}B)
   \end{align}
\end{proposition}
不难想到证明的思路是给$A$的投射解消。因为$S^{-1}$是正合函子,所以保$H^*$。因此用$\Hom$的同构性即可给出上述同构。

\begin{corollary}[Ext的局部化]{Ext-loc}
   设$R$是交换Noether环且$A$是有限生成$R$模.则下面的命题之间对于任意$B$和$n$都等价:
   \begin{enumerate}
      \item $\Ext_R^n(A,B)=0$
      \item 对于$R$的任何素理想$p$,$\Ext_{R_p}^n(A_p,B_p)=0$
      \item 对于$R$的任何极大理想$m$,$\Ext_{R_m}^n(A_m,B_m)=0$.
   \end{enumerate}
\end{corollary}
\section{Ext函子与扩张}
我们在这一节探讨Ext到底计算了什么。为此需要介绍扩张的概念。
\begin{definition}{extension}
   一个$A$过$B$的扩张$\xi$是指一个正合列$0 \to B \to X \to A \to 0$.称两个扩张$\xi,\xi'$是等价的,若存在交换图:
   % https://q.uiver.app/#q=WzAsMTAsWzAsMCwiMCJdLFsxLDAsIkEiXSxbMiwwLCJYIl0sWzMsMCwiQiJdLFs0LDAsIjAiXSxbMSwxLCJBIl0sWzIsMSwiWCciXSxbMywxLCJCIl0sWzQsMSwiMCJdLFswLDEsIjAiXSxbMCwxXSxbMSwyXSxbMiwzXSxbMyw0XSxbNSw2XSxbNiw3XSxbNyw4XSxbOSw1XSxbMSw1LCJcXGlkIiwxXSxbMyw3LCJcXGlkIiwxXSxbMiw2LCJcXGNvbmciXV0=
\[\begin{tikzcd}
	0 & A & X & B & 0 \\
	0 & A & {X'} & B & 0
	\arrow[from=1-1, to=1-2]
	\arrow[from=1-2, to=1-3]
	\arrow[from=1-3, to=1-4]
	\arrow[from=1-4, to=1-5]
	\arrow[from=2-2, to=2-3]
	\arrow[from=2-3, to=2-4]
	\arrow[from=2-4, to=2-5]
	\arrow[from=2-1, to=2-2]
	\arrow["\id"{description}, from=1-2, to=2-2]
	\arrow["\id"{description}, from=1-4, to=2-4]
	\arrow["\cong", from=1-3, to=2-3]
\end{tikzcd}\]

   一个扩张是分裂的,若其等价于$0 \to B \to A \oplus B \to 0$(典范的)。
\end{definition}
\begin{example}{}
   若$p$是素数,则仅存在$p$个等价的$\Z/p$过$\Z/p$的扩张。分别是分裂扩张和:
   \begin{align}
      0 \to \Z/p \xrightarrow{p} \Z/p^2 \xrightarrow{i}\Z/p \to 0, i=1,2,\dots,p-1
   \end{align}

   实际上$X$必须是$p^2$阶交换群。若$X$无$p^2$阶元,则根据$X=\Z/p\oplus \Z/p$。若$X$有$p^2$阶元,设该元为$b$。则$pb \in \Z/p=B$。于是有上述$p-1$种投射。
\end{example}
\begin{lemma}{}
   若$\Ext^1(A,B)=0$,则$A$过$B$的扩张总是分裂的。
\end{lemma}
\begin{proof}
   给定一个扩张$\xi$,根据$\Ext^*(A,-)$诱导的长正合列:
   \begin{align}
      \Hom(A,X) \to \Hom(A,A) \xrightarrow{\partial}\Ext^1(A,B)=0
   \end{align}
   所以$\id_A$有原像$\sigma:A \to X$。这就是一个$X \to A$的截面。所以$X=A \oplus B$分裂。
\end{proof}
如果$\Ext^1(A,B)$非$0$,为了给出截面,实际上可以计算$\partial(\id_A)=0$。我们把这个构造记作$\Theta(\xi)$.另外.如果两个扩张等价,那么他们的$\Theta(\xi)$相同.因此这个构造只依赖于$\xi$的等价类。

\begin{theorem}{}
   给定两个模$A,B$,映射$\Theta:\xi \mapsto \partial(\id_A)$给出了一个一一映射:
   \begin{align}
      \{\text{A过B的扩张的等价类}\} \to \Ext^1(A,B)
   \end{align}
\end{theorem}
因此这个定理给出了$\Ext^1(A,B)$的一个初步作用:确定$A$过$B$的扩张个数,并赋予一个群结构。
\begin{proof}
   对于$B$,固定一个正合列$0 \to B \to I \xrightarrow{\pi} N \to 0$.其中$I$内射。作用$\Hom(A,-)$,导出一个正合列:
   \begin{align}
      \Hom(A,I) \to \Hom(A,N) \xrightarrow{\partial} \Ext^1(A,B) \to 0
   \end{align}

   现在给定一个$x \in \Ext^1(A,B)$,选定$\beta \in \Hom(A,N)$使得$\partial(\beta)=x$.根据$\beta:A \to N$和$I \to N$,可以写出拉回$X$:
   % https://q.uiver.app/#q=WzAsMTAsWzAsMCwiMCJdLFsxLDAsIk0iXSxbMiwwLCJQIl0sWzMsMCwiQSJdLFs0LDAsIlxcYnVsbGV0Il0sWzAsMSwiMCJdLFsxLDEsIkIiXSxbMiwxLCJYIl0sWzMsMSwiQSJdLFs0LDEsIlxcYnVsbGV0Il0sWzAsMV0sWzIsM10sWzMsNF0sWzUsNl0sWzYsN10sWzcsOF0sWzgsOV0sWzEsNiwiXFxiZXRhIl0sWzIsN10sWzMsOCwiPSJdLFsxLDIsImoiXV0=
\[\begin{tikzcd}
	0 & B & X & A & 0 \\
	0 & B & I & N & 0
	\arrow[from=1-1, to=1-2]
	\arrow[from=1-3, to=1-4]
	\arrow[from=1-4, to=1-5]
	\arrow[from=2-1, to=2-2]
	\arrow[from=2-2, to=2-3]
	\arrow[from=2-3, to=2-4]
	\arrow[from=2-4, to=2-5]
	\arrow["{=}", from=1-2, to=2-2]
	\arrow[from=1-3, to=2-3]
	\arrow["\beta", from=1-4, to=2-4]
	\arrow[from=1-2, to=1-3]
\end{tikzcd}\]
这不仅是拉回,而且可以验证$0 \to B \to X \to A \to 0$是一个正合列。根据连接同态$\partial$的自然性,可以得到:
% https://q.uiver.app/#q=WzAsNCxbMCwwLCJcXEhvbShBLEEpIl0sWzEsMCwiXFxFeHReMShBLE0pIl0sWzAsMSwiXFxIb20oQSxBKSJdLFsxLDEsIlxcRXh0XjEoQSxCKSJdLFswLDFdLFswLDJdLFsyLDNdLFsxLDNdXQ==
\[\begin{tikzcd}
	{\Hom(A,A)} & {\Ext^1(A,B)} \\
	{\Hom(A,N)} & {\Ext^1(A,B)}
	\arrow[from=1-1, to=1-2]
	\arrow[from=1-1, to=2-1]
	\arrow[from=2-1, to=2-2]
	\arrow[from=1-2, to=2-2]
\end{tikzcd}\]
令上面的扩张是$\xi$,则$\Theta(\xi)=x$。于是我们通过给定$x\in \Ext^1(A,B)$给出一个扩张$\xi$使得$\Theta(\xi)=x$。

为了给出$\Ext^1(A,B)$到等价类的映射,我们还需要说明上述过程$\beta$的选取不改变$\xi$的等价类。实际上选取$\beta'\in \Hom(A,N)$使得$\partial{\beta'}=x$。于是$\beta'-\beta=\pi_*(\alpha),\alpha\in \Hom(A,I)$.于是可以绘制出下面的交换图:

% https://q.uiver.app/#q=WzAsNSxbMCwwLCJYIl0sWzEsMSwiWCciXSxbMiwxLCJBIl0sWzEsMiwiSSJdLFsyLDIsIk4iXSxbMCwxLCIiLDEseyJzdHlsZSI6eyJib2R5Ijp7Im5hbWUiOiJkYXNoZWQifX19XSxbMSwyLCJcXHNpZ21hJyIsMl0sWzAsMiwiXFxzaWdtYSIsMV0sWzEsMywicCciXSxbMyw0LCJcXHBpIiwyXSxbMiw0LCJcXGJldGEnIl0sWzAsMywicCtcXGFscGhhXFxjaXJjXFxzaWdtYSIsMl1d
\[\begin{tikzcd}
	X \\
	& {X'} & A \\
	& I & N
	\arrow[dashed, from=1-1, to=2-2]
	\arrow["{\sigma'}"', from=2-2, to=2-3]
	\arrow["\sigma"{description}, from=1-1, to=2-3]
	\arrow["{p'}", from=2-2, to=3-2]
	\arrow["\pi"', from=3-2, to=3-3]
	\arrow["{\beta'}", from=2-3, to=3-3]
	\arrow["{p+\alpha\circ\sigma}"', from=1-1, to=3-2]
\end{tikzcd}\](交换性已经在草稿纸上验证了)
根据拉回的泛性质,$X$到$X'$有一个态射.

通过具体到集合的验证,可以说明这是一个同构。所以$X$和$X'$是等价的扩张。

另一方面,给定$\xi$作为$A$过$B$的扩张,$I$的延拓性质表明存在一个$\tau:X \to I$满足:
% https://q.uiver.app/#q=WzAsMTAsWzAsMCwiMCJdLFsxLDAsIkIiXSxbMiwwLCJYIl0sWzMsMCwiQSJdLFs0LDAsIjAiXSxbMCwxLCIwIl0sWzEsMSwiQiJdLFsyLDEsIkkiXSxbMywxLCJOIl0sWzQsMSwiMCJdLFswLDFdLFsxLDJdLFsyLDNdLFszLDRdLFs1LDZdLFs2LDddLFs3LDhdLFs4LDldLFsyLDcsIlxcdGF1Il0sWzEsNiwiPSJdLFszLDgsIlxcYmV0YSIsMV1d
\[\begin{tikzcd}
	0 & B & X & A & 0 \\
	0 & B & I & N & 0
	\arrow[from=1-1, to=1-2]
	\arrow[from=1-2, to=1-3]
	\arrow[from=1-3, to=1-4]
	\arrow[from=1-4, to=1-5]
	\arrow[from=2-1, to=2-2]
	\arrow[from=2-2, to=2-3]
	\arrow[from=2-3, to=2-4]
	\arrow[from=2-4, to=2-5]
	\arrow["\tau", from=1-3, to=2-3]
	\arrow["{=}", from=1-2, to=2-2]
	\arrow["\beta"{description}, from=1-4, to=2-4]
\end{tikzcd}\]

其中$\beta$是$\tau$诱导的态射。我们断言$X$是$\beta$和$\pi:I \to N$的拉回。从而$\Psi(\Theta(\xi))=\xi$.
\end{proof}
如果我们可以给出扩张的运算,就能更好的理解上述的对应。
\begin{definition}[Baer和]{Baer-sum}
   设$\xi$和$\xi'$分别是$A$过$B$的两个扩张。设$X''$是$X \to A$和$X'  \to A$的拉回。则$X''$包含了三份$B$:$B \times 0,0 \times B,\{(-b,b):b\in B\}$。

   作$X''$对于对角线$B$的商运算,则$B \times 0$和$0 \times B$被对应为一个子群。而$X''/0\times B\cong X$和$X/B=A$,则我们得到正合列:
   \begin{align}
      \varphi: 0\to B \to Y \to A\to 0
   \end{align}
   $\varphi$的等价类被称为$\xi$和$\xi'$的Baer和。
\end{definition}
\begin{proposition}{Baer-sum-pro}
   扩张等价类的集合在Baer和的意义下生成了一个交换群,分裂扩张是该和的幺元。从而$\Theta$给出了一个群同构。
\end{proposition}
\begin{proof}
   我们说明$\Theta(\varphi)=\Theta(\xi)+\Theta(\xi')$.这说明了Baer和的良定性,也给出了命题成立。

   固定$0\to M \to P \to A\to 0$是一个正合列,且$P$是投射模。因为$P$投射,所以给出$\tau:P \to X$和$\tau':P\to X'$。
   
   接下来设$\tau'': P\to X''$是由$\tau:P \to X$和$\tau': P \to X'$诱导而来的态射。而设$\bar{\tau}:P \to Y$是诱导的态射。

   我们断言$\bar{\tau}$限制在$M$上由映射$\gamma+\gamma':M \to B$诱导。所以下面的交换图:
   % https://q.uiver.app/#q=WzAsMTAsWzAsMCwiMCJdLFsxLDAsIk0iXSxbMiwwLCJQIl0sWzMsMCwiQSJdLFs0LDAsIjAiXSxbMCwxLCIwIl0sWzQsMSwiMCJdLFsyLDEsIlkiXSxbMywxLCJBIl0sWzEsMSwiQiJdLFswLDFdLFszLDgsIj0iXSxbMyw0XSxbOCw2XSxbMiwzXSxbMSwyXSxbMiw3LCJcXGJhcntcXHRhdX0iXSxbNyw4XSxbOSw3XSxbMSw5LCJcXGdhbW1hK1xcZ2FtbWEnIl0sWzUsOV1d
\[\begin{tikzcd}
	0 & M & P & A & 0 \\
	0 & B & Y & A & 0
	\arrow[from=1-1, to=1-2]
	\arrow["{=}", from=1-4, to=2-4]
	\arrow[from=1-4, to=1-5]
	\arrow[from=2-4, to=2-5]
	\arrow[from=1-3, to=1-4]
	\arrow[from=1-2, to=1-3]
	\arrow["{\bar{\tau}}", from=1-3, to=2-3]
	\arrow[from=2-3, to=2-4]
	\arrow[from=2-2, to=2-3]
	\arrow["{\gamma+\gamma'}", from=1-2, to=2-2]
	\arrow[from=2-1, to=2-2]
\end{tikzcd}\]
成立。

因此我们有$\Theta(\varphi)=\partial(\gamma+\gamma')$.然而$\partial(\gamma+\gamma')=\partial(\gamma)+\partial(\gamma')=\Theta(\xi)+\Theta(\xi')$.所以命题成立。
\end{proof}

借助上述的命题,我们实际上可以思考这样的问题:如果一个Abelian范畴没有足够的投射模和内射模,我们也可以借助扩张生成的交换群来定义$\Ext^1$.当然这里的交换群仍需要证明。

相似的,我们也可以思考$\Ext^n$的含义。我们在这里建议大家阅读原书的79页到80页内容。
\section{逆向极限的导出函子}
设$I$是一个小范畴(即对象集和态射集都是集合)。$\mathcal{A}$是一个Abelian范畴。在第二章,我们说明了$\mathcal{A}^I$有足够多的内射对象。(至少是$A$完备且有足够多内射对象的时候)。另外,容易验证逆向极限是左正合函子(保核)。

因此我们可以定义从$\mathcal{A}^I$到$\mathcal{A}$的右导出函子$R^n\Lim_{i\in I}$。

我们在这一节关注$\mathcal{A}$是Ab且$I$是$\dots\to 2\to 1 \to 0$。我们把$\mathrm{Ab}^I$中的元素称作交换群的“塔”。他们的具体形式是:
\begin{align}
   \{A_i\}:\dots \to A_2\to A_1 \to A_0
\end{align}
这一节我们具体给出$\lim^1$的具体构造,并且证明$R^n\Lim=0,n\neq 0,1$。

我们自然想问这样的构造是否可以拓展为其他的Abelian范畴。Grothendieck告诉我们,在满足下面公理的情况下该范畴可以:

(AB$4^*$):$\mathcal{A}$是完备的,且任何集合的满射的乘积都是满射。

满足该公理的范畴大多是有underlying集合的范畴(交换群,模范畴,链复形范畴),但是在层范畴失效。

\begin{definition}{}
   给定Ab中的一个塔$\{A_i\}$。定义映射:
   \begin{align}
      \Delta:\prod_{i=0}^\infty \to \prod_{i=0}^\infty A_i
   \end{align}
   为:
   \begin{align}
      \Delta(\dots,a_i,\dots,a_0)=(\dots,a_i-\bar{a}_{i+1},\dots,a_1-\bar{a}_2,a_0-\bar{a}_1)
   \end{align}
   其中$\bar{a}_{i+1}$代表$a_{i+1}\in A_{i+1}$在$A_i$中的项。
   
   容易看出$\Delta$的$\ker$是$\Lim A_i$.我们定义$\Lim^1 A_i$是$\Delta$的余核,从而$\Lim^1$是从$\mathrm{Ab}^I$到$\mathrm{Ab}$的函子。我们定义$\Lim^0 A_i=\Lim A_i$,$\Lim^n A_i=0,n\geq 2$.
\end{definition}
上述定义给出了具体的构造。当然我们需要说明这是符合要求的函子。
\begin{lemma}{}
   函子$\{\Lim^n\}$给出了一个上同调$\delta$函子。
\end{lemma}
\begin{proof}
   设$0 \to \{A_i\} \to \{B_i\}\to \{C_i\} \to 0$是塔的一个短正合列。用蛇形引理:
   % https://q.uiver.app/#q=WzAsMTAsWzAsMCwiMCJdLFsxLDAsIlxccHJvZCBBX2kiXSxbMiwwLCJcXHByb2QgQl9pIl0sWzMsMCwiXFxwcm9kIENfaSJdLFs0LDAsIjAiXSxbMSwxLCJcXHByb2QgQV9pIl0sWzIsMSwiXFxwcm9kIEJfaSJdLFszLDEsIlxccHJvZCBDX2kiXSxbMCwxLCIwIl0sWzQsMSwiMCJdLFswLDFdLFsxLDJdLFsyLDNdLFszLDRdLFsxLDUsIlxcRGVsdGEiXSxbNSw2XSxbNiw3XSxbMyw3LCJcXERlbHRhIl0sWzIsNiwiXFxEZWx0YSJdLFs4LDVdLFs3LDldXQ==
\[\begin{tikzcd}
	0 & {\prod A_i} & {\prod B_i} & {\prod C_i} & 0 \\
	0 & {\prod A_i} & {\prod B_i} & {\prod C_i} & 0
	\arrow[from=1-1, to=1-2]
	\arrow[from=1-2, to=1-3]
	\arrow[from=1-3, to=1-4]
	\arrow[from=1-4, to=1-5]
	\arrow["\Delta", from=1-2, to=2-2]
	\arrow[from=2-2, to=2-3]
	\arrow[from=2-3, to=2-4]
	\arrow["\Delta", from=1-4, to=2-4]
	\arrow["\Delta", from=1-3, to=2-3]
	\arrow[from=2-1, to=2-2]
	\arrow[from=2-4, to=2-5]
\end{tikzcd}\]

就可以得到我们想要的自然长正合列。
\end{proof}
\begin{lemma}{}
   若所有的$A_{i+1}  \to A_i$都是满射,则$\Lim^1 A_i=0$.更多的,$\Lim A_i\neq 0$(除非每个$A_i$都是$0$),因为每个自然投射$\Lim A_i \to A_j$都是满射。
\end{lemma}
\begin{proof}
   给定$b_i \in A_i(i=0,\dots,n)$,以及任何$a_0\in A_0$。归纳的选择$a_{i+1}\in A_{i+1}$:使得$a_{i+1}$是$a_i-b_i  \in A_i$在$A_{i+1}$中的提升。

   从而$\Delta$将$(\dots,a_1,a_0)$映射到$(\dots,b_1,b_0)$.因此这种情况下$\Delta$是满射,$\Lim^1 A_i=0$。如果$b_i=0$,$(\dots,a_1,a_0)\in \Lim A_i$.
\end{proof}
\begin{corollary}{}
   $\Lim^1 A_i\cong (R^1\Lim)(A_i)$且$R^n \Lim=0,\forall n\neq 0,1$
\end{corollary}
\begin{proof}
   我们说明$\Lim^n$形成了一个泛$\delta$函子,从而根据泛性说明上述成立。我们只需要说明$\Lim^1$在足够多的内射对象(应付内射解消)上vanish。

   我们在第二章给出了足够多的内射对象:
   \begin{align}
      k_*E:\dots=E=E \to 0 \to 0 \dots\to 0
   \end{align}
   其中$E$内射。因此这里面所有的态射都是满射,因此$\Lim^1$在这些内射塔上都vanish。
\end{proof}
上述的证明在AB4*的情况下总是对的。我们给出反例(不满足AB4*)。
\begin{example}{}
   设$A_0=\Z$且$A_i=p^i\Z$是$p^i$生成的子群。对短正合列($p$是素数):
   \begin{align}
      0\to \{p^i\Z\}  \to \{\Z\} \to \{\Z/p^i\Z\}  \to 0
   \end{align}
   使用$\Lim$.

   从而$\Lim^1\{p^i\Z\}\cong\hat{\Z}_p/\Z$.

\end{example}
下面这个命题在原书上是习题。我们仅作记录,证明省略。(可以查找mathstackexchange)。

\begin{proposition}{}
   设$\{A_i\}$是一个塔,$A_{i+1}\to A_i$是包含映射。把$A=A_0$看作拓扑群,其中$a+A_i(a\in A,i\geq 0)$是开集。

   则$\Lim A_i=\cap A_i=0$当且仅当$A$是Hausdorff的.$\Lim^1 A_i=0$当且仅当$A$在下列意义是完备的:每个柯西列都有不一定唯一的极限点.
\end{proposition}
提示:证明$A$是完备的,当且仅当$A\cong \Lim(A/A_i)$
\begin{definition}{}
   我们称一个塔$\{A_i\}$满足Mittag-Leffler条件,若对于每个$k$都存在一个$j\geq k$使得$A_i \to A_k$的像等于$A_j\to A_k$,对于任意$i \geq j$成立。(即$A_i$在$A_k$的像满足降链条件)。

   例如,若$\{A_i\}$都是满射,该塔就满足M-L条件。

   有一种平凡的情况:若对于每个$k$都存在一个$j\geq k$使得$A_i \to A_k$的像是$0$,我们称该塔满足平凡M-L条件。
\end{definition}
\begin{proposition}{}
   若$A_i$满足M-L条件,则:$\Lim^1 A_i=0$
\end{proposition}
\begin{corollary}{}
   设$\{A_i\}$是有限Abel群的塔,或者是有限维向量空间上的塔,我们都有$\Lim^1 A_i=0$
\end{corollary}
下面的定理预示了下一节的泛系数定理。
\begin{theorem}{}
   设$\dots \to C_1 \to C_0$是Ab的链复形的塔链。(每个$C_i$都是链复形),且满足ML条件。设$C=\Colim C_i$。则对于每个$q$都存在一个正合列:
   \begin{align}
      0 \to \textstyle\Lim^1 H_{q+1}(C_i) \to H_q(C) \to \Lim H_q(C_i) \to 0
   \end{align}


若$\dots C_1\to C_0 \to 0$是上链复形的塔链且满足ML条件。则:
\begin{align}
   0 \to \textstyle\Lim^1 H^{q-1}(C_i) \to H^q(C) \to \Lim H^q(C_i) \to 0
\end{align}
正合。
\end{theorem}
在拓扑上,这个定理有一个类似的版本。考虑$X$是CW复形,而$X_i$是$X$的上升子复形链,使得$X=\cup X_i$.则存在一个正合列:
\begin{align}
   0 \to \textstyle\Lim^1 H^{q-1}(X_i) \to H^q(X) \to \Lim H^q(X_i) \to 0
\end{align}
可以一眼看出这个公式的便利之处:可以根据子群的同调群计算最大的群的同调群。
\begin{example}{}
   设$A$是$R$模且是子模$\dots \subset A_i \subset A_{i+1}\subset \dots$的并,则对于任何$R$模$B$和$q$,都存在列:
   \begin{align}
      0 \to \textstyle\Lim^1 \Ext_R^{q-1}(A_i,B) \to \Ext_R^q(A,B) \to \Lim \Ext_R^q(A_i,B)\to 0
   \end{align}
   是正合的。

   对于$\Z_{p^\infty}=\cup \Z/p^i$,上述列化为:
   \begin{align}
      0 \to \textstyle\Lim^1 \Hom(\Z/p^i,B) \to \Ext_R^1(\Z_{p^\infty},B) \to \Lim \Ext_R^1(\Z/p^i,B)=\hat{B}_p \to 0
   \end{align}
   其中$\hat{B}_p=\Lim(B/p^iB)$是$B$的$p$-adic的完备化。
   
   这相当于推广了计算:$\Ext^1_\Z(\Z_{p^\infty},\Z)\cong \hat{\Z}_p$.实际上,设$E$是一个不变的$B$内射解消,考虑上链复形的塔链:
   \begin{align}
      \Hom(A_{i+1},E) \to \Hom(A_i,E) \to \dots \Hom(A_0,E) 
   \end{align}
   因为每个$\Hom(-,E_n)$都是反变正合的,所以塔链中每一个映射都是满射。(单反过来就是满).而$\Hom(A_i,E)$的上同调是$\Ext^*(A_i,B)$,$\Ext^*(A,B)$是:
   \begin{align}
      \Hom(\cup A_i,E)=\Lim \Hom(A_i,E)
   \end{align}
   的上同调。
\end{example}
\begin{corollary}{}
   $Z[1/p]=\cup p^{-1}\Z$,从而$\Ext^1_\Z(\Z[1/p],\Z)\cong \hat{\Z}_p/\Z$.从而对于无挠群$B$,有$\Ext_\Z^1(\Q,B)=(\prod_p \hat{B}_p)/B$.
\end{corollary}

\section{泛系数定理}
这一节我们思考的问题是,在已知$P$的同调下,如何计算$P\otimes M$的同调。由于在拓扑中,有所谓$\Z$系数,$\R$系数,$R$系数的说法,所以我们实际上在思考不同系数情况下一个拓扑空间同调和上同调群的关系。因而这节的名字是泛系数定理。
\begin{theorem}[Kunneth公式]{Kunneth-formula}
   设$P$是由平坦右$R$模给出的链复形,且$d(P_n)$作为$P_{n-1}$的子模总是平坦的。则对于任何$n$和任何左模$M$,都存在正合列:
   \begin{align}
      0 \to H_n(P)\otimes_R M \to H_n(P\otimes_R M) \to \mathrm{Tor}_1^R(H_{n-1}(P),M)\to 0
   \end{align}
\end{theorem}
\begin{proof}
   考虑短正合列:
   \begin{align}
      0 \to Z_n \to P_n \to d(P_n)\to 0
   \end{align}
   对此使用$\Tor$函子,可以得知$Z_n$也是平坦模。考虑到$\Tor_1(d(P_n),M)=0$,则:
   \begin{align}
      0 \to Z_n \otimes M \to P_n\otimes M \to d(P_n)\otimes M \to 0
   \end{align}
   是正合的。从而我们给出了链复形的短正合列:
   \begin{align}
      0 \to Z\otimes M \to P\otimes M \to d(P)\otimes M \to 0
   \end{align}
   注意到$Z$和$d(P)$中的微分算子都是$0$,从而短正合列导引的长正合列为:
   \begin{align}
      H_{n+1}(dP\otimes M) \xrightarrow{\partial} H_n(Z\otimes M) \to H_n(P \otimes M) \to H_n(dP \otimes M) \xrightarrow{\partial}H_{n-1}(Z\otimes M)
   \end{align}
   其中$H_n(dP_n\otimes M)=dP_n \otimes M$,$H_n(Z_n\otimes M)=Z_n\otimes M$.

   设$i:d(P_{n+1}) \to Z_n$是包含映射。我们断言$\partial$实际上是$i\otimes M$。(实际上很容易给出)。另一方面,$0 \to d(P_{n+1}) \to Z_n \to H_n(P) \to 0$是$H_n(P)$的平坦解消,所以$\Tor_1(H_n(P),M)$可以使用:
   \begin{align}
      0 \to d(P_{n+1})\otimes Z_n\otimes M \to 0
   \end{align}
   计算。结合长正合列即可得到结果。
\end{proof}
\begin{theorem}[同调的泛系数定理]{homo-universal}
   设$P$是一个自由Abel群的链复形。则对于任意的$n$和每个交换群$M$而言,定理\ref{Kunneth-formula}中的正合列分裂。但是这个分裂并不典范。
   \begin{align}
      H_n(P\otimes M)\cong H_n(P)\otimes M \oplus \Tor_1^\Z(H_{n-1}(P),M)
   \end{align}
\end{theorem}
\begin{proof}
   众所周知,自由Abel群的子群还是自由的。考虑$d(P_n)$是$P_{n-1}$的子群,则$d(P_n)$是自由Abel群。不典范的,这说明:
   \begin{align}
      P_n=Z_n \oplus d(P_n)
   \end{align}
   从而$Z_n\otimes M$是$P_n\otimes M$的直和项,也是$\ker(d_n\otimes 1)$的直和项。

   商去$d_{n+1}\otimes 1$的像,我们有$H_n(P)\otimes M$是$H_n(P\otimes M)$的直和项。根据Kunneth公式可知另一个项是$\Tor_1^\Z(H_{n-1}(P),M)$.
\end{proof}
\begin{theorem}[复形的Kunneth公式]{Kunneth-formula-complex}
   设$P,Q$是右,左模链复形.如果$P$和$d(P)$都是平坦的,则存在正合列:
   \begin{align}
      0 \to \bigoplus_{p+q=n}H_p(P)\otimes H_q(Q) \to H_n(P\otimes Q) \to \bigoplus_{p+q=n-1}\Tor_1^R(H_p(P),H_q(Q)) \to 0
   \end{align}
\end{theorem}
\begin{proof}
   仿照定理\ref{Kunneth-formula}的证明,把$M$换成$Q$.
\end{proof}
为了节省时间,我们省略拓扑上的泛系数定理。

接下来我们攥写上同调版本的泛系数定理。
\begin{theorem}[上同调的泛系数定理]{cohomo-universal}
   设$P$是投射模给出的链复形,使得$d(P_n)$也是投射模。则对于每个$n$和$R$模$M$,存在一个非典范的分裂正合列:
   \begin{align}
      0 \to \Ext^1_R(H_{n-1}(P),M)\to H^n(\Hom_R(P,M)) \to \Hom_R(H_n(P),M)\to 0
   \end{align}
\end{theorem}
\begin{proof}
   因为$d(P_n)$投射,从而有非典范的分裂:$P_n=d(P_{n+1})\oplus Z_n$.从而$Z_n$也投射,并且有:
   \begin{align}
      0 \to \Hom(dP_{n+1},M) \to \Hom(P_n,M)\to \Hom(Z_n,M) \to 0
   \end{align}
   是正合的。所以$0 \to \Hom(dP,M) \to \Hom(P,M)\to \Hom(Z,M) \to 0$是链复形的正合列。导引的长正合列:
   \begin{align}
      H^{n-1}(\Hom(Z,M)) \xrightarrow{\partial} H^n(\Hom(dP,M)) \to H^n(\Hom(P,M)) \to H^n(\Hom(Z,M)) \xrightarrow{\partial} H^{n+1}(\Hom(dP,M))
   \end{align}
   注意到$dP$和$Z$的微分算子都是$0$,所以$\Hom(dP,M)$的微分也是$0$,因此$H^n(\Hom(dP,M))=\Hom(dP_n,M)$。同理$H^n(\Hom(Z,M))=\Hom(Z_n,M)$。并且这里的$\partial$右$d(P_{n+1})$到$Z_n$的嵌入给出。

   注意到$H_n(P)$有投射解消:
   \begin{align}
      0 \to d(P_{n+1}) \to Z_n \to H^n(P)
   \end{align}
   于是$\Ext^1(H_{n-1}(P),M)$和$\Hom(H_n(P),M)=\Ext^0(H_n(P),M)$都可以用:
   \begin{align}
      0 \to \Hom(Z_{n-1},M) \to \Hom(dP_n,M) \to 0
   \end{align}
   带入上面的长正合列即可得到正合结果。

   而分裂可依照\ref{homo-universal}的结果得出。
\end{proof}
\begin{example}{}
   设$X$是道路连通的,则$H_0(X)=\Z$,且$H^1(X;\Z)\cong \Hom(H_1(X),\Z)$.这是一个无挠的Abel群。($\Z$是投射模。)
\end{example}
\begin{theorem}[上双复形的泛系数定理]{}
   设$P$是一个链复形,$Q$是上链复形.
   
   则可以定义上双复形$\Hom(P,Q)$.用$H^*(\Hom(P,Q))$表示$\mathrm{Tot}(\Hom(P,Q))$的上同调。设$P_n$和$dP_n$总是投射的,则存在正合列:
   \begin{align}
      0 \to \prod_{p+q=n-1}\Ext^1_R(H_p(P),H^q(Q)) \to H^n(\Hom(P,Q)) \to \prod_{p+q=n}\Hom_R(H_p(P),H^q(Q)) \to 0
   \end{align}

\end{theorem}
最后我们给出右继承的概念以结束本节。一个环$R$称作右继承的,如果任何自由(右)模的子模都是投射(右)模。实际上,任何主理想整环都是继承环(他们都是交换的戴德金整环).

继承环这条良好的性质显然可以帮助我们把泛系数定理推广到任何继承环(直接的,主理想整环)。
 \ifx\allfiles\undefined
	
	% 如果有这一部分的参考文献的话,在这里加上
	% 没有的话不需要
	% 因此各个部分的参考文献可以分开放置
	% 也可以统一放在主文件末尾。
	
	%  bibfile.bib是放置参考文献的文件,可以用zotero导出。
	% \bibliography{bibfile}
	
	end{document}
	\else
	\fi
\chapter{Complex Manifold}

\chapter{K\"{a}hler Manifold}
\section{Difinition and K\"{a}hler Identity}
This section we introduce the basic concept of K\"{a}hler manifold and K\"{a}hler Identity.

Let $X$ be a complex manifold.We denote the induced almost complex structure by $I$. The following definition is natural.

\begin{definition}
A Riemann metric $g$ on $X$ is an hermitian structure on X if for any point $x \in X$,the scalar product $g_x$ on $T_x X$ is compatible with the almost complex structure $I_x$. That is, $g_x(I_x w,T_x v)=g_x(w,v)$ for any $x \in X,w,v \in T_x X$.

Then the induced form $\omega:=g(I(),())$ is called the fundamental form.
\end{definition}

\chapter{Vector Bundles}

\chapter{Applications of Cohomology}

\ifx\allfiles\undefined
	
	% 如果有这一部分的参考文献的话,在这里加上
	% 没有的话不需要
	% 因此各个部分的参考文献可以分开放置
	% 也可以统一放在主文件末尾。
	
	%  bibfile.bib是放置参考文献的文件,可以用zotero导出。
	% \bibliography{bibfile}
	
	\end{document}
	\else
	\fi
\ifx\allfiles\undefined

	% 如果有这一部分另外的package,在这里加上
	% 没有的话不需要
	
	\begin{document}
\else
\fi
\part{Morse理论}
设$f$是流形$M$上的光滑函数。我们定义:称一个点$p\in M$是$f$的临界点(critical point),若诱导映射$f_*:T_pM \to T_{f(p)}R$是$0$映射。

在流形上我们最好用各种各样的局部坐标讨论。设$(U;x_i,1\leq i \leq n)$是$p$附近的一个局部坐标系,则临界点的定义可以写为:
\begin{align}
	\pa{f}{x^1}=\pa{f}{x^2}=\dots=\pa{f}{x^n}=0
\end{align}

此时$f(p)$称为$f$的临界值。

在临界点处$f$的性质有着与非临界值完全不同的性质。Morse理论则是研究临界点处,$M$本身拓扑性质的改变的理论。
\chapter{流形上的非退化光滑函数}
\section{Morse函数}
我们先用一个引理说明在非临界点$M$的平凡性质。
\begin{lemma}[非临界点]
	设$M^a=\{p\in M|f(p)\leq a\}$。若$a$不是临界值,则$M^a$是带边的光滑流形。
\end{lemma}

引理的证明留作练习。主要使用到隐函数定理以及带边流形的定义。

\begin{definition}[非退化点]
	考虑$M$上的函数$f$.若在$f$的临界点$p$处存在一个局部坐标$(U;x^i)$使得矩阵:
	\begin{align}\label{Hess}
		(\frac{\partial^2 f}{\partial x^i\partial x^j}(p))
	\end{align}
	非奇异,则称$p$是一个非退化点。
\end{definition}
这里需要注意的是,$p$的非退化性显然与局部坐标$x^i$无关。因此$p$的非退化性是$f$内蕴的性质。

如果$p$是$f$的临界点,我们就可以定义在$T_pM$上的双线性函数$f_{**}$。若$v,w \in T_pM$,用$\tilde{v}$和$\tilde{w}$表示在$p$处值为$v,w$的向量场。定义:
\begin{align}
	f_{**}(v,w)=\tilde{v}_p(\tilde{w}f)
\end{align}

我们断言:
\begin{lemma}
	$f_**$是对称的良定双线性函数。
\end{lemma}
\begin{proof}
	考虑:
	\begin{align}
		\tilde{v}_p(\tilde{w}f)-\tilde{w}_p(\tilde{v}f)=[\tilde{v},\tilde{w}]_p(f)=0
	\end{align}
	最后一个等号成立,是因为$p$是$f$的临界点。

	从而$f_{**}$是对称的。因此,$\tilde{v}_p(\tilde{w}f)$与$\tilde{v}$的选取无关,$\tilde{w}_p(\tilde{v}f)$与$\tilde{w}$的选取无关.

	于是$f_{**}$与$\tilde{v},\tilde{w}$的选取都无关,因而是良定的双线性函数。
\end{proof}
\begin{definition}[Hessian,指数,零化度]
    \quad \quad 称$f_{**}$为函数$f$在$p$处的Hessian双线性函数。
	
	而$f_{**}$的指数定义为满足$f_**$限制在上为负定双线性函数的子空间$V$的最大维数。$f_{**}$的零化度定义为$f_{**}$的零空间$W$的维数,即子空间$W=\{v \in V|f_{**}(v,w)=0,\forall w\in V\}$的维数。
\end{definition}

可以验算$f_{**}$在坐标$(U;x^i)$下给出的就是矩阵\ref{Hess}.显然,$f$在$p$处非退化等价于$f_{**}$的零化度是$0$。$f_{**}$的指数也称为$f$在$p$处的指数。

\begin{lemma}[Morse引理]\label{Morse-lemma}
	设$p$是$f$的非退化点,则存在一个$p$处的局部坐标$(U;y^i)$满足$y^i(p)=0,\forall i$且:
	\begin{align}
		f(q)=f(p)-(y^1)^2-\dots-(y^\lambda)^2+(y^{\lambda+1})^2+\dots+(y^n)^2
	\end{align}
	在整个$U$上都成立.其中$\lambda$是$f$在$p$处的指数。
\end{lemma}
\begin{proof}
	我们首先说明如果$f$拥有这样的表达式,则指数为$\lambda$.

	对于坐标$(z^i)$,若:
	\begin{align*}
		f(q)=f(p)-(z^1(q))^2-\dots-(z^\lambda(q))^2+\dots+(z^n(q))^2
	\end{align*}
	则容易求出$f_{**}$在该坐标下的矩阵为$\mathrm{diag}(-2,\dots,-2,2,\dots,2)$.其中$-2$一共有$\lambda$个。

	因此存在一个$\lambda$维的子空间使得$f_{**}$是负定的,存在一个$n-\lambda$维的子空间$V$使得$f_{**}$是正定的。如果$p$处的指数大于$\lambda$,则对应的子空间与$V$相交不为空。但这是不可能的,因此$\lambda$是$f$在$p$处的指数。

	接下来我们说明$(y^i)$坐标存在。不妨设$p$是$\R^n$的原点且$f(p)=f(0)=0$.从而有:
	\begin{align}
		f(x_1,\dots,x_n)=\int_0^1\frac{\dd f(tx_1,\dots,x_n)}{\dd   t}\dd t=\int_0^1 \sum_{i=1}^n \pa{f}{x_i}(tx_1,\dots,tx_n)x_i\dd t
	\end{align}
	令$g_j=\int_0^1\pa{f}{x_i}(tx_1,\dots,tx_n)\dd t$,则$f=\sum_j  x_jg_j$在$0$处的一个邻域上成立。

	因为$0$是$f$的临界点,从而$\pa{f}{x)i}(0)=0$。这意味着$g_j(0)=\pa{f}{x_i}(0)$.因而对$g_j$作上述$f$同样的分解:
	\begin{align*}
		f(x_1,\dots,x_n)=\sum_{i,j} x_ix_jh_{ij}(x_1,\dots,x_n)
	\end{align*}
	
	不妨假设$h_{ij}$关于$i,j$对称。通过计算,不难验证矩阵$(h_{ij}(0))$等于:
	\begin{align*}
		(\frac{1}{2}\frac{\partial^2f}{\partial x^i\partial x^j}(0))
	\end{align*}
	因此$h_{ij}(0)$是非奇异的矩阵。仿照模仿有理标准型的构造,可以证明存在一组坐标$(y^i)$使得$f$呈现为引理中的形式。具体的构造办法详见Milnor原书。(附录)
\end{proof}

Morse引理的好处在于我们可以用指数唯一确定$f$在$p$处的一个标准形式。根据这个引理,可以得知$f$在非退化点$p$的一个邻域内只有$p$一个临界点。
\begin{corollary}
	非退化临界点是离散的。特别的,紧致流形$M$上的非退化临界点只有有限个。
\end{corollary}

\begin{definition}[Morse函数]
	若$f\in \mathcal{O}(M)$且只有非退化的临界点,则称该函数为Morse函数。
\end{definition}
Morse函数的好处是显而易见的。然而存在性则是一个问题。本节我们剩下的内容为下面的定理。
\begin{theorem}
	任何流形$M$上都存在一个可微的函数$f$,满足不存在退化临界点,且$M^a$对于任何$a \in \R$都是紧致的。
\end{theorem}

根据Whitney嵌入定理,任何流形都可以嵌入到维数足够高的欧氏空间。因而我们考虑$M$是$\R^n$的$k$维嵌入子流形(之后统称为“子流形”)。

定义$N \subset M\times \R^n$为:
\begin{align*}
	N=\{(q,v):q \in M,v\in T_q\R^n,v\perp M\}
\end{align*}
即$N$是$M$在$\R^n$中的法丛。不难验证$N$是$n$维的流形,且光滑的嵌入进$\R^{2n}$中。定义$E:N \to \R^n$为映射$(q,v)\mapsto q+v$.

\begin{definition}[焦点]
	称$e \in \R^n$是$(M,q)$重数为$\mu$的焦点,若$e=E(q,v),(q,v)\in N$且$E$在$(q,v)$处的Jacobian矩阵有零化度$\mu$.
\end{definition}

根据Sard定理,两个微分流形之间的可微映射的临界点是很有限的——临界值只有$0$测度。显然焦点是$E$的临界值,因而:
\begin{corollary}\label{0focus}
	对于几乎所有的$x \in \R^n$,$x$都不是$M$的焦点。
\end{corollary}

现在固定$p \in \R^n$.定义函数$f:M \to \R$为:
\begin{align}
	L_p=f:q\mapsto\|q-p\|^2
\end{align}
在坐标$(U;u^1,\dots,u^k)$下,$f$的表达式为:
\begin{align*}
	f(u^1,\dots,u^k)=\|\vec{x}(u^1,\dots,x^k)-\vec{p}\|^2=\vec{x}\vec{x}-2\vec{x}\vec{p}+\vec{p}\vec{p}
\end{align*}

因此可以计算:
\begin{align*}
	\pa{f}{u^i}=2\pa{\vec{x}}{u^i}\cdot(\vec{x}-\vec{p})
\end{align*}

因此$q$是$f$的临界点,当且仅当$q-p$垂直与$M$垂直。

考虑$f$的二阶导数。我们有:
\begin{align*}
	\frac{\partial^2 f}{\partial u^i\partial u^j}=2\pa{\vec{x}}{u^i}\pa{\vec{x}}{u^j}+\pa{\vec{x}}{u^i}u^j\cdot(\vec{x}-\vec{p})
\end{align*}

从而有
\begin{lemma}
	$q$是$f=L_p$的退化临界点等价于$p$是$(M,q)$的焦点。根据推论\ref{0focus},总存在这样的$L_p$使得该函数不存在退化的临界点。另外,若$q$是$L_p$的退化临界点,则该点的零化度是$p$的重数。
\end{lemma}

\section{临界值处的伦形}
\section{Morse不等式}


\chapter{Morse理论的应用——测地线变分}
\section{道路的能量积分}
\section{指标定理}
\section{道路空间的伦型}

\ifx\allfiles\undefined

	% 如果有这一部分另外的package,在这里加上
	% 没有的话不需要
	\newcommand{\id}{\mathrm{id}}
\newcommand{\Hom}{\mathrm{Hom}}
\newcommand{\N}{\mathbb{N}}
\newcommand{\Z}{\mathbb{Z}}
\newcommand{\Q}{\mathbb{Q}}
\newcommand{\R}{\mathbb{R}}
\newcommand{\C}{\mathbb{C}}
\newcommand{\HH}{\mathbb{H}}
\newcommand{\RP}{\mathbb{RP}}
	\begin{document}
\else
\fi
\chapter{Tor函子和Ext函子}
本章的目的是介绍Tor函子和Ext函子的诸多性质。他们是同调代数初等应用中的常客。
\section{Abel群的Tor函子}
我们首先观察一个经典的PID上的模——Abel群的Tor函子。其实,Tor函子的名字就来源于其对Abel群的研究。

\begin{example}{}
    对于Abel群$B$而言,$\mathrm{Tor}_0^{\Z}(\Z/p,B)=B/pB$,$\mathrm{Tor}_1^\Z(\Z/p,B)={}_pB=\{b \in B:pB=0\}$.对于$n\geq 2$,$\mathrm{Tor}_2^\Z(\Z/p,B)=0$.

    上述结果可以这么看。取$\Z/p$的投射解消
    \begin{align}
        0 \to \Z \stackrel{p}{\rightarrow}\Z \to \Z/p \to 0
    \end{align}
    从而我们计算的是:
    \begin{align}
        0 \to B \stackrel{p}{\rightarrow} B \to 0
    \end{align}
    的同调群。
\end{example}
 特殊情况下,Tor函子表现出$1$阶挠子群,高阶为$0$的特点。实际上,我们有下面的命题:
 \begin{proposition}{}
    对于两个Abel群$A$,$B$,我们有:
    
    (a)$\mathrm{Tor}_1^\Z(A,B)$是一个挠群。

    (b)$\mathrm{Tor}_n^\Z(A,B)$在$n \geq 2$的情况下为$0$.
 \end{proposition}
 \begin{proof}
    证明依赖Tor函子与滤过余极限交换性。$A$是其有限生成子群的滤过余极限,所以$\mathrm{Tor}_n(A,B)$是$\mathrm{Tor}_n(A_\alpha,B)$的滤过余极限。

    Abel群的余极限总是他们直和的商子群。所以我们只需要证明对于有限生成子群上述命题成立即可。

    设$A=\Z^m \oplus \Z/p_1 \oplus \Z/p_2 \dots \Z/p_r$。因为$\Z^m$是投射的,所以只用考虑:
    \begin{align}
        \mathrm{Tor}_n(A,B)=\mathrm{Tor}_n(\Z/p_1,B)\oplus \mathrm{Tor}_n(\Z/p_2,B) \oplus \dots \mathrm{Tor}_n(\Z_r,B)
    \end{align}
    于是根据之前的例子我们知道结论成立。
 \end{proof}
 \begin{proposition}{}
    $\mathrm{Tor}_1^\Z(\Q/\Z,B)$是$B$的挠子群。
 \end{proposition}
 \begin{proof}
    可以想见,$\Z/p$提取出$B$中挠性为$p$的元素。$\Q/\Z$是其有限子群的滤过极限,并且每个有限子群都同构于某个$\Z/p$($p$不一定是素数。)
    \begin{align}
        \mathrm{Tor}_*^\Z(\Q/\Z,B)\cong \Colim \mathrm{Tor}_1^\Z(\Z/p,B)\cong \Colim({}_pB)=\cup_p\{b\in B:pb=0\}
    \end{align}

 \end{proof}
 \begin{proposition}{}
    如果$A$是一个无挠交换群,则$\mathrm{Tor}_n(A,B)$对于$n \neq 0$和Abel群$B$总是$0$。
 \end{proposition}
 \begin{proof}
    $A$是有限生成子群的滤过余极限。然而$A$无挠意味着这些有限生成子群都是自由群。用Tor保滤过余极限即可。
 \end{proof}
 如果$R$是交换环,则张量积有典范的同构,因此$\mathrm{Tor}_*(A,B)\cong \mathrm{Tor}_*(B,A)$.

 \begin{corollary}{}
    $\mathrm{Tor}_1^\Z(A,-)=0$等价于$A$无挠等价于$\mathrm{Tor}_1^\Z(-,A)=0$.
 \end{corollary}
 但是Tor函子并非对于所有环都有这么好的性质。比如下面的例子就说明在$R=\Z/m$的情况下可能失败:
 \begin{example}{}
   设$R=\Z/m$,$A=\Z/d$。其中$d|m$。从而$A$是$R$模。

   我们考虑$A$周期性的自由解消:
   \begin{align}
      \dots \to \Z/m \to Z/m \to \Z/m \to \Z/d
   \end{align}
   其中从$\Z/m$到$\Z/d$的映射是商映射,而$\Z/m$各自之间交替出现$d$和$m/d$。所以对于任何一个$\Z/m$模$B$,我们都有:
   \begin{align}
      \mathrm{Tor}_n^{\Z/m}(\Z/d,B)=\begin{cases}
      B/dB,n=0\\ \{b\in B:db=0\}/(m/d)B,n \text{是奇数}\\ \{b \in B:(m/d)b=0\}/dB,n \text{是偶数且}>0
      \end{cases}
   \end{align}
 \end{example}
 然而我们可以尝试对下面特殊的情况进行一些讨论。
 \begin{example}{}
   设$r$是$R$的一个左非零除子。即${}_rR=\{s \in R|rs=0\}$是$0$。对于每个$R$模$B$,记${}_rB=\{b \in B:rb=0\}$。用$R/rR$代替上述$\Z/p\Z$,用相同的计算办法可以算的:
   \begin{align}
      \mathrm{Tor}_0(R/rR,B)=B/rB;\quad \mathrm{Tor}_1^R(R/rR,B)={}_r B; \quad \mathrm{Tor}_n^R(R/rR,B)=0, n\geq 0
   \end{align}
 \end{example}
 \begin{proposition}{}
   若${}_r R\neq 0$,我们只能得到一个并非投射的解消:
   \begin{align}
      0 \to {}_r R \to R \stackrel{r}{\rightarrow} R \to R/rR \to 0
   \end{align}
   然而第二章我们介绍了dimension shelfting办法\ref{dim-Shifting}。所以我们对于$n \geq 3$,存在:
   \begin{align}
      \mathrm{Tor}_n^R(R/rR,B) \cong \mathrm{Tor}_{n-2}^R({}_r R,B)
   \end{align}

   其次,还有正合列:
   \begin{align}
      0 \to \mathrm{Tor}_2^R(R/rR,B) \to {}_rR \otimes B \to {}_rB \to \mathrm{Tor}_1^R(R/rR,B) \to 0
   \end{align}
   因为$\mathrm{Tor}_2^R(R/rR,B)$是$0 \to {}_rR\otimes B \to R\otimes B=B$的核。而该映射的像就在${}_r B$中,所以上述正合列中第一个和第二个已经确实成立。

   考虑$\mathrm{Tor}_1(R/rR,B)$。根据导引长正合列:
   \begin{align}
      0 \to \mathrm{Tor}_1(R/rR,B) \to rR\otimes B \to B \to B/rB
   \end{align}
   为了定义${}_r B \to \mathrm{Tor}_1(R/rR,B)$.我们定义${}_r B \to rR\otimes B$.即$b \mapsto r \otimes b$。则该映射实际上打进$\mathrm{Tor}_1(R/rR,B)$.

   若$\sum (rr_i)\otimes b_i \in \mathrm{Tor}_1(R/rR,B)$且在$B$中像为$\sum r(1\otimes r_ib_i)=0$,则${}_r B$中$\sum r_ib_i$的像是$\sum (rr_i)\otimes b_i$。于是我们定义了满射。

   最后需要说明${}_r B$处的正合。若$r \otimes b=0$,则存在$r_i$和$b_i$使得$rr_i=0$,$b=\sum r_ib_i$.
 \end{proposition}
 \begin{proposition}{}
   设$R$是交换整环,分式域$F$。则$\mathrm{Tor}_1^R(F/R,B)$是$B$的挠子群:$\{b \in B:(\exists r\neq 0)rb=0\}$
 \end{proposition}
 \begin{proposition}{}
   $\mathrm{Tor}_1^R(R/I,R/J) \cong \dfrac{I\cap J}{IJ}$对于任何右理想$I$和左理想$J$都成立。特别的,对于双边理想$I$:
   \begin{align}
      \mathrm{Tor}_1(R/I,R/I)\cong I/I^2
   \end{align}
 \end{proposition}
 \begin{proof}
   % https://q.uiver.app/#q=WzAsMTYsWzAsMSwiMCJdLFsxLDEsIklKIl0sWzIsMSwiSSJdLFszLDEsIklcXG90aW1lcyBSL0oiXSxbNCwxLCIwIl0sWzEsMiwiSiJdLFsyLDIsIlIiXSxbMywyLCJSXFxvdGltZXMgUi9KIl0sWzAsMiwiMCJdLFs0LDIsIjAiXSxbMSwwLCIwIl0sWzEsMywiSi8oSUopIl0sWzIsMCwiMCJdLFsyLDMsIlIvSSJdLFszLDAsIlxca2VyIGkiXSxbMywzLCJSL0kgXFxvdGltZXMgUi9KIl0sWzAsMV0sWzEsMl0sWzIsM10sWzMsNF0sWzEsNV0sWzIsNl0sWzMsNywiaVxcb3RpbWVzXFxtYXRocm17aWR9Il0sWzgsNV0sWzUsNl0sWzYsN10sWzcsOV0sWzEwLDFdLFs1LDExXSxbMTIsMl0sWzYsMTNdLFsxNCwzXSxbNywxNV0sWzEwLDEyXSxbMTIsMTRdLFsxNCwxMSwiIiwxLHsic3R5bGUiOnsiYm9keSI6eyJuYW1lIjoiZGFzaGVkIn19fV0sWzExLDEzXSxbMTMsMTVdXQ==
\[\begin{tikzcd}
	& 0 & 0 & {\ker i} \\
	0 & IJ & I & {I\otimes R/J} & 0 \\
	0 & J & R & {R\otimes R/J} & 0 \\
	& {J/(IJ)} & {R/I} & {R/I \otimes R/J}
	\arrow[from=2-1, to=2-2]
	\arrow[from=2-2, to=2-3]
	\arrow[from=2-3, to=2-4]
	\arrow[from=2-4, to=2-5]
	\arrow[from=2-2, to=3-2]
	\arrow[from=2-3, to=3-3]
	\arrow["{i\otimes\mathrm{id}}", from=2-4, to=3-4]
	\arrow[from=3-1, to=3-2]
	\arrow[from=3-2, to=3-3]
	\arrow[from=3-3, to=3-4]
	\arrow[from=3-4, to=3-5]
	\arrow[from=1-2, to=2-2]
	\arrow[from=3-2, to=4-2]
	\arrow[from=1-3, to=2-3]
	\arrow[from=3-3, to=4-3]
	\arrow[from=1-4, to=2-4]
	\arrow[from=3-4, to=4-4]
	\arrow[from=1-2, to=1-3]
	\arrow[from=1-3, to=1-4]
	\arrow[dashed, from=1-4, to=4-2]
	\arrow[from=4-2, to=4-3]
	\arrow[from=4-3, to=4-4]
\end{tikzcd}\]
上图是蛇形引理\ref{snake}.验证$I/(IJ)$和$I \otimes R/J$有典范同构可以得出第一行正合。第二行则典范正合。

 最右边的列是计算$\mathrm{Tor}_1(R/I,R/J)$的定义式。感觉Dimesion Shifting,$\ker i$是$\mathrm{Tor}_1(R/I,R/J)$。根据snake引理,$\ker i$是$J/(IJ)  \to R/I$的核:$\dfrac{I\cap J}{IJ}$。
 \end{proof}
\section{Tor函子与平坦性}
我们在这一节着重研究Tor函子的ayclic对象——平坦对象。
\begin{definition}[平坦模]{flat-module}
   称一个左$R$模是平坦模,若函子$\otimes_R B$是正合函子。同样,对于右$R$模,也可以定义类似的平坦性。
\end{definition}
如果$A$是投射的,则$\mathrm{Tor}_n(A,B)=0$。不难说明$A$此时是平坦的。因为投射模一定是平坦模。然而平坦模不一定是投射模。例如$\Q$作为交换群而言是平坦的,但不是投射的。(为什么?)
\begin{theorem}{}
   若$S$是$R$中的乘法封闭集,则$S^{-1}R$是一个平坦模。
\end{theorem}
这个定理当然很交换代数,不过影响不大,我们可以尝试证明:
\begin{proof}
   构造一个滤过范畴$I$。对象是$S$中的元素,态射$\Hom_I(s_1,s_2)=\{s \in S:s_1s=s_2\}$。定义函子$F:I \to R$。$F(s)=R$,$F(s_1 \to s_2)$则定义为$R$上该态射自然给出的右乘法。

  
我们断言$F$的余极限$\Colim F(s) \cong S^{-1}R$。从而因为$S^{-1}R$是平坦模的滤过余极限,所以其是平坦的。

   下面计算$\Colim F$。首先定义$F(s) \to S^{-1}R$的映射为$r \mapsto r/s$.这样交换图显然成立:
   % https://q.uiver.app/#q=WzAsMyxbMCwwLCJGKHNfMSk9UjpyIl0sWzEsMCwiRihzXzIpPVI6cnMiXSxbMCwxLCJTXnstMX1SOnIvc18xPXJzLyhzXzFzKT1ycy9zXzIiXSxbMCwxLCJzIl0sWzAsMl0sWzEsMl1d
\[\begin{tikzcd}
	{F(s_1)=R:r} & {F(s_2)=R:rs} \\
	{S^{-1}R:r/s_1=rs/(s_1s)=rs/s_2}
	\arrow["s", from=1-1, to=1-2]
	\arrow[from=1-1, to=2-1]
	\arrow[from=1-2, to=2-1]
\end{tikzcd}\]

如果存在一个新的$B$使得余极限中关系成立,我们直接定义$S^{-1}R$中的元素$r/s$到$B$的态射为$F(s)=R$中$r$在$B$中的像即可。这是唯一的定义方式!
\end{proof}
\begin{proposition}[Tor和平坦]
   下面三个命题等价:

   (1)$B$是平坦模。

   (2)$\mathrm{Tor}_n^R(A,B)=0,\forall n\neq 0$

   (3)$\mathrm{Tor}_1^R(A,B)=0$
\end{proposition}
\begin{corollary}
   若$0 \to A \to B \to C \to 0$是正合列且$B,C$是平坦模,则$A$平坦。
\end{corollary}
\begin{proposition}
   设$R$是主理想整环,则$B$平坦等价于$B$无挠。
\end{proposition}
对于上述命题,我们给出一个反例。首先平坦显然无挠。但是无挠不一定平坦。设$k$是域且$R=k[x,y]$。$R$是经典的非主理想整环。设$I=(x,y)R$。考虑$k=R/I$有投射解消:
\begin{align}
   0 \to R \to R^2 \to R \to k
\end{align}
其中第一个$R$到$R^2$为$[-y,x]$.而$R^2$到$R$为$(x,y)$.从而$\mathrm{Tor}_1^R(I,k)\cong \mathrm{Tor}_2^R(k,k)\cong k$。于是$I$不是平坦模。

我们深入的研究一下平坦模。
\begin{definition}[Pontrjagin对偶]{Pontrjagi}
   左模$B$的Pontrjagin对偶模$B^*$是一个右模:
   \begin{align}
      B^*:=\Hom_{\mathrm{Ab}}(B,\Q/\Z); (fr)(b)=f(rb)
   \end{align}
\end{definition}
\begin{proposition}{}
   下面的命题等价。

   (1)$B$平坦。

   (2)$B^*$内射。

   (3)$I\otimes_R B\cong IB=\{x_1b_1+\dots+x_nb_n\in B:x_i\in I,b_i\in B\}$对于任何右理想$I$都成立。

   (4)$\mathrm{Tor}_I^R(R/I,B)=0$对于任何右理想$I$都成立。
\end{proposition}
\begin{proof}
   (3)和(4)的等价性来源于正合列:
   \begin{align}
      0 \to \mathrm{Tor}_1(R/I,B) \to I\otimes B \to B \to B/IB \to 0
   \end{align}
   现在考虑$A'$是$A$的子模。考虑:
   \begin{align}
      \Hom(A,B^*) \to \Hom(A',B^*)
   \end{align}
   $B^*$等价于说上述映射是满射。根据伴随关系,我们有:
   \begin{align}
      \Hom(A\otimes B,\Q/\Z) \to \Hom(A'\otimes B,\Q/Z)
   \end{align}是满射。即$(A\otimes B)^* \to (A'\otimes B)^*$是满射。

   用下面的\textbf{引理},可以知道此时$A' \otimes B \to A\otimes B$是单射,所以$B$是平坦模。同理也可以反推回去。所以(1)(2)等价。另外带入$A'=I,A=R$,可以推出$I\otimes B \to R\otimes B$是单射。于是$I\otimes B\cong IB$且根据Baer判别法,这是可逆的。所以(1)(3)等价。
\end{proof}
我们描述一个引理。
\begin{lemma}{}
   $f:A' \to A$是单射等价于$f^*:A^* \to A'^*$是满射。
\end{lemma}
\begin{proof}
   因为$\Q/\Z$是内射的$\Z$模,所以保正合。
\end{proof}
\begin{proposition}[Pontrjagin对偶与正合]{}
   $A \to B \to C$是正合的当且仅当对偶$C^* \to B^* \to A^*$是正合的。
\end{proposition}
\begin{proof}
   因为$\Q/\Z$是内射模,所以$\Hom(-,\Q/\Z)$是正合函子,因此$C^* \to B^* \to A^*$是正合的。

   如果$C^* \to B^* \to A^*$正合,则$A \to B \to C$首先复形。若$b \in B$且在$C$中的像为$0$,我们证明$b$在$A$的像中。若不然,则$b+\mathrm{im}A$是$B/\mathrm{im}A$中的非$0$元。我们定义$g:B/\mathrm{im}A \to \Q/\Z$使得$g(b+\mathrm{im}A)\neq 0$。则$g$也给出了$B^*$中的非$0$元且在$A^*$中的像为$0$。

   所以可以给出一个$f \in C^*$。剩下的就是显然了。
\end{proof}
这个证明写的比较模糊。

我们邀请读者回忆有限展示的概念。然后不加证明的给出有限展示与生成元的选取无关.

\begin{proposition}{}
   若$\varphi:F \to M$是满射且$F$是有限生成的,$M$是有限展示的,则$\ker \varphi$是有限生成的。
\end{proposition}
HINT:用蛇形引理。

仍然用$A^*$表示$A$的Pontrjagin对偶,则存在一个自然的映射$\sigma:A^* \otimes_R M \to \Hom_R(M,A)^*$
\begin{align}
   \sigma(f\otimes m)=h \mapsto f(h(m))
\end{align}
其中$f\in A^*,m \in M,h \in \Hom(M,A)$.我们的问题是,什么时候$\sigma$是一个同构?
\begin{theorem}{}
   对于任何有限展示的$M$,$\sigma$都是一个同构。
\end{theorem}
\begin{proof}
   若$M=R$,则自然有$\sigma$是同构。根据可加性,$M=\R^n$的时候也是如此。所以有:
   % https://q.uiver.app/#q=WzAsOCxbMCwwLCJBXipcXG90aW1lcyBSXm0iXSxbMCwxLCJcXEhvbShSXm0sQSleKiJdLFsxLDAsIkFeKlxcb3RpbWVzIFJebiJdLFsyLDAsIkFeKlxcb3RpbWVzIE0iXSxbMywwLCIwIl0sWzMsMSwiMCJdLFsyLDEsIlxcSG9tKE0sQSleKiJdLFsxLDEsIlxcSG9tKFJebixBKV4qIl0sWzAsMV0sWzAsMl0sWzIsM10sWzMsNF0sWzYsNV0sWzcsNl0sWzEsN10sWzIsN10sWzMsNl1d
\[\begin{tikzcd}
	{A^*\otimes R^m} & {A^*\otimes R^n} & {A^*\otimes M} & 0 \\
	{\Hom(R^m,A)^*} & {\Hom(R^n,A)^*} & {\Hom(M,A)^*} & 0
	\arrow[from=1-1, to=2-1]
	\arrow[from=1-1, to=1-2]
	\arrow[from=1-2, to=1-3]
	\arrow[from=1-3, to=1-4]
	\arrow[from=2-3, to=2-4]
	\arrow[from=2-2, to=2-3]
	\arrow[from=2-1, to=2-2]
	\arrow[from=1-2, to=2-2]
	\arrow[from=1-3, to=2-3]
\end{tikzcd}\]
    因为$\otimes$是右正合的,$\Hom$是左正合的,所以图中两个行正合.根据5引理\ref{5lemma}可知$\sigma$是同构。
\end{proof}
\begin{theorem}{}
   每个有限展示的平坦模是投射模。
\end{theorem}
\begin{proof}
   我们证明$\Hom(M,-)$是正合的。设$B\to C$是满射,则$C^* \to B^*$是单射。若$M$是平坦的,则:
   % https://q.uiver.app/#q=WzAsNCxbMCwwLCJDXipcXG90aW1lc19SIE0iXSxbMSwwLCJCXipcXG90aW1lcyBNIl0sWzAsMSwiXFxIb20oTSxDKV4qIl0sWzEsMSwiXFxIb20oTSxCKV4qIl0sWzAsMV0sWzAsMiwiXFxzaWdtYSJdLFsxLDMsIlxcc2lnbWEiLDJdLFsyLDNdXQ==
\[\begin{tikzcd}
	{C^*\otimes_R M} & {B^*\otimes M} \\
	{\Hom(M,C)^*} & {\Hom(M,B)^*}
	\arrow[from=1-1, to=1-2]
	\arrow["\sigma", from=1-1, to=2-1]
	\arrow["\sigma"', from=1-2, to=2-2]
	\arrow[from=2-1, to=2-2]
\end{tikzcd}\]
   给出了$\Hom(M,B)\to Hom(M,C)$的满射。所以$M$是投射模。
  \end{proof}  
   下面的引理来源于dimension shifting.
   \begin{lemma}[平坦解消引理]{}
      群$\mathrm{Tor}_*(A,B)$可以用平坦模进行计算。
   \end{lemma}
\begin{proposition}[Tor的平坦基变换]
    设$R \to T$是环同态,使得$T$成为了$R$模。从而对于所有的$R$模$A$,所有的$T$模$C$和所有的$n$:
    \begin{align}
      \mathrm{Tor}_n^R(A,C)\cong \mathrm{Tor}_n^T(A \otimes_R T,C)
    \end{align}
\end{proposition}
\begin{proof}
   选择$R$模的投射解消$P \to A$,则$\mathrm{Tor}_*^R(A,C)$是$P \otimes_R C$的同调。

   因为$T$是平坦的$R$模,所以$P_n\otimes T$是投射的$T$模且$P\otimes T \to A \otimes T$是$T$模的投射解消。所以$\mathrm{Tor}_n^T(A \otimes T,C)$是复形$(P\otimes_R T)\otimes_T C \cong P\otimes_R C$的同调。
\end{proof}
\begin{corollary}{}
   若$R$是交换环,$T$是平坦的$R$代数,则对于所有的$R$模$A,B$和所有的$n$:
   \begin{align}
      T\otimes_R \mathrm{Tor}_n^R(A,B)\cong \mathrm{Tor}_n^T(A\otimes_R T,T\otimes_R B)
   \end{align}
\end{corollary}
\begin{proof}
   设$C=T\otimes_R B$.根据上面的命题,我们只需要证明$\mathrm{Tor}_*^R(A,T\otimes B)=T\otimes \mathrm{Tor}_*^R(A,B)$.因为$T\otimes_R$是正合函子,所以$T\otimes \mathrm{Tor}_*^R(A,B)$是$T\otimes_R (P\otimes _R B)$的同调,从而为$\mathrm{Tor}_*^R(A,T\otimes B)$.
\end{proof}
为了使得$\mathrm{Tor}$给出模结构,我们必须假设$R$是交换环。原因是下面的引理:

\begin{lemma}{}
   设$\mu:A \to A$是左乘一个中心元$r$。则诱导的$\mu_*:\mathrm{Tor}_n^R(A,B)\to \mathrm{Tor}_n^R(A,B)$也是左乘$r$.
\end{lemma}
\begin{proof}
   选择$A$的投射解消$P \to A$。左乘$r$是一个$R$模的链复形映射$\tilde{\mu}:P \to P$.(因为$r$是一个中心元)。从而$\tilde{mu}\otimes B$是$P\otimes B$的$r$左乘。作为商群$\mathrm{Tor}$也是如此。
\end{proof}
\begin{corollary}{}
   若$A$是一个$R/r$模,则对于每个$R$模$B$,$R$模$\mathrm{Tor}_*^R(A,B)$也是$R/r$模。换句话说,$rR$乘在该模得$0$.
\end{corollary}
\begin{corollary}[Tor的局部化]{}
   若$R$是一个交换环且$A,B$都是$R$模。下面的命题对于所有$n$都成立:
   \begin{enumerate}
      \item $\mathrm{Tor}_n^R(A,B)=0$
      \item 对于$R$的任意素理想$p$,$\mathrm{Tor}_n^{R_p}(A_p,B_p)=0$
      \item 对于$R$的任意极大理想$m$,$\mathrm{Tor}_n^{R_m}(A_m,B_m)=0$.
   \end{enumerate}
\end{corollary}
\begin{proof}
   对于$R$模而言,$M=0$等价于任意素理想$p$,$M_p=0$等价于任意极大理想$m$,$M_m=0=0$.设$M=\mathrm{Tor}(A,B)$:
   \begin{align}
      M_p=R_p \otimes_R M=\mathrm{Tor}_n^{R_p}(A_p,B_p)
   \end{align}
\end{proof}
\section{性质较好的环的Ext函子}
讨论了Ext后,我们讨论Ext函子的性质。首先我们计算一些性质很好的环的Ext函子。

\begin{lemma}{}
   $\mathrm{Ext}_\Z^n(A,B)=0$,$\forall n \geq 2$和所有的交换群$A,B$.
\end{lemma}
\begin{proof}
   把$B$嵌入到一个内射的交换群$I^0$.其商群$I^1$是可除的,因而是内射的,所以我们给出了$B$的内射解消$0 \to B \to I^0 \to I^1 \to 0$.

   所以$\mathrm{Ext}^*(A,B)$可以计算为:
   \begin{align}
      0 \to \Hom(A,I^0) \to \Hom(A,I^1) \to 0
   \end{align}
   的上同调。
\end{proof}
因此我们只需要考虑$n=1$的情况。
\begin{example}{}
   $\mathrm{Ext}_{\Z}^0(\Z/p,B)={}_p B$.$\mathrm{Ext}_\Z^1(\Z/p,B)=B/pB$.

   可以使用$0 \to \Z \to \Z \to \Z/p$作为$\Z/p$的投射解消计算。
\end{example}

因为$\Z$是投射模,所以$\Ext^1(\Z,B)=0$对于任何$B$总是成立。我们可以依据这个结果和上述结果,在$A$是有限生成的Abel群时计算$\Ext(A,B)$:
\begin{align}
   A\cong \Z^m \oplus \Z/p  \Rightarrow \Ext(A,B)=\Ext(\Z/p,B)
\end{align}
然而无限生成的情况因为余极限不交换,要复杂得多。
\begin{example}[$B=\Z$]{}
   设$A$是一个挠群,用$A^*$表示Pontrjagin对偶。$\Z$有经典的内射解消:$0 \to \Z \to \Q \to \Q/\Z \to 0$。用这个解消计算$\Ext^*(A,\Z)$:
   \begin{align}
      0 \to \Hom(A,\Q) \to \Hom(A,\Q/\Z) \to 0 
   \end{align}
   从而$\Ext_\Z^0(A,\Z)=\Hom(A,\Z)=0$,$\Ext_\Z^1(A,\Z)=A^*$。

   为了对这个例子有更深的印象,注意到$\Z_{p^\infty}$是$\Z/p^n$的余极限(并).于是可以计算:
   \begin{align}
      \Ext_\Z^1(\Z_{p^\infty},\Z)=(\Z_{p^\infty})^*
   \end{align}
   这个群是$p$-adic整数的无挠群,$\hat{\Z}_p=\Lim (\Z/p^n)$。

   再考虑一个例子:$A=\Z[1/p],B=\Z$.此时:
   \begin{align}
      0 \to \Q=\Hom(\Z[1/p],\Q) \to \Hom(\Z[1/p],\Q/\Z) \to 0
   \end{align}
   $\Ext^0$比较容易,我们考虑$\Ext^1$.此时给定$f \in \Hom(\Z[1/p],\Q/\Z)$,筛出掉$\Hom(\Z[1/p],\Q)$的元素,本质上留存的是一个$p$-adic数。并且若两个$p$-adic数只差一个整数,与他们给出的$f$是一致的。因此$\Ext^1(\Z[1/p],\Z)=\Z_{p^\infty}$。

   这说明$\Ext$对于平坦模而言也不是vanish的。
\end{example}
\begin{example}[$R=\Z/m$,$B=\Z/p$]{}
   $\Z/p$在这种情况下有无穷的周期内射解消:
   \begin{align}
      0 \to \Z/p \xrightarrow{\iota} \Z/m \xrightarrow{p}  \Z/m \xrightarrow{m/p} \Z/m \xrightarrow{p} \dots 
   \end{align}

   于是$\Ext_{\Z/m}^n(A,\Z/p)$可以计算为:
   \begin{align}
      0 \to \Hom(A,\Z/m) \to \Hom(A,\Z/m) \to \Hom(A,\Z/m) \dots
   \end{align}
   的上同调。

   比如,若$p^2|m$,则$\Ext_{\Z/m}^n(\Z/p,\Z/p)=\Z/p$
\end{example}
\begin{proposition}{}
   对于所有的$n$和$R$:
   \begin{enumerate}
      \item $\Ext_R^n(\bigoplus_\alpha A,B)\cong \prod_{\alpha}\Ext_R^n(A_\alpha,B)$
      \item $\Ext_R^n(A,\prod_\beta B) \cong \prod_\beta \Ext_R^n(A,B_\beta)$
   \end{enumerate}
\end{proposition}
\begin{proof}
   设$P_\alpha$是$A_\alpha$的投射解消。于是$\oplus P_\alpha$是$\oplus A_\alpha$的投射解消。同理,$Q_\beta$是$B_\beta$的内射解消,则$\prod Q_\beta$是$\prod B_\beta$的内射解消。

   根据$\Hom$的性质,再加上:
   \begin{align}
      H^*(\prod C_\gamma)\cong \prod H^*(C_\gamma)
   \end{align}
   可得结果。
\end{proof}
\begin{lemma}{}
   设$R$是交换环,则$\Hom_R(A,B)$和$\Ext^*(A,B)$都是$R$模。若$\mu,\tau$分别是$r$的左乘($A,B$),则诱导的$\mu^*$和$\tau^*$也是左乘。
\end{lemma}
可以看到,这是Tor函子的相似版本,可用于给出Ext与局部化交换的性质。
\begin{proof}
   给$P \to A$投射解消.左乘$r$给出了$\tilde{mu}:P \to P$作为链复形映射。映射$\Hom(\tilde{mu},B)$是$\Hom(P,B)$上链复形,是左乘$r$.

   因此商群$\Ext^n(A,B)$被$\mu^*$作用也是$r$左乘。
\end{proof}
\begin{corollary}{}
   设$R$是交换环,$A$是$R/r$模。则对于$R$模$B$,$\Ext^*_R(A,B)$是$R/r$模。
\end{corollary}
接下来的引理,定理我们不写证明,读者可自查Weibel原书。

考虑$S^{-1}\Hom_R(A,B)$.其到$\Hom_{S^{-1}R}(S^{-1}A,S^{-1}B)$有一个自然的态射$\Phi$。但这个态射一般不是同构。
\begin{lemma}{}
   如果$A$是有限展示的$R$模,则对于每个中心可乘集合$S$,$\Phi$是同构。
\end{lemma}
不难想象证明用到的是5引理\ref{5lemma}。

\begin{proposition}{}
   设$A$是交换Noether环上的有限生成模.则$\Phi$也诱导了Ext的同构:
   \begin{align}
      \Phi:S^{-1}\Ext_R^n(A,B) \cong \Ext_{S^{-1}R}^n(S^{-1}A,S^{-1}B)
   \end{align}
\end{proposition}
不难想到证明的思路是给$A$的投射解消。因为$S^{-1}$是正合函子,所以保$H^*$。因此用$\Hom$的同构性即可给出上述同构。

\begin{corollary}[Ext的局部化]{Ext-loc}
   设$R$是交换Noether环且$A$是有限生成$R$模.则下面的命题之间对于任意$B$和$n$都等价:
   \begin{enumerate}
      \item $\Ext_R^n(A,B)=0$
      \item 对于$R$的任何素理想$p$,$\Ext_{R_p}^n(A_p,B_p)=0$
      \item 对于$R$的任何极大理想$m$,$\Ext_{R_m}^n(A_m,B_m)=0$.
   \end{enumerate}
\end{corollary}
\section{Ext函子与扩张}
我们在这一节探讨Ext到底计算了什么。为此需要介绍扩张的概念。
\begin{definition}{extension}
   一个$A$过$B$的扩张$\xi$是指一个正合列$0 \to B \to X \to A \to 0$.称两个扩张$\xi,\xi'$是等价的,若存在交换图:
   % https://q.uiver.app/#q=WzAsMTAsWzAsMCwiMCJdLFsxLDAsIkEiXSxbMiwwLCJYIl0sWzMsMCwiQiJdLFs0LDAsIjAiXSxbMSwxLCJBIl0sWzIsMSwiWCciXSxbMywxLCJCIl0sWzQsMSwiMCJdLFswLDEsIjAiXSxbMCwxXSxbMSwyXSxbMiwzXSxbMyw0XSxbNSw2XSxbNiw3XSxbNyw4XSxbOSw1XSxbMSw1LCJcXGlkIiwxXSxbMyw3LCJcXGlkIiwxXSxbMiw2LCJcXGNvbmciXV0=
\[\begin{tikzcd}
	0 & A & X & B & 0 \\
	0 & A & {X'} & B & 0
	\arrow[from=1-1, to=1-2]
	\arrow[from=1-2, to=1-3]
	\arrow[from=1-3, to=1-4]
	\arrow[from=1-4, to=1-5]
	\arrow[from=2-2, to=2-3]
	\arrow[from=2-3, to=2-4]
	\arrow[from=2-4, to=2-5]
	\arrow[from=2-1, to=2-2]
	\arrow["\id"{description}, from=1-2, to=2-2]
	\arrow["\id"{description}, from=1-4, to=2-4]
	\arrow["\cong", from=1-3, to=2-3]
\end{tikzcd}\]

   一个扩张是分裂的,若其等价于$0 \to B \to A \oplus B \to 0$(典范的)。
\end{definition}
\begin{example}{}
   若$p$是素数,则仅存在$p$个等价的$\Z/p$过$\Z/p$的扩张。分别是分裂扩张和:
   \begin{align}
      0 \to \Z/p \xrightarrow{p} \Z/p^2 \xrightarrow{i}\Z/p \to 0, i=1,2,\dots,p-1
   \end{align}

   实际上$X$必须是$p^2$阶交换群。若$X$无$p^2$阶元,则根据$X=\Z/p\oplus \Z/p$。若$X$有$p^2$阶元,设该元为$b$。则$pb \in \Z/p=B$。于是有上述$p-1$种投射。
\end{example}
\begin{lemma}{}
   若$\Ext^1(A,B)=0$,则$A$过$B$的扩张总是分裂的。
\end{lemma}
\begin{proof}
   给定一个扩张$\xi$,根据$\Ext^*(A,-)$诱导的长正合列:
   \begin{align}
      \Hom(A,X) \to \Hom(A,A) \xrightarrow{\partial}\Ext^1(A,B)=0
   \end{align}
   所以$\id_A$有原像$\sigma:A \to X$。这就是一个$X \to A$的截面。所以$X=A \oplus B$分裂。
\end{proof}
如果$\Ext^1(A,B)$非$0$,为了给出截面,实际上可以计算$\partial(\id_A)=0$。我们把这个构造记作$\Theta(\xi)$.另外.如果两个扩张等价,那么他们的$\Theta(\xi)$相同.因此这个构造只依赖于$\xi$的等价类。

\begin{theorem}{}
   给定两个模$A,B$,映射$\Theta:\xi \mapsto \partial(\id_A)$给出了一个一一映射:
   \begin{align}
      \{\text{A过B的扩张的等价类}\} \to \Ext^1(A,B)
   \end{align}
\end{theorem}
因此这个定理给出了$\Ext^1(A,B)$的一个初步作用:确定$A$过$B$的扩张个数,并赋予一个群结构。
\begin{proof}
   对于$B$,固定一个正合列$0 \to B \to I \xrightarrow{\pi} N \to 0$.其中$I$内射。作用$\Hom(A,-)$,导出一个正合列:
   \begin{align}
      \Hom(A,I) \to \Hom(A,N) \xrightarrow{\partial} \Ext^1(A,B) \to 0
   \end{align}

   现在给定一个$x \in \Ext^1(A,B)$,选定$\beta \in \Hom(A,N)$使得$\partial(\beta)=x$.根据$\beta:A \to N$和$I \to N$,可以写出拉回$X$:
   % https://q.uiver.app/#q=WzAsMTAsWzAsMCwiMCJdLFsxLDAsIk0iXSxbMiwwLCJQIl0sWzMsMCwiQSJdLFs0LDAsIlxcYnVsbGV0Il0sWzAsMSwiMCJdLFsxLDEsIkIiXSxbMiwxLCJYIl0sWzMsMSwiQSJdLFs0LDEsIlxcYnVsbGV0Il0sWzAsMV0sWzIsM10sWzMsNF0sWzUsNl0sWzYsN10sWzcsOF0sWzgsOV0sWzEsNiwiXFxiZXRhIl0sWzIsN10sWzMsOCwiPSJdLFsxLDIsImoiXV0=
\[\begin{tikzcd}
	0 & B & X & A & 0 \\
	0 & B & I & N & 0
	\arrow[from=1-1, to=1-2]
	\arrow[from=1-3, to=1-4]
	\arrow[from=1-4, to=1-5]
	\arrow[from=2-1, to=2-2]
	\arrow[from=2-2, to=2-3]
	\arrow[from=2-3, to=2-4]
	\arrow[from=2-4, to=2-5]
	\arrow["{=}", from=1-2, to=2-2]
	\arrow[from=1-3, to=2-3]
	\arrow["\beta", from=1-4, to=2-4]
	\arrow[from=1-2, to=1-3]
\end{tikzcd}\]
这不仅是拉回,而且可以验证$0 \to B \to X \to A \to 0$是一个正合列。根据连接同态$\partial$的自然性,可以得到:
% https://q.uiver.app/#q=WzAsNCxbMCwwLCJcXEhvbShBLEEpIl0sWzEsMCwiXFxFeHReMShBLE0pIl0sWzAsMSwiXFxIb20oQSxBKSJdLFsxLDEsIlxcRXh0XjEoQSxCKSJdLFswLDFdLFswLDJdLFsyLDNdLFsxLDNdXQ==
\[\begin{tikzcd}
	{\Hom(A,A)} & {\Ext^1(A,B)} \\
	{\Hom(A,N)} & {\Ext^1(A,B)}
	\arrow[from=1-1, to=1-2]
	\arrow[from=1-1, to=2-1]
	\arrow[from=2-1, to=2-2]
	\arrow[from=1-2, to=2-2]
\end{tikzcd}\]
令上面的扩张是$\xi$,则$\Theta(\xi)=x$。于是我们通过给定$x\in \Ext^1(A,B)$给出一个扩张$\xi$使得$\Theta(\xi)=x$。

为了给出$\Ext^1(A,B)$到等价类的映射,我们还需要说明上述过程$\beta$的选取不改变$\xi$的等价类。实际上选取$\beta'\in \Hom(A,N)$使得$\partial{\beta'}=x$。于是$\beta'-\beta=\pi_*(\alpha),\alpha\in \Hom(A,I)$.于是可以绘制出下面的交换图:

% https://q.uiver.app/#q=WzAsNSxbMCwwLCJYIl0sWzEsMSwiWCciXSxbMiwxLCJBIl0sWzEsMiwiSSJdLFsyLDIsIk4iXSxbMCwxLCIiLDEseyJzdHlsZSI6eyJib2R5Ijp7Im5hbWUiOiJkYXNoZWQifX19XSxbMSwyLCJcXHNpZ21hJyIsMl0sWzAsMiwiXFxzaWdtYSIsMV0sWzEsMywicCciXSxbMyw0LCJcXHBpIiwyXSxbMiw0LCJcXGJldGEnIl0sWzAsMywicCtcXGFscGhhXFxjaXJjXFxzaWdtYSIsMl1d
\[\begin{tikzcd}
	X \\
	& {X'} & A \\
	& I & N
	\arrow[dashed, from=1-1, to=2-2]
	\arrow["{\sigma'}"', from=2-2, to=2-3]
	\arrow["\sigma"{description}, from=1-1, to=2-3]
	\arrow["{p'}", from=2-2, to=3-2]
	\arrow["\pi"', from=3-2, to=3-3]
	\arrow["{\beta'}", from=2-3, to=3-3]
	\arrow["{p+\alpha\circ\sigma}"', from=1-1, to=3-2]
\end{tikzcd}\](交换性已经在草稿纸上验证了)
根据拉回的泛性质,$X$到$X'$有一个态射.

通过具体到集合的验证,可以说明这是一个同构。所以$X$和$X'$是等价的扩张。

另一方面,给定$\xi$作为$A$过$B$的扩张,$I$的延拓性质表明存在一个$\tau:X \to I$满足:
% https://q.uiver.app/#q=WzAsMTAsWzAsMCwiMCJdLFsxLDAsIkIiXSxbMiwwLCJYIl0sWzMsMCwiQSJdLFs0LDAsIjAiXSxbMCwxLCIwIl0sWzEsMSwiQiJdLFsyLDEsIkkiXSxbMywxLCJOIl0sWzQsMSwiMCJdLFswLDFdLFsxLDJdLFsyLDNdLFszLDRdLFs1LDZdLFs2LDddLFs3LDhdLFs4LDldLFsyLDcsIlxcdGF1Il0sWzEsNiwiPSJdLFszLDgsIlxcYmV0YSIsMV1d
\[\begin{tikzcd}
	0 & B & X & A & 0 \\
	0 & B & I & N & 0
	\arrow[from=1-1, to=1-2]
	\arrow[from=1-2, to=1-3]
	\arrow[from=1-3, to=1-4]
	\arrow[from=1-4, to=1-5]
	\arrow[from=2-1, to=2-2]
	\arrow[from=2-2, to=2-3]
	\arrow[from=2-3, to=2-4]
	\arrow[from=2-4, to=2-5]
	\arrow["\tau", from=1-3, to=2-3]
	\arrow["{=}", from=1-2, to=2-2]
	\arrow["\beta"{description}, from=1-4, to=2-4]
\end{tikzcd}\]

其中$\beta$是$\tau$诱导的态射。我们断言$X$是$\beta$和$\pi:I \to N$的拉回。从而$\Psi(\Theta(\xi))=\xi$.
\end{proof}
如果我们可以给出扩张的运算,就能更好的理解上述的对应。
\begin{definition}[Baer和]{Baer-sum}
   设$\xi$和$\xi'$分别是$A$过$B$的两个扩张。设$X''$是$X \to A$和$X'  \to A$的拉回。则$X''$包含了三份$B$:$B \times 0,0 \times B,\{(-b,b):b\in B\}$。

   作$X''$对于对角线$B$的商运算,则$B \times 0$和$0 \times B$被对应为一个子群。而$X''/0\times B\cong X$和$X/B=A$,则我们得到正合列:
   \begin{align}
      \varphi: 0\to B \to Y \to A\to 0
   \end{align}
   $\varphi$的等价类被称为$\xi$和$\xi'$的Baer和。
\end{definition}
\begin{proposition}{Baer-sum-pro}
   扩张等价类的集合在Baer和的意义下生成了一个交换群,分裂扩张是该和的幺元。从而$\Theta$给出了一个群同构。
\end{proposition}
\begin{proof}
   我们说明$\Theta(\varphi)=\Theta(\xi)+\Theta(\xi')$.这说明了Baer和的良定性,也给出了命题成立。

   固定$0\to M \to P \to A\to 0$是一个正合列,且$P$是投射模。因为$P$投射,所以给出$\tau:P \to X$和$\tau':P\to X'$。
   
   接下来设$\tau'': P\to X''$是由$\tau:P \to X$和$\tau': P \to X'$诱导而来的态射。而设$\bar{\tau}:P \to Y$是诱导的态射。

   我们断言$\bar{\tau}$限制在$M$上由映射$\gamma+\gamma':M \to B$诱导。所以下面的交换图:
   % https://q.uiver.app/#q=WzAsMTAsWzAsMCwiMCJdLFsxLDAsIk0iXSxbMiwwLCJQIl0sWzMsMCwiQSJdLFs0LDAsIjAiXSxbMCwxLCIwIl0sWzQsMSwiMCJdLFsyLDEsIlkiXSxbMywxLCJBIl0sWzEsMSwiQiJdLFswLDFdLFszLDgsIj0iXSxbMyw0XSxbOCw2XSxbMiwzXSxbMSwyXSxbMiw3LCJcXGJhcntcXHRhdX0iXSxbNyw4XSxbOSw3XSxbMSw5LCJcXGdhbW1hK1xcZ2FtbWEnIl0sWzUsOV1d
\[\begin{tikzcd}
	0 & M & P & A & 0 \\
	0 & B & Y & A & 0
	\arrow[from=1-1, to=1-2]
	\arrow["{=}", from=1-4, to=2-4]
	\arrow[from=1-4, to=1-5]
	\arrow[from=2-4, to=2-5]
	\arrow[from=1-3, to=1-4]
	\arrow[from=1-2, to=1-3]
	\arrow["{\bar{\tau}}", from=1-3, to=2-3]
	\arrow[from=2-3, to=2-4]
	\arrow[from=2-2, to=2-3]
	\arrow["{\gamma+\gamma'}", from=1-2, to=2-2]
	\arrow[from=2-1, to=2-2]
\end{tikzcd}\]
成立。

因此我们有$\Theta(\varphi)=\partial(\gamma+\gamma')$.然而$\partial(\gamma+\gamma')=\partial(\gamma)+\partial(\gamma')=\Theta(\xi)+\Theta(\xi')$.所以命题成立。
\end{proof}

借助上述的命题,我们实际上可以思考这样的问题:如果一个Abelian范畴没有足够的投射模和内射模,我们也可以借助扩张生成的交换群来定义$\Ext^1$.当然这里的交换群仍需要证明。

相似的,我们也可以思考$\Ext^n$的含义。我们在这里建议大家阅读原书的79页到80页内容。
\section{逆向极限的导出函子}
设$I$是一个小范畴(即对象集和态射集都是集合)。$\mathcal{A}$是一个Abelian范畴。在第二章,我们说明了$\mathcal{A}^I$有足够多的内射对象。(至少是$A$完备且有足够多内射对象的时候)。另外,容易验证逆向极限是左正合函子(保核)。

因此我们可以定义从$\mathcal{A}^I$到$\mathcal{A}$的右导出函子$R^n\Lim_{i\in I}$。

我们在这一节关注$\mathcal{A}$是Ab且$I$是$\dots\to 2\to 1 \to 0$。我们把$\mathrm{Ab}^I$中的元素称作交换群的“塔”。他们的具体形式是:
\begin{align}
   \{A_i\}:\dots \to A_2\to A_1 \to A_0
\end{align}
这一节我们具体给出$\lim^1$的具体构造,并且证明$R^n\Lim=0,n\neq 0,1$。

我们自然想问这样的构造是否可以拓展为其他的Abelian范畴。Grothendieck告诉我们,在满足下面公理的情况下该范畴可以:

(AB$4^*$):$\mathcal{A}$是完备的,且任何集合的满射的乘积都是满射。

满足该公理的范畴大多是有underlying集合的范畴(交换群,模范畴,链复形范畴),但是在层范畴失效。

\begin{definition}{}
   给定Ab中的一个塔$\{A_i\}$。定义映射:
   \begin{align}
      \Delta:\prod_{i=0}^\infty \to \prod_{i=0}^\infty A_i
   \end{align}
   为:
   \begin{align}
      \Delta(\dots,a_i,\dots,a_0)=(\dots,a_i-\bar{a}_{i+1},\dots,a_1-\bar{a}_2,a_0-\bar{a}_1)
   \end{align}
   其中$\bar{a}_{i+1}$代表$a_{i+1}\in A_{i+1}$在$A_i$中的项。
   
   容易看出$\Delta$的$\ker$是$\Lim A_i$.我们定义$\Lim^1 A_i$是$\Delta$的余核,从而$\Lim^1$是从$\mathrm{Ab}^I$到$\mathrm{Ab}$的函子。我们定义$\Lim^0 A_i=\Lim A_i$,$\Lim^n A_i=0,n\geq 2$.
\end{definition}
上述定义给出了具体的构造。当然我们需要说明这是符合要求的函子。
\begin{lemma}{}
   函子$\{\Lim^n\}$给出了一个上同调$\delta$函子。
\end{lemma}
\begin{proof}
   设$0 \to \{A_i\} \to \{B_i\}\to \{C_i\} \to 0$是塔的一个短正合列。用蛇形引理:
   % https://q.uiver.app/#q=WzAsMTAsWzAsMCwiMCJdLFsxLDAsIlxccHJvZCBBX2kiXSxbMiwwLCJcXHByb2QgQl9pIl0sWzMsMCwiXFxwcm9kIENfaSJdLFs0LDAsIjAiXSxbMSwxLCJcXHByb2QgQV9pIl0sWzIsMSwiXFxwcm9kIEJfaSJdLFszLDEsIlxccHJvZCBDX2kiXSxbMCwxLCIwIl0sWzQsMSwiMCJdLFswLDFdLFsxLDJdLFsyLDNdLFszLDRdLFsxLDUsIlxcRGVsdGEiXSxbNSw2XSxbNiw3XSxbMyw3LCJcXERlbHRhIl0sWzIsNiwiXFxEZWx0YSJdLFs4LDVdLFs3LDldXQ==
\[\begin{tikzcd}
	0 & {\prod A_i} & {\prod B_i} & {\prod C_i} & 0 \\
	0 & {\prod A_i} & {\prod B_i} & {\prod C_i} & 0
	\arrow[from=1-1, to=1-2]
	\arrow[from=1-2, to=1-3]
	\arrow[from=1-3, to=1-4]
	\arrow[from=1-4, to=1-5]
	\arrow["\Delta", from=1-2, to=2-2]
	\arrow[from=2-2, to=2-3]
	\arrow[from=2-3, to=2-4]
	\arrow["\Delta", from=1-4, to=2-4]
	\arrow["\Delta", from=1-3, to=2-3]
	\arrow[from=2-1, to=2-2]
	\arrow[from=2-4, to=2-5]
\end{tikzcd}\]

就可以得到我们想要的自然长正合列。
\end{proof}
\begin{lemma}{}
   若所有的$A_{i+1}  \to A_i$都是满射,则$\Lim^1 A_i=0$.更多的,$\Lim A_i\neq 0$(除非每个$A_i$都是$0$),因为每个自然投射$\Lim A_i \to A_j$都是满射。
\end{lemma}
\begin{proof}
   给定$b_i \in A_i(i=0,\dots,n)$,以及任何$a_0\in A_0$。归纳的选择$a_{i+1}\in A_{i+1}$:使得$a_{i+1}$是$a_i-b_i  \in A_i$在$A_{i+1}$中的提升。

   从而$\Delta$将$(\dots,a_1,a_0)$映射到$(\dots,b_1,b_0)$.因此这种情况下$\Delta$是满射,$\Lim^1 A_i=0$。如果$b_i=0$,$(\dots,a_1,a_0)\in \Lim A_i$.
\end{proof}
\begin{corollary}{}
   $\Lim^1 A_i\cong (R^1\Lim)(A_i)$且$R^n \Lim=0,\forall n\neq 0,1$
\end{corollary}
\begin{proof}
   我们说明$\Lim^n$形成了一个泛$\delta$函子,从而根据泛性说明上述成立。我们只需要说明$\Lim^1$在足够多的内射对象(应付内射解消)上vanish。

   我们在第二章给出了足够多的内射对象:
   \begin{align}
      k_*E:\dots=E=E \to 0 \to 0 \dots\to 0
   \end{align}
   其中$E$内射。因此这里面所有的态射都是满射,因此$\Lim^1$在这些内射塔上都vanish。
\end{proof}
上述的证明在AB4*的情况下总是对的。我们给出反例(不满足AB4*)。
\begin{example}{}
   设$A_0=\Z$且$A_i=p^i\Z$是$p^i$生成的子群。对短正合列($p$是素数):
   \begin{align}
      0\to \{p^i\Z\}  \to \{\Z\} \to \{\Z/p^i\Z\}  \to 0
   \end{align}
   使用$\Lim$.

   从而$\Lim^1\{p^i\Z\}\cong\hat{\Z}_p/\Z$.

\end{example}
下面这个命题在原书上是习题。我们仅作记录,证明省略。(可以查找mathstackexchange)。

\begin{proposition}{}
   设$\{A_i\}$是一个塔,$A_{i+1}\to A_i$是包含映射。把$A=A_0$看作拓扑群,其中$a+A_i(a\in A,i\geq 0)$是开集。

   则$\Lim A_i=\cap A_i=0$当且仅当$A$是Hausdorff的.$\Lim^1 A_i=0$当且仅当$A$在下列意义是完备的:每个柯西列都有不一定唯一的极限点.
\end{proposition}
提示:证明$A$是完备的,当且仅当$A\cong \Lim(A/A_i)$
\begin{definition}{}
   我们称一个塔$\{A_i\}$满足Mittag-Leffler条件,若对于每个$k$都存在一个$j\geq k$使得$A_i \to A_k$的像等于$A_j\to A_k$,对于任意$i \geq j$成立。(即$A_i$在$A_k$的像满足降链条件)。

   例如,若$\{A_i\}$都是满射,该塔就满足M-L条件。

   有一种平凡的情况:若对于每个$k$都存在一个$j\geq k$使得$A_i \to A_k$的像是$0$,我们称该塔满足平凡M-L条件。
\end{definition}
\begin{proposition}{}
   若$A_i$满足M-L条件,则:$\Lim^1 A_i=0$
\end{proposition}
\begin{corollary}{}
   设$\{A_i\}$是有限Abel群的塔,或者是有限维向量空间上的塔,我们都有$\Lim^1 A_i=0$
\end{corollary}
下面的定理预示了下一节的泛系数定理。
\begin{theorem}{}
   设$\dots \to C_1 \to C_0$是Ab的链复形的塔链。(每个$C_i$都是链复形),且满足ML条件。设$C=\Colim C_i$。则对于每个$q$都存在一个正合列:
   \begin{align}
      0 \to \textstyle\Lim^1 H_{q+1}(C_i) \to H_q(C) \to \Lim H_q(C_i) \to 0
   \end{align}


若$\dots C_1\to C_0 \to 0$是上链复形的塔链且满足ML条件。则:
\begin{align}
   0 \to \textstyle\Lim^1 H^{q-1}(C_i) \to H^q(C) \to \Lim H^q(C_i) \to 0
\end{align}
正合。
\end{theorem}
在拓扑上,这个定理有一个类似的版本。考虑$X$是CW复形,而$X_i$是$X$的上升子复形链,使得$X=\cup X_i$.则存在一个正合列:
\begin{align}
   0 \to \textstyle\Lim^1 H^{q-1}(X_i) \to H^q(X) \to \Lim H^q(X_i) \to 0
\end{align}
可以一眼看出这个公式的便利之处:可以根据子群的同调群计算最大的群的同调群。
\begin{example}{}
   设$A$是$R$模且是子模$\dots \subset A_i \subset A_{i+1}\subset \dots$的并,则对于任何$R$模$B$和$q$,都存在列:
   \begin{align}
      0 \to \textstyle\Lim^1 \Ext_R^{q-1}(A_i,B) \to \Ext_R^q(A,B) \to \Lim \Ext_R^q(A_i,B)\to 0
   \end{align}
   是正合的。

   对于$\Z_{p^\infty}=\cup \Z/p^i$,上述列化为:
   \begin{align}
      0 \to \textstyle\Lim^1 \Hom(\Z/p^i,B) \to \Ext_R^1(\Z_{p^\infty},B) \to \Lim \Ext_R^1(\Z/p^i,B)=\hat{B}_p \to 0
   \end{align}
   其中$\hat{B}_p=\Lim(B/p^iB)$是$B$的$p$-adic的完备化。
   
   这相当于推广了计算:$\Ext^1_\Z(\Z_{p^\infty},\Z)\cong \hat{\Z}_p$.实际上,设$E$是一个不变的$B$内射解消,考虑上链复形的塔链:
   \begin{align}
      \Hom(A_{i+1},E) \to \Hom(A_i,E) \to \dots \Hom(A_0,E) 
   \end{align}
   因为每个$\Hom(-,E_n)$都是反变正合的,所以塔链中每一个映射都是满射。(单反过来就是满).而$\Hom(A_i,E)$的上同调是$\Ext^*(A_i,B)$,$\Ext^*(A,B)$是:
   \begin{align}
      \Hom(\cup A_i,E)=\Lim \Hom(A_i,E)
   \end{align}
   的上同调。
\end{example}
\begin{corollary}{}
   $Z[1/p]=\cup p^{-1}\Z$,从而$\Ext^1_\Z(\Z[1/p],\Z)\cong \hat{\Z}_p/\Z$.从而对于无挠群$B$,有$\Ext_\Z^1(\Q,B)=(\prod_p \hat{B}_p)/B$.
\end{corollary}

\section{泛系数定理}
这一节我们思考的问题是,在已知$P$的同调下,如何计算$P\otimes M$的同调。由于在拓扑中,有所谓$\Z$系数,$\R$系数,$R$系数的说法,所以我们实际上在思考不同系数情况下一个拓扑空间同调和上同调群的关系。因而这节的名字是泛系数定理。
\begin{theorem}[Kunneth公式]{Kunneth-formula}
   设$P$是由平坦右$R$模给出的链复形,且$d(P_n)$作为$P_{n-1}$的子模总是平坦的。则对于任何$n$和任何左模$M$,都存在正合列:
   \begin{align}
      0 \to H_n(P)\otimes_R M \to H_n(P\otimes_R M) \to \mathrm{Tor}_1^R(H_{n-1}(P),M)\to 0
   \end{align}
\end{theorem}
\begin{proof}
   考虑短正合列:
   \begin{align}
      0 \to Z_n \to P_n \to d(P_n)\to 0
   \end{align}
   对此使用$\Tor$函子,可以得知$Z_n$也是平坦模。考虑到$\Tor_1(d(P_n),M)=0$,则:
   \begin{align}
      0 \to Z_n \otimes M \to P_n\otimes M \to d(P_n)\otimes M \to 0
   \end{align}
   是正合的。从而我们给出了链复形的短正合列:
   \begin{align}
      0 \to Z\otimes M \to P\otimes M \to d(P)\otimes M \to 0
   \end{align}
   注意到$Z$和$d(P)$中的微分算子都是$0$,从而短正合列导引的长正合列为:
   \begin{align}
      H_{n+1}(dP\otimes M) \xrightarrow{\partial} H_n(Z\otimes M) \to H_n(P \otimes M) \to H_n(dP \otimes M) \xrightarrow{\partial}H_{n-1}(Z\otimes M)
   \end{align}
   其中$H_n(dP_n\otimes M)=dP_n \otimes M$,$H_n(Z_n\otimes M)=Z_n\otimes M$.

   设$i:d(P_{n+1}) \to Z_n$是包含映射。我们断言$\partial$实际上是$i\otimes M$。(实际上很容易给出)。另一方面,$0 \to d(P_{n+1}) \to Z_n \to H_n(P) \to 0$是$H_n(P)$的平坦解消,所以$\Tor_1(H_n(P),M)$可以使用:
   \begin{align}
      0 \to d(P_{n+1})\otimes Z_n\otimes M \to 0
   \end{align}
   计算。结合长正合列即可得到结果。
\end{proof}
\begin{theorem}[同调的泛系数定理]{homo-universal}
   设$P$是一个自由Abel群的链复形。则对于任意的$n$和每个交换群$M$而言,定理\ref{Kunneth-formula}中的正合列分裂。但是这个分裂并不典范。
   \begin{align}
      H_n(P\otimes M)\cong H_n(P)\otimes M \oplus \Tor_1^\Z(H_{n-1}(P),M)
   \end{align}
\end{theorem}
\begin{proof}
   众所周知,自由Abel群的子群还是自由的。考虑$d(P_n)$是$P_{n-1}$的子群,则$d(P_n)$是自由Abel群。不典范的,这说明:
   \begin{align}
      P_n=Z_n \oplus d(P_n)
   \end{align}
   从而$Z_n\otimes M$是$P_n\otimes M$的直和项,也是$\ker(d_n\otimes 1)$的直和项。

   商去$d_{n+1}\otimes 1$的像,我们有$H_n(P)\otimes M$是$H_n(P\otimes M)$的直和项。根据Kunneth公式可知另一个项是$\Tor_1^\Z(H_{n-1}(P),M)$.
\end{proof}
\begin{theorem}[复形的Kunneth公式]{Kunneth-formula-complex}
   设$P,Q$是右,左模链复形.如果$P$和$d(P)$都是平坦的,则存在正合列:
   \begin{align}
      0 \to \bigoplus_{p+q=n}H_p(P)\otimes H_q(Q) \to H_n(P\otimes Q) \to \bigoplus_{p+q=n-1}\Tor_1^R(H_p(P),H_q(Q)) \to 0
   \end{align}
\end{theorem}
\begin{proof}
   仿照定理\ref{Kunneth-formula}的证明,把$M$换成$Q$.
\end{proof}
为了节省时间,我们省略拓扑上的泛系数定理。

接下来我们攥写上同调版本的泛系数定理。
\begin{theorem}[上同调的泛系数定理]{cohomo-universal}
   设$P$是投射模给出的链复形,使得$d(P_n)$也是投射模。则对于每个$n$和$R$模$M$,存在一个非典范的分裂正合列:
   \begin{align}
      0 \to \Ext^1_R(H_{n-1}(P),M)\to H^n(\Hom_R(P,M)) \to \Hom_R(H_n(P),M)\to 0
   \end{align}
\end{theorem}
\begin{proof}
   因为$d(P_n)$投射,从而有非典范的分裂:$P_n=d(P_{n+1})\oplus Z_n$.从而$Z_n$也投射,并且有:
   \begin{align}
      0 \to \Hom(dP_{n+1},M) \to \Hom(P_n,M)\to \Hom(Z_n,M) \to 0
   \end{align}
   是正合的。所以$0 \to \Hom(dP,M) \to \Hom(P,M)\to \Hom(Z,M) \to 0$是链复形的正合列。导引的长正合列:
   \begin{align}
      H^{n-1}(\Hom(Z,M)) \xrightarrow{\partial} H^n(\Hom(dP,M)) \to H^n(\Hom(P,M)) \to H^n(\Hom(Z,M)) \xrightarrow{\partial} H^{n+1}(\Hom(dP,M))
   \end{align}
   注意到$dP$和$Z$的微分算子都是$0$,所以$\Hom(dP,M)$的微分也是$0$,因此$H^n(\Hom(dP,M))=\Hom(dP_n,M)$。同理$H^n(\Hom(Z,M))=\Hom(Z_n,M)$。并且这里的$\partial$右$d(P_{n+1})$到$Z_n$的嵌入给出。

   注意到$H_n(P)$有投射解消:
   \begin{align}
      0 \to d(P_{n+1}) \to Z_n \to H^n(P)
   \end{align}
   于是$\Ext^1(H_{n-1}(P),M)$和$\Hom(H_n(P),M)=\Ext^0(H_n(P),M)$都可以用:
   \begin{align}
      0 \to \Hom(Z_{n-1},M) \to \Hom(dP_n,M) \to 0
   \end{align}
   带入上面的长正合列即可得到正合结果。

   而分裂可依照\ref{homo-universal}的结果得出。
\end{proof}
\begin{example}{}
   设$X$是道路连通的,则$H_0(X)=\Z$,且$H^1(X;\Z)\cong \Hom(H_1(X),\Z)$.这是一个无挠的Abel群。($\Z$是投射模。)
\end{example}
\begin{theorem}[上双复形的泛系数定理]{}
   设$P$是一个链复形,$Q$是上链复形.
   
   则可以定义上双复形$\Hom(P,Q)$.用$H^*(\Hom(P,Q))$表示$\mathrm{Tot}(\Hom(P,Q))$的上同调。设$P_n$和$dP_n$总是投射的,则存在正合列:
   \begin{align}
      0 \to \prod_{p+q=n-1}\Ext^1_R(H_p(P),H^q(Q)) \to H^n(\Hom(P,Q)) \to \prod_{p+q=n}\Hom_R(H_p(P),H^q(Q)) \to 0
   \end{align}

\end{theorem}
最后我们给出右继承的概念以结束本节。一个环$R$称作右继承的,如果任何自由(右)模的子模都是投射(右)模。实际上,任何主理想整环都是继承环(他们都是交换的戴德金整环).

继承环这条良好的性质显然可以帮助我们把泛系数定理推广到任何继承环(直接的,主理想整环)。
 \ifx\allfiles\undefined
	
	% 如果有这一部分的参考文献的话,在这里加上
	% 没有的话不需要
	% 因此各个部分的参考文献可以分开放置
	% 也可以统一放在主文件末尾。
	
	%  bibfile.bib是放置参考文献的文件,可以用zotero导出。
	% \bibliography{bibfile}
	
	end{document}
	\else
	\fi
\ifx\allfiles\undefined
	
	% 如果有这一部分的参考文献的话,在这里加上
	% 没有的话不需要
	% 因此各个部分的参考文献可以分开放置
	% 也可以统一放在主文件末尾。
	
	%  bibfile.bib是放置参考文献的文件,可以用zotero导出。
	% \bibliography{bibfile}
	
	\end{document}
	\else
	\fi
\ifx\allfiles\undefined

	% 如果有这一部分另外的package,在这里加上
	% 没有的话不需要
	
	\begin{document}
\else
\fi
\part{古典代数拓扑理论}
\chapter{基本群}
基本群读的时候,并没有做笔记,因而这一节暂时只能空缺了.有机会可以补上这一节的笔记.
\chapter{VK定理}
\section{广群(群胚):Groupoid}
为了介绍VK定理,我们首先有必要介绍广群.首先介绍普适一点的定义:
\begin{definition}
    称一个范畴$\mathcal{C}$是广群,若其每个态射都有逆,意味着其是同构的.所有广群的范畴用$\mathrm{Grpo}$来表示.其态射是两个范畴之间的函子.
\end{definition}
    如果广群的对象只有一个,那么该对象的所有态射显然构成一个群.这也是广群的名字的来源.

    我们聚焦于一类广群:拓扑广群.这类广群由拓扑空间$X$生成,其对象是$X$的每一个点,其态射是两个点之间的道路同伦类.这类广群我们统一记作$\Pi(X)$.因此我们不难想到由如下的函子:
    $$
    \Pi: \mathrm{Top} \to \mathrm{Grpo}
    $$
    这样的函子存在的关键条件在于是否有态射的对应关系.如果$f:X \to Y$是连续映射,那么$\Pi(X)\to \Pi(Y)$是否存在对应的函子呢?

    这样的函子是存在的.设为$F$,则$F(x)=f(x)\in Y$.对于$[x \to x']$的道路同伦类,我们考虑$f[x \to x']$.这是$Y$中的道路同伦类.我们需要验证良定性.若$a$和$b$是$X$中的同伦道路,则$f(a)$和$f(b)$显然是同伦的.这样的对应方式满足的复合性也很容易验证.

    \begin{proposition}
        $\Pi$是一个函子.
    \end{proposition}
    \begin{proposition}
        一个广群是连通的范畴,当且仅当其任意两个对象都是同构的.
    \end{proposition}
    因此$\Pi(X)$是连通范畴当且仅当$X$是道路连通空间.
    \section{VK定理的内容}
    我们有两个版本的VK定理,分别对应于群和广群的情况.先叙述广群的情况是更合适的.
    \begin{theorem}[van Kampen]
        设$\mathcal{O}$是$X$的一个道路连通开覆盖,使得其在任意有限交的操作下保持不变.把$\mathcal{O}$看作范畴,对象是里面的开集,态射由集合间的包含关系生成,那么该范畴到$\mathrm{Grpo}$有一个自然的函子.这个函子的余极限是$\Pi(X)$.即:
        $$
        \Pi(X)\cong \Colim_{U \in \mathcal{O}}\Pi(U)
        $$
    \end{theorem}
    \begin{proof}
        由于是余极限,则我们需要先给出$\Pi(U)$到$\Pi(X)$的函子.当然我们肯定先猜测是自然的映入所诱导的函子.

        那么根据余极限的定义,需要对于函子族$\{\eta_{U}:\Pi(U)\to \mathcal{G}\}$给出$\eta:\Pi(X)\to \mathcal{G}$.首先我们知道$U$是开覆盖,所以自然的定义每个点$p \in U$,使得$\eta(p)=\eta_U(p)$即可.当然这里需要验证良定性.事实上若$p \in U\cap V$,则$\eta_U(p)=\eta_{U \cap V}(p)=\eta_{V}(p)$,则良定.

        关键的是定义态射,即对于道路$a:x \to x'$的同伦类$[a]$,注意到$a$作为道路是一个紧集,从而有有限子覆盖$\{U_k\}$.这有限个开集可以把$a$裂成有限条道路的连接,使得这些道路都在某一个$U_k$中.而对于$U_k$中的道路$a_k$,自然可以得到$x_k$到$x_{k+1}$的道路所对应的态射$\eta_{U_k}(a_k) \to \eta_{U_k}(a_{k+1})$.然后所有态射都复合起来,就能得到$a$所对应的值.我们注意到这样做是必须的.

        问题是验证良定性.需要验证这几步操作均不影响:(1)取道路同伦类的代表元. (2)分裂道路 (3)分别取道路所在同伦类的被映射的态射进行复合.

        首先是同伦类.注意到不同的分裂方式实际上对应了同一个同伦类里面的道路,因此(1)(2)我们只用说明第一个.设$H$是连接同伦,$f$是上端,$g$是下端.则我们可以把$I \times I$分为若干个小正方形,使得每个正方形的像都只在某一个$U_k$中.从而$f$和$g$的每一个小段都是同伦的,这意味着$\eta_{U_k}$作用在上面的像相等.从而我们证明了余极限的成立.
    \end{proof}

    广群的版本虽然简洁,但是我们对于广群范畴并不熟悉,范畴的极限并不是简单的东西.因此我们还是要叙述群的版本.因为群范畴下的极限往往可以写出来的,在实际计算中也有大用.
    \begin{theorem}[van Kampen]
        假设$X$是道路连通空间,给$X$选择一个基点$x$.设$\mathcal{O}$是$X$的开覆盖,并且$x \in U, \forall U \in \mathcal{O}$,满足所有开集都是道路连通并且有限交封闭.同样,将$\mathcal{O}$看作范畴,那么函子:
        $$
        \pi_1(\cdot,x): \mathcal{O}\to \mathrm{Grp}
        $$
        对应的余极限是$\pi_1(X,x)$.即:
        $$
        \Colim_{U \in \mathcal{O}}\pi_1(U,x)=\pi_1(X,x)
        $$
    \end{theorem}
    我们首先不证明这个定理,而介绍一些推论和应用.
    \begin{corollary}
        设$X_i$是一些道路连通的带基点空间,$X$是他们的一点并,$x$是基点.对于$x \in X$和$X_i$,存在$V_i$是包含在$X_i$里面$x$的可缩邻域.则$\pi_1(X)$是所有$\pi_1(X_i)$的余积(自由积)
    \end{corollary}
    \begin{proof}
        设$U_i=X_i \cup \bigcup_{j\neq i}V_j$.$U_i$与任何$X_j$的交集都是开集,因而是开集.此时$U_i$成为$X$的开覆盖,并且以$x$为公共点.构造$X$的开覆盖,这样的开覆盖由$\{U_i\}$和所有的有限交构成.由于任意$U_i$和$U_j$的有限交都是可缩空间,因此这些有限交对于余极限而言毫无帮助.

        另一方面,$U_i$的基本群显然是$X_i$的基本群(这可以用VK定理显而易见的得到),因此$X$的基本群是$\pi_1(X_i,x)$的自由积.
    \end{proof}
    \begin{lemma}
        对于带基点的空间$X,Y$,$\pi_1(X \times Y)=\pi_1(X)\times \pi_1(Y)$.
    \end{lemma}
    \begin{proof}
        这来自于积的泛性质.如果我们能构造交换图成立:
        \[\begin{tikzcd}
            &&& {\pi_1(X)} \\
            {\pi_1(X) \times \pi_1(Y)} && {\pi_1(X \times Y)} && {} \\
            &&& {\pi_1(Y)}
            \arrow[from=2-1, to=1-4]
            \arrow[from=2-3, to=1-4]
            \arrow[from=2-3, to=3-4]
            \arrow[from=2-1, to=3-4]
            \arrow["{\exists !}"{description}, dashed, from=2-1, to=2-3]
        \end{tikzcd}
        \]

    事实上,选取$X \times Y$上的道路同伦类,其代表元投射到$X$和$Y$上自然有$X$和$Y$上的道路,然后得到同伦类.这就是$\pi_1(X \times Y)$到$\pi_1(X)$和$\pi_1(Y)$的投射定义.我们需要说明这是良定义的.这显然,因为$X \times Y$上的同伦限制在$X$上也是同伦.(复合一个$p_X$就万事大吉) 
    
    现在选择$([x],[y]) \in \pi_1(X)\times \pi_1(Y)$,那么什么样的在$X \times Y$上的道路投射到$\pi_1(X)$上才是$[x]$呢.其必然是$x,y$所复合而成的道路.因此这里的唯一存在是自然的.
    \end{proof}
    \begin{definition}
        称$X$是单连通空间,当且仅当其道路连通并且满足$\pi_1(X)=0$.
    \end{definition}
    \begin{proposition}
        设$X =U \cup V$,$U,V,U\cap V$是道路连通空间,并且共用$x \in X$作为同一个基点.若$V$是单连通空间,那么$\pi_1(U)\to \pi(X)$是一个满同态,其核是包含了$\pi_1(U \cap V)$像的最小正规子群.
    \end{proposition}
    \begin{proof}
        设$N$是命题中的正规子群.考虑交换图:
        \[\begin{tikzcd}
                & {\pi_1(U)} \\
                {\pi_1(U\cap V)} &&& {\pi_1(X)} & {\pi_1(U)/N} \\
                & {\pi_1(V)=0}
                \arrow["\xi"{description}, dashed, from=2-4, to=2-5]
                \arrow[from=1-2, to=2-4]
                \arrow[from=2-1, to=1-2]
                \arrow[from=2-1, to=3-2]
                \arrow[from=3-2, to=2-4]
                \arrow[from=1-2, to=2-5]
                \arrow[from=3-2, to=2-5]
            \end{tikzcd}
        \]

        由于$\pi_1(X)$是左半边图的余极限,因此根据泛性质有$\xi:\pi_1(X)\to \pi_1(U)/N$.我们要说明这是一个同构.根据代数学知识,$\pi_1(U)/N$是同态$\pi_1(U \cap V)\to \pi_1(U)$和$\pi_1(U \cap V)\to 0$的推出,因此反过来据泛性质又有$\xi$的逆.从而这是同构.
    \end{proof}
    \section{VK定理 群版本的证明}
    \begin{theorem}[van Kampen]

        假设$X$是道路连通空间,给$X$选择一个基点$x$.设$\mathcal{O}$是$X$的开覆盖,并且$x \in U, \forall U \in \mathcal{O}$,满足所有开集都是道路连通并且有限交封闭.同样,将$\mathcal{O}$看作范畴,那么函子:
        $$
        \pi_1(\cdot,x): \mathcal{O}\to \mathrm{Grp}
        $$
        对应的余极限是$\pi_1(X,x)$.即:
        $$
        \Colim_{U \in \mathcal{O}}\pi_1(U,x)=\pi_1(X,x)
        $$
    \end{theorem}
    \begin{proof}
        我们先证明$\mathcal{O}$有限的情况.

        设$G$是任何一个群,且$\eta:\pi_1:\mathcal{O}\to \mathrm{Grp}$是一系列$\mathcal{O}$形状的群的交换图.为了说明泛性质,我们必须说明存在$\tilde{\eta}:\pi_1(X,x)\to G$.并且根据交换图成立,$\tilde{\eta}|_{U}=\eta_{U}$.

        把$\pi_1(X,x)$看作范畴.则$\pi_1(X,x)\to \Pi(X)$存在内射,并且是一个等价.设$J:\Pi(X)\to \pi_1(X,x)$是逆的函子即$F \circ J=\mathrm{id}$.

        因为$\mathcal{O}$是有限的,并且在有限交的运算下保持封闭,因此我们可以做到下面的论断:
        
        若$x,y$都处于$U$中,那么我们可以选择一条从$x$到$y$并且全程在$U$的道路,使得这条道路成为$J$中把$\pi_1(X,y)$映射到$\pi_1(X,x)$的同构.

        从而:
        $$
        \Pi(U)\to \pi_1(U,x) \to G
        $$
        有函子的复合.这样的函子复合满足一个$\mathcal{O}$形的图:$\Pi|\mathcal{O}\to G$.这是因为只要$x,y \in U$,连接他们的道路(以保证同构)就一定在$U$中.

        由于$\Pi(X)$是上面这个图的余极限,因此自然有$\Pi(X) \to  G$的函子.不难想到$\tilde{\eta}:\pi_1(X,x)\to \Pi(X) \to G$是我们想要证明唯一存在的同态(函子).为了说明我们的想法是正确的,需要计算$\tilde{\eta}|_U$.
        $$
        \tilde{\eta}|_U=( F_U \circ J_U \circ \eta_U)=\eta_U
        $$
        因而泛性质验证完毕.

        接下来考虑一般情况.假设$\mathcal{F}$是这样一个集族:其中的集合是$\mathcal{O}$的有限子集,并且满足有限交封闭.对于$\mathcal{P} \in \mathcal{F}$,设$U_{\mathcal{P}}$是$\mathcal{P}$中开集的并,则$\mathcal{P}$是$U_{\mathcal{P}}$的开覆盖,并且满足有限情况的VK定理.因此:
        $$
        \Colim_{U \in \mathcal{P}}\pi_1(U,x)\cong \pi_1(U_{\mathcal{P}},x)
        $$

        现在把$\mathcal{F}$看作一个范畴,态射由$U_{\mathcal{P}} \subset U_{\mathcal{Q}}$决定.我们断言:
        $$
        \Colim_{\mathcal{P}\in \mathcal{F}} \pi_1(U_{\mathcal{P}},x) \cong \pi_1(X,x)
        $$
        
        这一断言和广群版本的证明很类似.因为每条道路都可以存在有限覆盖,以及$I \times I$的正方形可以拆分为若干个小正方体.

        接下来我们断言:
        $$
        \Colim_{ U \in \mathcal{O}}\pi_1(U,x)\cong \Colim_{\mathcal{P}\in \mathcal{F}}\pi_1(U_{\mathcal{P}},x)
        $$
        
        也就是证明:
        $$
        \Colim_{ U \in \mathcal{O}}\pi_1(U,x)\cong \Colim_{\mathcal{P}\in \mathcal{F}}\Colim_{U\in \mathcal{P}}\pi_1(U,x)
        $$
        对于范畴中的指标范畴换序问题,我们已经了解了.因此我们意图说明下面的极限:
        $$
        \Colim_{(U,\mathcal{P})\in (\mathcal{O},\mathcal{F})} \pi_1(U,x)
        $$
        这个极限和$\Colim_{U \in \mathcal{O}} \pi_1(U,x)$实际上并无二异.按照原书的话来讲,“the system on the right differs
        from the system on the left only in that the homomorphisms $\pi_1(U, x)\to \pi_1(V, x)$
        occur many times in the system on the right, each appearance making the same
        contribution to the colimit.”
    \end{proof}
    这个证明仍有需要注意的地方.为什么不能直接挪用广群的方法证明,而是要这样费尽周折呢?$S_1$是紧集合,那么回路当然也是紧集合.

    答案是我也不知道.这里留一个小小的疑问.
\chapter{覆叠空间}
\section{覆叠空间的定义}
     覆叠空间是代数拓扑中比较重要的一种关系.其提升定理与之后的纤维,上纤维等关系密切,因而学习覆叠空间是必要的.在本章中,我们假设所有的空间都是连通且局部道路连通的,当然这意味着空间本身道路连通.
\begin{definition}[覆叠空间]
    \quad

    设$B$是一个拓扑空间.如果存在拓扑空间$E$和连续满射$p:E \to B$,并且满足:对于$\forall b \in B$,存在$b$的开邻域$V$使得$p^{-1}(U)$是$E$中若干不相交开集$\{U_\alpha\}_{\alpha \in \Lambda}$的并$\bigcup_{\alpha \in \Lambda}U_\alpha$,并且映射$p|_{U_\alpha}:U_\alpha \to U$是同胚,则称$E$是$B$的\textbf{覆叠(复叠,复迭,覆迭)空间},连续映射$p$是对应的覆叠映射.

    我们称满足上述条件的$E$是全空间,$B$是底空间,$F_b=p^{-1}(b)$是覆叠映射$p$的纤维.
\end{definition}
\begin{proposition}
    若$f:A \to B$是连续映射,$D$是$f$和$p$的拉回,即交换图:
    \begin{tikzcd}
        A && D \\
        \\
        B && E
        \arrow["f", from=1-1, to=3-1]
        \arrow["p"', from=3-3, to=3-1]
        \arrow[from=1-3, to=1-1]
        \arrow[from=1-3, to=3-3]
    \end{tikzcd}
    中的$D=\{(a,e)\in A\times E|f(a)=p(e)\}$.此时投射$D \to A$也是覆叠的.
\end{proposition}
\begin{proof}
    首先验证满射.这一点可以由$p$是满射得到:$\forall a \in A,p^{-1}(f(a))$中的任何一个$e$都可以与$a$配对.从这里也可以得到,如果真的是覆叠映射,两个覆叠映射的叶数一样.

    考虑$f(a)$的邻域$V$,则$p^{-1}(V)=\bigcup U_\alpha \subset E$.取其中一个$U_\alpha$,则$p|_{U_\alpha}$是同胚.$U_\alpha$在$D$中的原像是:
    $$
    D_\alpha=\{(a,e)|f(a)=p(e) \in V, e\in U_\alpha\}
    $$
    显然$D_\alpha$投射到$A$上等于$f^{-1}(V)$.连续性是显然的,我们接下来说明限制在$D_\alpha$的投射是同胚,为此只用说明双射.若$(a,e)$和$(a,e')$都在$D_\alpha$中,则$p(e)=p(e')$.由于$p$是同胚,从而$p$是双射,于是$e=e'$.因此单射.满射容易得到.

    因此$D_\alpha$是$f^{-1}(V)$的一个叶子.从而$D \to A$的投射是覆叠映射.
\end{proof}
    这个命题容易让人联想到接下来的上纤维和纤维关于拉回推出的推论.
\section{道路提升唯一定理}
\begin{theorem}[提升唯一定理·道路与基本群]
    设$p:E \to B$是覆叠空间.设$b \in B$,$e,e' \in F_b=p^{-1}(b)$.则:
    \begin{enumerate}
        \item 道路$f:I \to B$,$f(0)=b$可以被唯一提升到满足$g(0)=e$的道路$g:I \to E$.其中$e$是给定的$F_b$中的元素.
        \item 同伦的,且起点都在$b$的道路$f \thicksim f'$被提升到同伦的,且起点都在$e$的道路$g \thicksim g'$,因此$g(1)=g'(1)$.
        \item $p_*:\pi_1(E,e) \to \pi(B,b)$是单同态.
        \item $p_*(\pi_1(E,e'))$与$p_*(\pi_1(E,e))$共轭.
        \item 随着$e'$遍历$F_b$,群$p_*(\pi_1(E,e'))$遍历$p_*(\pi_1(E,e))$所有的共轭类.
    \end{enumerate}
  \end{theorem}
  \begin{proof}
    唯一提升定理的证明就不写了,因为重复这样的验证工作没什么意思(指被验证烂了).
  \end{proof}
  假设$E$的基本群是平凡的,即是单连通空间.此时有一个很有意思的结论:
      
      \begin{proposition}
        设$E$是单连通空间.则$F_b$中的元素与$\pi_1(B,b)$有一个一一对应.
      \end{proposition}
      \begin{proof}
        规定$E$中起点$e$,可以定义映射$\sigma:F_b \to \pi_1(B,b)$,$\sigma(e')$是$E$中$e$到$e'$的道路同伦类.

        这个映射显然是双射.因为$\pi_1(B,b)$的元素只能提升后终点只有一个.并且大家都能提升.
      \end{proof}
      \begin{example}
        $S^n$是$RP^n$的单连通覆叠空间,且$F_b$只有两个元素.因此$\pi_1(RP^2)$只有两个元素,因此$\pi_1(RP^2)=\Z_2$
      \end{example}
      \begin{definition}
        覆叠$p:E \to B$是正则的,若$p_*(\pi_1(E,e))$是正规子群.覆叠是万有的,如果$E$是单连通空间.
      \end{definition}
\section{广群的覆叠}
接下来介绍的理论很牛批.因为广群的覆叠本质上就是为了描述覆叠空间中的代数本质.把道路简化为态射,把点简化为对象,这样的想法惊为天人,鬼斧神工般就得到了覆叠空间的代数本质!太厉害辣!
\begin{definition}
设$\mathcal{C}$是一个范畴,$x$是一个对象.范畴$x\backslash \mathcal{C}$是这样一个范畴:其对象是从$x$出发的态射$f:x \to y$.其态射$\varphi$是满足$f\circ \varphi:x \to y \to z=g$的$\mathcal{C}$中的态射.复合和恒等态射都是显然的.若$\mathcal{C}$是一个广群,此时$\varphi=g\circ f^{-1}$,从而范畴$x\backslash \mathcal{C}$完全由其对象决定.

设$\mathcal{C}$是一个小范畴(所有对象可以构成集合).定义$x$的星集$\mathrm{St}(x)$为$x\backslash \mathcal{C}$的对象集.记$x$的自态射集合为$\pi(\mathcal{C},x)$.
\end{definition}
\begin{definition}[广群的覆叠]
    设$\mathcal{E}$和$\mathcal{B}$是两个小的连通广群.称函子$p:\mathcal{E}\to \mathcal{B}$是一个覆叠,若其在对象集合上是一个满射,并且当限制在$\mathrm{St}(e),\forall e$时得到一个双射:
    $$
    p:\mathrm{St}(e)\to \mathrm{St}(p(e))
    $$
    对于$\mathcal{B}$的对象$b$,记$F_b$是其在$\mathcal{E}$中的原对象集,则$p^{-1}(\mathrm{St}(b))$是以$e$为指标的$\mathrm{St}(e)$集合的不交并.
\end{definition}
\begin{proposition}
    若$p:E \to B$是覆叠空间,那么诱导的函子$\Pi(p):\Pi(E)\to \Pi(B)$是广群的覆叠.
\end{proposition}
\begin{proof}
    对于$e \in E$,$p(e)$是$B$中的元素.$e$往外伸展开的道路同伦类映射到$B$上形成以$b$为起点的道路同伦类.由于道路提升唯一定理和同伦道路提升定理可知,这是一个双射.
\end{proof}
\begin{proposition}
    设$p:\mathcal{E}\to \mathcal{B}$是广群的覆叠,设$b$是$B$的对象,$e$和$e'$是$F_b$中的$\mathcal{E}$的对象.
    \begin{enumerate}
        \item $p:\pi(\mathcal{E},e) \to \pi(\mathcal{B},b)$是单同态.
        \item $p(\pi(\mathcal{E},e'))$与$p(\pi(\mathcal{E},e))$共轭.
        \item 随着$e'$遍历$F_b$,群$p(\pi(\mathcal{E},e'))$遍历$p(\pi(\mathcal{E},e))$所有的共轭类.
    \end{enumerate}
\end{proposition}
\begin{proof}
    这完全是前面定理的对应.不过由于是纯粹的代数问题,我们需要给出证明.

    对于1,由于$p:\mathrm{St}(e) \to \mathrm{St}(b)$是双射,限制在$\pi(\mathcal{E},e)$也应该是双射.从而这里的$p$是单同态.

    对于2,由于$\mathcal{E}$连通,因此选取态射$f:e \to e'$.此时两个群的同构由$f$诱导.$p(f) \in \pi(\mathcal{B},b)$,从而这个诱导的同构由$p$诱导为共轭.

    对于3,$p(\pi(\mathcal{E},e))$的共轭类均由一个$g:b\to b$诱导.根据双射关系,有$g':e \to e'$对应$g$,$e' \in F_b$.从而$g'$被认为是诱导该共轭关系,说明该共轭类是$p(\pi(\mathcal{E},e'))$.
\end{proof}
 
我们注意到$F_b$和$F_b'$之间可能会有很不错的关系.下面的定义和定理深入的研究了这样的关系.
\begin{definition}
    广群覆叠的记号如上.定义纤维变换函子$T=T(p):\mathcal{B} \to \mathrm{Set}$:$T(b)=F_b$.对于$f:b \to b'$,定义映射$T(f):F_b \to F_{b'}$,$T(f)(e)=e'$,其中$g:e \to e'$是$f$的提升.

    $T$成为函子是因为单位态射的提升自然是单位态射,而$T(f\circ g)=T(f)\circ T(g)$也是不难验证的.因为$f \circ g$的提升与$f$,$g$分别提升后复合的末端是一样的.
\end{definition}
纤维变换函子是描述广群覆叠的又一重要工具.比如,其函子性就说明了$T(f)$是$\mathrm{Set}$中的可逆态射,即双射.另外,由于$\mathrm{B}$的对象都是同构的,所以其函子像也都是同构的.换句话说,$F_b$和$F_{b'}$的基数相同.
\section{群作用和轨道范畴}
为了研究广群覆叠的关系,下面还要介绍一些东西.

首先我们不再阐述群作用的定义.(最近在每个课程里面都要强调好几次,抽象代数课程再怎么记不住都该熟悉了)称一个群作用是自由的,若$G_s$($s$的迷向子群是平凡的).也就是$G$对应的同态是单的.

在这里我们主要考虑的是传递作用.这种情况下,$G/G_s$构成的陪集集合与$S$同构.对于$G$的子群$H$,其正规化子$NH$定义为$NH=\{n\in G|n^{-1}Hn=H\}$.容易看出来$H$是$NH$的正规子群,因此我们定义Weyl群:$WH=NH/H$.

下面的引理纯粹是群论性质的(对,意思是我不想证明):
\begin{lemma}
    设$G$传递作用在集合$S$上.记$s \in S$,并且记$H=G_s$.则$WH$同构于群$\mathrm{Aut}_G(S)$.这个记号表示$G$-集合$S$的自同构群.即若$\varphi \in \mathrm{Aut}_G(S)$,则$\varphi(gs)=g\varphi(s),\forall g \in G $.
\end{lemma}
\begin{proof}[我是傲娇]
    对于$[n] \in WH$,定义$\varphi([n])$是这样一个自同构:$\varphi([n])(gs)=gns$.(因为传递作用代表可以记$S=\{gs|g\in G\}$.

    首先验证良定.$gn_1s=gn_2hs=gn_2s$,若$n_1=n_2h,h\in H$.从而良定行显然.

    我们说明这样的对应是同构的.双射:单:若$gn_1s=gn_2s$对于任何$g$都成立,则$n_1s=n_2s$,于是$n_1^{-1}n_2s=s$,即$n_1^{-1}n_2 \in H$,从而$[n_1]=[n_2]$.满射:对于$\varphi(s)=ns$.我们要证明$n \in NH$.注意到$hns=h\varphi(s)=\varphi(hs)=\varphi(s)=ns$,从而$n^{-1}hn\in H$,则$n \in NH$.

    同态:$\varphi([n_1][n_2])(gs)=gn_1n_2s=\varphi([n_2])(gn_1s)=\varphi([n_2])(\varphi([n_1])(gs))$.
\end{proof}

接下来的问题是$G$集合之间的映射
\begin{lemma}
    一个$G-$映射$\alpha:G/H \to G/K$拥有形式$\alpha(gH)=g\gamma K$,其中$\gamma \in G$满足$\gamma^{-1}h \gamma \in K, \forall h \in H$
\end{lemma}
\begin{definition}
    对于群$G$,定义轨道范畴$\mathcal{O}(G)$:对象是$G/H$,态射是$G-$映射.
\end{definition}

因此上面的引理似懂非懂的给出了$\mathcal{O}(G)$的结构.我们用下面的引理来具体化:
\begin{lemma}
    范畴$\mathcal{O}(G)$同构于范畴$\mathcal{G}$,这个范畴的定义是:对象是$G$的子群,态射是不同的子共轭关系:$\gamma^{-1} H \gamma \subset K$,$\gamma \in G$.即$\gamma:H \to K$代表着$\gamma^{-1} H \gamma \subset K$.
\end{lemma}
\begin{proof}
    要定义两个范畴之间的同构函子.对象的定义是自然的.对于$\gamma:G/H \to G/K$,引理说明$\gamma^{-1} H \gamma \subset K$.反之,若$\gamma$满足子共轭关系,那么定义$\alpha(gH)=g\gamma K$,容易验证这是一个$G-$映射.
\end{proof}

我们把群作用的定义拓展到广群.如果广群只有一个对象,则是一个群.此时群作用恰好是$G \to \mathrm{Set}$的一个函子.因此定义小广群的作用为函子$T:\mathcal{B}\to \mathrm{Set}$.对于每个$b$是$\mathcal{B}$的对象,显然$\pi(\mathcal{B},b)$有一个在$T(b)$上的作用.

称$T$是传递的,当且仅当每个$T(b)$上都是传递的.如果$\mathcal{B}$是连通小广群,这个定义可以优化为“存在一个$T(b)$上的作用是传递的.”

\begin{example}
    对于覆叠广群:$p:\mathcal{E}\to \mathcal{B}$,纤维变换函子是从$\mathcal{B}\to \mathrm{Set}$的函子.因此是广群的作用.也就是$\pi(\mathcal{B},b)$作用在$F_b$上.

    对于$e \in F_b$,$e$的迷向子群是:
    $$
    \{g:b \to b|p^{-1}(g):e \to e \in \pi(\mathcal{E},e)\}=p(\pi(\mathcal{E},e))
    $$
    
    这个作用是传递的,因为任何$h:e \to e' \in F_b$都会被$p$映射到$\pi(\mathcal{B},b)$上.

    因此$F_b$实际上是一个$\pi(\mathcal{B},b)$集合,并且$F_b \cong \pi(\mathcal{B},b)/p(\pi(\mathcal{E}),e)$.
\end{example}
\begin{definition}
    广群的覆叠是正则的当且仅当$p(\pi(\mathcal{E},e))$是$\pi(\mathcal{B},b)$的正规子群.是万有的,如果$p(\pi(\mathcal{E},e))=\{e\}$.

    显然覆叠空间的正则和万有与其对应的基本广群覆叠的正则与万有保持一致.

\end{definition}
\begin{proposition}
    广群的覆叠是万有的,当且仅当$\pi(\mathcal{B},b)$自由作用在$F_b$上.此时$F_b$与$\pi(\mathcal{B},b)$同构.这也验证了我们之前所说的,万有覆叠的情况下$\pi_1(B,b)$和$F_b$基数相同.
\end{proposition}
\section{广群覆叠的分类定理}
对于一个小的连通广群$\mathcal{B}$,我们想要研究其所有的覆叠广群.因此在之后的几个小节,我们均固定广群$\mathcal{B}$.
\begin{theorem}[覆叠提升基本定理·广群]
    设$p:\mathcal{E}\to \mathcal{B}$是广群的覆叠,并设$\mathcal{X}$是一个广群,$f:\mathcal{X}\to \mathcal{B}$是一个函子.选择一个基对象$x_0\in \mathcal{X}$,设$b_0=f(x_0)$,并设$e_0\in F_{b_0}$.则存在一个函子$g:\mathcal{X}\to \mathcal{E}$使得$g(x_0)=e_0$并且$p\circ g=f$这一事实等价于:
    $$
    f(\pi(\mathcal{X},x_0))\subset p(\pi(\mathcal{E},e_0))
    $$
    当这个条件满足的时候,$g$的存在还是唯一的.
    
    \begin{tikzcd}
        && {\mathcal{E}:e_0} \\
        \\
        {\mathcal{X}:x_0} && {\mathcal{B}:b_0}
        \arrow["p", from=1-3, to=3-3]
        \arrow["f"', from=3-1, to=3-3]
        \arrow["{\exists ! g}", dashed, from=3-1, to=1-3]
    \end{tikzcd}
\end{theorem}
\begin{proof}
    如果$g$存在,则$f(\pi(\mathcal{X}),x_0)=p(g(\pi(\mathcal{X}),x_0))\subset p(\pi(\mathcal{E},e_0))$.因此我们需要说明的是该条件满足时$g$一定唯一存在.

    对于$\mathcal{X}$中的对象$x$,和一个态射$\alpha:x_0 \to x$,设$\tilde{\alpha}$是$\mathrm{St}(e_0)$中唯一的元素满足:$p(\tilde{\alpha})=f(\alpha)$.(提升即可).如果$g$存在,那么$g(\alpha)$就必须是$\tilde{\alpha}$,因此$g(x)$必须是$\tilde{\alpha}$的终点,即$T(f(\alpha))(e_0)$.

    然而问题是$T(f(\alpha))(e_0)$在$\alpha$变化的过程中会变化吗?换句话说,从$x_0 \to x$的态射很可能不止$\alpha$一个,因此若$\beta:x_0 \to x$,那么$T(f(\beta))(e_0)=T(f(\alpha))(e_0)$能满足吗?

    这一点由条件:$f(\pi(\mathcal{X},x_0))\subset p(\pi(\mathcal{E},e_0))$保证.事实上,此时$\beta^{-1}\circ \alpha$是$\pi(\mathcal{X},x_0)$的元素.从而存在$\gamma \in \pi(\mathcal{E},e_0)$,使得$p(\gamma)=f(\beta^{-1})\circ f(\alpha)$.因此$T(f(\alpha))(e_0)=T(f(\beta))(T(p(\gamma))(e_0))=T(f(\beta))(e_0)$.

    因此我们给出了$g$在对象上的对应.对于态射而言,是显然的.因此$g$不仅存在而且唯一.
\end{proof}
如果是覆叠空间,这个定理反而比较复杂.但我们一旦简化到代数本质上,就显得容易了许多.

接下来我们正式研究覆叠的分类问题.所谓分类,我们可以理解为构建一个范畴关系.对象是所有覆叠广群,而目的是建立态射,最后研究这个范畴的终对象和始对象.那么首先我们定义两个覆叠之间的态射(映射)
\begin{definition}
    称一个函子$g$是广群覆叠的映射,若满足$g:\mathcal{E}\to \mathcal{E}'$并且交换图成立:
    \begin{tikzcd}
        {\mathcal{E}} && {\mathcal{E}'} \\
        \\
        && {\mathcal{B}}
        \arrow["{p'}", from=1-3, to=3-3]
        \arrow["p"', from=1-1, to=3-3]
        \arrow["g", from=1-1, to=1-3]
    \end{tikzcd}
\end{definition}
在这样的定义下,不难验证所有的$\mathcal{E}$构成范畴$\mathrm{Cov}(\mathcal{B})$,我们称之为覆叠范畴.我们用$\mathrm{Cov}(\mathcal{E},\mathcal{E}')$表示两个覆叠空间之间所有的态射.

\begin{lemma}
    $g$作为两个覆叠空间的映射本身也是覆叠.
\end{lemma}
\begin{proof}
    首先说明$g$是满射.取$e' \in \mathcal{E'}$.我们选一个$e\in \mathcal{E}$,则$f:g(e) \to e'$是$\mathcal{E}'$中的态射.把$f$投射到$\mathcal{B}$中,则有$p'(f):p' \circ g(e) \to p'(e')$.这个态射可以被提升到$\mathcal{E}$上,并且以$e$为起点.这个态射的终点被$g$作用后必然是$e'$.
    
    其次,$e$,$e'$,$b$三个点的星集是具有双射的集合.
\end{proof}
      覆叠基本定理给出了两个覆叠之间存在态射的充要条件.叙述如下:
      
\begin{theorem}
        设$p:\mathcal{E}\to \mathcal{B}$和$p':\mathcal{E}' \to \mathcal{B}$都是广群覆叠.选择$b \in \mathcal{B},e \in \mathcal{E},e'\in \mathcal{E'}$使得$p(e)=b=p'(e')$.那么存在一个映射$g:\mathcal{E}\to \mathcal{E'}$并且满足$g(e)=e'$当且仅当:
        $$
        p(\pi(\mathcal{E},e)) \subset p'(\pi(\mathcal{E'},e')) 
        $$  
        并且此时$g$的存在是唯一的.也就是说,如果$g,g'$同时满足$g(e)=g'(e)$,则$g=g'$.
        
        $g$是同构当且仅当上述的包含是相等.因此$\mathcal{E}$和$\mathcal{E}'$是同构的当且仅当$p(e)=p'(e')$的时候有$p(\pi(\mathcal{E},e))$和$p'(\pi(\mathcal{E}',e'))$是共轭的.
\end{theorem}

因此万有覆叠是任何覆叠的覆叠,并且在同构意义下唯一.我们接下来要做的是证明其存在.

放在一边.下面的定理仍然是值得注意的:
\begin{theorem}
    覆叠广群的记法如上.选择基点$b\in \mathcal{B}$,并记$G=\pi(\mathcal{B},b)$.则下面的两个集合存在双射:
    \begin{enumerate}
        \item $F_b$到$F_b'$的所有$G$映射.
        \item $\mathrm{Cov}(\mathcal{E},\mathcal{E'})$.
    \end{enumerate}
    并且这个双射是由$g$诱导的.即$g$限制在$F_b$上成为$G$映射.
\end{theorem}
\begin{proof}
    需要说明两点:1.任何$g$限制在$F_b$都成为到$F_{b'}$的$G$映射.2.任何$G$映射都能被延拓为某个$g$.

    先说明第一点.设$g$是一个广群覆叠之间的映射.取$e'=g(e),e\in F_b,e' \in F_{b'}$.则$F_b$与$F_{b'}$被表示为:$G/p(\pi(\mathcal{E},e))$和$G/p'(\pi(\mathcal{E}',e'))$.我们知道$G$映射的形式是$\alpha(gH)=g\gamma K$,$\gamma^{-1} H \gamma \subset K$.

    对于$he$,$g$作用在上面得到的结果是$hg(e)=he'$.因为$e \to he$,$h:b \to b$,$e' \to he'$是三个对应的态射.从而其末尾也相互对应.这说明$g$限制在上面是$G$映射.

    现在考虑双射问题.如果$g$与$g'$不同,那么在每个$e$上$g$与$g'$的值都不同.从而诱导出的$g(he)$不是$F_b'$中的同一个元素.这意味着限制在$F_b$上$g$与$g'$不同.对于任何$F_b$与$F_b'$的$G$映射$\alpha$,意味着$\alpha(ge)=ge'$.从而$\alpha(e)= e' \in F_b'$.

    根据前面的定理,我们要验证$p(\pi(\mathcal{E},e))$在$p(\mathcal{E}',e')$中.这是$e$和$e'$的迷向子群.由于$\alpha(e)=e'$,因此$e$的迷向子群显然是$e'$迷向子群的子集.故根据定理知存在延拓后的$\alpha$.
\end{proof}

这个定理说明$F_b$和$F_b'$中的$G$映射就唯一决定了所有的$g$覆叠.由纤维决定了映射,这是覆叠广群巧妙的性质.

我们用$\mathrm{Aut}(\mathcal{E})$表示$\mathcal{E}$自身的覆叠映射全体.这显然构成一个群.这不一定是同胚,因为$H$和$H'$作为共轭子群,可能存在$H <H'$.

下面的推论在上述定理的支持比较显然.只用验证其运算保持不变即可.
\begin{corollary}
    设$p$是覆叠,选择一个$b$作为$\mathcal{B}$的对象.简记$G=\pi(\mathcal{B},b)$和$H=p(\pi(\mathcal{E},e))$.那么$\mathrm{Aut}(\mathcal{E})$同构于$F_b$作为$G$集合的自同态群,即$WH$.

    如果$p$是正则覆叠,即$H$是正规子群,此时其正规化子为$G$,因此$WH=G/H$.如果$p$是万有覆叠,则$H$平凡,正规化子也为$G$,$WH=G$.
\end{corollary}
\section{广群覆叠的构造}
我们研究清楚了广群覆叠之间的关系.但对于具体的范畴,只研究态射而不研究对象,就好像是母单天天打galgame而不去恋爱实践一样,只是空中楼阁.因此我们要解决的问题是,到底有哪些广群覆叠?

首先根据之前的研究,我们知道最多只有一个同构类的覆叠广群对应$\pi(\mathcal{B},b)$的共轭子群类.但实际上至多这个词可以被删去.我们将证明所有的男生都可以如愿找到女(男)朋友(指所有的共轭类都能对应一个广群的覆叠).

\begin{theorem}
    对于广群$\mathcal{B}$,选择基对象$b$,记$G=\pi(\mathcal{B},b)$.则存在一个函子$\mathcal{E}()$:
    $$
    \mathcal{E}():\mathcal{O}(G)\to \mathrm{Cov}(B)
    $$
    使得上述两个范畴是等价范畴.对于$G$的子群$H$,覆叠$p:\mathcal{E}(G/H) \to \mathcal{B}$有一个典范的基对象$e$,使其处于$b$的纤维中,并且:
    $$
    p(\pi(\mathcal{E}(G/H),e))=H
    $$

    此时,$F_b=G/H$,并且对于映射$\alpha:G/H \to G/K$,把$\mathcal{\alpha}$限制在$F_b$正好是$\alpha$本身.
\end{theorem}
\begin{proof}
    我们只说明基本思路.具体的验证工作省略.注意到$\mathrm{St}_{\mathcal{E}(G/H)}(e)$与$\mathrm{St}_\mathcal{B}(b)$作为集合来说有同构,因此不管$H$如何,$\mathrm{St}(e)$在双射的意义下是一样的.但是这些态射的终点可以根据$H$的情况发生变化.即我们可以重复掉一些态射来“节省”空间.例如若$H$是平凡群,此时$\mathcal{E}(G)$是万有覆叠,从而$e$和$e'$只有一个态射.因此每个终点和每个态射一一对应.若$H$是$G$本身,此时$\mathcal{E}(G/G)$是$\mathcal{B}$本身.因此$e=b$,从而$b$到$b'$的态射集合是$\pi(\mathcal{B},b)$的复制.终点和态射的关系也是明朗的.

    对于一般的$H$,定义$G$作用在$\mathrm{St}(b)$上.方式为先转一圈再发射出去.则等价关系$x \thicksim y:hx=y,h \in H$存在.商去这个关系:$\mathrm{St}(b)/H$,我们将该集合作为$\mathcal{E}(G/H)$的对象集合.

    接下来定义这个范畴的态射.取某个对象的代表元$f:b \to b'$和$g: b \to b''$.两个对象之间的态射定义为$ g \circ h \circ f^{-1}$,$h \in H$.从而态射集合与$H$有一一对应.容易验证上述定义满足范畴的要求.并且形成广群.

    我们需要典范的基对象.设$e$是$\mathrm{id}_b$所在的等价类,则$\pi(\mathcal{E}(G/H),e)$是$\mathrm{id}_b$到自身的态射集合,即$H$本身.

    设覆叠$p$是把$fH$映射到其末尾,把态射映射为$\mathcal{B}$中一样的态射.则$p(H)=H$.只需要验证$p$是覆叠.为此,从$fH$出发的态射与$b'$出发的态射需要存在一一对应.前者到后者通过$p$对应,而取$g:b' \to b''$是后者的某一个态射,则$g \circ f$是从$b$到$b''$的态射.其所在的等价类与$fH$的等价类之间显然有态射$g$.因此这是双射.

    最后说明$G/H$和$G/K$的映射$\alpha$.定义$\mathcal{E}(\alpha)$把$fH$映射到$fgK$.其中$\alpha(aH)=agK,\forall aH \in G/H$.把态射$f' \circ h \circ f^{-1}$映射到同样的态射$f'g \circ g^{-1}hg \circ g^{-1}f^{-1}$.因为$g^{-1}hg \in K$.接下来的验证:函子性,限制$F_b$上等问题就省略了.
\end{proof}
\section{覆叠空间的分类定理}
覆叠空间的分类定理,即考虑同一个空间$B$的所有覆叠空间$\mathrm{Cov}(B)$之间的关系.注意到我们在研究关系,因此可以来猜测这样的关系本质是代数的,即之前广群的定理全部可以照搬过来.也就是说,我们完全可以把推gal的经验用在和现实的女孩子谈恋爱上.(如果已经有女朋友了)

可惜的是,空间的构造并非代数的,也就是之后的一节:覆叠空间的构造本身是拓扑的构造,即代数并不能有一个很好的相似方法来做覆叠空间.也就是说,就算你gal打的再多,要是连女朋友都找不到,也白搭.
\begin{theorem}[覆叠提升基本定理·拓扑空间]
    设$p:E\to B$是拓扑空间的覆叠,并设$X$是一个拓扑空间,$f:X\to B$是连续映射.选择一个基点$x_0\in X$,设$b_0=f(x_0)$,并设$e_0\in F_{b_0}$.则存在一个连续映射$g:X\to E$使得$g(x_0)=e_0$并且$p\circ g=f$这一事实等价于:
    $$
    f_*(\pi_1(X,x_0))\subset p_*(\pi_1(E,e_0))
    $$
    当这个条件满足的时候,$g$的存在还是唯一的.
    
    \begin{tikzcd}
        && {E:e_0} \\
        \\
        {X:x_0} && {B:b_0}
        \arrow["p", from=1-3, to=3-3]
        \arrow["f"', from=3-1, to=3-3]
        \arrow["{\exists ! g}", dashed, from=3-1, to=1-3]
    \end{tikzcd}
\end{theorem}
\begin{proof}
    若$g$存在,显然有包含关系.

    若包含关系满足,则广群覆叠$\Pi(p):\Pi(E)\to \Pi(B)$也有类似的包含关系,因此存在函子$g$.$g$可以看作映射,但是需要验证$g$的连续性.若$y \in X$,$g(y) \in U$,$U$是$E$的开集.则存在更小的$g(y)$的开邻域$U'$使得$p$同胚的把$U'$映射到$B$上开集$V$.若$W$是任何$y$的道路连通邻域使得$f(W)\subset V$,则$g(W)\subset U'$.这说明$g$是连续的.
\end{proof}
\begin{definition}
    称一个连续函数$g$是拓扑空间覆叠之间的映射,若满足$g:E\to E'$并且交换图成立:
    \begin{tikzcd}
        {E} && {E'} \\
        \\
        && {B}
        \arrow["{p'}", from=1-3, to=3-3]
        \arrow["p"', from=1-1, to=3-3]
        \arrow["g", from=1-1, to=1-3]
    \end{tikzcd}
\end{definition}
同样有类似的记号$\mathrm{Cov}(B)$,$\mathrm{Cov}(E,E')$.

\begin{lemma}
    覆叠空间$E \to E'$的映射自身也是覆叠.
\end{lemma}
\begin{proof}
    $e'$和$g(e)$都是$E'$的点,因此有道路连接.该道路投射到$B$上,再提升到$E$上并要求起点是$e$.则终点的$g$像是$e'$.从而$g$满射.$p'(e')$是$B$的点,从而有一个$U$满足$p^{-1}(U)$和$p'^{-1}(U)$是基本的.$e'$属于$p'^{-1}(U)$中的某个部分$V$,则$V$是基本的.这一点可以由道路连通的角度来看.
\end{proof}

覆叠基本定理给出了两个覆叠之间存在态射的充要条件.叙述如下:
      
\begin{theorem}
        设$p:E\to B$和$p':E' \to B$都是覆叠.选择$b \in B,e \in E,e'\in E'$使得$p(e)=b=p'(e')$.那么存在一个映射$g:E\to E'$并且满足$g(e)=e'$当且仅当:
        $$
        p_*(\pi_1(E,e)) \subset p'_*(\pi_1(E',e')) 
        $$  
        并且此时$g$的存在是唯一的.也就是说,如果$g,g'$同时满足$g(e)=g'(e)$,则$g=g'$.
        
        $g$是同胚当且仅当上述的包含是相等.因此$E$和$E'$是同胚的当且仅当$p(e)=p'(e')$的时候有$p_*(\pi_1(E,e))$和$p'_*(\pi_1(E',e'))$是共轭的.
\end{theorem}
\begin{corollary}
    万有覆叠在存在的情况下是同胚唯一的.
\end{corollary}

基本群能够决定覆叠空间之间的关系,这是代数拓扑中少有的事情.这很不典型.但是下面的定理或许能让我们初见其端倪:
\begin{theorem}
    基本广群函子诱导双射:
    $$
    \mathrm{Cov}(E,E') \to \mathrm{Cov}(\Pi(E),\Pi(E'))
    $$
\end{theorem}
\begin{proof}
    由上述定理显然.
\end{proof}

把覆叠间的映射限制在$F_b$上自然得到$F_b$到$F_b'$的映射.这是$G$映射.而$F_b$到$F_b'$的$G$映射都自然能诱导覆叠间的映射.这是局部与整体的巧妙联系.
\begin{theorem}
    覆叠空间记法如上.选择基点$b\in B$,并记$G=\pi_1(B,b)$.则下面的两个集合存在双射:
    \begin{enumerate}
        \item $F_b$到$F_b'$的所有$G$映射.
        \item $\mathrm{Cov}(E,E')$.
    \end{enumerate}
    并且这个双射是由映射$g$诱导的.即$g$限制在$F_b$上成为$G$映射.
\end{theorem}
\begin{proof}
    $G$作用在$F_b$上,意味着一种提升.即$e$到$f e$的道路同伦类被$p_*$映射为$f$.考虑$g(fe)$与$f g(e)$.$g(e)$到$f g(e)$的道路同伦类,$e$到$fe$的道路同伦类,被映射为$B$上的$f$.自然有$g(fe)=fg(e)$.这说明$g$限制后是$G$映射.

    不同的$g$限制在$F_b$上结果必然不同.因为若$g(e)=g'(e)=e'$,则,满足该条件的$g$只能有一个.

    对于$F_b$到$F_b'$的$G$映射$\alpha$,我们需要验证包含关系.这是$F_b$和$F_b'$的迷向子群.由于$\alpha$是$G$映射,因此迷向子群显然包含关系.
\end{proof}
类似的,我们用$\mathrm{Aut}(E)$表示$E$自身的覆叠映射全体.这显然构成一个群.这不一定是同胚,因为$H$和$H'$作为共轭子群,可能存在$H <H'$.
\begin{corollary}
    设$p$是覆叠,选择一个$b$作为$B$的对象.简记$G=\pi_1(B,b)$和$H=p(\pi_1(E,e))$.那么$\mathrm{Aut}(E)$同构于$F_b$作为$G$集合的自同态群,即$WH$.

    如果$p$是正则覆叠,即$H$是正规子群,此时其正规化子为$G$,因此$WH=G/H$.如果$p$是万有覆叠,则$H$平凡,正规化子也为$G$,$WH=G$.
\end{corollary}

\section{覆叠空间的构造}
这一节我们将介绍几个定理.证明是省略的,因为其本身并没有太多的帮助.只需要记住这些定理的结果,覆叠空间的基本性质就被拿捏了.
\begin{definition}
    称空间$B$是局部半单连通的,若任意$b \in B$都有一个邻域$U$使得$\pi_1(U,b) \to \pi_1(B,b)$是平凡同态.
\end{definition}
该定义显然是万有覆叠存在的必要条件.因为其基本邻域与$E$中某开集同胚,而该开集到$B$的映射将所有道路同伦类变为$1$.

但实际上这个条件保证了万有覆叠的存在:
\begin{theorem}
    若$B$连通,局部道路连通,局部半单连通,则$B$有万有覆叠.
\end{theorem}
\begin{proof}
    只说一句,考虑广群,我们起码会知道这个万有覆叠空间的点都是$b$到其他点的道路同伦类.问题的难点在于拓扑的构造.因为代数只能得到广群,广群不给拓扑是形成不了空间的.
\end{proof}

\begin{lemma}
    设$X$是$G$空间,则轨道范畴到$\mathrm{Top}$有函子$X/(-)$.
\end{lemma}
\begin{proof}
    $G/H$将被映射为$X/H$.若$G/H \to G/K$有$G$映射$\alpha$,则$X/H \to X/K$有连续函数$\alpha$:$\alpha(Hx)=K\gamma^{-1}x$.其中$\alpha(H)=\gamma K$.
\end{proof}
\begin{proposition}
    设$p:E \to B$是覆叠,$\mathrm{Aut}(E)$传递的作用在$F_b$上.则覆叠是正则的,并且$E/\mathrm{Aut}(E)$同胚于$B$.
\end{proposition}
\begin{proof}
    
\end{proof}
\begin{theorem}
    对于$B$,选择基点$b$,记$G=\pi_1(B,b)$.则存在一个函子$E(-)$:
    $$
    E(-):\mathcal{O}(G)\to \mathrm{Cov}(B)
    $$
    使得上述两个范畴是等价范畴.对于$G$的子群$H$,覆叠$p:E(G/H) \to B$有一个典范的基点$e$,使其处于$b$的纤维中,并且:
    $$
    p_*(\pi_1(E(G/H),e))=H
    $$

    此时,$F_b=G/H$,并且对于映射$\alpha:G/H \to G/K$,把$E(\alpha)$限制在$F_b$正好是$\alpha$本身.
\end{theorem}
因此$\pi_1(B,b)$实际上承担了许多责任.下面有典范的范畴等价:
\begin{tikzcd}
	& {\mathcal{O}(\pi_1(B,b))} \\
	{\mathrm{Cov}(B)} && {\mathrm{Cov}(\Pi(B))}
	\arrow["{E(-)}"', from=1-2, to=2-1]
	\arrow["{\mathcal{E}(-)}", from=1-2, to=2-3]
	\arrow["\Pi", from=2-1, to=2-3]
\end{tikzcd}


\chapter{紧致生成空间——CGWH空间}
拓扑空间繁多复杂.许多奇怪的拓扑并不违背点拓的规则,但是于具体问题无济于事.代数拓扑对空间进行了若干要求,以保证研究问题的流畅性以及理论的协调性.


\textbf{注意:若无特殊说明,之后约定所谓“紧空间”是指紧且Hausdorff的空间}.

\section{紧致生成空间的定义}
\begin{definition}
    $X$被称作弱Hausdorff的,若对于任何紧空间$K$和连续映射$g:K \to X$,$g(K)$是$X$的闭子集.
\end{definition}
\begin{lemma}
    当$X$是弱Hausdorff空间的时候,$g(K)$也是紧空间.
\end{lemma}
\begin{proof}
    紧性显然.问题是验证Hausdorff性质.如果$g$是闭映射,那么容易知道$T_4$空间是闭连续保持的,因而$g(K)$是Hausdorff空间.$g$是闭映射,因为$F \subset K$是闭集,则$F$也是紧Hausdorff空间,因此$g(F)$是$X$中闭集.而$g(K)$也是闭集,则$g(F)$是$g(K)$中闭集.
\end{proof}

弱Hausdorff性质是介于$T_1$和Hausdorff之间的分离性质.显然Hausdorff能推得弱Hausdorff,而$T_1$要求单点是闭集.由于单点必然是紧Hausdorff的,因此其必然是闭集.

\begin{definition}
    $X$的子空间$A$被称作是紧致闭的,若对于任意紧空间$K$和$g:K \to A$,$g^{-1}(A)$是$K$中的闭集.若$X$本身是弱Hausdorff的,那么该条件等价于$A$与$X$的任意紧子集相交都是闭集.
\end{definition}
\begin{proof}
    
\end{proof}
\begin{definition}
    $X$是k空间,若每个紧致闭的子集都是闭集.注意这里的紧致闭集不是闭+紧,是上面的定义.我们约定使用紧致且闭来表达闭+紧的含义.
\end{definition}

\begin{definition}
    $X$是紧致生成的空间,若其是一个弱Hausdorff的k空间.
\end{definition}
\begin{example}
    任何一个局部紧,第一可数的空间是紧致生成的空间.任何弱Hausdorff且第一可数的空间是紧致生成的空间.
\end{example}
\begin{proof}
    
\end{proof}
\begin{lemma}
    若$X$是紧致生成的空间,$Y$是任意空间,那么函数$f:X \to Y$是连续的当且仅当对于$X$的任何紧子集$K$,$f|_K$是连续的.
\end{lemma}
\begin{proof}
    
\end{proof}
引理有力的说明了紧致生成空间名字的由来:这个空间的拓扑和紧集有一种相容.所有的性质只由紧集决定.很多时候我们遇到的空间不是$k$空间,但是经过以下处理可以得到$k$空间.
\begin{proposition}
    设$X$是一个空间.给$X$赋予一个新的拓扑:$A$是$X$中的闭集,当且仅当其在原拓扑中是紧致闭的.这个拓扑的定义是良定的.
\end{proposition}
\begin{proof}
    
\end{proof}
我们把上述操作称为$k$化,得到的空间记为$kX$.从而有自然恒等映射$kX \to X$.这个映射显然是连续的.
\begin{proposition}
    若$X$是弱Hausdorff空间,则$kX$也是弱Hausdorff空间.从而弱Hausdorff空间的$k$化空间是紧致生成空间.
\end{proposition}
\begin{proof}
    回忆弱Hausdorff空间的定义.设$g:K \to kX$是连续函数且$K$是紧空间.我们想要证明$g(K)$是闭集.注意到$g(K)$作为$X$的子集来说是闭集($\mathrm{id}\circ g$同样是连续函数,且$\mathrm{id}\circ g(K)=g(K)$),则$g(K)$显然是$kX$中的闭集.
\end{proof}
\begin{proposition}
    $X$和$kX$拥有完全相同的紧子集.
\end{proposition}
\begin{proof}
    
\end{proof}

现在我们考虑弱Hausdorff和Hausdorff的相似之处.用$X \times_c Y$表示$X$和$Y$的一般乘积空间,用$X \times Y$表示$X \times _c Y$的$k$化.
\begin{proposition}
    若$X$和$Y$都是弱Hausdorff空间,则$X \times Y=kX \times kY$.若$X$是局部紧空间且$Y$是紧致生成空间,则$X \times Y=X \times_c Y$
\end{proposition}
\begin{proof}
    
\end{proof}
\begin{lemma}
    设$X$是$k$空间,那么$X$是弱Hausdorff空间当且仅当$\Delta X=\{(x,x)\} \subset X \times X$是闭集.
\end{lemma}
\begin{proof}
    
\end{proof}
\section{紧致生成空间范畴}
我们先说明本章的大标题——CGWH(compactly generated weak Hausdorff).由于不同书对紧致生成空间,紧的定义有区别.如果不认为紧致=紧+Hausdorff,那么我们也没有理由认为紧致生成蕴含着弱Hausdorff.紧致生成空间有另外的定义:
\begin{definition}[不作为本笔记的一般定义]
    称一个空间是紧致生成空间,若$A\subset X$是闭集当且仅当对于$X$的任何紧子集$X$,$A\cap K$都是闭集.称一个空间是豪斯多夫紧致生成空间(k空间),若$A\subset X$是闭集当且仅当对于$X$的任何紧豪斯多夫子集$X$,$A\cap K$都是闭集.称一个空间是紧致生成豪斯多夫空间,若其是紧致生成的,并且豪斯多夫.
\end{definition}
因而在这种前提下,CGWH空间,即紧致生成弱Hausdorff空间应该有定义:豪斯多夫紧致生成且弱Hausdorff的空间.这一个定义与之前的定义是一致的,即在弱Hausdorff的条件下:
\begin{proposition}
    若$X$是弱Hausdorff空间,则$X$是$k$空间当且仅当其的每个紧致闭集合都是闭集.
\end{proposition}

暂且不表.以后我们记录紧致生成的时候均代表CGWH空间.

一般的拓扑范畴$\mathrm{Top}$有一些不好的性质.比如商空间中.但考虑紧致生成空间可以避免:
\begin{proposition}
    设$X$是紧致生成空间,$\pi:X \to Y$是商映射,那么$Y$是紧致生成的当且仅当$(\pi \times \pi)^{-1}(\Delta Y)$是$X \times X$中的闭集.
\end{proposition}
可以对该命题做如下通俗的解释:一个紧致生成空间的商空间,若通过一个“闭的等价关系”诱导,则是紧致生成的空间.

\begin{proposition}
    若$X$和$Y$是紧致生成空间,$A$是$X$的闭子集,并且$f: A \to Y$是任意连续映射,则推出空间$Y \cup_{f} X$是紧致生成的空间.
\end{proposition}

\begin{proposition}
    若$\{X_i\}$是一系列紧致生成空间,并且包含映射$X_i\to X_{i+1}$均为闭的像集,则余极限$\Colim X_i$是紧致生成的.
\end{proposition}

接下来我们给出若干范畴的定义.这样是为了更好的说明CGWH空间的优点.
\begin{definition}
    重新定义$\mathrm{Top}$为所有紧致生成空间的范畴,$\mathrm{Top*}$为对应的带基点的范畴.记$w\mathrm{Top}$是所有弱Hausdorff空间的范畴,从而我们有函子:$k:w \mathrm{Top} \to \mathrm{Top}$和遗忘函子$j:\mathrm{Top}\to w\mathrm{Top}$.
\end{definition}
\begin{proposition}
    $(j,k)$是伴随函子.即$k$是$j$的右伴随,$j$是$k$的左伴随.
\end{proposition}
\begin{proof}
    
\end{proof}
注意到$w\mathrm{Top}$中,极限存在.即按照最一般的拓扑空间来看,弱Hausdorff空间的极限仍然是弱Hausdorff的.但是k空间不具有这样的性质.由于$k$是右伴随,因此保极限.从而我们可以先作遗忘函子,然后得到极限,再作用$k$.这样就可以得到$\mathrm{Top}$中的极限.这样的极限做法在集合上没有变化,唯一要注意的是拓扑发生的变化.

\begin{definition}
    对于两个紧致生成空间$X$,$Y$,记$Y^X$表示从$X$到$Y$的所有连续函数组成的集合.其拓扑由标准的紧开拓扑k化得到.所谓紧开拓扑,是指由下列集合作为子基组成生成的拓扑:
    $$
    \{U\subset Y,K \subset X|\{f|f(K)\subset U\}\}
    $$
    其中$U$是$Y$的开集,$K$是$X$的紧子集.
\end{definition}
\begin{proposition}[乘积空间的伴随]
    对于紧致生成空间$X,Y,Z$,典范的双射:
    $$
    Z^{X \times Y}\cong (Z^Y)^X
    $$
    是一个同胚.
\end{proposition}

从下节开始,所有的空间都被认为是紧致生成空间(CGWH),由此我们不再论述其前提.关于CGWH空间范畴,可以见Wiki.有详细的性质论述.
\chapter{上纤维}
纤维与上纤维是对偶的两组概念,其中的证明很多都是可以通过改变箭头方向,及乘积为指数等办法而复制证明.我们只证明对偶中的一个,另一个在纸上会做验证.

\section{上纤维的定义}
\begin{definition}
    称一个连续函数$i:A \to X$是上纤维,若其满足同伦拓展性质(homotopy extension property:HEP).这意味着若下列交换图中,若已经有:$h \circ i_0=f \circ i$,则存在$\tilde{h}$使得该图交换.
\[\begin{tikzcd}
        A && {A\times I} \\
        \\
        X && {X \times I} \\
        &&& Y
        \arrow["i", from=1-1, to=3-1]
        \arrow["{i_0}", from=3-1, to=3-3]
        \arrow["{i_0}", from=1-1, to=1-3]
        \arrow["{i\times \mathrm{id}_I}"', from=1-3, to=3-3]
        \arrow["f", from=3-1, to=4-4]
        \arrow["h", from=1-3, to=4-4]
        \arrow["{\exists \tilde{h}}"{description}, dashed, from=3-3, to=4-4]
    \end{tikzcd}
\]
\end{definition}
简单地说,如果$A$到$Y$有同伦,并且起点函数可以延拓为整个$X$的函数,那么整个同伦都可以被提升.

我们并不要求$\tilde{h}$具有唯一性.事实上其很多时候都不唯一.这个图也可以改写为紧凑的形式:\begin{tikzcd}
	A && {Y^I} \\
	\\
	X && Y
	\arrow["i", from=1-1, to=3-1]
	\arrow["f", from=3-1, to=3-3]
	\arrow["{p_0}"', from=1-3, to=3-3]
	\arrow["{\tilde{h}}"{description}, dashed, from=3-1, to=1-3]
	\arrow["h"', from=1-1, to=1-3]
\end{tikzcd} 原因是$A \times I \to Y$与$A \to Y^I$是同胚的.$p_0(\xi)=\xi(0)$.

\begin{lemma}
    上纤维映射的推出也是上纤维.即对于$i: A \to X$是上纤维,$g:A \to B$是连续函数,那么诱导映射$B \to B\cup_g X$也是上纤维.
\end{lemma}
\begin{proof}
    交换图表道尽一切:
    \[\begin{tikzcd}
            & {} \\
            A &&&& {A \times I} \\
            & B && {B\times I} \\
            && Y \\
            & {B \cup_g X} && {(B\cup_g X)\times I} \\
            X &&&& {X \times I}
            \arrow[from=3-2, to=5-2]
            \arrow[from=3-4, to=5-4]
            \arrow[from=3-2, to=3-4]
            \arrow[from=5-2, to=5-4]
            \arrow[from=5-2, to=4-3]
            \arrow[from=3-4, to=4-3]
            \arrow[from=2-1, to=3-2]
            \arrow[from=2-1, to=6-1]
            \arrow[from=6-1, to=5-2]
            \arrow[from=6-1, to=6-5]
            \arrow[from=2-1, to=2-5]
            \arrow[from=2-5, to=6-5]
            \arrow[from=6-5, to=5-4]
            \arrow[from=2-5, to=3-4]
            \arrow[curve={height=-18pt}, dashed, from=6-5, to=4-3]
            \arrow[dashed, from=5-4, to=4-3]
        \end{tikzcd}
    \]

    事实上,弯曲的虚线是$A \to X$的上纤维定义.直的虚线来自于$(B\cup_g X) \times I=(B\times I)\cup_{g\times \mathrm{id}_I} (X \times I)$,从而右边的题型是推出,因此根据推出的性质有该直虚线.
\end{proof}
\section{映射柱}
映射柱和映射锥是相当重要的概念.其来自于上纤维.对于映射$i:A\
to X$,定义其映射柱$Mi$是$i$和$A \to A \times I$的推出:
$$
Mi\equiv X\cup_i(A \times I)
$$
一般我们把$a \mapsto (a,0)$,意味着粘的地方是$A \times I$的底层(0层).

如果交换图成立,我们立马就可以意识到上纤维的成立:
\begin{tikzcd}
	A && {A\times I} \\
	& Mi \\
	X && {X \times I}
	\arrow[from=1-1, to=3-1]
	\arrow[from=3-1, to=2-2]
	\arrow[from=1-1, to=1-3]
	\arrow[from=1-3, to=2-2]
	\arrow[from=3-1, to=3-3]
	\arrow[from=1-3, to=3-3]
	\arrow["j", shift left=1, from=2-2, to=3-3]
	\arrow["r", shift left=1, dashed, from=3-3, to=2-2]
\end{tikzcd}

因为从$X \times I$的映射复合$Mi$到$Y$的映射即可.原因是$Mi$是推出,从而只要满足条件的$Y$存在,那么从$Mi$到$Y$就有唯一映射.

由于$X \times I$本身也是$Y$的一种类型,因此$Mi$到$X \times I$也有唯一的映射$j$.我们可以考虑$j$是怎么射的:$j(a,t)=(i(a),t)$,$j(x)=(x,0)$.注意到$r \circ j$是从$Mi$到$Mi$出发的映射,其满足推出的交换图.从而根据唯一性$r \circ j=\mathrm{id}_{Mi}$.这似乎提示我们$Mi$与$X \times I$在同伦上有神秘的关系.暂且不表.

下面的引理告诉我们在大多数情况下我们可以把上纤维看作映入映射,并且$A$是$X$的闭子集.
\begin{lemma}
    设$i:A \to X$是上纤维映射,则$i$是$i:A \to i(A)$同胚,并且$i(A)$是$X$的闭集.
\end{lemma}
\begin{proof}
    首先说明$i$是单射.$(a,t) \mapsto (i(a),t) \mapsto r(i(a),t) =(a,t), \forall t \in [0,1]$.这说明$r(i(a),tv)=(a,t)$.因此$i$是单射.

    其次要说明这是从$A$到$i(A)$的同胚.不妨考虑$A \times \{1\}$,其到$Mi$的像为$A \times \{1\}$.如果走$X \times I$,则$r(i(A)\times \{1\})$.于是$r(i(A)\times \{1\})=A \times \{1\}$这表明$i$有连续的逆($r$略做限制即可).因而为同胚.

    最后说明$i(A)$为闭集.我们直接假定$A$是$X$的子集.此时对于$A$的闭包$\overline{A}$,我们有:
    $$
    r(\overline{A}\times \{1\})=r(\overline{A\times \{1\}})\subset \overline{r(A \times \{1\})}=\overline{A \times \{1\}}=A \times \{1\}
    $$

    最后的等号是因为$A \times \{1\}$是$Mi$的子集.因此存在$\rho:\overline{A}\to \overline{A}$,满足$\rho(a)=a,\forall a \in A$.并且$\rho(\overline{A})=A$.

    $A$是CGWH空间.经过验证$\overline{A}$也是.则$\Delta \overline{A}$是$\overline{A} \times \overline{A}$的闭集.定义$\mathrm{id}_{\overline{A}} \times \rho:\overline{A} \to \overline{A}\times \overline{A}$.则$\Delta \overline{A}$的原像集是$A$.从而这是一个闭集.
\end{proof}
\section{上纤维的替换性}
\begin{theorem}
    任何一个映射$f:X \to Y$都可以分解为一个上纤维映射和一个同伦于$\mathrm{id}$的复合.
\end{theorem}
\begin{proof}
    考虑分解:
    $$
    j:X \to Mf   \qquad \qquad r:Mf \to Y \qquad \qquad f=r \circ j
    $$
    其中,$j(x)=(x,1)$,$r(y)=y$,$r(x,s)=f(x)$.我们断言$r$是同伦等价.事实上,考虑$Y \to Mf$的嵌入映射$i$,则$r \circ i:Y \to Y$是恒等映射,$i \circ r$与恒等映射同伦.可以定义$h:Mf \times I \to Mf$:
    $$
    h(y,t)=y, \qquad \qquad h((x,s),t)=(x,(1-t)s)
    $$

    剩下的是说明$j$是上纤维.即$Mj=(X \times I)\cup_j Mf=Mf$是$Mf \times I$的形变收缩核.这再显然不过了,直接压缩即可.
\end{proof}
\section{判定准则}
本节的内容实际意义不大.但我们还是做记录.我们试图寻找一个准则来判定上纤维.测试图当然是一个可行的办法.另外,判定从$X \times I$到$Mi$有一个形变收缩也是可行的办法.但是这样的办法还不足够.

考虑对$(X,A)$.上纤维对是那些与$X/A$行为“同调”的空间对.(原文:Cofibration pairs will be those pairs that “behave homologically” just like the associated
quotient spaces $X/A$).

\begin{definition}
    一个空间偶对$(X,A)$被称为是一个NDR对(neighbourhood deformation retract pair,可简单翻译为邻域形变收缩对),若存在映射$u:X \to I$使得$u^{-1}(0)=A$,以及同伦$h: X \times I \to X$使得$h_0=\mathrm{id}$,$h(a,t)=a,\forall a \in A,t \in I$,$h(x,1)\in A,\forall u(x)<1$.称$(X,A)$是DR对,如果$u(x)<1,\forall x \in X$,此时$A$是$X$的一个形变收缩.
\end{definition}
看得出来$NDR$的定义和我们想象的,邻域可以收缩到$A$是一致的.这里的$A$显然还是闭集.

\begin{lemma}
    若$(h,u)$和$(j,v)$代表$(X,A)$和$(Y,B)$作为NDR对的两个映射,则$(k,w)$代表$(X\times Y,X \times B \cup A \times Y)$作为NDR对的两个映射.在这里$w(x,y)=\min(u(x),v(y))$,
    $$
    k(x,y,t)=(h(x,t),j(y,tu(x)/v(y))),\quad \mathrm{if} \quad v(y)\geq v(x)
    $$
    $$
    k(x,y,t)=(h(x,tv(y)/u(x)),j(y,t)),\quad \mathrm{if} \quad v(y)\leq v(x)
    $$
    其次,若其中有一个对是DR的,那么乘积对也是DR的.
\end{lemma}
\begin{proof}
    同伦的表达式总有一种把人拒之门外的痛苦感.但其实不用担心,我们要理解的是这样复杂的表达式是怎么构造出来的.

    显然$w(x,y)$的零点意味着$u(x)$和$v(y)$中至少一个为$0$.因此此时$(x,y)\in X \times B\cup A \times Y$.同样后者也能推出$w(x,y)=0$.因此$w(x,y)$合理.

   对于$k$.一个一个考虑.若$t=0$,则$k(x,y,0)=(h(x,0),j(y,0))=(x,y)$.若$ (x,y)\in X \times B \cup A \times Y$,则至少有一个为$0$.不妨设$u(x)=0 \leq v(y)$,则$k(x,y,t)=(h(x,t),j(y,0))=(x,y)$.若$t=1$,$w(x,y)<1$,即有一个$u(x)$或者$v(y)$小于$1$,此时$k(x,y,1)$显然也在收缩核里面.

   因此得证(连续性显然)

   如果有一个是
\end{proof}
由引理我们知道
\begin{theorem}
    设$A$是$X$的闭子集,则下面命题等价:
    \begin{enumerate}
        \item $(X,A)$是NDR对.
        \item $(X\times I,X \times \{0\}\cup A \times I)$是DR对.
        \item $X \times\{0\}\cup A \times I$是$X \times I$的收缩.
        \item $A$到$X$的嵌入映射是上纤维.
    \end{enumerate}
\end{theorem}
\begin{proof}
    事实上我们只需要证明前两个的等价性.因为2,3,4显然等价.如果$(X,A)$是NDR对,那么$(X\times I,X \times \{0\}\cup A \times I)$是DR对.因为$(I,\{0\})$是DR对.$u(t)=\dfrac{1}{2}t$.

    如果$(X\times I,X \times \{0\}\cup A \times I)$是DR对,则我们直接考虑这是一个形变收缩核.
    $$
    r:X \times I \to X \times \{0\}\cup A \times I
    $$
    考虑$X\times I$投射在$X$和$I$上的映射为$\pi_1$和$\pi_2$.定义$u:X \to I$:
    $$
    u(x)=\mathrm{sup}\{t-\pi_2r(x,t)|t \in I\}
    $$
    以及$h:X \times I \to X$:
    $$
    h(x,t)=\pi_1r(x,t)
    $$
    可以验证$(h,u)$就对应为$(X,A)$的NDR对.若$x \in A$,显然$u(x)=0$.若$u(x)=0$,考虑$r(x,0)=(x,0)$,从而$u(x) \geq 0$.因此$\forall t \in I,\pi_2r(x,t)\geq t$.如果本身$t>0$,则$\pi_2r(x,t)>0$,因此$r(x,t)\in A\times I$.连续性告诉我们$t=0$时,$r(x,0)\in \overline{A \times I}=A \times I$.由于$r(x,0)=(x,0)$,所以$x \in A$. 
\end{proof}
\section{上纤维同伦等价}
我们引入新的概念:在集合$A$下的同伦:
\begin{definition}
    称资料$(X,i:A\to X)$是在A下的一个空间.两个这样的空间之间的连续映射$f$是指满足交换图的连续映射:
    \begin{tikzcd}
        & A \\
        X && Y
        \arrow["i"', from=1-2, to=2-1]
        \arrow["j", from=1-2, to=2-3]
        \arrow["f"', from=2-1, to=2-3]
    \end{tikzcd}
    
\end{definition}
这个定义倒是很范畴.定义对象,定义态射是满足交换的映射.
\begin{definition}
    设$(X,i)$,$(Y,j)$是两个以A为底的空间.$f$,$f'$是$X$到$Y$的两个映射.若存在同伦$H$连接$f$和$f'$并且$H(\cdot,t)$对于任何$t$来说都是$(X,i)$和$(Y,j)$的映射,那么称这是以$A$为底的同伦.
\end{definition}
一般情况下,如果两个空间只有一般的同伦等价,不一定能满足以$A$为基底的同伦等价.然而这个问题被上纤维解决:
\begin{proposition}
    设$i:A \to X$,$j:A \to Y$都是上纤维.设$f:X \to Y$是同伦等价映射,并且满足$f \circ i=j$.那么$f$是以$A$为基底的同伦等价.
\end{proposition}
\begin{proof}
    
\end{proof}
\begin{example}
设$i:A \to X$是上纤维.熟知的交换图:\begin{tikzcd}
	& A \\
	Mi && X
	\arrow["j"', from=1-2, to=2-1]
	\arrow["i", from=1-2, to=2-3]
	\arrow["r"', from=2-1, to=2-3]
\end{tikzcd}
中,$r$是同伦等价.其同伦逆为$\iota:X \to Mi$,$\iota(x)=(x,0)$.因为$r(i(x))=x$,$i(r(x,0))=(x,0)$,$i(r(a,t))=(a,0)$.从而$i \circ r$是把整个$A \times I$啪的一下拍扁在$X$面上.自然是同伦于$\mathrm{id}_{Mi}$.

然而$\iota$并不能满足上述交换图.即$\iota(i(x))=(x,0)$,$j(x)=(x,1)$.因此这不是一个以$A$为底的同伦逆.

然而根据定理,$i$和$j$都是上纤维,从而$r$只要是同伦等价,就是以$A$为基底的同伦等价.这意味着其有同伦逆满足交换图成立——尽管我们不一定知道具体的表达.

\end{example}
\begin{proposition}
    我们给定一个交换图:
    \[\begin{tikzcd}
            A && B \\
            \\
            X && Y
            \arrow["i"', from=1-1, to=3-1]
            \arrow["d", from=1-1, to=1-3]
            \arrow["j", from=1-3, to=3-3]
            \arrow["f"', from=3-1, to=3-3]
        \end{tikzcd}
    \]

    其中$i,j$都是上纤维映射,$d,f$是同伦等价.那么$(f,d):(X,A)\to (Y,B)$是空间偶对的同伦等价.
\end{proposition}
\begin{proof}
\end{proof}
\section{习题}
本节的习题意义非凡,因此做摘录如下:
\begin{proposition}
    设$i:A \to X$是上纤维,$A$是一个可缩空间(与单点集合同伦的空间).试证明商映射$X \to X/A$是同伦等价.
\end{proposition}
\begin{proof}
    
\end{proof}
\chapter{纤维}
\section{纤维的定义}
纤维的定义与覆叠空间类似.但其与上纤维呈现一个完美的对偶关系:
\begin{definition}
    满射$p:E \to B$是纤维,若其满足同伦提升性质(HLP).即若交换图:\begin{tikzcd}
        Y && E \\
        \\
        {Y\times I} && B
        \arrow["{i_0}", from=1-1, to=3-1]
        \arrow["f"', from=1-1, to=1-3]
        \arrow["p"', from=1-3, to=3-3]
        \arrow["h", from=3-1, to=3-3]
        \arrow["{\tilde{h}}"{description}, dashed, from=3-1, to=1-3]
    \end{tikzcd}
    
    中的$h\circ i_0=p\circ f$,则存在$\tilde{h}$使得上述交换图成立.
\end{definition}
我们也可把上面的交换图改写为另外的形式.只需要记住$\mathrm{Hom}(X \times I,Y)\cong \mathrm{Hom}(X,Y^I)$
\[
    \begin{tikzcd}
        E &&& {E^I} \\
        && Y \\
        \\
        B &&& {B^I}
        \arrow["p"', from=1-1, to=4-1]
        \arrow["{p_0}"', from=4-4, to=4-1]
        \arrow["h"', from=2-3, to=4-4]
        \arrow["f", from=2-3, to=1-1]
        \arrow["{\tilde{h}}"{description}, dashed, from=2-3, to=1-4]
        \arrow["p", from=1-4, to=4-4]
        \arrow["{p_0}", from=1-4, to=1-1]
    \end{tikzcd}\]

这里$p_0(\xi)=\xi(0)$.
\begin{lemma}
    如果$p:E \to B$是纤维,$g:A \to B$是任何映射,则两者的拉回$E \times_g A \to A$也是纤维映射.
\end{lemma}
这个引理自然就从略证明了.只要简单画图即可.

除了上述万有测试图外,我们当然可以构造特殊的测试图.记$Np\equiv E \times_p B^I$.$Np$是映射$p_0:B^I \to B$和$p:E \to B$的拉回.如果把上述图的$Y$换为$Np$,则$f$和$h$都是相应的投射.

如果有映射$s:Np \to E^I$满足$k \circ s=\mathrm{id}$,$k:E^I \to Np$是终对象对应的映射,则此时$s$被称为一个道路提升函数.从而:
$$
s(e,\beta)(0)=e \quad \quad \quad p \circ s(e,\beta)=\beta
$$

同理,$s$的存在就可以保证纤维的存在.
\begin{lemma}
    若$p:E\to B$是一个覆叠空间,则$p$是纤维,并且有唯一的提升函数.
\end{lemma}
\begin{proof}
    给定$Np$的某个点$(e,\beta)$,注意到$\beta(0)=p(e)$,于是相当于给定了$E$中一个点,和$B$上一条道路.此时为了使得$k \circ s=\mathrm{id}$,则$E^I$中的对应元素实际上是$\beta$的提升,并且起点是$e$.这样的提升当然只有一个(覆叠空间的性质).

    因此提升函数存在就唯一.我们要说明$s$的连续性.即$s(e,\beta)$是$\beta$的提升,该提升以$e$为起点.当然这一点现在很难说明,因为我们实际上不熟悉$E^I$的拓扑.只能默认其暂时正确.
\end{proof}
\begin{proposition}
    若$i:A \to X$是上纤维,$B$是一个空间,则$p=B^i:B^X \to B^A$是一个纤维.
\end{proposition}
\begin{proof}
    $B^{Mi}=B^{X\times\{0\} \cup A \times I}c\cong B^X \times_p (B^A)^I=Np$

    若$r:X \times I \to Mi$是一个收缩,则:
    $$
    B^r:Np \cong B^{Mi} \to B^{X \times I}\cong (B^X)^I
    $$

    因此我们诱导出了一个道路提升函数.
\end{proof}
同理,如果$p:E \to B$是一个纤维,那么$p*:B^X \to E^X$也是上纤维(尚未真正考虑)

与上纤维相同,同样可以把任何一个映射分解为一个同伦等价和一个纤维的复合.我们考虑$Nf=X \times_f Y^I$.从而$f$与下面的映射相同:
$$
X \to Nf \to Y
$$
其中$x \mapsto (x,c_{f(x)}),(x,\xi) \mapsto \xi(1)$.先说明$X \to Nf$是一个同伦等价.设这个映射为$\mu$,$\tau:Nf \to X$是投射.我们断言两个映射互为同伦逆.显然$\tau \circ \mu$是$\mathrm{id}_X$,而$\mu \circ \tau$将$(x,\xi)$映射为$(x,c_{f(x)})$.

注意到$\xi(0)=f(x)$.从而定义$h: Nf \times I \to Nf:h((x,\xi),t)=(x,\xi_t)$.其中$\xi_t(s)=\xi((1-t)s)$.$h$是符合要求的同伦映射.

接着我们说明$\omega:Nf \to Y$是纤维.首先这是一个满射.对于任意$y$,我们需要找到$x\in X$使得$f(x)$与$y$在一条道路上.我们默认$Y$道路连通,从而这条道路存在.

接着说明HLP性质.我们验证测试图:
\begin{tikzcd}
	A && Nf \\
	\\
	{A \times I} && Y
	\arrow["\omega"{description}, from=1-3, to=3-3]
	\arrow["{i_0}"{description}, from=1-1, to=3-1]
	\arrow["g"{description}, from=1-1, to=1-3]
	\arrow["h"{description}, from=3-1, to=3-3]
	\arrow["{\tilde{h}}"{description}, dashed, from=3-1, to=1-3]
\end{tikzcd}

考虑$g(a)=(g_1(a),g_2(a))$.显然$g_1(a)$是$X$中的点,$g_2(a)$是$Y$上的道路.$g_2(a)(0)=f(g_1(a))$.从而交换图告诉我们$h(a,0)=g_2(a)(1)$.

现在直接构造$\tilde{h}:A \times I \to Nf$.令$\tilde{h}(a,t)=(g_1(a),j(a,t))$,并且$j(a,t)$的表示式为:
$$
j(a,t)(s)=g_2(a)(s+st) \quad \quad ,\mathrm{if} \quad 0\leq s \leq 1/(1+t) 
$$
$$
j(a,t)(s)=h(a,s+ts-1) \quad \quad ,\mathrm{if} \quad 1/(1+t)\leq s \leq 1
$$

注意这里的构造是怎么写出来的.首先$f_1(a)$是简单化的处理,其次,我们要构造$j(a,t)$这个道路,使得$j(a,0)$是$g_2(a)$,$j(a,t)(1)$是$h(a,t)$.等于说开头固定好道路,并且要求这样的道路变化在末尾以某种方式变化.

于是我们把这两个道路连接起来,并且要求新的道路在$[0,1]$里面沿着这条长道路,走到不同的位置即可.

\section{纤维判定准则}
不幸的是,在这里上纤维的对偶效果失败了,我们没办法使用对偶的方式来判定一个映射是否是上纤维.换言之,我们决定使用覆叠的理论来判定.

考虑$B$的开覆盖$\mathcal{O}$.如果$p$是覆叠,我们选取典范的开覆盖,从而对于$U \in \mathcal{O}$,投射$U \times F \to U$同胚于$p$,$F$是一个离散集.

这引导出丛的概念(bundle).
\begin{definition}
    映射$p:E \to B$是一个丛映射,若给定$B$的开覆盖$\mathcal{O}$,存在同胚$\phi: U \times F \to p^{-1}(U)$,使得$p \circ \phi=\pi_1$,(即$p(\phi(b,f))=b$).这里$F$是一个给定的拓扑空间.

    我们要求$\mathcal{O}$是可数的,意味着存在若干连续函数$\lambda_U:B \to I$使得$\lambda_U^{-1}(0,1]=U$,并且$\mathcal{O}$是局部有限的.
\end{definition}
 
当然,任何仿紧空间的开覆盖都有可数的开加细.

丛的定义说明覆叠就是一个丛.其次,覆叠空间本身是纤维,因此我们会关注纤维和丛的关系.
\begin{theorem}
    设$p:E \to B$是映射且$\mathcal{O}$是可数的开覆盖,则$p$是纤维,当且仅当$p:p^{-1}(U)\to U$是纤维对于$\forall U \in \mathcal{O}$都成立.
\end{theorem}
\begin{proof}
    
\end{proof}
\section{纤维同伦等价}
这部分的内容与上纤维完全一致,不写了.(喜)
\section{纤维变换}
在覆叠空间中,$F_b$和$F_{b'}$之间的函数由$b \to b'$的道路诱导.纤维导出了一个类似的理论,但是只是在同伦的情况下.

设$p:E \to B$是一个纤维,$b \in B$,$i_b:F_b \to E$是嵌入映射.对于道路$\beta:I \to B$,$\beta(0)=b,\beta(1)=b'$,则同伦提升给出了:
\begin{tikzcd}
	{F_b \times\{0\}} && E \\
	\\
	{F_b \times I} & I && B
	\arrow[from=1-1, to=3-1]
	\arrow["{i_b}"{description}, from=1-1, to=1-3]
	\arrow[from=1-3, to=3-4]
	\arrow["{\pi_2}"{description}, from=3-1, to=3-2]
	\arrow["\beta"{description}, from=3-2, to=3-4]
	\arrow["{\tilde{\beta}}"{description}, dashed, from=3-1, to=1-3]
\end{tikzcd}中的$\tilde{\beta}$.

我们来详细考量这个$\tilde{\beta}$.注意到$\tilde{\beta}(\cdot,t)$把$F_b$的点全部映射到$F_{\beta(t)}$.因而若$t=1$:
$$
\tilde{\beta_1}:F_b \to F_{b'}
$$
该映射被称为由$\beta$诱导的纤维变换.

\begin{proposition}
    对于同伦的$\beta$和$\beta'$,$\tilde{\beta_1}$和$\tilde{\beta_1'}$也是同伦映射.因而记$\tilde{\beta_1}$的同伦类为$\tau[\beta]$.其中$[\beta]$是一个同伦类.
\end{proposition}
\begin{proof}
    
\end{proof}

处处都是同伦!我们苦恼着引入了新的范畴——同伦范畴:$h\mathrm{Top}$.该范畴的特点是,里面的对象是仍然是拓扑空间,但是态射都是同伦等价类.而两个空间在该范畴下同构意味着两个空间同伦.

因此在该范畴下,$\tau[c_b]=[\mathrm{id}]$,$\tau[\gamma \circ \beta]=\tau[\gamma]\circ \tau [\beta]$.并且$\tau[\beta]$自然有逆$\tau[\beta^{-1}]$.

\begin{theorem}
    对于纤维而言,$B$中的道路同伦类被提升为纤维变换的同伦类,这实际上说明了一个函子$\lambda:\Pi(B)\to h\mathrm{Set}$.因此,若$B$是道路连通的,那么$B$中的任何两个纤维都是同伦等价的.
\end{theorem}

\chapter{带基点上纤维和纤维的正合列}
我们将使用带基点空间的纤维和上纤维来生成两个空间正合列.之所以要在带基点空间做这样的工作,是因为虽然上纤维的情况可以简单的推广到不带基点的空间,但是纤维则很困难.

值得一提的是,本节的大部分命题都是“拓扑结论”.我们不会太花费精力于上,尽量以陈述为主.
\section{带基点的同伦映射类}
对于带基点的空间$X,Y$.我们默认其基点为$x_0,y_0$(省略后续的记号).记$[X,Y]$为带基点映射的同伦等价类集合.这个集合有一个自然的基点,其代表元为$f:X \to Y,f(X)=y_0$.$y_0$是$Y$的基点.

接下来我们要考虑的带基点空间范畴.有:
\begin{definition}
    对于带基点空间$X,Y$,定义其Smash积$X \wedge Y$为空间$X \wedge Y=X \times Y/X \vee Y$.其中$X \vee Y$被看作$X \times Y$的子空间.$X\wedge Y$的基点是$(x_0,y_0)$所在的等价类.
\end{definition}
我们接下来的陈列的命题都是不给出证明的事实:
\begin{proposition}
    设$F(X,Y)$表示带基点空间之间的带基点连续函数集合,其作为$Y^X$的子空间自然有紧开拓扑,基点为常值映射.则有下面空间的同胚:
    \begin{align*}
        F(X \wedge Y,Z)\cong F(X,F(Y,Z))
    \end{align*}
\end{proposition}
\begin{proposition}
    用$\pi_0$表示空间的道路连通分支集合.则$\pi_0(F(X,Y))=[X,Y]$.
\end{proposition}
\section{Cone与Suspension}
这一小节我们给出若干概念.
\begin{definition}
    设$X$是带基点空间.$X$的映射锥定义为:$CX=X \wedge I$.$I$作为单位区间,其基点默认为$1$.于是$CX$具体的表达式:
    \begin{align*}
        CX=X\times I/(\{x_0\}\times I \cup X \times \{1\})
    \end{align*}

    将$S^1$视作$I/\partial I$.用$1$代表其基点.$X$的Suspension定义为:$\Sigma X=X \wedge S^1$.表达式同理,不再赘述.

    一般情况下,上述两个概念也被称为“约化锥”和“约化三角锥”.
\end{definition}
\begin{definition}
    $X$的道路空间定义为$PX=F(I,X)$.$I$的基点定义为$0$.从而$PX$中的道路都是从$x_0$出发的.$X$的回路空间定义为$\Omega X=F(S^1,X)$.
\end{definition}
\begin{proposition}
    下面的共轭(同胚显然成立):
    \begin{align*}
        F(\Sigma X,Y)=F(X,\Omega Y),[\Sigma X,Y]=[X,\Omega Y]
    \end{align*}
\end{proposition}
实际上,由于$S^1$特殊的结构(后面会说,其是上群),我们可以在$[\Sigma X,Y]=[S^1,F(X,Y)]$上定义群结构.我们可以具体的给出运算:
\begin{align*}
    (g+f)(x \wedge t)=f(x \wedge 2t),0\leq t \leq 1/2;g(x\wedge 2t-1),1/2\leq t \leq 1,g,f:\Sigma X \to Y
\end{align*}

\begin{lemma}
    $[\Sigma X,Y]$是群.$[\Sigma^2 X,Y]$还是交换群.
\end{lemma}
第一个结论略去了.和基本群是群的做法完全一致.关于第二个,可参见教材.其本质上是乾坤大挪移(悲)
\section{带基点上纤维}
带基点的上纤维本质上与不带基点没有太大的区别.我们要求$A$的基点必须要映射到$X$的基点.

\begin{definition}
    若$X$的基点$x_0$满足:$i_0:\{x_0\} \to X$是上纤维映射,则称其为非退化的带基点空间.
\end{definition}
用$Y_{+}$表示给$Y$塞一个毫无关系的基点构成的空间.记号上有:
\begin{align*}
    X \wedge Y_{+}=X \times Y/\{x_0\}\times Y
\end{align*}
从而用这个记号可以方便的表示不约化的映射柱和映射锥.不再赘述.
\begin{proposition}
    带基点映射$i:A \to X$是上纤维当且仅当$Mi=X \cup_i (A \wedge I_{+})$是一个$X \wedge I_{+}$的收缩.
\end{proposition}

\section{上纤维正合列}
对于带基点的映射$f:X \to Y$,定义同伦上纤维:$Cf$.
\begin{align*}
    Cf=Y\cup_f CX=Mf/j(X),j:X \to Mf, x \mapsto (x,1)
\end{align*}

设$i:Y \to Cf$是映射映射.其是一个上纤维,因为其为$f$和$X \to CX,x \mapsto (x,0)$的推出.继续,考虑$pi:Cf \to Cf/Y=\Sigma X$是商映射.下面的列:
\begin{align*}
    X \to Y \to Cf \to \Sigma X \to \Sigma Y \to \Sigma Cf \Sigma^2 X \to \Sigma^2 Y
\end{align*}
\begin{definition}
    空间列整合意蕴$S' \stackrel{f}{\to} S \stackrel{g}{\to} S''$有:$g(s)=\{*\} \Leftrightarrow s=f(s'),\exists s'$
\end{definition}

\begin{theorem}
    对于任何的带基点空间$Z$,诱导的列:
    \begin{align*}
        \to [\Sigma Cf,Z] \to [\Sigma Y,Z] \to [\Sigma X,Z] \to [Cf,Z]\to [Y,Z] \to [X,Z]
    \end{align*}
    是正合的.含有$\Sigma$的是群的正合,含有$\Sigma^2$的交换群的正合.
\end{theorem}

证明详见教材.用到的引理有:
\begin{lemma}
    对于$i:A \to X$是一个上纤维,商映射$Ci \to Ci/CA\cong X/A$是同伦等价.
\end{lemma}
\begin{proof}
    首先$Ci \to X/A$的映射已经写出.我们构造$X/A$到$Ci$的映射.如果能给出$X$到$Ci$的映射,并且$A$的像均为$\{*\}$即可.实际上根据上纤维所得到的:
    \begin{align*}
        r: X\wedge I_{+} \to M_i=X \cup_i (A\wedge I_{+})
    \end{align*}
    就可以给出我们想要的映射.
\end{proof}
\begin{lemma}
    设$A$是$X$的可缩子空间,则$X$与$X/A$同伦等价.
\end{lemma}

\section{带基点纤维}
同样的,我们可以定义$Nf$作为纤维映射的判定:
\begin{align*}
    Nf=\{(x,\chi)|\chi(1)=f(x)\} \subset X \times Y^I
\end{align*}
一个带基点映射$p:E \to B$是纤维映射,当且仅当存在带基点的道路提升函数$s$:
$$
s:Np \to F(I_{+},E)
$$
\section{纤维正合列}
设$Ff$是$f:X \to Y$的拉回:
\begin{tikzcd}
	Ff && PY \\
	\\
	X && Y
	\arrow["f", from=3-1, to=3-3]
	\arrow["{p_1}"', from=1-3, to=3-3]
	\arrow[from=1-1, to=1-3]
	\arrow["\pi", from=1-1, to=3-1]
\end{tikzcd}.因为是纤维的拉回,$\pi$自然是纤维.



\chapter{同调公理与胞腔同调}
\section{同调公理}
固定交换群$\pi$,考虑空间偶对$(X,A)$.称$(X,A) \to (Y,B)$是弱等价的,如果$A \to B$和$X \to Y$同时都是弱等价.

我们给出若干条要求,下面的定理说的是其是不矛盾的(即有模型使得其存在)
\begin{theorem}[同调公理的存在性]
    对于整数$q$,存在从同伦范畴到Abel群范畴的函子$H_q(X,A;\pi)$,以及自然变换:$\partial H_q(X,A;\pi) \to H_{q-1}(A;\pi)$.其中$H_q(X;\pi)$定义为$H_q(X,\emptyset;\pi)$.同时,这些函子还满足下面的公理:
    
    (1)维数公理:$H_0(*;\pi)=\pi$,$H_q(*;\pi)=0,q>0$.

    (2)正合公理:下面的列正合:
    \begin{align*}
        \rightarrow H_q(A;\pi) \rightarrow H_q(X;\pi) \rightarrow H_q(X,A;\pi) \rightarrow H_{q-1}(A;\pi)
    \end{align*}

    (3)切除公理(excision):设$(X;A,B)$是分割三元组.即$X=A^o \cup B^o$.则包含关系$(A,A\cap B)\to (X,B)$诱导了同构:
    \begin{align*}
        H_*(A,A\cap B;\pi)\longrightarrow H_*(X,B;\pi)
    \end{align*}

    (4)可加公理:$(X,A)$是一些$(X_i,A_i)$的不交并,则包含$(X_i,A_i)\to (X,A)$给出同构:
    \begin{align*}
        \sum_i H_*(X_i,A_i;\pi)\to H_*(X,A;\pi)
    \end{align*}

    (5)弱等价公理:$f:(X,A)\to (Y,B)$是空间偶对的弱等价,则$f_*:H_*(X,A;\pi)\to H_*(Y,B;\pi)$给出了一个同构.
\end{theorem}
该定理的证明,实际上是考虑胞腔逼近.假设我们有下面定理的成立:
\begin{theorem}
    对于整数$q$,存在从CW对同伦范畴到Abel群范畴的函子$H_q(X,A;\pi)$,以及自然变换:$\partial H_q(X,A;\pi) \to H_{q-1}(A;\pi)$.其中$H_q(X;\pi)$定义为$H_q(X,\emptyset;\pi)$.同时,这些函子还满足下面的公理:
    
    (1)维数公理:$H_0(*;\pi)=\pi$,$H_q(*;\pi)=0,q>0$.

    (2)正合公理:下面的列正合:
    \begin{align*}
        \rightarrow H_q(A;\pi) \rightarrow H_q(X;\pi) \rightarrow H_q(X,A;\pi) \rightarrow H_{q-1}(A;\pi)
    \end{align*}

    (3)切除公理(excision):设$(X;A,B)$是分割三元组.即$X=A^o \cup B^o$.则包含关系$(A,A\cap B)\to (X,B)$诱导了同构:
    \begin{align*}
        H_*(A,A\cap B;\pi)\longrightarrow H_*(X,B;\pi)
    \end{align*}

    (4)可加公理:$(X,A)$是一些$(X_i,A_i)$的不交并,则包含$(X_i,A_i)\to (X,A)$给出同构:
    \begin{align*}
        \sum_i H_*(X_i,A_i;\pi)\to H_*(X,A;\pi)
    \end{align*}
\end{theorem}
只需要定义一般的$H_*(X,A;\pi)=H_*(\Gamma X,\Gamma A;\pi)$即可给出一般空间的同调公理.

因此我们接下来讨论胞腔同调.同时,给出其定义和一些例子
\section{胞腔同调}
胞腔同调是满足同调公理的一个重要模型.在具体计算中,我们常常利用其计算空间$(X,A)$的同调群.

我们把系数设置为$\pi=\Z$,并且省略掉$H_*(X,A;\pi)$中的$\pi$.设$X$是一个CW复形.我们先给出链复形$C_*(X)$.
\section{构造}
\begin{definition}
    $C_*(X)$的第$n$阶自由Abel群为$C_n(X)$,每有一个$n$维胞腔$j:(D^n,S^{n-1})$,就有生成元$[j]$.
\end{definition}
问题是定义微分算子$d$.也就是定义$d_n:C_n(X)\to C_{n-1}(X)$.这里的核心是利用映射度.

我们知道,$\pi_n(S_n)=\Z$.因此一个映射$f:S^n\to S^n$诱导的$f_*:\Z \to \Z$,若$f_*(a)=na$,则定义其映射度为$n$.

然而对于球面$S^n$来说,我们经常把$S^n \to D^{n+1}$,$D^n/S^{n-1}$,$\Sigma S^{n-1}$混为一谈.我们下面规范这三个球面,用同胚给出他们之间的关系.首先,标准球面$S^n$记为:
\begin{align*}
    S^n=\{(x_1,\dots,x_{n+1})|x_1^2+\dots+x_{n+1}^2=1\}
\end{align*}

1.定义$\nu_n:D^n/S^{n-1}\to S^n$:
\begin{align*}
    \nu_n(tx_1,\dots,tx_n)=(ux_1,\dots,ux_n,2t-1), u=(1-(2t-1)^2)^{1/2}
\end{align*}

2.定义:$\iota_n:S^n \to \Sigma S^{n-1}$:
\begin{align*}
    \iota(x_1,\dots,x_{n+1})=(vx_1,\dots,vx_n)\wedge (x_{n+1}+1)/2, v=1/(\sum_{i=1}^n x_i^2)^{1/2}
\end{align*}

3.定义$\xi_n:(D^n,S^{n-1})\rightarrow (CS^{n-1},S^{n-1})$:
\begin{align*}
    \xi_n(tx_1,\dots,tx_n)=(x_1,\dots,x_n)\wedge (1-t)
\end{align*}
则:
\begin{align*}
    \iota_n \circ \nu_n=-\xi_n
\end{align*}
\begin{lemma}
    下面的图在同伦意义下交换:\begin{tikzcd}
	{D^n \cup_{S^{n-1}}CS^{n-1}} && {\Sigma S^{n-1}} \\
	\\
	{D^n/S^{n-1}} && {S^n}
	\arrow["\pi", from=1-1, to=1-3]
	\arrow["\psi"', from=1-1, to=3-1]
	\arrow["{\iota_n}"', from=3-3, to=1-3]
	\arrow["{\nu_n}"', from=3-1, to=3-3]
	\arrow["{-\xi_n}"{description}, from=3-1, to=1-3]
\end{tikzcd}
\end{lemma}
\begin{proof}
    考虑$CS^{n-1} \cup_{S^{n-1}}CS^{n-1}$即可.
\end{proof}
回到胞腔的情况.把$n$维胞腔$j$:$(D^n,S^{n-1})\to (X^n,X^{n-1})$.则同胚:
$$
\alpha:\bigvee_j D^n/S^{n-1}\to X^n/X^{n-1}
$$
定义$\pi_j:X^n/X^{n-1} \to S^n$为$\alpha^{-1}$复合:$j$处胞腔为$\nu_n$,其余为基点常值映射.

于是若有$n$胞腔$j$和$n-1$胞腔$i$,得到映射$f_{ij}:(n\geq 2)$
$$
S^{n-1} \to X^{n-1} \to X^{n-1}/X^{n-2} \to S^{n-1}
$$
若$n=1$,补充$\rho:X^{n-1}\to X^{n-1}/X^{n-2}$为包含映射$X^0 \to X^0_{+}$.

现在,若$n\geq 2$,定义$a_{ij}$是$f_{ij}$的映射度,定义$d_n([j])=\sum_i a_{ij}[i]$.若$n=1$,$d_1([j])=[j(1)]-[j(0)]$.

我们只用说明$d_n \circ d_{n-1}=0$即可构造$C_*(X)$和$H_*(X)=H_*(C_*(X))$.
\section{验证是链复形}
定义拓扑边缘算子$\partial_n:X^n/X^{n-1}\to \Sigma(X^{n-1}/X^{n-2})$
$$
X^n/X^{n-1} \stackrel{\psi^{-1}}{\rightarrow} Ci \stackrel{\pi}{\rightarrow} \Sigma X^{n-1} \stackrel{\Sigma \rho}{\rightarrow}\Sigma(X^{n-1}/X^{n-2}) 
$$
其中$i:X^{n-1}\to X^n$是包含映射.我们断言$d_n$在某种意义下由$\partial_n$诱导.为此,我们先验的给出约化同调函子(什么是约化以后再说)
\begin{definition}
    设$X$是带基点的$n-1$连通空间.定义$\tilde{H}_n'(X)$是一个交换群:

    1.若$n=0$,由集合$\pi_0(X)-\{*\}$生成的自由Abel群.

    2.若$n=1$,定义为$\pi_1(X)/[\pi_1(X),\pi_1(X)]$.即基本群的交换化.

    3.若$n\geq 2$,定义为$\pi_n(X)$.
\end{definition}
\begin{definition}
    设$X$是带基点$n-1$连通空间.定义$\Sigma$:
    $$
    \Sigma:\tilde{H}'_n(X)\to \tilde{H}'_{n+1}(\Sigma X)
    $$
    $$
    \Sigma[f]=[\Sigma f \circ \iota_{n+1}],f:S^n \to X
    $$

    但$n\geq 1$时上面的定义才有定义.若$n=0$,对于$x \in X$,可以定义$\Sigma[x]=[f_*^{-1}\cdot f_x]$,$* \in X$并且$f_x$是道路$t \mapsto x\wedge t$.
\end{definition}
\begin{lemma}
    若$X$是一些$n$维球面的一点并,则$\Sigma:\tilde{H}_n'(X)\to \tilde{H}_{n+1}'(\Sigma  X)$是同构.
\end{lemma}
\begin{proof}
    我们断言$\tilde{H}'_n(X)$是由$S^n \to \vee_i S^n$生成的自由Abel群.

    由于$S^n$是紧的,因此我们只用考虑有限的一点并情况.当$n\geq 2$时,$(\times_i S^n,\vee_i S^n)$有自然的CW结构(比如$(S^1\times S^1,S^1\vee S^1)$).并且没有$q<2n-1$的胞腔.

    从而根据胞腔逼近定理,结论成立(总是同伦于胞腔的映射).因此引理得证,原因是单个球面的情况是显然的.
\end{proof}
回到CW复形$X$.我们定义$[j\circ \nu_n^{-1}]$同伦等价类:
\begin{align*}
    S^n \stackrel{\nu_n^{-1}}{\longrightarrow}D^n/S^{n-1} \stackrel{j}{\longrightarrow} X^n/X^{n-1}
\end{align*}
以作为$\tilde{H}_n'(X^n/X^{n-1})$的典范基.原因自然是$X^n/X^{n-1}$是球面的一点并.

\begin{lemma}
    微分算子$d_n:C_n(X) \to C_{n-1}(X)$可以被等同于下面群同态的复合:
    \begin{align*}
        d_n':\tilde{H}_n'(X^n/X^{n-1}) \stackrel{(\partial_n)_*}{\longrightarrow}\tilde{H}_n'(\Sigma X^{n-1}/X^{n-2}) \stackrel{\Sigma^{-1}}{\longrightarrow} \tilde{H}_{n-1}'(X^{n-1}/X^{n-2})
    \end{align*}
\end{lemma}
\begin{proof}
    首先容易验证最左边和最右边的群同构于$C_n$和$C_{n-1}$.对应为$[j\circ \nu_n^{-1}] \sim [j]$.交换图表道尽一切.
    \[\begin{tikzcd}
	{S^n} &&& {S^n} \\
	\\
	{D^n/S^{n-1}} & {D^n \cup CS^{n-1}} & {\Sigma S^{n-1}} & {\Sigma S^{n-1}} \\
	\\
	{X^n/X^{n-1}} & {X^n \cup CX^{n-1}} & {\Sigma X^{n-1}} & {\Sigma (X^{n-1}/X^{n-2})}
	\arrow["j"', from=3-1, to=5-1]
	\arrow["{\psi^{-1}}", from=5-1, to=5-2]
	\arrow["\pi", from=5-2, to=5-3]
	\arrow["{\Sigma \rho}", from=5-3, to=5-4]
	\arrow["{\iota_n}", from=1-4, to=3-4]
	\arrow["{\Sigma \pi_i}"', from=5-4, to=3-4]
	\arrow[from=3-1, to=3-2]
	\arrow[from=3-2, to=3-3]
	\arrow["{a_{ij}}", from=3-3, to=3-4]
	\arrow["{j \cup Cj}"{description}, from=3-2, to=5-2]
	\arrow["{\Sigma j}"{description}, from=3-3, to=5-3]
	\arrow["{a_{ij}}", from=1-1, to=1-4]
	\arrow["{\nu_n^{-1}}"', from=1-1, to=3-1]
	\arrow["{\iota_n}", from=1-1, to=3-3]
    \end{tikzcd}\]
 该图表示了$a_{ij}$的由来.另外考虑$d_n'([j\circ \nu_n^{-1}])=\sum_i a_{ij}'[i \circ \nu_{n-1}^{-1}]$.我们考虑$a_{ij}'$的由来.可以看出来其就是右下角方块.从而$a_{ij}=a_{ij}'$.
\end{proof}
\begin{lemma}
    $d_n \circ d_{n-1}=0$
\end{lemma}
\begin{proof}
    我们只用说明$\Sigma \partial_{n-1}\circ \partial_n$同伦于平凡映射.下面的交换图在同伦意义下交换是自然的.
    \[\begin{tikzcd}
	{X^n \cup CX^{n-1}} & {\Sigma X^{n-1}} & {\Sigma(X^{n-1}\cup CX^{n-2})} & {\Sigma^2 X^{n-2}} \\
	{X^n/X^{n-1}} & {\Sigma(X^{n-1}/X^{n-2})} & {\Sigma(X^{n-1}/X^{n-2})} & {\Sigma^2 X^{n-2}/X^{n-3}}
	\arrow[from=1-1, to=1-2]
	\arrow["{\Sigma i}", from=1-2, to=1-3]
	\arrow[from=1-3, to=1-4]
	\arrow[from=1-1, to=2-1]
	\arrow[from=2-1, to=2-2]
	\arrow["{=}"', from=2-2, to=2-3]
	\arrow[from=2-3, to=2-4]
	\arrow[from=1-2, to=2-2]
	\arrow["{\Sigma \psi}", from=1-3, to=2-3]
	\arrow[from=1-4, to=2-4]
\end{tikzcd}\]

    而$\Sigma \pi \circ \Sigma i$本身是平凡映射.
\end{proof}
\section{验证同调公理}
接下来我们验证给出的胞腔同调满足四条同调公理.首先定义CW对的同调群.$H_*(X,A)$定义为$C_*(X,A)=C_*(X)/C_*(A)$的同调群.结果为:
\begin{align*}
    H_*(X,A)\cong H_*(X/A)
\end{align*}

1.正合公理:我们用蛇形引理给出正合公理.即:
\begin{align*}
    0 \to C_*(A) \to C_*(X) \to C_*(X/A) \to 0
\end{align*}
给出:
\begin{align*}
    \partial:H_q(X/A) \to H_{q-1}(A)
\end{align*}

2.函子性:即给定$f:X \to Y$是胞腔映射,存在$f_*:H_q(X)\to H_q(Y)$的群同态.事实上,胞腔映射说明有$f:X^n/X^{n-1} \to Y^n/Y^{n-1}$.这个映射在同伦意义下与拓扑边界映射是交换的,从而诱导了$f_n:C_n(X)\to C_n(Y)$,与微分算子交换.因此有同调群的群同态.
\begin{tikzcd}
	{X^n/X^{n-1}} && {Y^{n}/Y^{n-1}} \\
	\\
	{\Sigma (X^{n-1}/X^{n-2})} && {\Sigma (Y^{n-1}/Y^{n-2})}
	\arrow["{\partial_n}", from=1-1, to=3-1]
	\arrow[from=1-1, to=1-3]
	\arrow["{\partial_n}"', from=1-3, to=3-3]
	\arrow[from=3-1, to=3-3]
\end{tikzcd}

函子性还要求胞腔之间的同伦映射诱导同调群的同构.我们只用说明同伦$X \times I \to Y$给出了$C_*(X) \otimes \mathscr{I} \to C_*(Y)$.这个留作下文.

3.维数公理和可加性公理:显然.

4.切除公理:若$X =A \cup B$,那么$A/A\cap B$和$X/B$是CW复形之间的同构.

最后我们讨论一般系数的同调.
\begin{definition}
    对于一般系数的群$\pi$,定义:
    $$
    C_*(X,A;\pi):=C_*(X,A) \otimes \pi
    $$
    则可以照猫画虎的给出所有同调的公理和条件.本质上没有区别.
\end{definition}
若$\pi \to \rho$还存在一个群同态,则我们有自然的变换:
\begin{align*}
    H_q(X,A;\pi) \to H_q(X,A;\rho)
\end{align*}
这是因为我们可以定义$C_*(X,A) \otimes \pi \to C_*(X,A) \otimes \rho$.
\section{乘积空间的链复形}
胞腔同调有一个很强的事实:可以直接给出笛卡尔乘积下空间的同调.
\begin{theorem}
    $X$和$Y$是CW复形,则$X \times Y$有自然的CW结构,使得:
    \begin{align*}
        C_*(X \times Y) \cong C_*(X) \otimes C_*(Y)
    \end{align*}
\end{theorem}
让我们计算$I$的同调群.注意到$I$有零维胞腔$[0],[1]$,1维胞腔$I$.给出$S^0 \to X^0 \to X^0_+ \to S^0$.若考虑$[0]$,则$0 \mapsto 0,1 \mapsto *(1)$.$[1]$也如此考虑.于是$d([I])=[0]-[1]$.

从而$C_*(I) \cong \mathscr{I}$.定理补充了函子性最后的说明.
\begin{proof}
    定义$\kappa$作为群同态:
    \begin{align*}
        \kappa:C_*(X)\times C_*(Y) \to C_*(X \times Y) \quad \kappa([i]\otimes [j])=(-1)^{pq}[i \times j]
    \end{align*}
    对于每一阶群来讲,这确实是群同构.因此我们只需要说明微分算子交换性即可.

    这里我们就不做类似的工作了,可自行查看教材.
\end{proof}
\section{一些例子}
\begin{example}[$S^n$]
    平凡:$\tilde{H}_n(S^n)=\Z$,$\tilde{H}_q(S^n)=0,q \neq n$.
\end{example}
设$j$表示$T$,$K$和$\R P^2$的2胞腔.
\begin{example}[$T$]
    $T$有一个0胞腔,两个1胞腔,一个2胞腔.仔细计算,微分映射全部为$0$.所以$H_*(T;\Z)=C_*(T)$.
\end{example}
\begin{example}[$\R P^2$]
    $\R P^2$有两个0胞腔两个1胞腔,一个2胞腔.计算有:$H_0=\Z$,$H_1=\Z_2$,$H_q=0,q \geq 2$.
\end{example}

接下来计算$\R P^n$的情况.首先:
\begin{lemma}
    对角映射$a_n:S^n \to S^n$的映射度为$(-1)^{n+1}$.
\end{lemma}
\begin{proof}
    归纳.$n=1$时显然为$1$.注意到$\iota_n \circ a_n=-(\Sigma a_{n-1})\circ \iota_n$.而suspension不改变映射度,所以归纳即可.
\end{proof}
我们在这里不对$\R P^n$的情况做过多阐述.其本质是无聊的.
\chapter{同调公理的若干推论}
胞腔同调是一个很好的同调模型.但是我们这章不考虑具体模型,而是从同调公理出发研究同调.与此同时,维数公理并不是最关键的.所以我们采取“广义同调”的情况.即仅删去维数公理,保持其他公理不变而得到$E_q$函子.同理,CW复形也有类似的拓展.

\section{约化同调}
空间对的同调与带基点空间的约化同调之间的关系.后者是更常用的.

对于带基点空间$X$,定义$X$的约化同调:
\begin{align*}
    \tilde{E}_q(X)=E_q(X,*)
\end{align*}

显然$*$是$X$的收缩,所以我们有直和分解:(正合列中,如果)
\begin{align*}
    E_*(X)\cong \tilde{E}_*(X)\oplus E_*(*) 
\end{align*}
这是因为我们本来有长正合列:
\begin{align*}
    \dots  \to E_q(*) \to E_q(X) \to E_q(X,*) \stackrel{\partial}{\rightarrow} E_{q-1}(*)\to \dots
\end{align*}
此时拉回一个$X \to \{*\}$,则该长正合列分裂.

再次注意到,在正合列中,$E_*(*)$被同构映射.(观察空间对的交换图).即$E_*(*)$在下面的列中,从$E_q(A)$完全同构的到$E_q(X)$.而映射到$(X,A)$时,被当作零映射.
\begin{align*}
    \dots  \to E_q(A) \to E_q(X) \to E_q(X,A) \stackrel{\partial}{\rightarrow} E_{q-1}(A)\to \dots
\end{align*}
于是我们有含约化同调的正合列:
\begin{align*}
    \dots  \to \tilde{E}_q(A) \to \tilde{E}_q(X) \to E_q(X,A) \stackrel{\partial}{\rightarrow} \tilde{E}_{q-1}(A)\to \dots
\end{align*}

我们也可以从约化同调的情况得到非约化的情况.对于不带基点空间$X$,有$E_*(X)=\tilde{E}_*(X_+)$.这根据可加性公理得到.

同理,$f:X \to Y$可以给出带基点空间映射$f:X_+ \to Y_+$,而$f$诱导的$f_*$在带基点情况和不带绩点的情况下正好一致.(还是交换图给出)

这一章我们关注带基点空间的约化同调.如果我们讨论不带基点的情况,则统一作用函子$()_{+}$.

注意,$X$的不约化双角锥$\Sigma X$并非$X_+$的约化三角锥!!!事实上,$\Sigma(X_+)$同伦等价于$\Sigma(X)$和一个$S^1$的单点并.
\section{上纤维和空间对的同调}
\begin{theorem}
    对于任何上纤维$i:A \to X$,商映射$q:(X,A) \to (X/A,*)$给出同调群的同构:
    \begin{align*}
        E_*(X,A) \to E_*(X/A,*)=\tilde{E}_*(X/A)
    \end{align*}
\end{theorem}
\begin{proof}
    做经典的拆分.令$Ci=B\cup C$.$B=X\cup_A A\times [0,2/3]$,$C=A\times [1/3,1]/A \times \{1\}$.根据excisive公理:
    \begin{align*}
        H_*(B,B\cap C)=H_*(X\cup_i A\times [0,2/3],A\times [1/3,2/3]) \cong H_*(X,C)=H_*(Ci,A\times [1/3,1]/A \times \{1\})
    \end{align*}
    显然$(B,B\cap C)\sim (X,A)$.而$(Ci,A \times [1/3,1]/A\times\{1\})$与$(X/A,*)$同伦.并且交换图:
    
    \begin{tikzcd}
	{(X\cup_iA \times [0,2/3],A\times[1/3,2/3]} & {(Ci,A\times[1/3,1]/A\times\{1\})} \\
	{(X,A)} & {(X/A,*)}
	\arrow["r", from=1-1, to=2-1]
	\arrow["q", from=2-1, to=2-2]
	\arrow["\psi"', from=1-2, to=2-2]
	\arrow[from=1-1, to=1-2]
\end{tikzcd}成立.
\end{proof}
\section{Suspension和长正合列}
\begin{theorem}
    对于非退化的带基点空间$X$,存在自然的同构:
    \begin{align*}
        \Sigma:\tilde{E}_q(X)\cong \tilde{E}_{q+1}(\Sigma X)
    \end{align*}
\end{theorem}
\begin{proof}
    注意到$CX$是可缩空间,于是其约化同调与$*$相同,为$0$.从而根据约化同调的长正合列,可以得到:
    \begin{align*}
        \tilde{H}_q(X) \cong \tilde{H}_{q+1}(CX/X)=\tilde{H}_{q+1}(\Sigma X)
    \end{align*}
    其中第一个同构由连接同态$\partial$给出.
\end{proof}
\begin{corollary}
    设$A$是$X$的带基点上纤维.在长正合列:
    \begin{align*}
        \dots \rightarrow \tilde{E}_q(A) \rightarrow \tilde{E}_q(X) \rightarrow \tilde{E_q}(X/A) \stackrel{\partial}{\rightarrow} \tilde{E}_{q-1}(A) \to \dots
    \end{align*}
    中,连接同态$\partial$可以分解为:
    \begin{align*}
        \tilde{E}_q(X/A) \stackrel{\partial_*}{\rightarrow}\tilde{E}_q(\Sigma A) \stackrel{\Sigma^{-1}}{\rightarrow}\tilde{E}_q(A)
    \end{align*}
\end{corollary}
\begin{proof}
    考虑$\partial$的函子性.显然$E_q(\Sigma A)$到$E_{q-1}(A)$的同构就包含了$\partial$.我们给出两个空间对$(X,A)$和$(CA,A)$.显然,我们只需要构造$X$到$CA$保持$A$的映射.
    \begin{align*}
        (X,A) \to (Ci,CA) \to (\Sigma A,\Sigma A) \to  
    \end{align*}
\end{proof}
由于$S^0$实际上是两个孤立点,$\tilde{E}_*(S^0)=E_*(*)$.而$S^n$是$S^{n-1}$的Suspension,则我们有同构
\begin{corollary}
    \begin{align*}
       \tilde{E}_q(S^n)\cong  \tilde{E}_{q-n}(*)
    \end{align*}
    
\end{corollary}
\section{MV序列}
本小节我们给出三个有用的结论.其都涉及的是空间对(甚至是三元对).
\begin{theorem}[辫引理]
    考虑三元组$(X,A,B)$,其中$B \subset A \subset X$.则存在下面的正合列:
    \begin{align*}
        \dots \rightarrow E_q(A,B)\rightarrow E_q(X,B)\rightarrow E_q(X,A) \stackrel{\partial}{\rightarrow} E_{q-1}(A,B) \to \dots
    \end{align*}
    其中$\partial$由:
    $$
    E_q(X,A) \to E_{q-1}(A) \to E_q(A,B)
    $$
    给出.
\end{theorem}
\begin{proof}
    我们不采取追图的方式.而是考虑CW逼近.此时我们有$X,A,B$被CW逼近.而$X/A \cong (X/B)/(A/B)$,于是根据约化同调的理论考虑$(X/B,A/B)$即可.
\end{proof}
\begin{theorem}
    设$(X,A,B)$是excisive三元组,令$C=A\cap B$.则映射:
    \begin{align*}
        E_*(A,C)\oplus E_*(B,C) \to E_*(X,C)
    \end{align*}
    给出了同构.
\end{theorem}
\begin{proof}
    我们采取代数的办法.
    \[\begin{tikzcd}
	{(A,C)} && {(B,C)} \\
	& {(X,C)} \\
	{(X,B)} && {(X,C)}
	\arrow[from=1-1, to=3-1]
	\arrow[from=1-1, to=2-2]
	\arrow[from=2-2, to=3-1]
	\arrow[from=1-3, to=2-2]
	\arrow[from=2-2, to=3-3]
	\arrow[from=1-3, to=3-3]
\end{tikzcd}
    \]
    图中,左边右边均为群同构.而对角线均为正合列.注意到该正合列分裂.

也可以考虑CW逼近.即$X/C=A/C \wedge B/C$.从而得到答案.
\end{proof}
\begin{theorem}[Mayer-Vietoris Sequence]
    设$(X;A,B)$是excisive三元组,$C=A\cap B$.则我们有下面的正合列:
    \begin{align*}
        \dots \longrightarrow E_q(C) \stackrel{\psi}{\longrightarrow}E_q(A)\oplus E_q(B) \stackrel{\phi}{\longrightarrow} E_q(X) \stackrel{\Delta}{\longrightarrow} E_{q-1}(C) \longrightarrow \dots
    \end{align*}
    其中,$\psi(c)=(i_*(c),j_*(c))$.而$\phi(a,b)=k_*(a)-l_*(b)$.而$\Delta$是下列映射的复合:
    \begin{align*}
        E_q(X)\to E_q(X,B)\cong E_q(A,C) \to E_{q-1}(C)
    \end{align*}
\end{theorem}
我们不打算给出这个定理的证明.实际上是一个追图的过程.可以参考教材的图示.

最后给出一个重要的定理:同调与特殊的余极限可交换.同样,证明的过程略显繁琐,我们也略去.
\begin{theorem}
    设$X_i$是一系列空间,满足$X_i \subset X_{i+1}$.则$E(X_i)$到$E(X)$有自然的同态.

    从而根据余极限泛性质,有:
    \begin{align*}
        \Colim E(X_i) \to E(X)
    \end{align*}
    这是一个典范的同构.
\end{theorem}
\chapter{Hurewicz定理和唯一性定理}
同调公理有一个急需解决问题:是否满足这样公理的同调是唯一的?(假设我们有维数公理).

同时,我们也思考,在这种情况下,同调群和同伦群的关系是什么?这就是Hurewicz定理.
\chapter{奇异同调}

\section{奇异链复形}
我们定义标准的拓扑$n$单形:
\begin{definition}
    设$\Delta_n$是如下$\R^{n+1}$的子集:
    \begin{align*}
        \Delta_n=\{(t_0,\dots,t_n)|0 \leq t_i \leq 1,\sum t_i=1\}
    \end{align*}
\end{definition}
\begin{definition}[面映射]
    对于$\Delta_n$,存在$n+1$个面映射$d_i,0\leq i \leq n$:
    \begin{align*}
        \delta_i:\Delta_{n-1} \to \Delta_n, \delta_i(t_0,\dots,t_{n-1})=(t_0,\dots,t_{i-1},0,t_i,\dots,t_{n-1})
    \end{align*}
\end{definition}
对于定义,请注意其指标关系.即是在第$i$号位置后添加$0$.
\begin{definition}[退化映射]
     对于$\Delta_n$,存在$n+1$个退化映射$d_i,0\leq i \leq n$:
    \begin{align*}
        \delta_i:\Delta_{n+1} \to \Delta_n, \delta_i(t_0,\dots,t_{n+1})=(t_0,\dots,t_{i-1},t_i+t_{i+1},\dots,t_{n+1})
    \end{align*}
\end{definition}
对于定义,请注意其指标关系.即是在第$i$号位置后,合并$i+1$号和$i+2$号.

我们要考虑的是,对于一般的拓扑空间$X$,是否可以做一些剖分,使得其也拥有单纯复形类似的结构.这里我们自然想到定义$X^{\Delta_n}$
\begin{definition}
    对于空间$X$,定义$S_nX$是所有连续映射$f:\Delta_n \to X$的几何.特别的,$S_0X \cong X$.这里$1 \mapsto x$是$S_0X$中元素的形式.
\end{definition}
根据交换图,我们自然诱导$d_i:S_nX \to S_{n-1}X$和$s_i:S_nX \to S_{n+1}X$:
\begin{align*}
    d_i(f)(u)=f(\delta_i(u)),u \in \Delta_{n-1}   \quad s_i(f)(v)=f(\sigma_i(v)),v \in \Delta_{n+1}
\end{align*}
\[\begin{tikzcd}
	{\Delta_n} && X && X && {\Delta_n} \\
	\\
	{\Delta_{n-1}} &&&&&& {\Delta_{n+1}}
	\arrow["{\delta_i}", from=3-1, to=1-1]
	\arrow["f", from=1-1, to=1-3]
	\arrow["{d_if}"', from=3-1, to=1-3]
	\arrow["f"', from=1-7, to=1-5]
	\arrow["{s_if}"', from=3-7, to=1-5]
	\arrow["{\sigma_i}", from=3-7, to=1-7]
    \end{tikzcd}
\]

\begin{proposition}\label{pro:1}
    下面的命题被称为单纯集公理.但是对于上述定义,其确为可以直接导出的性质:
    \begin{enumerate}
        \item $d_i \circ d_j=d_{j-1}\circ d_i$
        \item $d_i \circ s_j= s_{j-1}\circ d_i ,i<j$
        \item $d_i \circ s_j=\mathrm{id},i=j,j+1$
        \item $d_i \circ s_j=s_j \circ d_{i-1},i>j+1$
        \item $s_i\circ s_j=s_{j+1}\circ s_i,i\leq j$
    \end{enumerate}
\end{proposition}
\begin{proof}
    Can be easily checked by every puple. 
\end{proof}
\begin{definition}
    设$f \in S_nX$.称$f$是\textbf{非退化的},若不存在$g$和$i$使得$f=s_ig$.换言之,$f$是原原本本从$\Delta_n$出发的映射,并非由$S_{n-1}X$上的映射提升后伪装而来.

    称$f:\Delta_n \to X$是一个$n$维奇异单形.
\end{definition}
\begin{proposition}
    设$C_n(X)$是所有非退化$n$维奇异单形所生成的自由Abel群,将$C_n(X)$看作由所有$n$单形生成的自由Abel群商掉所有退化的$n$单形生成的自由Abel群.定义:
    \begin{align*}
        d=\sum_{i=0}^n(-1)^i d_i:C_n(X)\to C_{n-1}(X)
    \end{align*}
    则$d$良定且满足$d \circ d=0$.
\end{proposition}
\begin{proof}
    $d$良定的意思是,若$f$退化,则$df$中非平凡元也是退化的.此时两个商群之间可以定义良定的同态.设$f=s_jg$,则$d_if=d_i \circ s_j g$.注意到映射$i=j,j+1$时$d_if$虽然为$g$,但是$d$表达式中抵消.而其他可用公式换$s$出来,所以$d$良定.
    \begin{align*}
        d\circ d=\sum_{i=0}^{n-1}\sum_{j=0}^n (-1)^{i+j}d_i \circ d_j=0
    \end{align*}
    上述等式只是一个$d_i \circ d_j$的交换公式和一个计数.
\end{proof}
\begin{definition}[奇异同调]
    $X$的奇异同调被定义为下列链复形的同调:
    \begin{align*}
        H_*(X;\pi)=H_*(C_*(X)\otimes \pi)
    \end{align*}
\end{definition}
\section{几何实现}
尽管通过验证同调公理,我们可以很快的说明奇异同调和之前定义的胞腔同调相同.但是使用下面的方式时更有教义的:几何实现.
\begin{definition}
    给定空间$X$,定义$X$的几何实现为拓扑空间:
    \begin{align*}
        \Gamma X=\coprod_{n \geq 0}(S_n X \times \Delta_n)/(\sim) 
    \end{align*}
    其中等价关系定义为:
    \begin{align*}
        (d_if,u)\sim (f,\delta_i u) \quad (s_if,v)\sim (f,\sigma_i v)
    \end{align*}
    $\Gamma X$的拓扑定义为商拓扑+并拓扑.
\end{definition}
从$\Gamma X$到$X$有自明的映射:$\gamma:\Gamma X \to X:\gamma(|f,u|)=f(u)$.
\begin{theorem}\label{thm:iso}
    对于任何$X$,$\Gamma X$是CW复形,其中每个$n$胞腔对应一个非退化$n$奇异单形.$\Gamma X$的胞腔同调与$X$的奇异同调有自然的同构.
\end{theorem}
\begin{theorem}\label{thm:weake}
    对于任何$X$,映射$\gamma:\Gamma X \to X$是一个弱等价.
\end{theorem}
由上述的两个定理,我们自然得到$X$的奇异同调与$X$的CW逼近的胞腔同调同构.这是非常有价值的断言.
\begin{proof}[定理\ref{thm:iso}]
    我们定义空间$\overline{X}=\coprod_{n\geq 0}S_nX \times \Delta_n$.定义函数:
    \begin{align*}
        \lambda:\overline{X} \to \overline{X} \text{  and  } \rho:\overline{X} \to \overline{X}
    \end{align*}
    其中若$f=s_{j_p}\dots s_{j_1}g$,则$\lambda(f,u)=(g,\sigma_{j_1}\dots \sigma_{j_p}u)$,其中$g$非退化点.若$u=\delta_{i_q}\dots \delta_{i_1}v$,则$\rho(f,u)=(d_{i_1}\dots d_{i_q}f,v)$,其中$v$是其所在的单形的内部.

    称$(f,u)$是非退化的,若$f$非退化且$u$是内部点.我们断言:

    \textbf{函数的复合:$\lambda \circ \rho$将$\overline{X}$的每个点都映射到该点所在的等价类中唯一的非退化点.}即每个等价类都有唯一的非退化点.

    实际上,非退化点的存在是显然的.(函数的复合得到的结果就是非退化点).我们证明若$(f,u) \sim (f',u')$且两点均为非退化点,则$f=f',u=u'$.考量这个等价关系,我们有$f=s_{j_p}\dots s_{j_1}\dots d_{i_1}\dots d_{i_q}f'$.注意到两个都是非退化,则$s$不存在.即$f=d_{i_q}\dots d_{i_q}f'$.同样的断言说明$u=\sigma_{j_1}\dots \sigma_{j_p}u'$.

    于是$(f,u)=(d_{i_1}\dots d_{i_q}f',u)\sim (f',\delta_{i_q}\dots \delta_{i_1}u)\sim (f',u')$.于是$f=f',u=u'$.\textbf{断言证明完毕.}
     
    接下来我们直截了当的说明$\Gamma X$的胞腔结构.设$(\Gamma X)^n$是$\coprod_{m \leq n}S_m X \times \Delta_m$的商映射像.则:
    \begin{align*}
        (\Gamma X)^n-(\Gamma X)^{n-1}=\{ \text{非退化n奇异单形}\} \times \{\Delta_n -\partial \Delta_n\}
    \end{align*}
    于是$\Gamma X$是CW复形,其$n$胞腔:
    \begin{align*}
        \tilde{f}:(\Delta_n,\partial \Delta_n) \to ((\Gamma X)^n,(\Gamma X)^{n-1}) \quad \tilde{f}(u)=|f,u|, f\text{是一个非退化的n单形.}
    \end{align*}
    我们接下来计算$\Gamma X$的胞腔同调.显然$C_n(X)$已经同构,我们需要计算$d$是否有同样的对应.即对于$n$胞腔$f$和$n-1$胞腔$g$,映射度仅在$g=d_if$的情况下为$(-1)^i$.

    胞腔同调的映射度定义为:

    \begin{tikzcd}
	{S^{n-1}} & {X^{n-1}} & {X^{n-1}/X^{n-2}\cong \bigvee_\alpha S^{n-1}} & {S^{n-1}}
	\arrow["{\tilde{f}}", from=1-1, to=1-2]
	\arrow["\pi", from=1-2, to=1-3]
	\arrow["g", from=1-3, to=1-4]
\end{tikzcd}
 
    复合后的映射度.注意到$g$若不为$d_if$的形式,则上述映射复合为零映射.若$g=d_if$,在考虑定向的情况下(不妨以$n=2$的情况为例子理解,次数省略)可以计算为$[f,g]=(-1)^i$.

    因而$H_*(\Gamma X;\pi)\text{(cell homology)} \cong H_*(X;\pi)(\text{singular homology})$
\end{proof}
\begin{proof}[定理\ref{thm:weake}]
    弱等价的含义是,对于任何阶的同伦群,映射$\gamma:\Gamma X \to X$诱导的同伦群同态都是同构.即下面的交换图:
    \[\begin{tikzcd}
	{(\Delta_n,\partial \Delta_n)} \\
	\\
	{(\Gamma X,|x,1|)} && {(X,x)}
	\arrow["f"', from=1-1, to=3-1]
	\arrow["\gamma", from=3-1, to=3-3]
	\arrow["{\gamma_*f}", from=1-1, to=3-3]
    \end{tikzcd}
    \]

    是单满射.满射是显然的,因为任取$g:(\Delta_n,\partial \Delta_n) \to (X,x)$,我们定义$\tilde{g}:(\Delta_n,\partial \Delta_n) \to (\Gamma X,|x,1|)$为:
    \begin{align*}
        \tilde{g}(u)=|g,u| \text{即可}
    \end{align*} 

    但单射并不容易.我们简要记录如下:
    
    设$g:(\Delta_n,\partial \Delta_n) \to (\Gamma X,|x,1|)$满足$\gamma \circ g$同伦于常值映射$c_x$.我们要说明的是$g$同伦于$c_{|x,1|}$.

    首先对$g$使用胞腔逼近.注意到$\Delta_n$可以分划为若干更小的$\Delta_n$.则我们可以得到一个同伦$g \sim g'$使得$g'$是单纯的.所谓单纯的,是指$g'\circ e$是$\Gamma X$的一个胞腔,若$e$是$\Delta_n$的的划分的胞腔:$g'\circ e:S^{n-1} \to \Delta_n^\alpha \to \Delta_n \to \Gamma X$.从而$\gamma \circ g'$也同伦于$c_x$.我们得到同伦映射:
    \begin{align*}
        h:(\Delta_n\times I,\partial \Delta_n \times I\cup \Delta_n \times \{1\}) \to (X,x)
    \end{align*}
    
    同样的,我们可以分割$\Delta_n \times I$使得$\Delta_n \times \{0\}$是子复形.此时,我们可以一个单形一个单形的把$h$提升到$\Gamma X$上:
     \begin{align*}
        \tilde{h}:(\Delta_n\times I,\partial \Delta_n \times I\cup \Delta_n \times \{1\}) \to (\Gamma X,|x,1|)
    \end{align*}
    使得$\tilde{h}$限制在$\Delta_n \times \{0\}$上是$g'$并且$\gamma \circ \tilde{h}$.于是$g'$与$c_{|x,1|}$同伦.
\end{proof}

\section{单纯范畴和几何实现}
注意到上述的所有证明我们实际上都没有真正的关注过$S_nX$具体的结构.我们关注的实际上是面映射和退化映射.所以我们可以推广$S_nX$:
\begin{definition}
    称一族集合$K_*=\{K_n\}$是单纯集族,若对于$K_n$,装备了$n+1$个面映射和$n+1$个退化映射.并且这些映射满足命题\ref{pro:1}中的关系.
\end{definition}
\begin{definition}[单纯范畴]
    所有单纯集族可以构成一个范畴——单纯范畴.其中态射定义为一族函数$\{f_n\}$使得$f_n:K_n \to J_n$并且所有的$f_n$与所有的面映射,退化映射交换.
\end{definition}
\begin{definition}[几何实现函子与奇异函子]
    对于任何单纯集族$K_*$,都可以如同$S_n X$般的定义几何实现空间$|K_*|$.注意到$f:K_* \to J_*$自然诱导$|f|:|K_*| \to |L_*|$,则几何实现是一个从单纯范畴到拓扑范畴的函子.

    对于空间$X$,如之前所言,定义$S_*X$为其奇异集族,这是一个单纯集族.并且$X \to Y$也自然诱导$S_*X \to S_* Y$.称这个函子为奇异函子.
\end{definition}

这些定义后,我们给出一个比较深刻的结果:
\begin{theorem}
    几何实现函子和奇异函子是一对伴随函子.即:
    \begin{align*}
        \mathrm{Hom}(|K_*|,X) \cong \mathrm{Hom}(K_*,S_*X)
    \end{align*}
    其中,取$f:|K_*| \to X$,则对于$K_n$,定义$\tilde{f}_n:K_n \to S_nX$为$\tilde{f}_n(k)(u)=f(k,u),k \in K_n, u \in \Delta_n$.

    取$g:K_* \to S_*X$,定义$\tilde{g}=\gamma\circ |g|: |K_*| \to |S_*X| \to X$.
\end{theorem}
\begin{proof}
    我们做简要的复合验证.即验证两个映射互为逆.实际上从构造的方式来说自然性比较显然.

    取$f:|K_*| \to X$,且设$\tilde{f}$是对应的$K_* \to S_*X$.则$\tilde{f}$回拉:
    \begin{align*}
        |K_*| \to |S_*X| \to X: |k,u| \to |\tilde{f}_n(k),u| \to \tilde{f}_n(k)(u)=f(k,u) \text{正好回拉为}f
    \end{align*}

    取$g:K_* \to S_*X$,设$g'$是回拉的映射.考虑:
    \begin{align*}
        K_* \to S_*X \text{的第}n\text{层}: g'_n(k)(u)=\tilde{g}(|k,u|)=g_n(k)(u) 
    \end{align*}
\end{proof}

同样的,我们可以对单纯集定义同调.这里不再赘述细节.

几何实现有一个很重要的性质:其对单纯集的乘法保持交换:
\begin{proposition}
    设$X_*$和$Y_*$是两个单纯集.定义乘法:$X_* \times Y_*$的第$n$层是$X_n \times Y_n$,而面映射和退化映射也是乘法.

    注意到投射$X_* \times Y_* \to X_*$,$X_* \times Y_* \to Y_*$自然给出映射$|X_* \times Y_*| \to|X_*|$和$|X_* \times Y_*| \to |Y_*|$.于是根据积的泛性质,有唯一的:
    \begin{align*}
        |X_* \times Y_*| \to |X_*| \times |Y_*|
    \end{align*}
    我们断言:这是空间的同胚.即连续映射:
    \begin{align*}
        |(f,g),u| \mapsto |f,u| \times |g,u| 
    \end{align*}
\end{proposition}
\begin{proof}
    
\end{proof}
但是上述同胚并不是胞腔系的同胚.至少其把$n$胞腔映射到了$2n$胞腔.其同伦于一个胞腔映射,但胞腔映射不再是一个同胚!

但是我们可以知道,这至少诱导了一个经典的链同伦等价:
\begin{align*}
    C_*(|K_* \times L_*|) \to C_*(|K_*|)\times C(|L_*|)
\end{align*}

\section{分类空间和Eilenberg-MacLane空间}
我们想定义一个这样的空间.其除了某阶同伦群是我们指定的abel群外,其它全是$0$.这样的空间称为Eilenberg-MacLane空间.这个空间很容易定义,但是实际上具有很抽象的拓扑.

我们假定有一个拓扑群$G$.给出两个单纯集:
\begin{align*}
    E_*(G):E_n(G)=G^{n+1}\quad B_*(G):B_{n}(G)=G^n \quad p_*:E_*(G)\to B_*(G)\text{为投射}
\end{align*}
关键是定义面映射和退化映射.我们摘录如下:

此时$p_*$是自然的单纯集映射.

所谓分类空间,是指$E_*$和$B_*$的几何实现$E(G)$和$B(G)$.此时自然有空间之间的连续映射$p:E(G) \to B(G)\quad p(|(e_0,\dots,e_n),u|)=|e_0,\dots,e_{n-1},u|$.

我们仔细考察一下两个空间的结构.$E(G)$满足:
\begin{align*}
    E(G)^n -E(G)^{n-1}=(G^n-W)\times G \times (\Delta_n-\partial \Delta_n)\\
    B(G)^n -B(G)^{n-1}=(G^n-W)\times (\Delta_n-\partial \Delta_n)\\
\end{align*}
其中$W$是这样的$G^n$的子集:至少有一个坐标为$e$的点.自然,我们会有猜想:
\begin{proposition}
    $(p,E(G),B(G),G)$是一个纤维丛.
\end{proposition}
这个猜想正确的可能性很大——若$e$是$G$的非退化点,即$e \to G$是上纤维.这里我们不给出证明.而下一个命题则没有那么自然,尽管我们一样不能给出证明.
\begin{proposition}
    $E(G)$是可缩空间.
\end{proposition}

根据纤维丛诱导的长正合列:
\begin{align*}
    \dots \to 0=\pi_{q+1}(E(G)) \to \pi_{q+1}(B(G)) \to \pi_{q}(G) \to \pi_q(E(G))=0 \to \dots
\end{align*}
我们得到:
\begin{align*}
    \pi_{q+1}(B(G))\cong \pi_q(G)
\end{align*}

对于拓扑群$G$和$H$,下面的群同构,空间同胚显然:
\begin{align*}
    (G \times H)^n \cong G^n \times H^n
\end{align*}

于是单纯集合的同构显然:
\begin{align*}
    E_*(G \times H)\cong E_*(G) \times E_*(H),\quad B_*(G \times H)\cong B_*(G) \times B_*(H)
\end{align*}

由于几何实现是同胚,因此:
\begin{align*}
    B(G\times H)\cong B(G)\times B(H)
\end{align*}

现在假设$G$是交换的群,因此乘法$G \times G \to G$和取逆$G \times G$都是群同态.
\begin{proposition}
    $B(G)$和$E(G)$是交换的拓扑群.
\end{proposition}
\begin{proof}
    仅证明$B(G)$.我们直接考虑:
    \begin{align*}
        B(G)\times B(G) \cong B(G \times G) \to B(G)
    \end{align*}

    因为$G$交换,所有上述乘法交换.实际上我们需要讨论幺元和逆元.但是这涉及到几何实现的同胚.这是我们没有讨论的,因而暂且作罢.
\end{proof}

此外,$p:E(G) \to B(G)$也是同态.$G$对$E(G)$的嵌入(以$B_0(G)$唯一点作为原像集合生成的$G$)也是同态.

\begin{definition}
    设$B^0(G)=G$.$B^n(G)=B^{n-1}(G)$.对于abel群$\pi$,定义其拓扑为离散拓扑,则其零阶同伦群为$\pi$.我们定义Eilenberg-MacLane空间:
    \begin{align*}
        K(\pi,n)=B^n(\pi)
    \end{align*}
    不难根据$\pi_{q+1}(B(G)) \cong \pi_q(G)$逐步降低,最后得到$K$满足我们对Eilenberg-MacLane空间的期待.
\end{definition}
\chapter{Poincare对偶定理}
在把代数拓扑的结果运用到几何拓扑的研究中,Poincare对偶定理是一个关键的里程碑.其给出紧致流形的同调与上同调的关系.这完全是拓扑意义上的.
\section{前置知识}
在叙述Poincare对偶定理前,我们需要给出一些前置的知识.
\section{帽积(cap product)}
我们已经知道,上同调具有cup product.其本质是复合$H^q(X)$和$H^{p}(X)$而得到$H^{p+q}(X)$.这里的关键是对角映射$X \to X\times X$使得我们可以把$X \times X$的上同调映射到$X$中.

设$X$是CW复形.同样考虑$\Delta:X \to X \times X$.其并非是胞腔映射,于是我们考虑其CW逼近$\Delta'$.因此:
\begin{align*}
    \Delta'_*: C_*(X) \to C_*(X \times X) \cong C_*(X) \otimes C_*(X)
\end{align*}
具体的说,则$C_n(X)$的元素被映射到$\sum C_p(X) \otimes C_{n-p}(X)$中.

与$R$做张量积:
\begin{align*}
    \Delta'_*: C_*(X;R) \to C_*(X \times X;R) \cong C_*(X;R) \otimes C_*(X;R)
\end{align*}
对于一个$R$模,我们定义$\cap:C^*(X;\pi)\otimes C_*(X;R) \to C_*(X;\pi)$:
\begin{align*}
    C^*(X;\pi) \otimes_R C_*(X;R) \stackrel{\mathrm{id}\otimes \Delta_*'}{\longrightarrow}C^*(X;R)\otimes C_*(X;R) \otimes C_*(X;R) \stackrel{\epsilon \otimes \mathrm{id}}{\longrightarrow}C_*(X;\pi)
\end{align*}
具体的,我们把:
\begin{align*}
    Hom_R(C_p(X;R),\pi)\otimes_R C_p(X;R) \to \pi
\end{align*}
从而:
\begin{align*}
    \cap:C^p(X;\pi) \otimes_R C_n(X;R) \longrightarrow C_{n-p}(X;\pi)
\end{align*}
如果我们把上同调的链复形看作负的次数,则我们希望$\cap$是一个链复形之间的映射.实际上,只需要验证$\epsilon$是.然而$\epsilon \circ d=0$,所以即可.

因此$\cap$随即诱导了一个重要的映射:
\begin{align*}
    \cap: H^*(X;\pi) \otimes_R H_*(X;R) \rightarrow H_*(C^*(X;\pi)\otimes_R C_*(X;R)) \to H_*(X;\pi)
\end{align*}
这就是帽积的定义.

注意到帽积和杯积的最核心映射都是$\Delta:X \to X\times X$对角映射.所以我们自然猜想其有很强的关系.首先,杯积的定义为:
\begin{align*}
    \Delta'^*:C^*(X;R)\otimes C^*(X;R) \cong C^*(X \times X;R)\to C^*(X;R)
\end{align*}
我们考虑下面的交换图:\begin{tikzcd}
	{C^*(X\times X;R)\otimes_R C_*(X;R)} && {C^*(X\times X;R)\otimes_R C_*(X\times X;R)} \\
	\\
	{C^*(X;R)\otimes_RC_*(X;R)} && R
	\arrow["{id \otimes \Delta'_*}", from=1-1, to=1-3]
	\arrow["{\Delta'^*\otimes id}"', from=1-1, to=3-1]
	\arrow["\epsilon"', from=3-1, to=3-3]
	\arrow["\epsilon", from=1-3, to=3-3]
\end{tikzcd}

其中$\epsilon$是结合映射.交换的原因是我们把$C^*(X\times X;R)$写为$\mathrm{Hom}(C_*(X \times X),R)$.因此$\Delta'^*$和$\Delta'_*$有显而易见得关系.

我们断言杯积和帽积的关系是:
\begin{proposition}[基本恒等式]
    \begin{align*}
        \langle \alpha \cup \beta,x\rangle=\langle \beta,\alpha \cap x\rangle
    \end{align*}
\end{proposition}
\begin{proof}
    根据上述交换图.设$\langle \alpha \cup \beta,x\rangle$意指先走下,再走右.$\langle \beta,\alpha \cap x\rangle$意指先走右,再走下面.(有一个简单的交换关系,可参见教材.)
\end{proof}
在Poincare对偶定理的证明中,我们还需要用到相对帽积.对于$(X,A)$,我们有两种相对帽积:
\begin{align*}
     \cap: H^p(X,A;\pi) \otimes_R H_n(X,A;R) \to H_{n-p}(X;\pi)\quad \cap: H^p(X;\pi) \otimes_R H_n(X,A;R) \to H_{n-p}(X,A;\pi)
\end{align*}
假定$(X,A)$是CW复形对,$\Delta':A \to A \times A$同伦于对角映射.($M\otimes M /N \otimes N\cong M/N \otimes M$)
\begin{align*}
    \Delta'_*:C_*(X,A;R) \to C_*(X,A;R)\otimes C_*(X;R) \quad \Delta'_*:C_*(X,A;R) \to C_*(X;R)\otimes C_*(X,A;R)
\end{align*}

帽积的定义是明晰的,但是其困难在于难以用图的方式去追踪具体的点.
\section{定向和基本类}
在流形课程中我们学习了定向的定义.这里我们引入以某个交换环$R$为系数的定向.原来的定向可以看为$\R$系数的定向.我们先给出一个事实,然后介绍定向的定义.
\begin{proposition}
任何流形总是$\Z_2$可定向的.
\end{proposition}

对于$x \in M$,选取局部坐标邻域$U$,则根据正合公理和Excision公理:
\begin{align*}
    H_i(M,M-x)\cong H_i(U,U-x) \cong \tilde{H_{i-1}}(U-x) \cong \tilde{H}_{i-1}(S^{n-1})
\end{align*}
于是如果$i$不等于$n$,则$H_i(M,M-x)$是$0$.而$H_n(M,M-x)$是$R$.因此我们可以把$H_n(M,M-x)$看作一个生成元的$R$自由模.

\begin{definition}
    $M$的一个在子空间$X$处的$R$基本类意指$z_n \in H_n(M,M-X)$使得对于$\forall x \in X$,映射:
    \begin{align*}
        H_n(M,M-X)\to H_n(M,M-x)
    \end{align*}
    给出的$z_n$的像是$H_n(M,M_x)$的一个生成元.若$X=M$,则$z \in H_n(M)$简称为$M$的一个基本类.

    $M$的一个$R$-定向意指$M$的一个开覆盖$\{U_i\}$,使得每个$U_i$都有一个基本类$z_i$满足:$U_i \cap U_j$非空,则$z_i$和$z_j$被映射到$H_n(M,M-U_i\cap U_j)$的同一个元素.

    若$M$有一个$R$-定向,称其是可定向的.若$R=\Z$,则上述定义与流形的定向定义一致.
\end{definition}

关于定向的相关拓展,可查看\href{http://www.map.mpim-bonn.mpg.de/Orientation_of_manifolds}{网站}.

如果$M$本身有一个$R$-基本类,则其拥有一个自然的$R$-定向.我们只需要把开覆盖$\{U_i\}$中的$z_i$设置为$z$在映射$H_n(M) \to H_n(M,M-U_i)$下的像即可.根据$H_n$函子性,则自然给出定向.然而反过来则不那么容易.在紧致流形的情况下,这是正确的.为了说明此事,我们先给出定理:
\begin{theorem}[Vanishing]
    $M$是流形.对于系数群$\pi$,若$i>n$,则$H_i(M;\pi)=0$.若$M$连通但不紧致,则$\tilde{H}_n(M;\pi)=0$.
\end{theorem}
证明在下节给出.根据消失定理和MV序列,我们给出:

\begin{theorem}
    设$K$是$M$的紧子空间.则对于任何系数群$\pi$,$H_i(M,M-K;\pi)=0$,若$i>n$.并且一个$M$的$R$-定向唯一决定$M$在$K$上的一个基本类.特别的,若$M$是紧的,则$R$-定向决定一个基本类.
\end{theorem}
\begin{proof}
    首先假定$K$被包含在一个局部坐标邻域$U$中.于是:
    \begin{align*}
        H_i(M,M-K;\pi)\cong H_i(U,U-K;\pi) \cong \tilde{H}_{i-1}(U-K;\pi)
    \end{align*}
    由于$U-K$是开集,所以根据消失定理,在$i>n$的时候,$\tilde{H}_{i-1}(U-K;\pi)=0$.想要给出$K$处的基本类,则考虑$M$在$U$处的基本类.$H_n(M,M-U) \to H_n(M,M-K)$给出$K$处的基本类.

    为了给出一般的结果,我们只需要说明,若结论已经对$K,L,K\cap L$成立,则对$K\cup L$也成立.为此,考虑MV序列:
    \begin{align*}
        H_{i+1}(M,M-K\cap L) \to H_i(M,M-K\cup L) \stackrel{\psi}{\rightarrow} H_i(M,M-K)\oplus H_i(M,M-L) \stackrel{\phi}{\rightarrow} H_i(M,M-K\cap L)
    \end{align*}
    若$i>n$,则长正合列给出$H_i(M,M-K\cup L)=0$.设$i=n$,则根据$H_{i+1}=0$得知$\psi$是单射.显然同一个定向给出的$K$和$L$的基本类$z_K$和$z_L$是相容的,则$\phi(z_K,z_L)=0$.根据正合性存在唯一的$z_{K\cup L}$使得$\psi(z_{K\cup L})=(z_K,z_L)$.

    显然$z_{K \cup L}$是$K\cup L$的基本类.
\end{proof}

在定向方面,我们还有一个有意思的结论:
\begin{corollary}
    设$M$是连通的紧致流形,维度大于$0$.则要么$M$是不可定向的,并且$H_n(M;\Z)=0$.要么$M$可定向且:
    \begin{align*}
        H_n(M;\Z) \rightarrow H_n(M,M-x;\Z)\cong \Z
    \end{align*}
\end{corollary}
\begin{proof}
    观察$H_n(M-x)$.这是一个连通但不紧致的流形,所以消失定理告诉我们这是0调.因此根据正合公理:
    \begin{align*}
        H_n(M;\pi) \rightarrow H_n(M,M-x;\pi)\cong \pi
    \end{align*}
    是一个典范的单射.由于$\Z_q$是域,根据泛系数定理,我们有单射:
    \begin{align*}
        H_n(M;\Z)\otimes Z_q \to H_n(M,M-x;\Z)\otimes Z_q \cong Z_q,
    \end{align*}
    对于任何$q$都成立.若$H_n(M;\Z)$不是$0$,则上述单射是同构.定向的关系是一目了然的.
\end{proof}
\section{消失定理的证明}
本节我们简单证明消失定理.当然详细的证明请参见教材.

\begin{lemma}[同调的紧支性]
    对于任何空间$X$和元素$x \in H_q(X)$,存在$X$的紧子空间$K$和$k \in H_q(K)$使得$k$被映射到$x$.
\end{lemma}
\begin{proof}
    考虑$Y$是$X$的CW逼近.记$x=\gamma_*(y)$.考虑$y$在$C_q(X)$中的代表元$z$.则$z$是有限多个$q$胞腔的和,是$C_q(L)$($L$是$Y$的一个有限子复形)中的元素.设$K=\gamma(L)$并且$k$是$z$的同调类在$\gamma$下的像,则$K$是紧空间,并且$k$的像是$x$.(CW逼近是一个函子)
\end{proof}
\begin{lemma}
    设$U$是$\R^n$的开集,则$H_i(U)=0$,若$i \geq n$.
\end{lemma}
\begin{proof}
    设$s \in H_i(U)$.我们说明$i \geq n$的时候$s=0$.根据上面的引理,存在紧集$K$满足$k \in H_i(K)$且$i_*:H_i(K)\to H_i(U)$有$i_*(k)=s$.

    考虑$\R^n$的一个精细的CW结构:由小的n-立方体构成n胞腔,则$K$是紧集意蕴存在一个有限的子复形$L:K \subset L \subset U$.考虑交换图:
    \begin{tikzcd}
	{H_{i+1}(\R^n,L)} & {H_i(\R^n,U)} \\
	{H_i(L)} & {H_i(U)}
	\arrow[from=1-1, to=2-1]
	\arrow[from=2-1, to=2-2]
	\arrow[from=1-1, to=1-2]
	\arrow[from=1-2, to=2-2]
\end{tikzcd}

    注意到当$i \geq n-1$的时候$(\R^n,L)$的同调显然是$0$.从而交换图左边为$0$.而$s$在$H_i(L)$的像中(先映射到$H_i(L)$,再到$H_i(U)$).从而$s=0$.
\end{proof}
\begin{lemma}
    设$U$是$\R^n$中的开集.假定$t \in H_n(\R^n,U)$,且任意$x \in \R^n-U$,都有$t$被映射到$H_n(\R^n,\R^n-x)$中的$0$.则$t=0$.
\end{lemma}
\begin{proof}
    定理的约化版本:若$s \in \tilde{H}_{n-1}(U)$且对于任意$x \in \R^n-x$,都有$s$被映射到$\tilde{H}_{n-1}(\R^n-x)$中的$0$,则$s=0$.

    同样考虑$K$紧集.则$K$包含在一个略大的开集$\tilde{V}$中,且$\overline{V}$是紧集,$\overline{V} \subset K$.于是我们有一个开集$V$使得$s$在$r \in \tilde{H}_{n-1}(V)$的像中.我们断言$r$的像是$0$.

    我们考虑$V$被一个可缩空间$T$包含,且$T$的闭包仍是紧集.记$L=T-(T\cap U)$.对于$x \in \overline{L}$,考虑一个闭的方体$D$包裹$x$.则有限覆盖告诉我们,存在$\{D_1,\dots,D_q\}$满足覆盖$\overline{L}$.

    设$C_i=D_i \cap T$.我们断言,$r$在$\tilde{H}_{n-1}(T-(C_1\cup \dots \cup C_p))$,$0 \leq p \leq q$的像是$0$.这通过归纳得到.

    若$p=0$,则$\tilde{H}_{n-1}(T)$是$0$.于是显然.对于归纳步骤,我们给出MV序列:
    \begin{align*}
        \tilde{H}_{n-1}(T-(C_1\dots C_p)) \to \tilde{H}_{n-1}(T-(C_1\dots C_{p-1}))\oplus \tilde{H}_{n-1}(\R^n-D_p)
    \end{align*}
    是一个单射.因为$\tilde{H}_n(T-(C_1\cup \dots C_{p-1}))\cup (\R^n-D_p)$是$0$.(开集的同调在$i \geq n$时为$0$.)

    注意到$s$映射到右边两个同调群都是$0$(归纳假设和$D_p$可以缩到$U$外的点).所以归纳步骤完成.
    \begin{align*}
        V \subset T-(C_1\cup \dots \cup C_q)\subset T\cap U \subset U
    \end{align*}
    告诉我们$r$的像是$0$.
\end{proof}
用上面三个引理,我们给出消失定理的sketch of proof.
\begin{proof}[sketch of proof of vanishing theorem]
    设$s \in H_i(M)$.我们必须说明若$i >n$,$s=0$.若$M$连通但不紧且$i=n$,$s=0$.

    考虑$M$的紧子空间使得$s$属于$H_i(K)$的像中.则$K$被有限个局部坐标邻域$\{U_i\}$覆盖.根据函子性,要说明$s=0$,只需要说明$H_i(U_1\cup \dots \cup U_q)=0$.

    我们自然可以做数学归纳.因此我们要说明对于特定的$i$值,$H_i(U \cup V)=0$.条件是$U$是局部坐标,而$V$是在$i$特定值情况下$H_i(V)=0$的开集.自然是考虑MV序列:
    \begin{align*}
        H_i(U) \oplus H_i(V) \to H_i(U \cup V) \to \tilde{H}_{i-1}(U\cap V) \to \tilde{H}_{i-1}(U)\oplus \tilde{H}_{i-1}(V)
    \end{align*}
    首先考虑$i>n$的情况.此时,$U\cap V$是$\R^n$开集,所以$H_i(U\cup V)$两边总是$0$.因此$H_i(U\cup V)=0$.

    现在假设$M$是连通的不紧开集,并且赋值$i=n$.我们首先有$H_n(U)=0$,$H_n(V)=0$,$\tilde{H}_{n-1}(U)=0$.根据MV序列,$H_n(U\cup V)=0$等价于$\tilde{H}_{n-1}(U\cap V) \to \tilde{H}_{n-1}(V)$是单射.这个映射由映入映射$U\cap V \to V$诱导.

    我们断言,$H_n(M) \to H_n(M,M-y)$对于$\forall y \in M$都是零同态.若$x \in M$并且$L$是一条连接$x$和$y$的道路,则交换图:
    \begin{tikzcd}
	&& {H_n(M,M-x)} \\
	{H_n(M)} & {H_n(M,M-L)} \\
	&& {H_n(M,M-y)}
	\arrow["\cong", from=2-2, to=1-3]
	\arrow["\cong", from=2-2, to=3-3]
	\arrow[from=2-1, to=2-2]
    \end{tikzcd}成立.
    
    上述交换图表明,如果$s \in H_n(M)$被映射到$H_n(M,M-x)$中的$0$,则也映射到$H_n(M,M-y)$中的$0$.因此我们只需要说明有一个$x$即可.如果$s$处于$H_n(K)$的像中,$K$是紧集,我们可以选择$x \in M-k$使得$K \to M \to (M,M-x)$正合.因此$s$被映射到$H_n(M,M-x)$中的$0$.

    考虑下面的交换图.
    \[
        \begin{tikzcd}
	&& {H_n(U\cup V)} & {H_n(M)} \\
	{H_n(V,U\cap V)} & {H_n(U\cup V,U\cap V)} && {H_n(M,M-y)} \\
	{\tilde{H}_{n-1}(U\cap V)} & {H_n(U,U\cap V)} && {H_n(U,U-y)} \\
	{\tilde{H}_{n-1}(V)}
	\arrow[from=1-3, to=1-4]
	\arrow["0", from=1-4, to=2-4]
	\arrow["\cong"', from=3-4, to=2-4]
	\arrow[from=3-2, to=3-4]
	\arrow["\partial"{description}, from=3-2, to=3-1]
	\arrow["{i_*}", from=3-1, to=4-1]
	\arrow["\partial"{description}, from=2-1, to=3-1]
	\arrow[from=2-1, to=2-2]
	\arrow[from=1-3, to=2-2]
	\arrow["\partial"{description}, from=2-2, to=3-1]
	\arrow[from=3-2, to=2-2]
	\arrow[from=2-2, to=2-4]
    \end{tikzcd}
    \]
    我们的目的是说明图中$i_*$是单射.设$r \in \ker i_*$.注意到$\tilde{H}_{n-1}(U)=0$,则最下方的$\partial$是满射.所以存在$s \in H_n(U,U\cap V)$使得$\partial(s)=r$.我们的目的是说明$s=0$.

    观察$s$,其位于$H_n(U,U\cap V)$.我们只需要说明对于任何$y \in U-(U\cap V)$,都有$s$映射到$H_n(U,U-y)$中的$0$.根据引理可以得到$s=0$.

    在$H_n(V,U\cap V)$中存在$t$使得$\partial(t)=r$.用$t'$和$s'$代表原字母在$H_n(U\cup V,U\cap V)$中的像.于是$\partial(t'-s')=0$,意蕴着$w \in H_n(U\cup V)$.

    从$w$出发,映射到$H_n(M,M-y)$为$0$意味着$s'-t'$也是.

    注意到$t$映射到$H_n(M,M-y)$是$0$,所以$s$也是.因此$s$映射到$H_n(U,U-y)$也是$0$.从而命题得证.
\end{proof}
\section{Poincare对偶定理}
\begin{theorem}[Poincare对偶定理]
    设$M$是紧致$R$-定向流形.则对于任何的$R$模$\pi$,下面的映射$D$是一个同构:
    \begin{align*}
        D:H^p(M;\pi)\cong H_{n-p}(M;\pi)
    \end{align*}
    其中,$D$由$D(\alpha)=\alpha \cap z$给出,$z$是$M$的基本类.
\end{theorem}
回忆基本恒等式(仅仅在系数为$R$的情况下成立):
$$
\langle \alpha \cup \beta,x\rangle=\langle \beta,\alpha \cap x\rangle
$$
因此$D$甚至是杯积的伴随.

现在我们证明定理.
\begin{proof}
    略
\end{proof}
\section{一些应用}
\section{方向覆盖}
这小节我们采取整系数的同调.
\begin{proposition}
    设$M$是连通$n$流形.则存在一个2覆叠$p:\tilde{M}\to M$使得$\tilde{M}$是连通的当且仅当$M$不是可定向的.
\end{proposition}
\begin{proof}
    仅仅记录构造方法:
    \begin{align*}
        \tilde{M}=\{(x,\alpha)|x \in M,\alpha(\text{生成元}) \in H_n(M,M-x)\}, \quad p(x,\alpha)=x
    \end{align*}
    $\tilde{M}$的拓扑基为:$U$为开集,$\beta \in H_n(M,M-U)$
    $$
    \langle U,\beta\rangle=\{(x,\alpha)|x \in U,\beta \text{被映射为}\alpha\}
    $$
    事实上,若$(x,\alpha)\in \langle U,\beta\rangle \cap \langle V,\gamma \rangle$,则可以选择坐标邻域$W:x \in W$.并且存在唯一的$\alpha'$使得$\alpha' \in H_n(M,M-W)$映射到$\alpha$,且$\alpha,\beta$映射到$\alpha'$.

    显然$\tilde{M}$是流形且是2覆叠.我们接下来说明$\tilde{M}$可定向.实际上,如果$U$是一个局部坐标,$(x,\alpha)\in \langle U,\beta \rangle$,下面的映射图诱导了同构:
    \[
        \begin{tikzcd}
	    {(\tilde{M},\tilde{M}-\langle U,\beta\rangle)} & {(M,M-U)} \\
	    {(\tilde{M},\tilde{M}-(x,a))} & {(M,M-x)} \\
	    {(\langle U,\beta\rangle,\langle U,\beta\rangle-(x,\alpha))} & {(U,U-x)}
	    \arrow[from=1-1, to=2-1]
	    \arrow[from=3-1, to=2-1]
	    \arrow["{p }", from=3-1, to=3-2]
	\arrow[from=3-2, to=2-2]
	\arrow[from=1-2, to=2-2]
    \end{tikzcd}
    \]
    通过这个图,$\beta \in H_n(M,M-U)$具体了一个$\tilde{\beta}\in H_n(\tilde{M},\tilde{M}-\langle U,\beta\rangle)$,并且独立于$(x,\alpha)$的选择.

    对于开集$U$,只需要指定其定向为$\beta$即可.

    根据定义,$M$的一个定向是一个特殊的截面$s:M \to \tilde{M}$.改变符号,$-s$表明$\tilde{M}=\mathrm{im}(s)\cup \mathrm{im}(-s)$且不交.如果$M$可定向,则$\tilde{M}$写为两个同胚于$M$的不交并.$\tilde{M}$不连通.
    
    如果$M$不可定向,则$\tilde{M}$连通.假设$\tilde{M}$不连通,则存在$(x,\alpha)$和$(x,-\alpha)$在两个连通分支.容易说明这两个连通分支给出了定向.
\end{proof}
\begin{proposition}
    如果$M$是单连通的,或者$\pi_1(M)$不包含指数为$2$的子群,则$M$是可定向的.如果$M$可定向,则$M$有且仅有两个方向.
\end{proposition}
\begin{proof}
    如果$M$不可定向,则$p_*(\pi_1(\tilde{M}))$是$\pi_1(M)$的指数为$2$的子群.着给出了第一个命题.第二个命题实属显然.
\end{proof}
\section{一些推论}
考虑整数系数,则根据对偶,$H^p(M)\cong H_{n-p}(M)$.若$p=0$,则$H^0(M)\cong H_n(M)$.于是$H_n(M)=\Z$.注意到同构由:
\begin{align*}
    D:H^0(M) \to H_n(M): D(\alpha)=\alpha \cap z
\end{align*}
给出.此时$H_n(M)$的生成元就是其基本类.因为基本类必须映射到$\Z$的$1$.

若$p=n$,则$H_0(M)\cong H^n(M)=\Z$.我们也得到了$H^n(M)$的生成元$\zeta$.并且$\zeta$和$z$对偶:$\langle \zeta,z\rangle=1$.(用奇异同调清楚知道$H_0$和$H^0$的情况.)

\begin{corollary}
    设$T_p$是$H^p(M)$的挠子群.杯积$\alpha \otimes \beta$给出$\langle \alpha\beta,z\rangle$给出了非奇异的双线性:
    \begin{align*}
        H^p(M)/T_p \otimes H^{n-p}(M)/T_{n-p}\to \Z
    \end{align*}
\end{corollary}
\begin{proof}
    若$\alpha \in T_p$,则存在$ra=0$.从而$r(\alpha \cup \beta)=0$.因为$H^n(M)\cong \Z$,则$\alpha \cup \beta=0$.因此杯积在挠元的时候为$0$.上述映射是良定的.

    注意到$\mathrm{Ext}_{\Z}^1(\Z_r,\Z)=\Z_r$,并且$H_p(M)$一定是有限生成的,则$\mathrm{Ext}_{\Z}^1(H_*(M),\Z)$是一个挠群.从而根据泛系数定理,由:
    \begin{align*}
        H^p(M)/T_p\cong \mathrm{Hom}(H_p(M),\Z)
    \end{align*}
    若$\alpha \in H^p(M)$投射到自由abel群$H^p(M)/T_p$的一个生成元,则存在$\alpha \in H_p(M)$使得$\langle \alpha,a\rangle=1$.用对偶,则存在$\beta \in H^{n-p}(M)$使得$\beta \cap z=a$:$\langle \beta \cup \alpha,z\rangle=\langle \alpha,\beta \cap z\rangle=1$
\end{proof}

借此,我们甚至可以给出$\C P^n$的杯积.(给出上同调环)
\begin{corollary}
    作为分次环,$H^*(\C P^n)$是多项式代数$Z[\alpha]/)\alpha^{n+1}$,其中$\alpha$的度是$2$.
\end{corollary}
\begin{proof}
显然$H^{2q}(\C P^n)$是自由Abelian群,其生成元只有一个.$0\leq q \leq n$.我们要说明的是$H^{2q}(\C P^n)$的生成元是$\alpha^q$.

注意到$\C P^{n-1}$是$\C P^{n}$的$2n-1$骨架,所以显然有同构:$H^{2q}(\C P^n) \to H^{2q}(\C P^{n-1})$在$q<n$时候的同构.根据$n$进行归纳.当$n=1$,$\C P^1 \cong S^2$.此时结论是平凡的.

现在假定$n-1$成立.从而$q<n$时,$\alpha^q$生成了$H^{2q}(\C P^n)$.根据上面的推论,注意到存在$H^{2q-2}(\C P^n)$使得$\alpha \cup \beta$与$z$做内积为$1$.而$\alpha \cup \beta$显然是$H^{2n}(M)$的生成元,因此我们只需要说明$\beta$是$H^{2q-2}$的生成元.实际上这是明显的,因为$\beta$来自于$H_2(M)$的生成元.
\end{proof}
考虑到$M$的上同调中的挠元,更方便的是考虑$R$是一个域.
\begin{proposition}
    任何可定向流形$M$对于交换环$R$来说都是$R$可定向的.
\end{proposition}
\begin{proposition}
    设$M$是连通的紧致$R$定向流形,其中$R$是域.则$\alpha \otimes \beta \to \langle \alpha\cup \beta,z\rangle$定义了非奇异的映射:
    \begin{align*}
        H^p(M;R)\otimes_R H^{n-p}(M;R)\to R
    \end{align*}
\end{proposition}
最后的推论是:
\begin{corollary}
    作为分次环,$H^*(\R P^n;\Z_2)$是多项式$Z_2[\alpha]/(\alpha^{n+1})$.其中$\alpha=1$,即$\alpha^q$是$H^q(\R P^n;\Z_2)$的非零元.
\end{corollary}
\chapter{流形的指标,带边流形}
Poincar\'{e}对偶给出了一个流形$M$的欧拉示性数的强烈限制.同时它也给出了流形的指标概念.

我们先回忆欧拉示性数的概念.
\begin{definition}
    拓扑空间$X$的欧拉示性数$\chi(X)$定义为:
    \begin{align*}
        \chi(X)=\sum_{i}(-1)^i \mathrm{rank}H_i(X;\Z)
    \end{align*}
\end{definition}
这里的系数并不是紧要的.根据泛系数定理,如果我们给定一个PID环$R$:
\begin{align*}
    H_i(X;R)=(H_i(X;\Z)\otimes R)\oplus \mathrm{Tor}_{1}^{R}(H_{i-1}(X,\Z),R)   \\ \Rightarrow \mathrm{rank}H_i(X;R)=\mathrm{rank}H_i(X;\Z)
\end{align*}

特别的,取$R$是一个域$F$,我们就有:
\begin{align*}
    \chi(X)=\sum_{i}(-1)^i \mathrm{dim}H_i(X;F)
\end{align*}

如果$X$拥有有限的CW结构,我们也可以使用一些代数的技术得到:
\begin{align*}
    \chi(X)=\sum_{i}(-1)^i \mathrm{rank}C_i(X;\Z)
\end{align*}

下面的命题说明紧致奇数维流形的欧拉示性数只能是$0$:
\begin{proposition}
    设$M$是奇数维紧致流形,则$\chi(M)=0$
\end{proposition}
\begin{proof}
    设$F=\Z_2$,从而泛系数定理和Poincar\'{e}对偶给出:
    \begin{align*}
        H_{i}(M;\Z_2)\cong H^i(M;\Z_2)\cong H_{n-i}(M;\Z_2)
    \end{align*}
    因为$n$是奇数,从而$i$与$n-i$两两配对.
\end{proof}

下面考虑紧致的可定向偶数维流形.维数$n=2m$.则:
\begin{align*}
    \chi(M)=\sum_{i=0}^{m-1}(-1)^i2\dim H_i(M)+(-1)^m \dim H_m(M) 
\end{align*}

现在我们取域为有理数$\Q$.实际上核心的考量是杯积:
\begin{align*}
    \phi:H^m(M)\otimes H^m(M)\to \Q
\end{align*}
其中$\phi(\alpha,\beta)=\langle \alpha \cup \beta,z\rangle$.这是非退化的二次型.如果$H_m(M)$的维数是偶数,则$\phi$对称.反之则$\phi$反对称.
\begin{proposition}
    若$M$是紧致定向$n$流形.若$n$模4余2,则$\chi(M)$是偶数.
\end{proposition}
\begin{proof}
    反对称性给出$H_m(M)$的维数是偶数维.
\end{proof}

因此我们需要关注的是维数为$4$的倍数的流形的欧拉示性数.现在我们取域系数为$\R$.假定我们都熟悉$\R$上对称二次型的标准形式.

\begin{definition}
    设$M$是$4k$维紧致可定向流形.定义$M$的指标$I(M)$维杯积给出的二次型:
    \begin{align*}
        H^{2k}(M;\R)\otimes JH^{2k}(M;\R) \to \R
    \end{align*}
    的符号差(正指数减去负指数).若$M$的维数不为$4k$维,则定义$I(M)=0$.
\end{definition}

\chapter{同调,上同调与分类空间$K(\pi,n)$}
\ifx\allfiles\undefined
	
	% 如果有这一部分的参考文献的话,在这里加上
	% 没有的话不需要
	% 因此各个部分的参考文献可以分开放置
	% 也可以统一放在主文件末尾。
	
	%  bibfile.bib是放置参考文献的文件,可以用zotero导出。
	% \bibliography{bibfile}
	
	\end{document}
	\else
	\fi



\ifx\allfiles\undefined

	% 如果有这一部分另外的package,在这里加上
	% 没有的话不需要
	
	\begin{document}
\else
\fi

\part{流形的代数拓扑}

\chapter{De Rham理论}
\section{}
\begin{proposition}{}
   
\end{proposition}
\chapter{\u{C}ech-de Rham复形}

\chapter{代数拓扑与谱序列}

\chapter{示性类}
\section{}
\begin{theorem}[Borsuk-Ulam定理]{Borsuk-Ulam定理1}
    设$f:S^n \to \mathbb{R}^n$是连续函数。则一定存在点$p \in S^n$使得$f(p)=f(-p)$。
\end{theorem}
\ifx\allfiles\undefined
	
	% 如果有这一部分的参考文献的话,在这里加上
	% 没有的话不需要
	% 因此各个部分的参考文献可以分开放置
	% 也可以统一放在主文件末尾。
	
	%  bibfile.bib是放置参考文献的文件,可以用zotero导出。
	% \bibliography{bibfile}
	
	\end{document}
	\else
	\fi
\ifx\allfiles\undefined

	% 如果有这一部分另外的package,在这里加上
	% 没有的话不需要
	
	\begin{document}
\else
\fi
\part{指标理论}
\chapter{Clifford代数及其表示}
\section{Clifford表示}
 Most of the important applications of Clifford algebras come through a detailed understanding of their representations.We can deduce them with a deep understanding of the classification in Section 4.

 Now we begin with a general definition.
    
  \begin{definition}\label{def:reps}
  Let $K \supset k$ be a field containing $k$.Then a \textbf{K-representation} of Clifford algebra $\mathrm{Cl}(V,q)$ is a $k$-algebra homomorphism
    \begin{align*}
        \rho:\mathrm{Cl}(V,q) \to \mathrm{Hom}_{K}(W,W)
    \end{align*}

    where $\mathrm{Hom}_{K}(W,W)$ is the algebra of linear transformations of a finite dimensional vector space $W$ of $K$.

    We call $W$ a $\mathrm{Cl}(V,q)$-module over $K$ and write $\rho(\varphi)(w)=\varphi \cdot w$ to simplify the notation.
  \end{definition}



We shall be interested in the cases that $K=\R,\C,\HH$.Note that a complex vector space is just a real space $W$ together with a linear map $J:W \to W$ such that $J^2=-\mathrm{Id}$.So a complex reps. is just a real resp. $\rho:\mathrm{Cl}(V,q)$ such that:
\begin{align*}
    \rho(\varphi)\circ J=J \circ \rho(\varphi)
\end{align*}

Thus the image of $\rho$ commmutes with the subalgebra $\mathrm{span}{\mathrm{Id},J}\cong $. This algebra is called a \textbf{commmuting subalgebra} of $\rho$.

Similar method applies to quaternionic resp. of $\mathrm{Cl}(V,q)$.


    Any complex resp. can be extended to a reps. of $\mathrm{Cl}_{r,s}\otimes $.This is also similar for quaternionic resp.


\begin{definition}
   Let $V,q,k \subset K$ be as in definition \ref{def:reps}.A resp. of $\mathrm{Cl}(V,q)$ will be said to be reducible if the vector space $W$ can be written as a non-trivial direct sum( over $K$):
   \begin{align*}
    W=W_1 \oplus W_2
   \end{align*}

   such that $\rho(\varphi)(W_i)\subset W_i, \forall \varphi \in \mathrm{Cl}(V,q)$.

   A reps. is called irreducible if it is not reducible.
\end{definition}
  


This definition is not conventional as "irreducible" is often regarded as a property that there are no proper invariant subspaces. However.

\chapter{Spin几何和Dirac算子}



\section{向量丛上的Spin结构}
\section{Spin流形}
\section{Clifford丛,旋量丛}
In this section, we suppose that readers have known some basic propertys of principal bundle and the associated bundle construction.We will briefly introduce these conceptions in appendix B.




\section{联络}
\section{Dirac算子}
\section{基本椭圆算子}
\section{$Cl_k$线性Dirac算子}
\section{消灭定理}
\chapter{指标定理}

\chapter{Chern-Weil理论}
\chapter{热核}



%\problemset
\begin{problemset}
  \item Solve the Riemann Conjecture.
\end{problemset}


\chapter{复几何上的指标定理}
\ifx\allfiles\undefined
	
	% 如果有这一部分的参考文献的话,在这里加上
	% 没有的话不需要
	% 因此各个部分的参考文献可以分开放置
	% 也可以统一放在主文件末尾。
	
	%  bibfile.bib是放置参考文献的文件,可以用zotero导出。
	% \bibliography{bibfile}
	
	\end{document}
	\else
	\fi


\nocite{en2,en3}

\printbibliography[heading=bibintoc, title=\ebibname]
\appendix


\chapter{示性类与拓扑K理论}

This appendix covers some of the basic property of characteristic classes used in this note.We discuss these conceptions in a pure topological viewpoint.

\section{Grassmannians流形的上同调环}

\textbf{Summation Operator} is an abbreviation used to express the summation of numbers, it plays an important role in statistics and econometrics analysis. If $\{x_i: i=1, 2, \ldots, n\}$ is a sequence of $n$ numbers, the summation of the $n$ numbers is:

\begin{equation}
\sum_{i=1}^n x_i \equiv x_1 + x_2 +\cdots + x_n
\end{equation}
\chapter{Principal G-bundle}

\end{document}

\end{document}
