\def\allfiles{}
\documentclass[lang=cn,newtx,10pt,scheme=chinese]{elegantbook}

\title{几何与拓扑笔记}
\subtitle{主体:微分几何与代数拓扑}

\author{夏目凉}
\institute{陈省身数学研究所 Chern Institute of Mathematica}
\date{\today}
\version{1.0}
\bioinfo{Bio}{Information}

\extrainfo{\textcolor{red}{黑格尔在某个地方说过,一切伟大的世界历史事变和人物,可以说都出现两次。他忘记补充一点:第一次是作为悲剧出现,第二次是作为笑剧出现。}
}
\setcounter{tocdepth}{3}

\logo{logo-blue.png}
\cover{cover.jpg}
%记号命名
\newcommand\til[1]{\tilde{#1}}
\def\g{\mathfrak{g}}
\newcommand{\II}{\mathrm{II}}
\newcommand{\U}{\mathcal{U}}
\newcommand{\id}{\mathrm{id}}
\newcommand{\Hom}{\mathrm{Hom}}
\newcommand{\Ext}{\mathrm{Ext}}
\newcommand{\Tor}{\mathrm{Tor}}
\newcommand{\N}{\mathbb{N}}
\newcommand{\OO}{\mathcal{O}}
\newcommand{\Z}{\mathbb{Z}}
\newcommand{\Q}{\mathbb{Q}}
\newcommand{\F}{\mathbb{F}}
\newcommand{\R}{\mathbb{R}}
\newcommand{\PP}{\mathbb{P}}
\newcommand{\C}{\mathbb{C}}
\newcommand{\HH}{\mathbb{H}}
\newcommand{\RP}{\mathbb{RP}}
\newcommand{\dd}{\mathrm{d}}
\newcommand{\HdR}[1]{H_{\mathrm{dR}}^#1}
\newcommand{\ii}{\sqrt{-1}}
\newcommand{\la}{\langle}
\newcommand{\ra}{\rangle}
\newcommand{\pa}[3][]{
	\frac{\partial^{#1} #2}{\partial {#3}^{#1}}
	}
\newcommand{\bpa}{\bar{\partial}}
\newcommand{\npa}{\partial}

% 本文档命令
\usepackage{array}
\newcommand{\ccr}[1]{\makecell{{\color{#1}\rule{1cm}{1cm}}}}
%调用图表包
\usepackage{quiver}

%极限、余极限与滤过余极限的实现
\makeatletter
\newcommand{\Colim@}[2]{
  \vtop{\m@th\ialign{##\cr
    \hfil$#1\operator@font lim$\hfil\cr
    \noalign{\nointerlineskip\kern1.5\ex@}#2\cr
    \noalign{\nointerlineskip\kern-\ex@}\cr}}%
}
\newcommand{\Colim}{%
  \mathop{\mathpalette\Colim@{\rightarrowfill@\scriptscriptstyle}}\nmlimits@
}
\makeatother

\makeatletter
\newcommand{\Lim@}[2]{%
  \vtop{\m@th\ialign{##\cr
    \hfil$#1\operator@font lim$\hfil\cr
    \noalign{\nointerlineskip\kern1.5\ex@}#2\cr
    \noalign{\nointerlineskip\kern-\ex@}\cr}}%
}
\newcommand{\Lim}{%
  \mathop{\mathpalette\Lim@{\leftarrowfill@\scriptscriptstyle}}\nmlimits@
}
\makeatother


\makeatletter
\newcommand{\colim@}[2]{%
  \vtop{\m@th\ialign{##\cr
    \hfil$#1\operator@font oli~$\hfil \cr
    \noalign{\nointerlineskip\kern1.5\ex@}#2\cr
    \noalign{\nointerlineskip\kern-\ex@}\cr}}%
}
\newcommand{\colim}{%
  \mathop{\mathrm{c}\mathpalette\colim@{\rightarrowfill@\scriptscriptstyle}\mathrm{\!\!m}}\nmlimits@
}
\makeatother

\makeatletter
\newcommand{\cone@}[1]{%
  \vtop{\m@th\ialign{##\cr
    \hfil$#1\operator@font cone$\hfil\cr
    \noalign{\nointerlineskip\kern1.5\ex@}\cr
    \noalign{\nointerlineskip\kern-\ex@}\cr}}%
}
\newcommand{\cone}{%
  \mathop{\mathpalette\cone@{\scriptscriptstyle}}\nmlimits@
}
\makeatother



% 修改标题页的橙色带
\definecolor{customcolor}{RGB}{32,178,170}
\colorlet{coverlinecolor}{customcolor}
\usepackage{cprotect}

\addbibresource[location=local]{reference.bib} % 参考文献,不要删除

%编译顺序
%\includeonly{title,1,2,3}

\begin{document}

\maketitle

\frontmatter

\tableofcontents

\mainmatter
\include{part/Riemann Geometry/Riemann geometry.tex}
\ifx\allfiles\undefined

	% 如果有这一部分另外的package,在这里加上
	% 没有的话不需要
	
	\begin{document}
\else
\fi

\part{李群}
\chapter{2023.02.16}
\section{一些定义}
\begin{definition}
\textbf{拓扑群}$(G, \cdot )$:$G$是群且是一个拓扑空间,满足$\cdot:G \times G \to G$和$()^{-1}:G \to G$都是连续的。

\textbf{李群}:$(G,\cdot)$:$G$是拓扑群,且本身是一个微分流形,满足$\cdot: G \times G \to G$和$()^{-1}:G \to G$是光滑的。
\end{definition}

为什么没有拓扑群专门的课程呢?
\begin{proposition}[Hilbert第五问题]
    拓扑群+局部欧氏能否成为李群呢?

    叙述如下:

    任意局部欧的拓扑群是李群且微分结构是唯一实解析的。
\end{proposition}
该问题在1950年代已经被证明了。从而拓扑群方向基本没有人研究了。
\begin{proposition}
    任意连通的微分流形是道路连通的。
\end{proposition}
\begin{proof}
    
\end{proof}
\begin{definition}[李子群]
    设$H$是李群$G$的子群。若$H$是$G$的浸入子流形,则$H$是$G$的李子群。
\end{definition}
回忆:什么是浸入子流形?

\begin{proposition}[Yamabe]
    李群的道路连通的子群是李子群。
\end{proposition}
但是李群的连通子群就不一定是李子群了。
\begin{example}[李子群但不是嵌入李子群]
    考虑$T^2$作为李群(显然是李群)。考虑子流形:
    $$
    H^n=\{(e^{it},e^{int})|n \in \N,t \in \R\}
    $$
    $$
    H^a=\{(e^{it},e^{iat})|a \in \R/\Q,a>0,t \in \R\}
    $$
    
    $H^n$是嵌入的李子群,其同胚于$S^1$。但$H^a$是非嵌入的李子群,只是浸入李子群。($\overline{H^a}=T^2$.)
\end{example}
\section{李群与矩阵李群}
一般线性群:
$$
\mathrm{GL}(n,\C)=\{A \in \C^{n\times n}||A|\neq 0\}
$$
$$
\mathrm{GL}(n,\R)=\{A \in \R^{n\times n}||A|\neq 0\}
$$
其中$\mathrm{GL}(n,\C)$是复李群(乘法是全纯的),实李群。另外一个是实李群。
\begin{definition}[矩阵李群]
    $\mathrm{GL}(n,\C)$的闭子群$G$称为矩阵李群。换句话说,若$\mathrm{GL}(n,\C)$满足若序列$\{A_m\}\subset G$,则$\lim_{n \to \infty}A_m =A \in M_n(\C)$,有$A \in G$或者不可逆。
\end{definition}
注意:

1.李群不一定是矩阵李群。但紧李群一定是矩阵李群。(Peter-Weyl)

2.矩阵李群是$\mathrm{GL}(n,\C)$的闭子群,而不是$M_n(\C)$的“闭子群”。

3.任何闭子群都是嵌入李子群(Cartan)。

\begin{example}
    $\mathrm{GL}(n,\R)$是矩阵李群,但不是$M_n(\C)$的闭子集。

    $\mathrm{GL}(n,\Q)$是群但并非矩阵李群。其不是闭集。
\end{example}
\begin{example}[矩阵群的例子]
    A.一般线性群。$\mathrm{GL}(n,\R)$,$\mathrm{GL}(n,\C)$。

    B.特殊线性群。$\mathrm{SL}(n,\C)=\{A \in \mathrm{GL}(n,\C)||A|=1\}$,$\mathrm{SL}(n,\R)=\{A \in \mathrm{GL}(n,\R)||A|=1\}$。

    C.正交群:$\mathrm{O}(n,\R)=\{A \in \mathrm{GL}(n,\R),A^T A=I_n\}$.$\mathrm{O}(n,\C)=\{A \in \mathrm{GL}(n,\C)|A^T A=I_n\}$

    D.特殊正交群:$\mathrm{SO}(n,\R)=\{A \in \mathrm{O}(n,\R)||A|=1\}$。

    E.酉群:$A \in M_n(\C)$,使得$A*A =I_n$。其中$A*=(\overline{A})^T$。这样的矩阵称为酉矩阵。酉群:$U(n)=\{A \in \mathrm{GL}(n,\C):A^*A=I_n\}$。酉矩阵变换保证酉内积的不变。\textbf{注意}:$U(n)$是实李群而非复李群。特殊辛群:$SU(n)=\{A \in U(n)|\mathrm{det}A=1\}$

    F.辛群:$S_p(n)$。$\mathrm{GL}(n,\mathbb{H})=\{A\in \mathbb{H}^{n \times n}:\mathrm{det} A\neq 0\}$是实李群。注意,由于四元数一些神秘的性质,虽然行列式,迹是良定义的,但是有$\mathrm{det}(AB)\neq \mathrm{det}(BA)$,$\mathrm{Tr}(AB)\neq \mathrm{Tr}(BA)$。
 
    $S_p(n)=\{A \in \mathrm{GL}(n,\mathbb{H})|A*A=I\}$。称为辛群,不是复李群。

    $S_p(1)\cong SU(2) \cong S^3$。

    G.实辛群。$\R^{2n}$的反对称双线性型。$w(x,y)=\sum_{j=1}^n(x_jy_{n+j}-x_{n+j}y_j)$。

    $$
    S_p(n,\R)=\{A \in M_{2n}(\R)|:w(Ax,Ay)=w(x,y),\forall x,y \in \R^{2n}\}=\{A \in \mathrm{SL}(2n,\R):\Omega A^T \Omega=A^{-1}\}, \Omega=\begin{bmatrix}
        0&I\\-I&0
    \end{bmatrix}
    $$
\end{example}
\section{球面上的李群结构}
$S^n$有李群结构等价于$n=1$或者$n=3$。并且$S^1 \cong SO(2,\R)$,$S^3 \cong \mathrm{SU}(2) \cong S_p(1)$.

\begin{proposition}
    李群上的切丛是平凡的。即对于$n$维李群,有$TG \cong G \times \R^n$。
\end{proposition}
\begin{proof}
    $$
    G\times T_e G \to TG:(g,v)\mapsto (g,(L_g)_* v)
    $$
    $L_g$是左平移作用,是一个微分同胚。详细证明可以见梅加强。
\end{proof}

对于$S^n$,若$S^n$的切丛是平凡的,则$n=1$,$n=3$,$n=7$。因而想要关注$S^n$是否为李群,只需要考虑这三个。

\chapter{2023.02.23}
\section{李群的局部性质}
\subsection{单位元邻域生成连通子群}
\begin{proposition}
    
连通李群$G$可以由$e$处任意邻域生成。
\end{proposition}
\begin{lemma}
    设$H$是李群$G$的开子群,则$H$是$G$的闭子群。
\end{lemma}
\begin{proof}
    这一点在拓扑群的考量中就可实现。固定$g \in G$,$L_g:G \to G$是一个微分同胚。故任意左陪集$gH$是开集。考虑$G \times H \to G$群作用,则$G=\bigcup_{g \in G}gH$且是不交并。从而$H$的余集是开集,$H$是闭子群。
\end{proof}
因此开子集是闭子群。是即开又闭的集合。
\begin{lemma}
    设$U$是$e$处的开邻域,则$U$的生成子群$H$是开集。
\end{lemma}
\begin{proof}
    根据定义,群$H$包含所有的乘积$x_1^{\epsilon_1}\dots x_n^{\epsilon_n}$。故$H=\bigcup_{x \in U}xU$是开集。
\end{proof}
两个引理立马就得到了命题。
\begin{proposition}
    设$G$是李群,$G_0$是$G$的单位连通分支,则$G_0$是$G$的子群。
\end{proposition}
\begin{proof}
    对给定的$x \in G_0$,其中$x$可以是任意指定的。由于$e \in G_0 \cap x^{-1}G_0$,从而$G_0=x^{-1}G_0$。于是$\forall y\in G_0$,$x^{-1}y \in G_0$.$G_0$是子群。
\end{proof}
\begin{corollary}
    $\forall x_1,x_2 \in G$,则$x_1G_0 \cap x_2G_0$要么是空集,要么是$x_1G_0=x_2G_0$
\end{corollary}
\begin{proof}
    如果相交非空,则两者都同胚于$G_0$。
\end{proof}
\begin{corollary}
    $G_0$是$G$的正规子群。
\end{corollary}
\begin{proof}
    考虑群作用$G\times G \to G:(g,h)\mapsto ghg^{-1}$。于是$G$成为了若干共轭类的并。

    由于$e \in gG_0g^{-1} \cap eG_0 e^{-1}$,因此轨道$gG_0g^{-1}=G_0$。于是$G_0$是正规子群。
\end{proof}
正规子群意味着可以做商。那么:
$$
1 \to G_0 \to G \to G/G_0 \to G
$$
是正合列。

\subsection{单位元的切空间具有Lie代数结构}
\begin{definition}[李代数]
    $V$是有限维$k$向量空间。$[,]$是$k$双线性:$[,]V \times V \to V$,满足反对称和Jacobbi恒等式:
    $$
    [X,[Y,Z]]+[Y,[Z,X]]+[Z,[X,Y]]=0
    $$

\end{definition}
\begin{example}
    $\mathrm{GL}(n,\C)$是李代数。$[A,B]=AB-BA$。此时这是Lie代数。
\end{example}
\begin{theorem}[Ado定理]
    有限维李代数是$(\mathrm{GL}(n,\C),[,])$的李子代数。
\end{theorem}

我们自然关注李群的李代数问题。
\begin{definition}[左不变向量场]
    李群$G$上的向量场$X$称为左不变的向量场,如果$L_g(Xh)=X(gh)$对于任意的$g,h$都成立。其中$L_g$是左平移作用。
\end{definition}
\begin{proposition}
    设$g$是$G$上左不变向量场的集合,$T_eG$是$G$在$e$处的切空间,则$I:g \to T_eG$,$X \to X(e)$是向量空间的同构。
\end{proposition}
\begin{proof}
    根据左不变的定义,左不变向量场由$T_eG$中的元素确定。即$X(g)=L_g(X(e))$,从而$I$是双射。

    根据向量场的运算$(X+Y)(e)=X(e)+Y(e)$,$(kX)(e)=k(X(e))$知道$I$是同构。
\end{proof}
由于$g$是有限维的李代数,其中$[X,Y]=XY-YX$,则$I$诱导了$T_e G$上的李括号结构:
$$
[X(e),Y(e)]:=[XY-YX](e)
$$
\begin{definition}
    $(T_e G,[,])$称为李群$G$的李代数。
\end{definition}

我们借此计算一下一些李群的李代数。
\begin{example}
    $\mathrm{GL}(n,\R)$的李代数为$\mathrm{GL}(n,\R)$。李括号为$XY-YX$.

    $\mathrm{GL}(n,\R)$在$I$处的切空间为$\mathrm{GL}(n,\R)$。左不变向量场由$\tilde{X}(I)=X \in G$决定。
\end{example}
\begin{example}
    $\mathrm{SL}(n,\R)$的李代数。$A \in \mathrm{SL}(n,\R)$,则$|A|=1$。

    $\mathrm{det}(I+\epsilon X)=1+\epsilon tr(X)+o(\epsilon^2)=1$,则
\end{example}
\begin{example}
    
\end{example}
\subsection{Hall-Wilt恒等式}
令$[x,y]=x^{-1}y^{-1}xy$是群上的交换子。考虑伴随作用$x^y=y^{-1}xy$.则下面有恒等式:
$$
[[x,y^{-1}],z]^y[[y,z^{-1}],x]^z[[z.x^{-1}],y]^x=1
$$
\begin{example}
    
\end{example}
\section{李群与李代数的关系——指数映射}
1.矩阵(复数元)的指数映射。
$$
e^X:=\sum_{n=0}^\infty \frac{X^m}{m!}, \quad X \in \mathrm{GL}(n,\C)
$$

\begin{proposition}
     对于任何$X \in \mathrm{GL}(n,\C)$,上述级数收敛。    
\end{proposition}
\begin{proof}
    分析方法:

    考虑$\mathrm{GL}(n,\C)$上的范数是所有元素的平方和的平方根。若$\lim X_n \to X$,则有$\lim |X_n -X|=0$。

    从而转化为:
    $$
    \|\sum_{m=0}^\infty \frac{X^m}{m!}\| \leq \sum_{m=0}^\infty \frac{\|X\|^m}{m!}=e^{\|X\|}
    $$
    收敛。(代数范数)

    代数方法:考虑可对角化矩阵
    $$
    X=CDC^{-1}
    $$
    若$D$对角,则显然收敛。若$X$幂零,则显然也收敛。一般的情况而言,由于任意的$X$可唯一分解为$X=S+N$,且$SN=NS$。从而$e^X=e^{N+S}=e^N e^S$。故收敛。
\end{proof}
$(R,+) \to (\mathrm{GL}(n,\C))$是李群同态。

\begin{lemma}
    
    设$\mathrm{Sym}_n(\R)$是$n$阶实对称矩阵,$\mathrm{Sym}_n^+(\R)$是正定矩阵。则$\mathrm{Sym}_n(\R) \to \mathrm{Sym}_n^+(\R)$是微分同胚。
\end{lemma}

\begin{proposition}
    对于$A \times $
\end{proposition}

\section{矩阵李群的性质与李代数}
\chapter{2023.03.02}
\section{李群的指数映射}
\begin{definition}[李群同态]
    设$H,G$是李群。若$\varphi:H \to G$是光滑的群同态,则称$\varphi$是李群同态。
\end{definition}
\begin{definition}[李代数同态]
    设$g,h$是李代数,线性映射$\varphi:h \to g$称为李代数同态,若$\varphi[x,y]_h =[\varphi(x),\varphi(y)]_g$.
\end{definition}
\begin{definition}[单参数变换群]
    李群同态$\varphi:(\R,+) \to G$称为单参数变换群。
\end{definition}
\begin{proposition}\label{pro:eee}
    设$G$是李群且李代数是$g$。对于任意给定的$X \in g$,存在唯一的单参数变换子群$\varphi_x:\R \to G$满足:
    $$
    \dfrac{d}{dt}|_{t=0}\varphi_x(t)=X(e)
    $$
\end{proposition}
\begin{proof}[积分曲线+ODE解的完备性]
    只需证明任意给定的$X \in g$,存在完备的积分曲线$\varphi_X$使得:$\varphi_X(t+s)=\varphi_X(t)\varphi_X(s)$。

    对于任意的$X \in g$,存在$\epsilon>0$使得在$(-\epsilon,\epsilon)$,$X$的积分曲线$\varphi_X(t)$存在,且满足$\varphi_X(0)=e$,$\dfrac{d}{dt}|_{t=0}\varphi_x(t)=X(e)$.这是可以做到的,因为是局部的性质。

    我们验证这是同态。设$\varphi_1(t)=\varphi_X(s+t),\varphi_2(t)=\varphi_X(s)\varphi_X(t)$。则$\dfrac{d}{dt}|_{t=0}\varphi_1(t)=X(\varphi_X(s))$,$\dfrac{d}{dt}|_{t=0}\varphi_2(t)=\dfrac{d}{dt}|_{t=0}L_{\varphi_X(s)}\varphi_X(t)=(L_{\varphi_X(s)})_*\dfrac{d}{dt}|_{t=0}\varphi_x(t)=(L_{\varphi_X(S)})_* X(e)=X(\varphi_X(s))$。

    根据ODE解的存在唯一性,则$\varphi_1(t)=\varphi_2(t)$。从而这是同态。

    再证明完备性。作曲线$\varphi_x^{\R}(t):=\varphi_X(\epsilon/2)\varphi_X(t-\epsilon/2)$。则根据同态性,$\varphi_X^{\R}$与$\varphi_X(t)$在$(-\epsilon/2,\epsilon)$是重合的。由此我们把区间延拓到了$(-\epsilon,3/2\epsilon)$.类推可以延拓$(-\epsilon,+\infty)$。同理可以延拓到$(-\infty,\epsilon)$。因此这是完备的曲线。
\end{proof}
\begin{theorem}
    给定李代数的同态$\phi:g \to h$,若$G$是单连通的,则存在唯一的李群同态$\Phi:G \to H$满足$\Phi_{*e}=\phi$:
    \begin{tikzcd}
	{\Phi:G} && H \\
	\\
	{\phi:g} && h
	\arrow[from=1-1, to=1-3]
	\arrow[from=3-1, to=3-3]
	\arrow[from=3-1, to=1-1]
	\arrow[from=3-3, to=1-3]
     \end{tikzcd}
\end{theorem}
\begin{proof}[命题\ref{pro:eee}李代数同态的提升]
    对任意的$X$是$g$里的元素,$\phi_X:\R \to g$,$t \mapsto tX$是李代数的同态。这里$\R$的李代数结构为$[x,y]=0$.

    由于$\R$单连通,$\exists$唯一的李群同态$\varphi_x:\R \to G$使得$(\varphi_X)_{*e}=\phi_X$。即为所求的单参数变换群。
\end{proof}
\begin{definition}[李群的指数映射]
    设$G$是李群,李代数为$g$。考虑映射$\mathrm{exp}:g \to G$,$X \mapsto \varphi_X(1)$称为$G$的指数映射。
\end{definition}
\begin{remark}
    指数映射一般非满射。\textbf{$G$是紧李群,则$\exp$是满射。这一点暂时不证明。}
\end{remark}
\begin{example}
    考虑$\R$是李群,则$g=\R$。我们计算指数映射。对于给定的$a \in \R$,其单参数变换群为$\varphi_a(t)=ta$。从而指数映射$\exp(a)=a$。
\end{example}
\begin{example}
    设$G=S^1$。$g=\R$。给定$a \in \R$,则$\varphi_a(t)=e^{2\pi i at}$,则$\exp(a)=e^{2\pi i a}$
\end{example}
\begin{example}
    $G=\mathrm{GL}(n,\C)$。任取$A$是可逆矩阵,$\varphi_A(t)=e^{tA}$于是$\exp(A)=e^A$.
\end{example}
我们自然的给出指数映射的性质。
\section{指数映射的性质}
\begin{proposition}
    存在$(g,+)$的单位元邻域$U(0)$以及$(G,\cdot)$的单位元邻域$V(e)$使得$\exp:U(0) \to V(e)$是微分同胚。且满足$\exp_{*0}=\mathrm{id}$,$T_0g\cong g$.从而$T_eG =g$。
\end{proposition}
\begin{proof}
    首先证明$\exp$是光滑映射。考虑$G \times g$上由向量场$(X,0)$诱导的流。
    $$
    \Phi:\R \times G \times g \to G \times g, (t,g,X)\mapsto (g \exp{tX},X)
    $$
    这是光滑映射。设$G \times g \to G$是自然投影,从而也使光滑的。因此$\exp=P\circ \Phi(0,e,X)$也是光滑的。

    由定义$\exp_{*0}(X)=\dfrac{d}{dt}|_{t=0} \exp{tX}=X(e)$,得$\exp_{*e}=\mathrm{id}$。

    从而根据反函数定理可知,存在两个邻域使其为微分同胚。
\end{proof}
\begin{remark}
    该性质可以定义$e$处得一个局部坐标系:
    $$
    \phi:V(e) \to \R^n \quad \exp(t_1x_1+\dots+t_nx_n)\mapsto (t_1,\dots,t_n)
    $$
    其中$X_i$是$g=T_e G$的一组基。
\end{remark}
\begin{proposition}\label{pro:extension}
    设$n \geq 1$,$X_1,\dots,X_n$是$g$里面的元素。当$\|t\|$充分小的时候,有:
    \begin{align}
        \exp(tX_1)\exp(tX_2)\dots \exp(tX_n)=\exp(t\sum_{1\leq i \leq n}X_i+\dfrac{t^2}{2}\sum_{1\leq i \leq j\leq n}[x_i,x_j]+o(t^3))
    \end{align}
\end{proposition}
先给出一个引理:
\begin{lemma}
    设$f$是$G$上的光滑函数,当$\|t\|$充分小的时候,有:
    \begin{align}
        f( \exp(tX_1)\exp(tX_2)\dots \exp(tX_n))=f(e)+t\sum_{i}X_if(e)+\frac{t^2}{2}(\sum_i X_i^2f(e)+2\sum{i<j}X_iX_jf(e))+o(t^3)
    \end{align}
\end{lemma}
\begin{proof}
    对于$\forall f \in C^\infty(G)$,$X \in g$有:
    \begin{align}
        (Xf)(a)=X(a)f=(L_a)_*X(e)(f)=X(e)((L_a)^*f)=\dfrac{d}{dt}|_{t=0}f(a\exp{tX})
    \end{align}
    于是对于任意的$t \in \R$,有:
    \begin{align}
        (Xf)(a\exp tX)=\frac{d}{ds}f(\exp(t+s)X)=\frac{d}{ds}|_{s=t}f(a\exp sX)
    \end{align}
    对于$X_1,\dots,X_k \in \g$,有:
    \begin{align}
        (X_1X_2f)(a)=\frac{d}{dt_1}|_{t_1=0}(X_2f)(a\exp t_1X_1)=\frac{d}{dt_1}\frac{d}{dt_2}f(a\exp(t_1X_1)\exp(t_2X_2))
    \end{align}
    以此可以类推,从而可知$(X_1X_2\dots X_k f)a$的情况。取$a=e$可得上述引理。
\end{proof}
\begin{proof}[命题\ref{pro:extension}]
    由于足够小的邻域内$\exp$是同胚,因此构造其逆映射$\log$。这里我们要求$\|t\|$足够的小。由于$\exp(0)=e$,则$\log(e)=0$。且对于任意的$X \in \g$,有:
    \begin{align}
        Xf(e)=\frac{d}{dt}|_{t=0}f(\exp tX)=\frac{d}{dt}|_{t=0} tX=X
    \end{align}
    对于任意$n>1$,$X^n f(e)=\dfrac{d}{dt^n}|_{t=0}(tX)=0$.

    注意到$\sum X_i^2 +\sum 2X_iX_j= (X_1+\dots+X_n)^2+\sum_{i<j}[X_i,X_j]$。
    
    对$\exp(tX_1)\dots \exp(tX_n)$作用$\log$。只要$t$足够小,那么就有右边式子结论。
\end{proof}
\begin{proposition}
    设$G$是李群,李代数$\g$。$H$是$G$的闭子群,则$\mathfrak{h}:=\{X \in \g|\exp tX\in H,\forall t \in \R\}$是$\g$的子代数。
\end{proposition}
\begin{proof}
    首先我们说明$\mathfrak{h}$是子空间。由定义$\forall X \in \mathfrak{h},s \in \R$有$sX \in \mathfrak{h}$。

    由上述命题可知:
    \begin{align}
        \exp(t/n X)\exp(t/n Y)=\exp(t/n(X+Y)+t^2/2n^2[X,Y]+o(1/n^3))
    \end{align}
    上式$n$次方,即可得到:
    \begin{align}
        (\exp(t/n X)\exp(t/n Y))^n=\exp(t(X+Y)+t^2/2n[X,Y]+o(1/n^2))
    \end{align}
    对$n$取极限$n \to \infty$,则右式自然有为$\exp(t(X+Y))$为左式的极限。而$H$是闭子群,从而极限也属于$H$。这就说明$X+Y \in \mathfrak{h}$.

    再证明$[X,Y]\in \mathfrak{h}$。根据上式的估计:
    \begin{align}
        (\exp(-t/n X)\exp(-t/n Y)\exp(t/n X)\exp(t/n Y))^{n^2}=\exp(t^2[X,Y]+o(1/n))
    \end{align}
    同样给极限,从而$[X,Y] \in \mathfrak{h}$.
\end{proof}
我们可以看到,在进行指数映射的计算性质前,我们常常会使用关于乘积的估计。

\begin{proposition}
    设$\|\cdot\|$是$\g$上的范数,$\{X_i\}$是$\g$中的序列满足:
    \begin{enumerate}
        \item $X_i \to 0, i \to \infty$.
        \item $\exp X_i \in H,\forall i$.
        \item $\lim_{i \to \infty}\dfrac{X_i}{\|X_i\|}=X \in g$
    \end{enumerate}
    则$X \in \mathfrak{h}$如上面性质的定义。
\end{proposition}
\begin{proof}
    给定$t\neq 0$,取$n_i:=\max\{n \in \Z,n\leq \dfrac{t}{\|X_i\|}\}$
    \begin{align}
        \exp tX=\exp(\lim_{i \to \infty}\dfrac{tX_i}{\|X_i\|})=\lim_{i \to \infty}\exp(n_iX_i)=\lim_{i \to \infty}(\exp X_i)^{n_i}\in H
    \end{align}
    于是根据上述性质得到$X \in \mathfrak{h}$。
\end{proof}
\begin{proposition}
    $\mathfrak{h},H$的定义如上。$(\mathfrak{h},+)$存在单位元邻域$U(0)$和$(H,\cdot)$的单位元邻域$V(e)$使得:
    \begin{align}
        \exp_{G}|_{U(0)}:U(0)\to V(e)
    \end{align}
    是微分同胚。
\end{proposition}
\begin{proof}
    设$\mathfrak{h}'$是$\g$的子空间,使得$g=\mathfrak{h}\oplus \mathfrak{h}'
    $。令$\Phi:\g \to G$,使得$\Phi(X+Y)=\exp_G X\exp_G Y$。显然$\Phi_{*0}(X+Y)=X+Y$,则$\Phi$是局部的微分同胚。

    注意到$\exp|_h=\Phi|_h$,只需要证明$\Phi$将$\mathfrak{h}$中的单位邻域微分同胚的映射到$H$中的单位邻域。

    假设对于$U(0)\subset \mathfrak{h}$,$\Phi$都无法将其微分同胚的映射到$V(e)\subset H$。即需依赖$h'$中分量$Y_i \neq 0$的元素,才能通过$\Phi$得到$V(e)$。
    \begin{align}
        \exp X_i \exp Y_i \in H \Rightarrow \exp(Y_i)\in H
    \end{align}
    又因为$Y_i \to 0$,所以$Y_i/\|Y_i\|$的极限$Y \in \mathfrak{h}\cap \mathfrak{h'}$。这产生了矛盾,因为$\| Y\|=1$。

\end{proof}

最后我们给出闭子群定理。
\begin{theorem}[Cartan]
    李群$G$的闭子群$H$是$G$的嵌入李子群。
\end{theorem}
\begin{proof}
    对$G$的单位元邻域$U(e)$,存在$(g,+)$的单位元邻域$V(0)$使得:$\log:U(e) \to V(0)$是微分同胚。

    根据上述命题,自然有:$\log(U(e)\cap H)=V(0)\cap \mathfrak{h}$。于是$U(e)$的坐标使得$H$中$e$的邻域是嵌入子流形。

    对于$\forall g \in G$,由于$L_g$是微分同胚,故给定$h \in H$:
    \begin{align}
        U(h)\to U(e) \to V(0)
    \end{align}
    是$h$邻域$U(h):=(L_h)(U(e))$的坐标。

    这意味着每个点$h \in H$,都有邻域$U(h)$使得$U(h)\cap H \to L_h^{-1}(U(h)\cap H) \to \log(L_h^{-1}(U(h)\cap H))=V(0)\cap \mathfrak{h}$。于是这意味着$H$是嵌入子流形。
\end{proof}
\chapter{2023.03.09}
\section{闭子群定理应用}
\begin{proposition}
    $\varphi: G \to H$是李群同态,则$\ker \varphi$是嵌入(正规)李子群。
\end{proposition}

\begin{proof}
$\varphi$是李群同态,则$\ker \varphi=\varphi^{-1}(e)$是闭子群。根据闭子群定理,$\ker \varphi$是嵌入李子群。

法2:常秩定理。引理:李群同态$\varphi: G \to H$是常秩映射。这是因为$\mathrm{rank}\varphi_{*e}=\mathrm{rank}\varphi_{*g}$。

引理证明:$\phi \circ L_g=L_{\varphi(g)}\circ \varphi$得到$\varphi_{*g}\circ(L_g)_{*e}=(L_{\phi(g)})_{*e}\phi_{*e}$。由于$L_g$是微分同胚,所以$\forall g \in G$,有$\mathrm{rank}\phi_{*g}=\mathrm{rank}\phi_{*e}$。

\end{proof}
\begin{theorem}[秩的整体性定理(Global rank theorem)]
    $\phi$是$M$到$N$的光滑映射,则$\phi$是常秩的。则有:

    (1)$\phi$是单射意味着$\phi$是浸入。

    (2)$\phi$是满射意味着$\phi$是淹没。

    (3)$\phi$是双射意味着$\phi$是微分同胚。
\end{theorem}
\begin{theorem}
    两个李群的连续同态是李群同态。
\end{theorem}
\begin{proof}
    设$\varphi:G \to H$是连续同态。则$\Gamma_\varphi:=\{(g,\varphi(g))|g \in G\}$是$G \times H$的闭子群。

    由闭子群定理知$\Gamma_\varphi$是李子群。

    故$P:\Gamma_\varphi \to G \times H \to G, \quad (g,\varphi(g)) \mapsto (g,\varphi(g))\mapsto g$是一个光滑映射,并且作为抽象群的同构,并且是李群的同态。

    现在只用说明$\Gamma_\varphi$到$G$的映射$P$的逆是光滑的。由于$P$是常秩的映射,从而根据Global rank theorem知这是微分同胚。则$\varphi:P_2 \circ P^{-1}$是李群同态。
\end{proof}
\begin{proposition}
    任何拓扑群都有唯一的光滑结构使之成为李群。但群上的拓扑结构不一定是唯一的。
\end{proposition}
我们考虑闭子群定理的逆命题:
\begin{proposition}
    设$G$是李群,$H$是$G$的嵌入李子群,则$H$是闭子群。
\end{proposition}
\begin{proof}
设$H$是嵌入李子群,对于$\forall g \in G$,存在邻域$U(g)$使得$U(g) \cap H=U(g)\cap \overline{H}$。(嵌入李子群的局部性质)。

令$g=e$,则$U(e)\cap H=U(e)\cap \overline{H}$。下证$\overline{H}\subset H$。

对于$ h\in \overline{H}$,$hU(e)\cap H \neq \emptyset$.取$h'\in hU(e)\cap H$,则$h'h \in U(e)$,又$h \in \overline{H}$,存在序列$\{h_n\} \subset H$使得$h_n \to h$。于是$\{h_n^{-1}h'\}$收敛于$h^{-1}h'$。于是$h{-1}h' \in U(e)\cap \overline{H}=U(e)\cap H$。故$h \in H$
\end{proof}
\section{李群同态和李代数同态}
\begin{proposition}
    设$\varphi: H \to G$是李群同态,则$\varphi_{*e} h \cong T_eH \to g \cong T_e G$是李代数同态。
\end{proposition}
\begin{lemma}
    $f:M \to N$的光滑映射。若$M$上的向量场$X_1,X_2$与$N$上的向量场$Y_1,Y_2$是$f$相关的(即$f_{*m}X_i(m)=Y_i(f(m))$),则$[X_1,X_2]$和$[Y_1,Y_2]$是$f$相关的。
\end{lemma}
\begin{proof}[命题3.4]
    只需要证明由$v \in T_eH$所诱导的左不变向量场$X$与$\varphi_{*e}(v) \in T_e G$诱导的左不变向量场$Y$是$\varphi$相关的。
    $$
    \varphi_{*g}(X(g))=\varphi_{*g}(Lg)_{*e}v=(L_{\varphi(g)})_* \circ \varphi_{*e}v=Y(\varphi(g))
    $$
\end{proof}
\begin{proposition}
    设$\phi$是李群同态$H \to G$。则图标可换:
\[\begin{tikzcd}
	H && G \\
	\\
	h && g
	\arrow["\phi", from=1-1, to=1-3]
	\arrow["{\mathrm{exp}}", from=3-1, to=1-1]
	\arrow["{\mathrm{exp}}"', from=3-3, to=1-3]
	\arrow["{\phi_*}"', from=3-1, to=3-3]
\end{tikzcd}\]
\end{proposition}
\begin{proof}
    考虑$\psi(t)=\phi(\mathrm{exp}tX)$。由于$\phi$是李群同态,则$\psi$是单参数变换群$R \to G$使得$\dfrac{d}{dt}|_{t=0} \psi(t)=\phi_{*e}\circ \mathrm{exp}_{*0}(X(e))$。从而:
    $$
    \mathrm{exp}_G(t\phi_{*e}(X))=\mathrm{exp}^{(H)}_{*0}(X(e))
    $$
    根据单参数变换群的唯一性。

    令$t=1$,就得到$\mathrm{exp}_G(\phi_*(X))=\phi(\mathrm{exp}_H(X))$。
\end{proof}
\section{李子群与李子代数}
\begin{proposition}
    $H$是$G$的李子群,则$\mathrm{Lie}(H):=h$是$g$的李子代数。
\end{proposition}
\begin{proof}
    设$i :H \to G$是包含映射。则$i_{*e}:h \to g$的李代数单同态。从而$i_{*(e)}h \cong h \subset g$是李子代数。
\end{proof}
\begin{proposition}
    设$G$是李群且$h$是$\mathrm{Lie}G=g$的子代数。则存在唯一的连通李群$H\subset G$使得:$\mathrm{Lie}H =h$。
\end{proposition}
\begin{proof}
    设$X_1,\dots,X_k$是$h \subset g$的基底,由于$X_i$是左不变的且$X_i(e)$的值确定了$X_i$,并且$\{X_i(e)\}$是线性无关的,则$\{X_i(g)\}$对于任意的$g$都是线性无关的。

    故$D_g=\mathrm{Span}\{X_1(g),\dots,X_k(g)\}$是$G$上的$k$维-分布。

    由于$[X_i,X_j] \in h=\mathrm{Span}\{X_1,\dots,X_k\}$。根据Frobenius定理,存在唯一的$D_g$的极大连通积分子流形$H \subset G$。

    下证$H$具有群结构。由于$X_i$左不变,$(L_h)_*(S_g)=S_g$。

    故$L_h H=H,\forall h \in H$。

    最后证明唯一性。设$K$亦是$V_g$的连通积分子流形。则$K\subset H$。由于$T_e K=T_eH$。由反函数定理,存在$U(e)\subset K,V(e)\subset H$使得$U(e)$和$V(e)$是微分同胚。由$H,K$群乘法相同且$H,K$连通,则$H$与$K$相同。
\end{proof}
\section{李的基本定理}
\begin{definition}
    设$G,H$是李群,$U(e) \subset G$,$V(e) \subset H$是邻域。

    (1)$f:U(e)\subset G \to V(e) \subset H$若满足$\forall g_1,g_2 \in U(e)$,使得$g_1g_2\in U(e)$且$f(g_1g_2)=f(g_1)f(g_2)$。则称$f$是$G,H$是局部同态。

    (2)若$f$还是(局部)微分同胚,称$f$是局部同构。 
\end{definition}
\begin{theorem}[李的第一基本定理]
    设$G$和$H$是局部同构的李群,则$\mathrm{Lie}G:=g$,$h$是同构的李代数。$(f: G \to H)$。
\end{theorem}
\begin{proof}
    局部同构能得到$f_{*e}$是双射且为李代数的同态。
\end{proof}
\begin{theorem}[李的第二定理]
    $g,h$是$G,H$的李代数。$\varphi:g \to h$同构推到出$G,H$的局部同构。
\end{theorem}
\begin{proof}
    令$a=\mathrm{Graph}(\rho)=\{(x,\rho(x))|x\in g\}$。
    $$
    [(x_1,\rho(x_1)),(x_2,\rho(x_2))]=([x_1,x_2],[\rho(x_1),\rho(x_2)])=([x_1,x_2],\rho[x_1,x_2])
    $$
    从而$a$是$g \oplus h$上的子代数。存在唯一连通的李子群$A \subset G\times H$使得$\mathrm{Lie}(A)=a$。

    设$i: A \to G \times H$是包含映射,则$\varphi:A \to G\times H \to G$是李群同态且$\varphi_{*e}$是$\mathrm{id}$。

    根据反函数定理及$\varphi$是同态,则$\varphi:A \to G$是局部同构。

    同理$\psi: A \to G\times H \to H$是李群同态,由于$\psi_{*e}:(x,\rho(x))\mapsto \rho(x)$是李代数同构,从而$\psi:A \to H$是局部同构。

    令$w(e)=\varphi^{-1}(l)\cap \psi^{-1}(v)$.则$\psi \circ \varphi^{-1}:\varphi(w(e)) \to \psi(w(e))$是局部同构。
    \end{proof}
    \begin{theorem}[李的第三定理]
        设$g$是有限维李代数,则存在唯一的单连通李群$\tilde{G}$使得$\mathrm{Lie}(\tilde{G})=g$。

        从而李代数和单连通有着一一对应的关系。
    \end{theorem}
    \begin{proof}
        根据Ado引理,$g$是$\mathrm{gl}(n,\C)$的子代数。则存在唯一连通的李子群$G \subset \mathrm{gl}(n,\C)$使得$\mathrm{Lie}(G)=g$。


    \end{proof}
    \chapter{2023.03.16}
    \section{覆盖群及其应用}
    \begin{definition}
        $G$是连通李群,$G$的覆叠空间$\tilde{G}$且$\tilde{G} \to G$是李群同态,则称$\tilde{G}$是$G$的一个覆盖群。
    \end{definition}
    \begin{proposition}\label{pro:cover}
        连通李群$G$的覆叠空间$\tilde{G}$自然蕴含李群结构且使得$\tilde{\pi}:\tilde{G}\to G$是李群同态。
    \end{proposition}
    \begin{lemma}
        设$\pi:X \to M$是连通流形上的覆盖。$Z$是连通流形且满足对于任何光滑映射$\alpha,\pi$,有$\alpha_* (\pi(Z))\subset \pi_*(\pi_1(X))$且$\alpha(z_0)=m_0$。对于$\forall x_0 \in$ 
    \end{lemma}
    \begin{proof}[命题\ref{pro:cover}]
        我们说明$\tilde{G}$有群结构。考虑图表:\begin{tikzcd}
	&& {\tilde{G}} \\
	\\
	{\tilde{G}\times\tilde{G}} && G
	\arrow["\pi", from=1-3, to=3-3]
	\arrow["{\tilde{\alpha}}", from=3-1, to=1-3]
	\arrow["\alpha"', from=3-1, to=3-3]
\end{tikzcd}

    其中$\alpha(\tilde{g_1},\tilde{g_2})=\pi(\tilde{g_1})\pi(\tilde{g_2})^{-1}$。由$\alpha$定义得到:
    \begin{align}
        \alpha_*(\pi_1(\tilde{G}\times \tilde{G}))\subset \pi_*(\pi_1(\tilde{G}))
    \end{align}
    任取$\tilde{e}\in \pi^{-1}(e)$,则存在唯一的$\tilde{\alpha}:\tilde{G} \times \tilde{G} \to \tilde{G}$使得其为提升且$\tilde{\alpha}(\tilde{e},\tilde{e})=\tilde{e}$。

    我们定义$\tilde{G}$中元素的逆元。对于任意的$\til{g},\til{g_1},\til{g_2}$,定义$\til{g}$的逆元为$\til{\alpha}(\til{e},\til{g})$,$\til{g_1}\til{g_2}=\til{\alpha}(\til{g_1},\til{g_2}^{-1})$.
\end{proof}
\begin{example}
    $\mathrm{Sp}(1) \times \mathrm{Sp}(1)$是$\mathrm{SO}(4,\R)$的覆盖群。

    对于$a,b \in \mathbb{H}$,考虑$T_{ab}:\mathbb{H} \to \mathbb{H} \cong \R^4,v \mapsto avb$.可以验证$T_{a,b}\in \mathrm{SO}(4,\R)$。
\end{example}
\begin{example}

\end{example}
\begin{theorem}
    $G,H$连通子群,$\Phi: G \to H$是李群同态,则$\Phi$是李群覆盖等价于$\Phi_{*e}:g \to h$是李代数同构。
\end{theorem}
\begin{theorem}
    $G,H$是李群,$G$是单连通的。若$\varphi:g  \to h$是李代数同态,则存在唯一的李群同态$\Phi:G \to H$满足$\Phi_{*e}=\varphi$。
\end{theorem}
\begin{proof}[证法2:BCH公式]
    \textbf{BCH公式是指:}
    
    设$G$是李群,设$X,Y\in g$,$\|X\|$和$\|Y\|$足够小,则定义:
    \begin{align}
        X *Y:=\mathrm{log}(\mathrm{exp}X \mathrm{exp} Y)=X+Y=\sum_{m\geq 2}P_m(X,Y)
    \end{align}
    其中$P_m(X,Y)$是由$m-1$层$X,Y$李括号的线性求和:
    \begin{align}
        P_2(X,Y)=1/2[X,Y] \quad P_3=1/12([X,[X,Y]]-[Y,[X,Y]])\quad P_4=1/24 [X,[Y,[Y,X]]]
    \end{align}
     
    接下里阐述证明:

    首先构造局部的同态:令$\Phi:=\mathrm{exp}\varphi \mathrm{log}:U \to H$,$U$是满足BCH公式的足够小邻域。

    对于$A,B \in U$,令$X=\log A,Y=\log B$,则:
    \begin{align}
        \Phi(AB)=\Phi(\mathrm{exp})
    \end{align}
\end{proof}
\begin{definition}[李群中心]
    $Z(G)$定义为$G$的交换李子群。$Z(g)=\{Z \in g|[Z,X]=0,\forall X \in g\}$是李代数的中心。
\end{definition}
\begin{theorem}
    设$G,\til{G}$是连通李群:
    \begin{enumerate}
        \item 若$\Phi:\til{G}\to G$是李群覆叠,则$\ker \Phi$是$Z(G)$的离散子群。
        \item 若$\Gamma$是$Z(G)$的离散子群,则$G/\Gamma$是李群且$\Phi$是李群覆盖。
    \end{enumerate}
\end{theorem}
\begin{proof}
    令$\Gamma$是$\ker \Phi$。由于$\Phi$是局部微分同胚,则存在$U(e)$是$\til{G}$的开子集,满足$U(e)\cap \Gamma=\{e\}$。对于$\forall r\in \Gamma$,$rU\cap \Gamma=\{r\}$。则$\Gamma$是离散子群。

    对$\forall g \in \til{G}$,$\gamma \in \Gamma$
\end{proof}
\begin{corollary}
    $G$是连通李群,$\mathrm{Lie}G=g$,则$G \cong \til{G}/\Gamma$,$\Gamma$是$Z(G)$中的离散子群。


\end{corollary}
\begin{example}
    设$G$是$\mathrm{Sl}(n,\R)$的覆盖群。则其不是矩阵李群。换言之,不存在单的李群同态:$\varphi:G \to \mathrm{GL}(n,\R)$。 

    为了说明这一事实,我们采取反证法。即假设存在$\varphi$。考虑图表:
    
\end{example}
\chapter{2023.03.23}
\section{李群的基本群求法}
以$\mathrm{SL}(n,\R)$为例。考虑$n \geq 3$的情况。

第一步使用极分解。即$\mathrm{SL}(n,\R)=\cong \mathrm{SO}(n,\R)\times \R^m$。从而:
\begin{align}
    \pi_1(\mathrm{SL}(n,\R))=\pi_1(\mathrm{SO}(n,\R))
\end{align}
在已知同伦群的作用下构造可迁群作用。(轨道唯一,纤维从):
\begin{align}
    \mathrm{SO}(n+1,\R)\times S^n \to S^n: A \times (e_1,\dots,e_{n+1})^T=A(e_1,\dots,e_{n+1})^T
\end{align}
考虑稳定化子:$\mathrm{Stab}(e_1,0,0,\dots,0)=\begin{pmatrix}
    1& \quad \\ \quad &\mathrm{SO}(n,\R)
\end{pmatrix}$
从而$S^n\cong \mathrm{SO}(n+1,\R)/\mathrm{SO}(n,\R)$得到正合列:$\mathrm{SO}(n) \to \mathrm{SO}(n+1) \to S^n$。从而诱导长正合列:
\begin{align}
    \to \pi_{i+1}(C) \to \pi_i(A)\to \pi_1(B) \to \pi_1(C) \to \pi_{i-1}(C) \to
\end{align}
已知$\pi_i(S^n)=0,i=1,2,\dots,i-1$.上式带入$n=3$,有:
\begin{align}
    0=\pi_2(S^3) \to \pi_1(\mathrm{SO}(3)) \to \pi_1(\mathrm{SO}(4)) \to \pi_1(S^3)=0
\end{align}
从而$\pi_1(\mathrm{SO}(3))\cong \pi_1(\mathrm{SO}(4))$。同理$\pi_1(\mathrm{SO}(3)) \cong \pi_1(\mathrm{SO}(n))$。

\section{李代数的复化和实形式}
复李代数$\C$向量空间,有李括号$[,]$。

\begin{proposition}
    一个复李代数$(\mathfrak{g},[,])$可以看为实李代数$(\mathfrak{g},[,],I)$。$I$是$R$线性变换.且$I^2=-\mathrm{id}$。且$[Iu,v]=[u,Iv]=I[u,v]$。
\end{proposition}
\begin{proof}
    先给定$(\mathfrak{g},[,])$作为复李代数。定义$I:\mathfrak{g}^{\R}\to \mathfrak{g}^{\R}$为$X \to iX$。

    给定$(\mathfrak{g}^{\R},[,])$,定义数乘:
    $$
    \C \times \mathfrak{g}^{\R} \to \mathfrak{g}^{\R},(a+bi)(u):=au+bIu
    $$
\end{proof}
李代数的复化。设$\mathfrak{g}$是实李代数,$\mathfrak{g}_{\C}=\mathfrak{g}\oplus i \mathfrak{g}$是向量空间的复化。定义:
\begin{align}
    [u+iv,x+iy]=[u,x]-[v,y]+i[u,y]+i[v,x]
\end{align}
则称$(\mathfrak{g}_{\C},[,])$称为$(\mathfrak{g},[,])$的复化。

\begin{proposition}
    $\mathfrak{g}_{\C}\cong ({\mathfrak{g}_{\C}}{\R},I)$。其中$I(u+vi)=-v+ui \in {\mathfrak{g}_{\C}}^{\R}$.
\end{proposition}
\begin{definition}[实形式]
    设$h$是实李代数$\mathfrak{h}_{\C}\cong \mathfrak{g}$,则是实形式。
\end{definition}
\begin{proposition}
    实形式等价于$\tau$是$\mathfrak{g} \to \mathfrak{g}$的共轭线性对合自同构。
\end{proposition}
\begin{remark}
    \begin{enumerate}
        \item 并非所有复李代数都有实形式。若实形式存在,则$(\g,I)\cong (\g,-I)$。
        \item 实形式不一定唯一。比如$\mathrm{SL}(n,\C)$:
        \begin{align}
            \tau(x)=\overline{x} \Rightarrow \g^{\tau}=\mathrm{SL}(n,\R)
        \end{align}
        分裂实形式。
        \begin{align}
            \tau(x)=-x^* \Rightarrow \g^\tau=\mathrm{SU}(n)
        \end{align}
        紧实形式。
    \end{enumerate}
\end{remark}
\section{复流形}
复流形是具有复结构的实流形。即$M$上有开覆盖$\{U_\alpha\}$,其中$\{U_\alpha\}$与$\C$中的开子集微分同胚。使得$U_\alpha$具有复坐标$(z_1,\dots,z_n)$,满足任意坐标变换的转移函数全体光滑。

然而复结构的流形很难做实际的验证。我们考虑$(M,J)$,$J$是一个$(1,1)$型张量,$J :TM \to TM$满足$J^2=-\mathrm{id}$。$J$称为近复结构。

当$M$是偶数维,$J_x:T_x M \to T_x M$,$J_x^2=-\mathrm{id}$意味着$\mathrm{det}(J_x)^2=(-1)^n>0$

近复结构是复结构等价$\forall X,Y \in \Gamma(TM)$,
\begin{align}
    [JX,JY]-J[JX,Y]-J[X,JY]-[X,Y]=0
\end{align}
\begin{definition}[复李群]
    复流形且是个群。群乘法和逆都是全纯的。
\end{definition}
\begin{theorem}
    一个连通李群$G$是复李群等价于$G$的李代数是复的李代数。
\end{theorem}
\begin{proof}
    $(G,J)$
\end{proof}

\section{泛包络代数}
对于域$\F$的李代数$\mathfrak{g}$,存在唯一的结合代数$U(\mathfrak{g})$以及双线性映射$i:\mathfrak{g}\to U(\mathfrak{g})$使得:
\begin{enumerate}
    \item $i[X,Y]=i(X)i(Y)-i(Y)i(X)$
    \item $U(g)$由$i(g)$生成。
    \item 
\end{enumerate}
\chapter{2023.3.30}
\section{$U(g)$}

定义$T(g)$为:
\begin{align}
    T(g)=\bigoplus_{k=0}^\infty g^{\oplus k}
\end{align}
其中加法是形式的加法,乘法我们只考虑基:单纯做张量积即可。于是上述结构其实是一个分次环。

定义$U(g)=T(g)/(e_i \otimes e_j-e_j \otimes e_i-[e_i,e_j])$。

\begin{theorem}[PBW]
    设$\{a_1,\dots,a_m\}$是$\g$的基底,则
\end{theorem}
\begin{remark}
    \begin{enumerate}
        \item PBW基形式上与$k[x_1,\dots,x_n]$保持一致。但是不交换。
        \item $U(g)$上有滤子$F_0 \subset F_1 \subset F_2 \dots$,其中:
        $$
        F_k=\mathrm{Span}_{\F}(a_1^{k_1}\dots a_n^{k_n},k_1+\dots+k_n \leq k)
        $$
        得到交换的$k$代数:
        $$
        F_0 \oplus F_1/F_0 \oplus \dots \cong S(g)\cong k[x_1,\dots,x_n] 
        $$
        其中$S(g)$是对称的张量集。
        \item 考虑单射$i:\g \to U(\g)$,则$i(a_1),\dots,i(a_n)$是线性无关的。
    \end{enumerate}
\end{remark}
\begin{definition}[$U(\g)$的上乘法]
    定义$\Delta:U(\g)\to U(\g)\otimes U(\g)$。我们给出基底的定义即可:
    设$e_i$是$\g$的基底,由:
    $$
    \Delta(e_i)=e_i \otimes 1+1 \otimes e_i
    $$
    并且直接同态的定义乘法:
    $$
    \Delta(e_i\otimes e_j)=\Delta(e_i)\Delta(e_j)
    $$

\end{definition}
\begin{definition}
    对于$r \in U(\g)$,若$\Delta(r)=r \otimes 1+1 \otimes r$,则称其为本原元(primitive)。若$\Delta(r)=r \otimes r$,则称其为类群元(grouplike)。 
\end{definition}
\begin{proposition}
    \begin{enumerate}
        \item 若$r,s$本原,则$[r,s]$是本原的。即本原元构成李代数。
        \item 若$r,s$是类群元,则$rs$是类群元。从而所有的类群元构成一个$\g$的形式李群。
        \item 若$r$是本原的,则$\exp r$是$\hat{U(\g)}$(定义为PBW基生成的形式幂级数)的类群元。
        \item 若$r$是类群元,且常数项是$1$,则$\log r$是$\hat{U(\g)}$的本原元。
    \end{enumerate}
\end{proposition}
\begin{theorem}
    设$\g$是特征为$0$域上的李代数,则$\g$是$U(\g)$中所有的本原元。
\end{theorem}
\begin{proof}
    先设$\g$是交换的基底。此时$U(\g)=k[x_1,\dots,x_n]$。考虑:
    $$
    \Delta(f)=1\otimes f+f \otimes 1 \Leftrightarrow f(x)+f(y)=f(x+y),\forall x,y \in k^n
    $$
    对于$f^{(k)}$中的$k$次齐次多项式:
    $$
    2^k f^{(k)}(x)=f^{(k)}(2x)=2f^{(k)}(x) \Rightarrow (2^k-2)f^{(k)}=0
    $$
    于是$\mathrm{deg}(f)=1$。

    接着考虑非交换。$U(\g):F_0 \subset F_1 \subset F_2$。接着考虑:
    $$
    \mathrm{Span}_k\{a_1^{k_1}\dots a_n^{k_n}:k_1\dots k_n \leq k\}
    $$
\end{proof}
\begin{example}
    反例:$\g$是一维的,$\mathrm{char}k=p>0$.基底$X \in \g$,则$X^p$也是本原元。
\end{example}
\begin{theorem}[BCH公式]
    $\exp X \exp Y=\exp(X+Y+1/2[X,Y]+[X,[X,Y]],\dots)$
\end{theorem}
\begin{proof}
    不妨假设$X,Y$线性无关。
\end{proof}
\begin{remark}
    BCH对特征为$p$的李代数不成立。
\end{remark}
\section{代数群和李群}
\begin{definition}
    代数群:$G$.$k$上的仿射态射簇(多项式零点集)且是个群。满足群乘法是态射(多项式函数)
\end{definition}
\begin{proposition}
    当$k=\R$,任何$k$上的代数群都是李群且是$\mathrm{Gl}(n,\C)$的闭子群。即矩阵李群。则非矩阵李群不是代数群。
\end{proposition}
代数群$G$的李代数是$k[G]$上的满足$\delta \circ L_g=L_g \circ \delta$的导子$\delta$。李括号受到$\mathrm{Char}k$的影响。若$k$的特征是$0$,则$[\delta_1,\delta_2]=\delta_1\circ \delta_2-\delta_2 \circ \delta_1$。
\section{李群李代数的表示}
$V$是复的向量空间。$\mathrm{GL}(V)$是$V$上的线性同构构成的集合,自然根据维数有:$\mathrm{GL}(V)=\cong \mathrm{GL}(n,\C)$。$\mathrm{gl}(V)$是$V$上线性变换。$\mathrm{gl}(V)\cong \mathrm{gl}(n,\C)$。

\begin{definition}
    设$G$是李群,光滑的李群同态:$\rho:G \to \mathrm{GL}(V)$称为$G$的复表示。

    设$\g$是李代数,则李代数同态:$\g \to \mathrm{gl}(V)$称为$\g$的表示。
\end{definition}
\begin{remark}
    \begin{enumerate}
        \item 李群表示等价于线性群作用
        \item 李代数表示等价于$\g$模。
        \item 对于任何的李代数表示$\pi:\g \to \mathrm{gl}(V)$都存在唯一的$U(\g)$上的同态是的交换图成立(泛性质):\begin{tikzcd}
            {U(\mathfrak{g})} \\
            \\
            {\mathfrak{g}} && {\mathrm{gl}(V)}
            \arrow[from=3-1, to=1-1]
            \arrow["\pi"', from=3-1, to=3-3]
            \arrow["{\tilde{\pi}}", from=1-1, to=3-3]
        \end{tikzcd}
    \end{enumerate}
\end{remark}
\begin{definition}[伴随表示]
    李代数:$\mathfrak{g} \to \mathrm{gl}(\g):X \to \mathrm{ad}_X \in \mathrm{gl}(\g)$称为$\g$的伴随表示。$\mathrm{ad}_X(Y)=[X,Y]$。
\end{definition}
Jacobbi恒等式:

\chapter{2023.04.06}
\subsection{李代数的表示和李代数模}
\begin{proposition}
    李代数$\g$的自同构群是$\mathrm{GL}(\g)$的嵌入李子群,李代数为$\mathrm{Der}(\g)$。其中$\mathrm{Der}(g)$是导子李代数。
\end{proposition}
\begin{proof}
    这是因为$\mathrm{Aut}(\g)$是由方程$A[x,y]=[Ax,Ay]$定义的。因此$\mathrm{Aut}(g)$是$\mathrm{GL}(\g)$的闭子群,从而是嵌入李子群。

    考虑$\mathrm{Aut}(\g)$的李代数:
    \begin{align}
        \mathrm{Lie}(\mathrm{Aut}(\g))=\{D \in \mathrm{gl}(\g)|e^{tD}\in \mathrm{Aut}(\g),\forall t\}
     \end{align}
    即$e^{tD}[X,Y]=[e^{tD}X,e^{tD}Y]$
    
    下面证明导子满足上述要求。

    考虑$\g$值函数$y_1(t)=e^{tD}[X,Y],y_2=[e^{tD}X,e^{tD}Y]$.当$t=0$,$y_1=y_2$。为了证明$y_1=y_2$,验证发现$y_1'=y_2'$。根据ODE解的存在唯一性,$y_1=y_2$。
    \end{proof}
    
\begin{proposition}
    设$G$是连通李群,则:(a)$\ker \mathrm{Ad}=Z(G)$ (b)$\mathrm{Int}(\g)= \mathrm{Im}(\mathrm{Ad}) \cong G/Z(G)$。
\end{proposition}

\begin{definition}
    \begin{enumerate}
        \item 如果表示是单射,则称为忠实表示。
        \item $V$的子空间$W$满足$g \cdot W=\{\pi(g)w, \forall w \in W\} \subset W$,$\forall g \in G$。
        \item 若表示$(G,\pi,V)$的不变子空间只有$0$,$V$,则称表示$\pi$是不可约的。
    \end{enumerate}
\end{definition}
\begin{example}[非矩阵李群]
    设$G=\R \times \R \times S^1$。定义乘法为$(x_1,y_1,u_1)(x_2,y_2,u_2)=(x_1+x_2,y_1+y_2,e^{ix_1y_2}u_1u_2)$.

    对于$G$的任意表示(有限维),$\pi_G:G \to \mathrm{GL}(V)$,$\pi_G$都不是忠实的。

    考虑海森堡群$$
    H=\{\begin{pmatrix}
        1&a&b\\0&1&c\\0&0&1
    \end{pmatrix}|a,b,c\in \R\}$$
    以及李群同态:$\Phi:H \to G$.$G \to \mathrm{GL}(V)$。

    其中
    $$
    \begin{pmatrix}
        1&a&b\\0&1&c\\0&0&1
    \end{pmatrix} \mapsto (a,c,e^{ib}) \mapsto \pi_G(a,c,e^{ib})
    $$

    $\varphi$的核是$\ker \Phi=\{\begin{pmatrix}
        1&0&2n \pi\\0&1&0\\0&0&1
    \end{pmatrix}|n \in  \Z\}$。是$Z(H)$的离散正规子群。

    于是$H$是$G$的覆盖群。李代数相同。为
    $$
    \mathfrak{h}=\{\begin{pmatrix}
        0&a&b\\0&0&c\\0&0&0
    \end{pmatrix}|a,b,c \in \R\}
    $$
    基底为$A=\begin{pmatrix}
        0&1&0\\0&0&0\\0&0&0
    \end{pmatrix},B=\begin{pmatrix}
        0&0&1\\0&0&0\\0&0&0
    \end{pmatrix},C=\begin{pmatrix}
        0&0&0\\0&0&1\\0&0&0
    \end{pmatrix}$.且$[A,C]=B,[A,B]=[C,B]=0$。
\end{example}
\begin{proposition}
    设$\pi$是$H$的表示。若$\ker \Phi \subset \ker \pi$,则$Z(H)\subset \ker \pi$.
\end{proposition}
若上述性质成立,假设存在忠实表示$\pi_G$,则$\ker \pi_H=\ker \Phi \subset Z(H) \subset \pi_H$矛盾!从而前面我们的$G$没有忠实表示。

\begin{proof}
    考虑两个引理:$\pi(B)$是幂零矩阵。$X$是非零幂零矩阵,则$e^{tX}=I$等价于$t=0$。

    我们说明根据两个引理可以得到上述命题。

    由于$e^{tB}=\begin{pmatrix}
        1&0&t\\0&1&0\\0&0&1
    \end{pmatrix}$,$e^{kn\pi(B)}=\pi(e^{knB})=I(\ker \Phi \subset \ker \pi)$。对所有的$n \in \Z$成立,则由引理1,2知$\pi(B)=0$。因此对于$t \in \R$,$\pi(e^{tB})=e^{t\pi(B)}=I(Z(H)\subset \ker \pi)$。

    对于两个引理的证明,我们放在下面。
\end{proof}
\begin{lemma}
    $\pi(B)$是幂零矩阵。
\end{lemma}
\begin{proof}
    
\end{proof}
\begin{lemma}
    $X$是非零幂零矩阵,则$e^{tX}=I$等价于$t=0$。
\end{lemma}
\begin{proof}
    由于$X$幂零,$e^{tX}$是关于$t$的多项式,因此存在$P_{jk}(t)$使得$(e^{tX})
    _{jk}=P_{jk}(t)$.

    假设$\exists t \neq 0$,使得$e^{tX}=I$。则$e^{ntX}=(e^{tX})^n=I$。从而$P_{jk}(nt)=\delta_{jk}$。于是$e^{tX}\equiv I$.这说明$e^{tX}$不显含$t$。对$e^{tX}$求导,则$Xe^{tX}=X=0$。因此$X=0$与题设矛盾!
\end{proof}
\chapter{2023.04.13}
\subsection{不变内积的存在性}
\begin{theorem}
    交换李群的表示都是1维的。
\end{theorem}
\begin{proof}
    设表示为$(G,\pi,V)$。对于$\forall g \in G$,$\pi(g): V \to V$是$G$可换的。则$\pi(g)=\lambda \mathrm{id}$。从而$V$的任何子空间一定是不变子空间,因此$V$是1维的。
\end{proof}
Haar测度:紧李群存在左右不变的积分(等价于测度)

即$\forall f \in C^{\infty}(G)$:
$$
\int_G f(g)\omega =\int_G f(hg)\omega= \int_G f(gh)\omega =\int_G f(g^{-1})\omega
$$
其中$\omega$是体积形式,即$\int 1\omega=1$。

第一步,在一般的李群上定义左不变形式。在$e \in G$,取定$\omega_e \in \wedge^n T_e^* G$,$n =\mathrm{dim}G$。在$G$上定义体积形式:
\begin{align*}
    \omega \in \Omega^n(G) \Leftrightarrow (L_h^*\omega)(g)=\omega(hg), \forall g,h \in G
\end{align*}
从而
\begin{align*}
    \int_G f(g)\omega(g)\text{是左不变的,即}\int_G f(hg)\omega(hg)=\int_G f(g)\omega(g)
\end{align*}
我们对$\omega$做正规化,即定义:
\begin{align*}
    \int_G 1\omega=1
\end{align*}
我们称正规的左不变测度为左Haar测度。

第二步,我们说明紧李群上模函数恒为$1$,这等价于左不变测度是右不变的。

由于对于$\forall g \in G$,$R_g^* \omega$仍然是左不变的,左不变$n$形式是$1$维向量空间。故存在$\Delta:G \to R_{>0}$(称为模函数)使得$\omega=\Delta(g)R_g^*(\omega)$。

下证:紧李群左Haar测度是右不变的等价于$\Delta \equiv 1$。

思路:证明$\Delta$是李群(反)同态。

考虑$\omega(hg_1g_2)=\Delta(g_1g_2)(R_{g_1g_2}^*)h=\Delta(g_1g_2)R_{g_1}^*(R_{g_2}^*\omega)h$。

又$R_{g_2}^*\omega$是左不变的,则$(R_{g_2}^*)(hg_1)=\Delta(g_1)(R_{g_1}^*)(R_{g_2}^*\omega)h$。

于是$\omega(hg_1g_2)=\Delta(g_2)R_{g_2}^*(hg_1)=\Delta(g_2)\Delta(g_1)R_{g_1}^*(R_{g_2}^*\omega)h$.

因此可以看出来$\Delta$是反同态。但是$R$是交换的,从而这也是同态。

由于$(R_{+},\times)$紧子群只有$\{1\}$,因此$\Delta(g)\equiv 1$.因此紧李群是右不变的。

\begin{theorem}
    紧李群表示$G \to \mathrm{GL}(V)$表示空间上有不变内积。
\end{theorem}
\begin{proof}
    取$V$的一个内积$\langle,\rangle$。在$V$上定义新的内积:\begin{align*}
        \langle v,u \rangle:=\int_G \langle g\cdot v,g\cdot u\rangle dg, g\text{是Haar不变测度}
    \end{align*}
    根据积分的左右不变性:
    $$
    \langle hv,hu\rangle_G =\langle u,v\rangle
    $$
\end{proof}
\begin{corollary}
    紧李群李代数$\g$上存在Ad不变的内积。
\end{corollary}
\begin{theorem}
    紧李群的表示完全可约。
\end{theorem}
\subsection{一个例子:$\mathrm{SU}(2)$}
首先考虑一个例子。这个例子本身很重要.
\begin{example}[$\mathrm{SU}(2)$的表示]
    \textbf{1.李代数方法:}

    目标给出了$\mathrm{SU}(2)$的不可约表示的分类和构造。
    $\mathrm{SU}(2)=\{M \in \mathrm{GL}(2,\C):M^*M=I,\mathrm{det}M=1\}\cong S^3$单连通。
\end{example}
考虑$\mathrm{SU}(2)$的

\chapter{2023.04.20}
令$W_{kl}=\{v \in V:D(\theta_1,\theta_2)v=e^{i(k\theta_1+l\theta_2)},k,l \in \Z\}$权为$(k,l)$的权空间。则$V=\oplus_{k,l \in \Z} W_{kl}$。

$\mathrm{SU}(3)$的李代数的复化为$\mathrm{sl}(3,\C)$。考察$\mathrm{sl}(3,\C)$基底在$W_{kl}$的作用。

李代数$\mathrm{sl}(3,\C)$的基底:
\begin{align*}
    H_1=\begin{pmatrix}
        1&0&0\\0&-1&0\\0&0&0
    \end{pmatrix},H_2=\begin{pmatrix}
        0&0&0\\0&1&0\\0&0&-1
    \end{pmatrix}
\end{align*}

\begin{remark}
    当$k,l$固定的时候,$\lambda:=k\theta_1+l\theta_2$可以看为线性函数:$\mathfrak{h} \to \C$,$\begin{pmatrix}
        \theta_1&0&0\\0&\theta_2&0\\0&0&-\theta_1\theta_2
    \end{pmatrix}\mapsto k\theta_1+l\theta_2$

    即$\lambda\in \mathfrak{h}^*$。

    此时,权空间$W_{kl}$记为:
    $$
    W_\lambda=\{v:H(\theta_1,\theta_2)v=\lambda(H(\theta_1,\theta_2))v,\text{所有}H(\theta_1,\theta_2)\in \mathfrak{h}\}
    $$
\end{remark}
\begin{example}[标准表示]
    设$\C^3=\mathrm{span}\{e_1,e_2,e_3\}$。
\end{example}
\begin{example}[$\mathrm{Sym}^2\C^3$]
    考虑$e_1^2=e_1\otimes e_1$
\end{example}
\begin{example}[伴随表示]
    设$H(\theta_1,\theta_2)$如上。设$\theta_3=\theta_1+\theta_2$。

    (a)$i=j$
\end{example}
\chapter{李群的表示}
\begin{definition}[基本权]
    1.设$\{H_i\}$为$(h_i,\langle \rangle)$的标准正交基,定义:
    \begin{align*}
        \lambda_i(H_j)=\delta_{ij},\lambda_i \in h^*
    \end{align*}
    $\lambda_i$称为基本权,$\lambda$是整支配权 等价于$\lambda=\sum k_i\lambda_i,k_i \in \N$.

    2.素根系的基本权:

    (a)当选定$h$和素根系$\Phi$,基本权$\lambda_i:=\dfrac{2\langle \lambda_i,\alpha_j\rangle}{\langle \alpha_j,\alpha_j\rangle}=\delta_{ij},\forall \alpha_i \in \Phi$
\end{definition}
\begin{example}
    $\mathrm{SU}(2)=\mathrm{sl}(2,\C)$。素根系$\Phi=\{L_1\}$。

    Cartan矩阵$A=(2)$。则$\alpha_1=2\lambda_1$,$\lambda_1=1/2\alpha_1$。
\end{example}
\textbf{基本权空间构造}
Type $A_n$:$\mathrm{SU}(n+1)$与$\mathrm{sl}(n+1,\C)$。

基本权:$\lambda_i=L_1+L_2+\dots+L_i$,$1 \leq i \leq n$.权系:$\Gamma_w=\{\sum c_i\lambda_i|c_i \in \Z\}=\{\sum k_i L_i|k_i \in \Z\}$。

整支配权:$\Gamma_w^d=\{\sum c_i \lambda_i|c_i \in \N\}=\{\sum k_i L_i|k_1 \geq k_2 \dots k_n\}$。

设$R_n$是$\mathrm{SU}(n+1)$的基本表示。
\section{Schur正交化定理}
\begin{theorem}
    设$(\pi_1,V_1)$和$(\pi_2,V_2)$是紧李群$G$的不可约表示。$\langle,\rangle_i$是$V_i$上的$G$不变内积,$i=1,2$。则有:
    \begin{align*}
        \int_G \langle \pi_1(x)u_1,v_1\rangle_1\langle \pi_2(x)u_2,v_2\rangle_2 dx=0
    \end{align*}
\end{theorem}
\begin{proof}
    设$l:V_2 \to V_1$是线性映射,定义新的线性映射:$L_2$
    \begin{align*}
        L_2:V_2 \to V_1,\quad L=\int_G \pi_1(x)\circ l \circ \pi_2(x^{-1})dx
    \end{align*}
    下面验证$L$是$G$可换的。这等价于:
    \begin{align*}
        \forall y\in G,\pi_1(y)\circ L \circ \pi_2(y^{-1})=L
    \end{align*}
    对于$v_2 \in V_2$,有:
    \begin{align*}
        \pi_1(y) \circ L \circ \pi_2(y^{-1})v=\pi_1(y)\int_G \pi_1(x) \circ l \circ \pi_2((yx)^{-1})v_2 dx=\int_G \pi_1(x) \circ l\circ \pi_2(x^{-1})v_2dx=Lv_2
    \end{align*}
    由于$\pi_1,\pi_2$是不等价的,根据Schur引理,这说明$L=0$。故$\langle Lv_2,v_1\rangle_1=0$。

    接下来我们令$l:V_2 \to V_1, \omega_2 \mapsto \langle \omega_2,u_2\rangle_2 u_1$。对于$\forall \omega_2 \in V_2$:
    \begin{align*}
        0=\langle Lv_2,v_1\rangle_1=\int_G \langle \pi_1(x)\circ l\circ \pi_2(x^{-1})v_2,v_1 \rangle_1 dx \text{带入}l,\text{根据内积不变得到结果。} vr
    \end{align*}
\end{proof}
\begin{definition}
    对于任意给定$v,L \in V^*$,$\phi:G \to \C$.$\phi(g)=L(\pi(g)v)$,称为$G$的矩阵系数。
\end{definition}
\begin{remark}
    当$(\pi,V,\langle,\rangle_G)$是酉表示(eg.$G$是紧李群):
    \begin{align*}
        \phi(g):=\langle \pi(g)v,u\rangle_{G'},\text{给定}u,v \in V
    \end{align*}
\end{remark}
\begin{theorem}
    $\phi$是矩阵系数当且仅当$\mathrm{Span}(R_g^*\phi:g \in G)$是有限维向量空间,其中$R_g: G \to G \forall g,h  \mapsto hg$。
\end{theorem}
\begin{theorem}[Schur正交化定理]
    $G$是紧李群。$\pi_1,\pi_2$是不等价的表示。设$\phi_1,\phi_2$分别是对应的矩阵系数,则有:
    \begin{align*}
        \int_G \phi_1(g)\phi_2(g)dg=0
    \end{align*}
\end{theorem}
\begin{corollary}
    \begin{align*}
    \int_G \chi_1(g)\overline{\chi_2(g)}dg=0
    \end{align*}
\end{corollary}
下面两个判别法可以判定是否有等价表示:
\begin{theorem}
    1.$(\pi_1,V_1)$不可约等价于:
    \begin{align*}
        \int_G |\chi_{\pi_1}(g)|^2dg=1
    \end{align*}
    
    2.两个表示等价等价于$\chi_{\pi_1}=\chi_{\pi_2}$。
\end{theorem}
\begin{remark}
    设$\pi=\bigoplus_{i=1}^n \pi_i$是紧李群$G$上的表示且$\tau$是$G$的不可约表示,则:
    \begin{align*}
        \int_G \chi_{\tau}(g)\chi_{\pi}(g)dg
    \end{align*}
    是与$\tau$等价的不可约表示的个数,即在$\pi$中的重数。
\end{remark}
\subsection*{类函数}
\begin{definition}
    类函数定义为$\phi: G \to \C$使得$\phi(ghg^{-1})=\phi(h)$,$\forall g,h \in G$.
\end{definition}
因而特征标是连续的类函数。我们用特征标来对表示进行分类。
\begin{example}[$T^n=(S^1)^n$]
    $T^n=(S^1)^2$的不可约分类。

    注意到有用的事实:$T^n$是交换李群,所以其不可约表示只可能是$1$维。考虑不可约表示族:
    \begin{align*}
        (e^{i\theta_1},\dots,e^{i\theta_n})\cdot v=e^{i(\sum_{k=1}^n m_i\theta_i)}\cdot v,m_i \in \Z
    \end{align*}
    我们证明$T^n$的不可约表示与$\Z^n$中的格点有一一对应。

    首先说明对于不同的对$n$元整数对,上面的表示都是不等价的。

    注意到这是1维表示,则特征标是明显的。因此特征标显然不同。

    由于$L^2((S^1)^n)$的基恰为:
    \begin{align*}
        \{e^{i(m_1\theta_1+\dots+m_n\theta_n)}:m_1,\dots,m_n \in \Z\}
    \end{align*}
   故不存在不可约表示的特征标$\chi$与上面所有的$\chi_{\pi}$都正交,则不存在其他不可约表示。
\end{example}
\begin{example}[$\mathrm{SU}(2)$]
    $\mathrm{Sym}^n \C^2$。

    基底$\{e_1^k,e_2^{n-k}\}=\mathrm{Span}\{z_1^kz+2^{n-k}:0\leq k\leq n\}$.

    表示可以由标准表示诱导的线性作用:
    \begin{align*}
        (g:p)(v):=p(g^{-1}v)
    \end{align*}

    \begin{proof}
        先证明不可约推导不等价。

        $\forall g \in \mathrm{SU}(2)$,特征根$e^{i\theta},e^{-i\theta}$,因此$g \sim \begin{pmatrix}
            e^{i\theta}&0\\0&e^{-i\theta}
        \end{pmatrix}:=t(\theta)$
       对于$\forall 0\leq k \leq n$,令$p_k:=z_1^{k}z_2^{n-k}$。由定义:$\pi_n(t(\theta))p_k$,所以算得$\chi_{\pi_n}(g)=\chi_{\pi_n}(t(\theta))=\sum_{k=0}^n e^{i(2k-n)\theta}=\dfrac{\sin (n+1) \theta}{\sin \theta}$。所以计算有:
     \begin{align*}
        \int_{\mathrm{SU}(2)}|\chi_{\pi_n}(g)|^2dg=\frac{1}{2\pi^2}\int_0^\pi \int_0^\pi \int_0^{2\pi} \dfrac{\sin (n+1) \theta}{\sin \theta}^2\sin^2 \varphi \sin \psi d\psi d\varphi d\theta=1
     \end{align*}
    
    下证$(\pi_n,V_n)$是所有不可约表示。等价于说明$\{\chi_{\pi_n}:n \in \N\}$在所有$\mathrm{SU}(2)$的连续类函数中稠密($L^2$-范数下).这等价于说明$\{t(\theta)\}\cong S^1$中的类函数。由于$t(\theta)\sim t(-\theta)$,则$S^1$上的所有偶函数。

    根据Fourier分析,$\{\cos(n\theta)\}$是$S^1$上的偶函数空间上的稠密子集,且$\chi_{\pi_n}(t(\theta))-\chi_{\pi_{n-2}}(t(\theta))=2\cos(n\theta)$。则$\{\chi_{\pi_n}\}$是$\mathrm{SU}(2)$类函数空间上的稠密子集。于是是所有的不可约表示。
    \end{proof}
\end{example}
\subsection*{Peter-Weyl定理}
\begin{theorem}
    分析:矩阵系数空间是$(C(G),\|\cdot\|_\infty)$的稠密子集。

    代数:紧李群是矩阵李群。
\end{theorem}
\begin{corollary}
    特征标空间是类函数空间的稠密子集。
\end{corollary}
\begin{proof}
    代数到分析(Stone-Weiseestrass定理)

    设$X$是紧拓扑空间,$C(X)$是连续复值函数构成的代数。若$A \subset C(X)$是子代数。满足1.$A$可分类,即$\forall x_1 \neq x_2$,$\exists f:f(x_1)\neq f(x_2)$。2.$A$中有常函数。3.若$f \in A$,则$\overline{f}\in A$。则$A$是$C(X)$的稠密子集。

    我们验证矩阵系数构成子代数。略。

    由于存在单同态$i:G \to \mathrm{GL}(n,\C)$的闭子群。对于$\forall g\in G$,$g \mapsto g_{ij}$,$g \mapsto overline{g_{ij}}$,$g \mapsto 1$都是矩阵系数。因此1,2,3成立,这意蕴着矩阵系数空间是稠密的。

    分析到代数:一个引理:
    \begin{lemma}
        设$G$是紧李群,对于$\forall g \neq e$,存在不可约表示$(\pi,v)$使得$\pi(g)$不是$\mathrm{id}$。
    \end{lemma}
    事实上,取$f \in C(G)$,使得$f(e)=0,f(g)=1$。则存在$\pi$以及矩阵系数$\phi$使得$\|f-\phi\|_{\infty}<\epsilon$.于是$\phi(e)\neq \phi(g)$推的$\pi(g)\neq \pi(e)=\mathrm{id}$。

    下面构造$G$的忠实表示。

    对于$\forall g \in G^0$(单位连通分支),$g_1 \neq e$。于是存在表示$\pi(g_1)\neq \mathrm{id}$从而$G^0$不是$\ker \pi_1$的子集。于是$\ker \pi_1$的维数小于$G$的维数。

    若$\ker$的维度不是$0$,则取$g_2 \neq e$使其在$(\ker \pi_1)^0$中。则表示的直和$\pi_1\oplus \pi_2$的$\ker$维度进一步降低。

    最终把$\ker$降为$0$维。设$\ker=\{c_1,\dots,c_n\}$是有限集合。取$\{\varphi_i\}$:$\varphi_i(c_i)\neq \mathrm{id}$。于是进一步减少$ker$的个数。
    \end{proof}
    \section{紧李群的分类}
    先考虑紧李代数。紧李代数的分类为:
    \begin{align*}
        \g=Z(\g)\oplus S_1 \oplus S_2 \oplus \dots S_m
    \end{align*}
    $S_i$是单李代数。

    但是以$\g$为李代数的李群$G$不一定是紧李群。问题出在交换的部分。

    然而$[G,G]:=\{ghg^{-1}h:g,h \in G\}$是以$\g$的交换子$[\g:\g]=S_1\oplus S_m$为李代数的连通紧半单李群。

    从而由Dynkin分类得到紧李代数的分类,然后得到紧李代数对应的连通李群$G$的分类:$G=\tilde{G}/\Gamma$。$\Gamma$是$Z(\tilde{G})$离散的正规子群。

    从而在得到连通紧半单李群的分类$[G:G]$。而连通紧李群的分类为$[G \times G] \times T^n$
    \begin{align*}
        S_1 \otimes S_2 \dots \otimes S_m/\Gamma \times T^n
    \end{align*}
    其中$\Gamma$是$Z(S_1\times \dots \times S_m)$的有限子群。

    从而Peter-Weyl定理对连通成立。如果不连通,则对每个连通分支:$G/G_0$是有限群。从而$\forall g \notin G_0$,存在表示$\rho$使得$\rho(g)$不是$\mathrm{id}$。

\ifx\allfiles\undefined
	
	% 如果有这一部分的参考文献的话,在这里加上
	% 没有的话不需要
	% 因此各个部分的参考文献可以分开放置
	% 也可以统一放在主文件末尾。
	
	%  bibfile.bib是放置参考文献的文件,可以用zotero导出。
	% \bibliography{bibfile}
	
	\end{document}
	\else
	\fi
\ifx\allfiles\undefined

	% 如果有这一部分另外的package,在这里加上
	% 没有的话不需要
	
	\begin{document}
\else
\fi
\part{黎曼曲面与复几何}
\ifx\allfiles\undefined

	% 如果有这一部分另外的package,在这里加上
	% 没有的话不需要
	
	\begin{document}
\else
\fi
\chapter{链复形}
\section{$R$-Mod上的链复形}
我们直接给出定义:
\begin{definition}{}
  一个$R$模上的链复形是一族$R$模$\{C_n\}$,与模同态$d_n:C_n \to C_{n-1}$,使得$d_n \circ d_{n-1}=0$.习惯上,我们把这些$d_n$称为微分(来源于微分拓扑),把$d_n$的核$\ker d_n$成为$C$的$n$圈,用$Z_n$表示。$d_{n+1}:C_{n+1}\to C_{n}$的像称为$C$的$n$边界,用$B_n$表示。

  显然$B_n \subset Z_n$($d_{n+1}\circ d_n=0$)。定义$C$的$n$阶同调模为$H_n(C)=Z_n/B_n$。
\end{definition}

实际上,存在范畴$Ch$($R$模下)。其对象为一般的链复形。态射$u:C \to D$定义为一族$R$模同态$u_n:C_n \to D_n$,使得与微分交换:$u_{n-1}d=du_{n}$。在这里,我们混用了$d$的记号,但是其意义并非是容易混淆的。请看交换图:

  \[\begin{tikzcd}
	{C_n} && {C_{n-1}} \\
	\\
	{D_n} && {D_{n-1}}
	\arrow["d", from=1-1, to=1-3]
	\arrow["{u_{n-1}}", from=1-3, to=3-3]
	\arrow["{u_n}"', from=1-1, to=3-1]
	\arrow["d"', from=3-1, to=3-3]
\end{tikzcd}\]
这条交换性质保证了下面的命题:
\begin{proposition}{}
  两个链复形之间的态射将圈映射到圈,将边界映射到边界。因此根据模同态定理,$u$诱导了映射$u:H_n(C) \to H_n(D)$。因此$H_n$是从$Ch$到$R$mod的函子。
\end{proposition}
\begin{proof}
  诱导模同态的证明略。要验证$H_n$是函子,需要说明其保$id$和复合。保id也是显然的,因此仅需要证明复合。对于$u,v$是链复形$C,D,E$之间的态射,我们自然有$(u \circ v)_n=u_n \circ v_n$。所以有诱导的映射满足复合关系。
\end{proof}
备注:这段证明说的比较含糊,但实际上是抽象代数的基本验证。我更建议读者自行验证这个命题。

\begin{example}{}
  考虑链复形$\{\mathrm{Hom}(A,C_n)\}$,其是$\Z$上的链复形。其中$A$是$R$模,$C_n$是已知的链复形的第$n$个模。假设$A=Z_n$($C$的第$n$阶圈),则若$H_n(\mathrm{Hom}(Z_n,C))=0$,则$H_n(C)=0$.

  当然我们需要验证其是一个链复形,以及给出其微分。这里省略。给定$a \in Z_n$,我们需要说明存在$b \in C_{n+1}$使得$db=a$。显然可以定义$f:Z_n \to C_n$使得$f(Ra)=Ra$。并且$d\circ f=0$。于是存在$g:Z_n \to C_{n+1}$满足$d \circ g=f$。于是定义$b=g(a)$,从而$db=f(a)=a$。
\end{example}

\begin{definition}{}
  态射$u:C \to D$被称之为拟同构态射,若其诱导的$H_n(C)\to H_n(D)$都是同构。
\end{definition}

把链复形的定义稍微倒错一下(把$C_n$写为$C^{-n}$),我们可以得到上链复形的概念:
\begin{definition}{}
  一个$R$模上的上链复形是指一族$R$模$\{C^n\}$,与模同态$d^n:C^n \to C^{n+1}$,使得$d^{n+1} \circ d^{n}=0$.习惯上,我们把这些$d^n$称为微分(来源于微分拓扑),把$d^n$的核$\ker d^n$成为$C$的$n$上圈,用$Z^n$表示。$d^{n-1}:C^{n-1}\to C^{n}$的像称为$C$的$n$上边界,用$B^n$表示。

  显然$B^n \subset Z^n$($d^{n+1}\circ d^n=0$)。定义$C$的$n$阶上同调模为$H^n(C)=Z^n/B^n$。

  上链复形的其他定义(态射,拟同构)与链复形一致。
\end{definition}

实际中,我们往往限制链复形和上链复形中不为$0$的模的指数(index)。对于一个链复形而言,若除有限个外其他$C_n$都是$0$,称之为有限链复形。如果对于$n>b(n<a)$有$C_n=0$,则称之为有上边界(下边界)链复形。

显然,有界(上有界,下有界)的链复形构成$Ch$的全子范畴。

上链复形有同样的定义。我们不多赘述。

接下来是一些代数拓扑计算同调群的例子。节省篇幅和时间,就不做记录了。
\section{链复形的运算}
我们希望在范畴论的角度下解释链复形,从而我们介绍Abelian范畴的定义。

\begin{definition}{}
一个范畴$\mathcal{A}$被称为$Ab$范畴,若其每个hom集$\mathrm{Hom}(A,B)$都被赋予了一个abel群的结构,使得态射的复合满足分配:$(f+f')\circ g=f \circ g+f'\circ g$。其中$f,f':B \to C$,$g:A \to B$。同理还有右分配。

显然,$Ch$范畴是一个$Ab$范畴。我们定义加法为模同态的相加$(f+g)_n=f_n+g_n$。

对于$Ab$范畴,我们可以定义加性函子$F:\mathcal{A} \to \mathcal{B}$,假如$F$给出了$\mathrm{Hom}(A,B)$到$\mathrm{Hom}(FA,FB)$的群同态。
\end{definition}
然而$Ab$范畴在范畴论上性质并不强。我们常用的许多结构:积,ker,coker都无法给出。所以我们给出下面的定义:
\begin{definition}{}
  一个可加范畴是指一个$Ab$范畴,外加拥有$0$对象和$A \times B$的积。从而在可加范畴上我们可以定义有限的积。

  显然$Ch$范畴也是一个可加范畴。
\end{definition}
\begin{proposition}
  直和与直积与同调操作交换。即$\oplus H_n(A_\alpha)\cong H_n(\oplus A_\alpha)$和$\otimes H_n(A_\alpha)\cong H_n(\otimes A_\alpha)$
\end{proposition}
\begin{proof}
链复形的直积,直和定义是自然的。对于直和而言,容易有$Z_n(\oplus A_\alpha)=\oplus Z_n(A_\alpha)$和$B_n(\oplus A_\alpha)=\oplus B_n(\oplus A_\alpha)$。因此同调群直接做商即可。
\end{proof}

\begin{definition}{}
  链复形$B$称为$C$的子复形,若$B_n \subset C_n$且$B$的微分算子是$C$微分算子在$B$上的限制。

  子复形的定义自然给出了商复形:
  \begin{align*}
    \dots \to C_{n+1}/B_{n+1} \to C_n/B_n \to C_{n-1}/B_{n-1} \to \dots
  \end{align*}
\end{definition}
\begin{proposition}
  若$f:B \to C$是链复形之间的映射。则可以良定义$\{ker f_n\}$是$B$的子复形。也可以良定义$\{\mathrm{coker} f_n\}$是$C$的商复形。
\end{proposition}
\begin{proof}
  显然。
\end{proof}
下面我们介绍一般可加范畴里面$\ker$和$\mathrm{coker}$的定义。
\begin{definition}{}
  对于态射$f:B \to C$,我们定义$\ker$为$i:A \to B$使得满足$fi=0$且任意$i':A' \to B$满足$fi'=0$,都有存在唯一的$g:A' \to A$使得$ig=i'$。定义$\mathrm{coker}f$为$p:C \to D$满足$pf=0$且使得任意满足$p'f=0$的$p':C \to D'$,存在唯一的$h:D \to D'$使得$p'=hp$。
\end{definition}
我们鼓励读者在这里使用交换图以描述核与余核的区别。

\begin{proposition}{}
  $\ker$是单态,$\mathrm{coker}f$是满态。
\end{proposition}
\begin{proof}
  仅对单态说明。读者可以自己尝试证明满态。回忆单态的定义是,对于$g:B \to C$,若任意$f,f':A \to B$满足$gf=gf'$,则$f=f'$。对于$\ker$而言,这一点由泛性质自然给出。
\end{proof}

在$R$模中,单态,单射,ker的概念是重合的。然而一般的范畴却不一定如此——ker是否能定义都是未决的问题。同样,在$R$模范畴中,满态,满射,coker的概念都是重合的,而一般的范畴则不一定。
\begin{proposition}{}
  对于$R$模上的$Ch$范畴,$f$是$A$到$B$的链映射。则之前定义的$\ker f$和$\mathrm{coker}f$确实是满足一般范畴定义的核和余核。
\end{proposition}
\begin{proof}
  套用定义,然后根据$R$模范畴中核,余核定义即可得到结果。
\end{proof}

定义核和余核是定义阿贝尔范畴的关键。
\begin{definition}{}
  称一个加性范畴是一个Abelian范畴,若其满足三个条件:

  1.每个态射都有核和余核。

  2.任何一个单态都是其余核态射的核。

  3.任何一个满态都是其核态射的余核。
\end{definition}
显然$R$模范畴是一个Abelian范畴。可以证明以下事实:
\begin{proposition}{}
  给定$\mathcal{A}$作为Abelian范畴,可以良定义态射的像$im(f)$。对于$f:B \to C$,我们有$\ker(\mathrm{coker}f)=\cong \mathrm{coker}(\ker f)$。因而定义$im(f)=\ker(\mathrm{coker}f)$
\end{proposition}
用交换图可以更加准确的描述命题里面的同构。
\[
  \begin{tikzcd}
	{\ker f} & B & C & {\mathrm{coker}f} \\
	& {\mathrm{coker}i} & {\ker p}
	\arrow["f", from=1-2, to=1-3]
	\arrow["i", from=1-1, to=1-2]
	\arrow["p", from=1-3, to=1-4]
	\arrow[from=1-2, to=2-2]
	\arrow[from=2-3, to=1-3]
	\arrow[dashed, from=2-2, to=2-3]
\end{tikzcd}\]
图中虚线的存在是因为泛性质。不妨考虑$fi=0$,所以根据$\mathrm{coker}$泛性质,存在唯一的$\mathrm{coker}i \to C$.由于$\mathrm{coker}i$是满态,所以构造的态射与$p$复合后是$0$。因此根据$\ker$泛性质,有虚线态射的存在。若此态射是同构,我们称这是一个严格的态射$f$。从而Abelian范畴有一个性质为:
\begin{proposition}{}
  Abelian范畴的态射都严格。
\end{proposition}
\begin{definition}{}

  1.称一个$\mathcal{A}$的列是正合列,若每个对象处都有$\ker=im$。

  2.考虑$\mathcal{B}$是$\mathcal{A}$的子Abelian范畴,若其本身是Abelian的,并且任何一个在$\mathcal{B}$的正合列,在$\mathcal{A}$中都正合。
\end{definition}

在Abelian范畴下可以讨论一般通过链复形的结构。因此我们给出了可加范畴$Ch(\mathcal{A})$。从而同调成为了从这个范畴到$\mathcal{A}$的函子。

\begin{theorem}{}
  $Ch(\mathcal{A})$是一个Abelian范畴。
\end{theorem}
\begin{proof}
  首先我们验证该范畴有核和余核。构造与$R$模范畴中的构造类似,因此留给读者做验证。(ker和coker的映射需要用泛性质给出)。

  如果$f:B \to C$是单态链映射,我们断言,$f$是单的,当且仅当$f_n$对于每个$n$都是单的。(可以构造一个非常简单的复形)。对于$f_n$,自然的$B_n$是$\ker (\mathrm{coker}f_n)$。我们断言$B\cong \ker (\{\mathrm{coker}f_n\})$。显然在$B_n$上类似。而对于微分算子:
\[
    \begin{tikzcd}
	{B_n} && {B_{n-1}} \\
	\\
	{\ker (\mathrm{coker} f_n)} && {\ker(\mathrm{coker}f_{n-1})}
	\arrow[from=1-1, to=1-3]
	\arrow[from=1-1, to=3-1]
	\arrow[from=1-3, to=3-3]
	\arrow[from=3-1, to=3-3]
\end{tikzcd}
\]
  这张图本身是交换的。因为同构$B_n\cong \ker (\mathrm{coker}f_n)$本身来自于$\ker$的泛性质(不妨自己研究一下)。

  满态的情况不予验证。
\end{proof}
\begin{proposition}{}
  链复形的正合(范畴论的定义)等价于每个列$0 \to A_n \to B_n \to C_n \to 0$都正和。
\end{proposition}
\begin{proof}
  考察链复形中的像。不妨考虑像定义为$\ker$。对于$0 \to A \to B \to C \to 0$,其中$i:A \to B$的像定义为$\ker (\mathrm{coker}i)$。同时,$\ker p$存在,并且根据泛性质(自行验证),存在$\ker (\mathrm{coker}i) \to \ker p$的态射。

  正合意味着这个态射是一个同构。不难发现这个态射本身为$\{\ker (\mathrm{coker}i_n) \to \ker p_n\}$。同构于是等价于每一个小的态射都是同构。
\end{proof}

接下来我们讨论双链复形。这一节会更多出现在谱序列的章节中,就之后记录。

链复形有着丰富的构造,这一节只做了最基本的描述。我们将在第五节见到更多构造。
\section{长正合列}
\begin{theorem}{}
  设$0 \to A \to B \to C \to 0$是链复形的正合列。那么存在一个自然的映射$\partial: H_n(C)\to H_{n-1}(A)$,称为连接同态,使得:
  \begin{align*}
    \dots \to H_{n+1}(C) \to H_n(A) \to H_n(B) \to H_n(C) \to H_{n-1}(A) \to \dots
  \end{align*}
  是一个正合列。

  同样,对于上链复形的正合列:$0 \to A \to B \to C  \to 0$,存在一个自然的$\partial: H^n(C)\to H^{n+1}(A)$和一个长正合列:
  \begin{align*}
    \dots \to H^{n-1}(C) \to H^n(A)\to H^n(B) \to H^n(C) \to H^{n+1}(A)
  \end{align*}
\end{theorem}
如果使用追图的技巧,这个定理的证明是相当容易的。(显得繁琐,但是没有思维难度)为此,我们试图介绍一些可能看起来不那么初等的证明。

\begin{proposition}{}
  对于链复形的正合列:$0 \to A \to B \to C \to 0$,如果其中有两个链复形是正合的,则第三个链复形也是正合的。
\end{proposition}
\begin{proof}
  在长正合列中考虑两个同调群为$0$。由正合性可以得到另外一个也是$0$。
\end{proof}

\begin{proposition}{}[33引理]
  给定交换图
  \[
    \centering
    \begin{tikzcd}
	& 0 & 0 & 0 \\
	0 & {A'} & {B'} & {C'} & 0 \\
	0 & A & B & C & 0 \\
	0 & {A''} & {B''} & {C''} & 0 \\
	& 0 & 0 & 0
	\arrow[from=1-2, to=2-2]
	\arrow[from=2-1, to=2-2]
	\arrow[from=3-1, to=3-2]
	\arrow[from=2-2, to=2-3]
	\arrow[from=3-2, to=3-3]
	\arrow[from=2-3, to=2-4]  
	\arrow[from=3-3, to=3-4]
	\arrow[from=4-2, to=4-3]
	\arrow[from=4-3, to=4-4]
	\arrow[from=4-1, to=4-2]
	\arrow[from=4-4, to=4-5]
	\arrow[from=3-4, to=3-5]
	\arrow[from=2-4, to=2-5]
	\arrow[from=1-3, to=2-3]
	\arrow[from=1-4, to=2-4]
	\arrow[from=2-2, to=3-2]
	\arrow[from=3-2, to=4-2]
	\arrow[from=4-2, to=5-2]
	\arrow[from=2-3, to=3-3]
	\arrow[from=3-3, to=4-3]
	\arrow[from=4-3, to=5-3]
	\arrow[from=2-4, to=3-4]
	\arrow[from=3-4, to=4-4]
	\arrow[from=4-4, to=5-4]
\end{tikzcd}
  \]
  这是一个在某个阿贝尔范畴的交换图,使得每一列都是正合的。则:

  1.若底部两行正合,则第一行也正合。

  2.若顶部两行正合,则第三行也正合。

  3.若顶部和底部两行正合,且复合$A \to C$是$0$,则中间一行也正合。
\end{proposition}
\begin{proof}
  追图即可。读者自证不难。或者也可以考虑上述定理,只需要说明这是链复形之间的映射。
\end{proof}

我们用蛇形引理证明上述的长正合列定理。然而我们不准备给出蛇形引理的证明。
\begin{lemma}[Snake]{snake}
  考虑交换图:
  \[
    \begin{tikzcd}
	& {\ker f} & {\ker g} & {\ker h} \\
	& {A'} & {B'} & {C'} & 0 \\
	0 & A & B & C \\
	& {\mathrm{coker} f} & {\mathrm{coker} g} & {\mathrm{coker} h}
	\arrow[from=2-2, to=2-3]
	\arrow[from=2-3, to=2-4]
	\arrow[from=2-4, to=2-5]
	\arrow["f", from=2-2, to=3-2]
	\arrow[from=3-1, to=3-2]
	\arrow["g", from=2-3, to=3-3]
	\arrow[from=3-3, to=3-4]
	\arrow["h", from=2-4, to=3-4]
	\arrow[from=3-2, to=3-3]
	\arrow[from=1-2, to=2-2]
	\arrow[from=1-3, to=2-3]
	\arrow[from=1-4, to=2-4]
	\arrow[from=3-2, to=4-2]
	\arrow[from=3-3, to=4-3]
	\arrow[from=3-4, to=4-4]
	\arrow[dashed, from=1-2, to=1-3]
	\arrow[dashed, from=1-3, to=1-4]
	\arrow[dashed, from=4-2, to=4-3]
	\arrow[dashed, from=4-3, to=4-4]
	\arrow[curve={height=24pt}, squiggly, from=1-4, to=4-2]
\end{tikzcd}
  \]
   如果图中实两行都是正合的,那么存在一个正合列:
   \begin{align*}
    \ker f \to \ker g \to \ker h \to \mathrm{coker}f \to \mathrm{coker} g \to \mathrm{coker} h
   \end{align*}
   其中$\ker h \to \mathrm{coker}f$需要构造。可以简单描述为:
   \begin{align*}
    \partial(c')=i^{-1}g p^{-1}(c') ,c' \in \ker h
   \end{align*}
\end{lemma}

蛇形引理在一般的abelian范畴里面也是成立的。原因是我们可以把一个小abelian范畴$\mathcal{A}$嵌入到$R$-mod范畴中。对于非小范畴$\mathcal{C}$,对于任何交换图,我们都可以找到小范畴包含这张交换图。所以在abelian范畴里面这也是成立的。

\begin{lemma}[5引理]{5lemma}
  考虑交换图
  \[
    \begin{tikzcd}
	{A'} & {B'} & {C'} & {D'} & {E'} \\
	A & B & C & D & E
	\arrow["a"', from=1-1, to=2-1]
	\arrow["b"', from=1-2, to=2-2]
	\arrow["c"', from=1-3, to=2-3]
	\arrow["d"', from=1-4, to=2-4]
	\arrow["e"', from=1-5, to=2-5]
	\arrow[from=1-1, to=1-2]
	\arrow[from=1-2, to=1-3]
	\arrow[from=1-3, to=1-4]
	\arrow[from=1-4, to=1-5]
	\arrow[from=2-1, to=2-2]
	\arrow[from=2-2, to=2-3]
	\arrow[from=2-3, to=2-4]
	\arrow[from=2-4, to=2-5]
\end{tikzcd}
  \]
  若$a,b,d,e$是同构,且行正合,则$c$也是同构。
\end{lemma}

现在我们讨论导引长正合列的办法。对于$0 \to A \to B \to C \to 0$是正合的链复形链。我们给出:
\[ \begin{tikzcd}
	& 0 & 0 & 0 \\
	0 & {Z_nA} & {Z_n B} & {Z_nC} \\
	0 & {A_n} & {B_n} & {C_n} & 0 \\
	0 & {A_{n-1}} & {B_{n-1}} & {C_{n-1}} & 0 \\
	& {A_{n-1}/dA_n} & {B_{n-1}/dB_n} & {C_{n-1}/dC_n} & 0 \\
	& 0 & 0 & 0
	\arrow[from=3-2, to=3-3]
	\arrow[from=3-3, to=3-4]
	\arrow[from=3-4, to=3-5]
	\arrow["d", from=3-2, to=4-2]
	\arrow[from=4-1, to=4-2]
	\arrow["d", from=3-3, to=4-3]
	\arrow[from=4-3, to=4-4]
	\arrow["d", from=3-4, to=4-4]
	\arrow[from=4-2, to=4-3]
	\arrow[from=2-2, to=3-2]
	\arrow[from=2-3, to=3-3]
	\arrow[from=2-4, to=3-4]
	\arrow[from=4-2, to=5-2]
	\arrow[from=4-3, to=5-3]
	\arrow[from=4-4, to=5-4]
	\arrow[from=5-2, to=6-2]
	\arrow[from=5-3, to=6-3]
	\arrow[from=5-4, to=6-4]
	\arrow[from=5-2, to=5-3]
	\arrow[from=5-3, to=5-4]
	\arrow[from=2-2, to=2-3]
	\arrow[from=2-3, to=2-4]
	\arrow[from=2-1, to=2-2]
	\arrow[from=1-3, to=2-3]
	\arrow[from=1-4, to=2-4]
	\arrow[from=1-2, to=2-2]
	\arrow[from=3-1, to=3-2]
	\arrow[from=5-4, to=5-5]
	\arrow[from=4-4, to=4-5]
\end{tikzcd}\]

以及:
\[
  \begin{tikzcd}
	& {A_n/dA_{n+1}} & {B _n/dB_{n+1}} & {C_n/dC_{n+1}} & 0 \\
	0 & {Z_{n-1}B} & {Z_{n-1}B} & {Z_{n-1}C}
	\arrow[from=1-2, to=1-3]
	\arrow[from=1-3, to=1-4]
	\arrow[from=1-4, to=1-5]
	\arrow[from=2-1, to=2-2]
	\arrow[from=2-2, to=2-3]
	\arrow[from=2-3, to=2-4]
	\arrow[from=1-2, to=2-2]
	\arrow[from=1-3, to=2-3]
	\arrow[from=1-4, to=2-4]
\end{tikzcd}\]


根据第一幅图我们知道第二幅图的第一列和第二列都是正合的。根据snake引理可知第二幅图给出了我们想要的长正合列。

当然也可以直接追图得到长正合列。这没有什么本质困难的东西。

最后我们说明长正合列中$\partial$是自然的。即对于两个正合链复形列之间的态射,我们能给出一个交换图。
\[
  % https://q.uiver.app/#q=WzAsMTAsWzAsMCwiMCJdLFsxLDAsIkEiXSxbMiwwLCJCIl0sWzMsMCwiQyJdLFs0LDAsIjAiXSxbNCwxLCIwIl0sWzMsMSwiQyciXSxbMiwxLCJCJyJdLFsxLDEsIkEnIl0sWzAsMSwiMCJdLFswLDFdLFsxLDJdLFsyLDNdLFszLDRdLFs5LDhdLFs4LDddLFs3LDZdLFs2LDVdLFsxLDhdLFsyLDddLFszLDZdXQ==
\begin{tikzcd}
	0 & A & B & C & 0 \\
	0 & {A'} & {B'} & {C'} & 0
	\arrow[from=1-1, to=1-2]
	\arrow[from=1-2, to=1-3]
	\arrow[from=1-3, to=1-4]
	\arrow[from=1-4, to=1-5]
	\arrow[from=2-1, to=2-2]
	\arrow[from=2-2, to=2-3]
	\arrow[from=2-3, to=2-4]
	\arrow[from=2-4, to=2-5]
	\arrow[from=1-2, to=2-2]
	\arrow[from=1-3, to=2-3]
	\arrow[from=1-4, to=2-4]
\end{tikzcd}\]
\begin{proposition}{}
  长正合列是一个从$\mathcal{S}$到$\mathcal{T}$的函子。其中$\mathcal{S}$是短链复形正合列范畴,$\mathcal{T}$是长正合列范畴。
\end{proposition}
\begin{proof}
  由于$H_n$是函子,所以我们只需要给出$\partial$的自然性。用嵌入定理可以只考虑$R$模范畴,从而追图。给定$z \in H_n(C)$,用$c$表示$z$在$Z_nC$的代表元。显然$z'$作为$H_n(C')$中$z$的像拥有$c'$作为代表元。

   用$b$表示$c$的一个原像,则$b'$作为$b$在$B'$中像是$c'$的一个原像。从而根据$\partial$的构造可以得到上图交换。
\end{proof}

\begin{proposition}{}
  正合列$0 \to Z \to C \to B[-1] \to 0$给出一个可以分裂为短正合列的长正合列。
\end{proposition}
\begin{proof}
  考虑$Z$的同调群为$Z_n$.而$B[-1]$的同调群是$B_{n-1}$。长正合列为$Z_n \to H_n(C) \to B_{n-1} \to Z_{n-1}$。其中$H_n(C) \to B_{n-1}$是$0$态射。
\end{proof}
\begin{proposition}{}
  作为链复形之间的态射$f$,若$\ker f$和$\mathrm{coker}f$零调,则$f$是一个拟同构。
\end{proposition}
\begin{proof}
  若ker和coker其中有一个平凡,则诱导的长正合列即可得到结果。

  现在考虑两个都不平凡。不妨考虑$\ker f \to A \to im(f) \to B \to \mathrm{coker} f$。其中前三个是短正合的,后三个也是短正合的。$\ker f$零调意味着$H_n(A)=H_n(im(f))$。$\mathrm{coker}f$零调意味着$H_n(im(f))=H_n(B)$

  余下要验证的是$f$确实可以如此分解。但是根据abelian范畴,上述两个同构的复合确实是$f_*$。
\end{proof}
\section{链同伦}
链同伦来自于代数拓扑。这是一个很好的概念(用于证明拟同构)。

\begin{definition}{}
  一个复形$C$称为分裂的,如果存在$s_n:C_n \to C_{n+1}$使得$dsd=s$.如果$C$还是零调的,则称$C$是分裂正合的。
\end{definition}
下面的例子说明零调的复形也可能不是分裂的。
\begin{example}{}
  $\to \Z/4 \to \Z/4 \to \Z/4 \to \dots$

  其中每个映射都是把元素乘$2$。这是一个零调的列(不管是作为$\Z$还是$\Z_4$模)。然而不是正合的,原因是不存在直和分解:$\Z_4 \cong \Z_2\oplus \Z_2$。
\end{example}
下面的命题涉及一些投射模的性质。不懂这个的读者可以先不看。
\begin{proposition}{}
  零调有下界,且均为自由模的链复形是分裂正合的。零调且均为自由生成abel群的链复形是分裂正合的。
\end{proposition}
\begin{proof}
  对于第一句话,考虑$R^k \to R^m \to R^n \to 0$。因为$R^n$是投射模,所以是$R^m$的直和项,于是有自然嵌入到$R^m$的映射。并且这个映射满足$dsd=d$。接着考虑$R^m$的直和项(分解$R^n$后剩下的),其也是投射模,并且$R^k$到其为满射,则其为$R^k$的直和项。定义$R^m$到$R^k$的映射为该直和打回去的映射。

  对于第二句话。因为$\Z$是主理想整环,所以有限生成自由模的子模是有限生成自由模。因此大家都是投射模。
\end{proof}

\begin{definition}{}
  对于链映射$f:C \to D$,称其为0伦的,若存在一族映射$s_n:C_n \to D_{n+1}$使得$f=ds+sd$。$\{s_n\}$称作$f$的链收缩。

  称$f,g:C \to D$是同伦的链映射,若$f-g$是零伦的。同伦等价的定义不再赘述。
\end{definition}
\begin{proposition}{}
  同伦的链映射诱导一样的同调群同态,因此同伦等价的链复形同调群同构。
\end{proposition}
\section{映射锥和映射柱}
设$f: B \to C$是链复形之间的映射。我们可以定义$f$的映射锥$\mathrm{cone}(f)$为一个新的链复形.

其$n$阶元为$B_{n-1}\oplus C_n$,微分定义为:
\begin{align*}
	d(b,c)=(-d(b),d(c)-f(b))
\end{align*}
我们省略验证$d \circ d$是一个复形的计算。同理,对于上链复形之间的映射$f: B \to C$,也可以定义$\mathrm{cone}(f)$.其第$n$阶元为$B^{n+1}\oplus C^n$,而微分:
\begin{align*}
	d(b,c)=(-db,dc-f(b))
\end{align*}
\begin{proposition}{}
	设$\mathrm{cone}(C)$定义为$C \to C$的恒同映射给出的映射柱。则$\mathrm{Cone}(C)$是分裂正合的。并且$s(b,c)=(-c,0)$是分裂映射。
\end{proposition}
\begin{proof}
	我们先考虑$0$调。实际上$d(c,c')=(-dc,dc'-c)$。因此$\ker d$满足$dc=0$且$dc'=c$.(自然可以只写为$dc'=c$)。所以自然有$d(-c',0)=(c,c')$,于是$\ker d =im(d)$所以零调。

	再考虑分裂正合。实际上$dsd(c,c')=ds(-dc,dc'-c)=d(c-dc',0)=(-dc,dc'-c)$。所以分裂正合成立。

\end{proof}

\begin{proposition}{}
	设$f$是$C$和$D$之间的链映射。$f$零伦当且仅当$f$可以延拓为$(-s,f):\mathrm{Cone}(C)\to D$的链映射。
\end{proposition}
\begin{proof}
	设$f$零伦,则$f=sd+ds$,$s:C_n \to D_{n+1}$。从而$(-s,f)(c,c')=-s(c)+f(c'),c \in C_n,c' \in C_{n+1}$.显然我们有$d(-s,f)(c,c')=-ds(c)+df(c')$,$(-s,f)d(c,c')=(-s,f)(-dc,dc'-c)=sdc+fdc'-c$.

	不难验证两个结果是相同的。

	另一方面,如果可以延拓为上述映射,则不难发现$f=ds+sd$。
\end{proof}

现在我们说明任何$f_*$都可以用下面的方式描述(导引长正合列)。让我们考虑短正合列:
\begin{align*}
	0 \to C \to \mathrm{cone}(f) \to B[-1] \to 0
\end{align*}
其中$c \mapsto (0,c)$,$(b,c) \mapsto -b$。这是正合列,所以导引长正合:
\begin{align*}
	\dots \to H_{n+1}(\mathrm{cone}(f)) \to H_n(B) \to H_n(C) \to H_n(\mathrm{cone} f) \to H_{n-1}(B) \to \dots
\end{align*}
其中连接同态$\partial$正是$H_n(B) \to H_n(C)$。因此下面命题就是自然的了。

\begin{proposition}{}
	$\partial=f_*$
\end{proposition}
\begin{proof}
	考虑$b \in B_n$是一个圈,那么$(-b,0)$在映射锥复形中提升了$b$。求一次微分,有$(db,fb)=(0,fb)$。于是:
	\begin{align*}
		\partial [b]=[fb]=f_*[b]
	\end{align*}
\end{proof}
\begin{corollary}{}
	$f:B \to C$是拟同构,当且仅当$\mathrm{cone}(f)$是正合的。因此我们把拟同构的问题化成了分裂的正合列的问题。
\end{corollary}

一个类似的构造是映射柱。我们用$\mathrm{cyl}(f)$表示。

对于$f:B \to C$,定义$\mathrm{cyl}(f)$的$n$阶元为$B_n \oplus B_{n-1}\oplus C_n$。定义微分:
\begin{align*}
	d(b,b',c)=(d(b)+b',-d(b'),d(c)-f(b'))
\end{align*}
我们最好用矩阵来描述:
\begin{align*}
	\begin{pmatrix}
	d_B & \mathrm{id}_B & 0\\
	0 & -d_B & 0\\
	0  & -f & d_C 
\end{pmatrix}
\end{align*}
通过计算该矩阵的平方,我们知道该微分满足$d^2=0$。

\begin{proposition}{}
	映射柱$\mathrm{cyl}(C)$表示恒同映射$\mathrm{id}_C$的映射柱。则$f,g$是$C \to D$的同伦的映射当且仅当存在$(f,s,g):\mathrm{cyl}(C) \to D$的链映射。
\end{proposition}
\begin{proof}
	这一点从拓扑上其实很好想。当然我们也可以计算一下交换性:
	\begin{align*}
		d(f,s,g)(c_1,c,c_2)=d(f(c_1)+s(c)+g(c_2))=df(c_1)+ds(c)+dg(c_2) \\
        (f,s,g)d(c_1,c,c_2)=(f,s,g)(dc_1+c,-dc,dc_2-c)=fd(c_1)+f(c)-sd(c)+gd(c_2)-g(c)
	\end{align*}
	相减得到:$ds(c)-sd(c)-f(c)+g(c)$。所以$f,g$同伦等价于说该式子等于$0$,也就意味着交换。
\end{proof}
\begin{lemma}{}
	由$(0,0,c)$生成的子复形同构于$C$。并且$\alpha:C \to \mathrm{cyl}(f)$是一个拟同构。
\end{lemma}
\begin{proof}
	这一条也是非常符合拓扑感受的(映射柱和底盘是同伦的)。首先,子复形是显然的。我们只需要说明存在这样一个正合
	\begin{align*}
		0 \to C \to \mathrm{cyl}(f) \to \mathrm{cone}(-\mathrm{id}_B) \to 0
	\end{align*}
	这个正合也是显然的。最后根据$\mathrm{cone}(-id_B)$的零调性,根据蛇形引理可得结果。
\end{proof}
事实上,这是一个链同伦等价。原因是我们定义$\beta(b,b',c)=f(b)+c$.则有:
\begin{align*}
	\alpha \beta(b,b',c)=(0,0,f(b)+c), \quad \beta(\alpha(c))=c
\end{align*}
只用说明第一个复合同伦于$\mathrm{id_{\mathrm{cyl(f)}}}$。 事实上,相减后得到$(0,0,f):\mathrm{cyl}(f) \to \mathrm{cyl}(f)$。定义$s(b,b',c)=(0,b,0)$.则容易得到$(0,0,f)=ds+sd$。

我们也可以用映射柱考量$f_*$。显然$(b,0,0)$生成的子复形同构于$B$,并且$\mathrm{cyl}(f)/B$同构于$\mathrm{cone}(f)$。

定义$B \to \mathrm{cyl}(f) \to C$,第二个映射是$\beta$。这个复合正好是$f$。所以$f_*$也分解开。我们可以构造下面的交换图:
\[\begin{tikzcd}
	&& C \\
	0 & B & {\mathrm{cyl}(f)} & {\mathrm{cone}(f)} & 0 \\
	& 0 & C & {\mathrm{cone}(f)} & {B[-1]} & 0
	\arrow[from=2-1, to=2-2]
	\arrow[from=2-2, to=2-3]
	\arrow[from=2-3, to=2-4]
	\arrow[from=2-4, to=2-5]
	\arrow["f", from=2-2, to=1-3]
	\arrow["\beta", from=2-3, to=1-3]
	\arrow["\alpha", from=3-3, to=2-3]
	\arrow[from=3-2, to=3-3]
	\arrow[from=3-3, to=3-4]
	\arrow["{=}", no head, from=2-4, to=3-4]
	\arrow["\delta", from=3-4, to=3-5]
	\arrow[from=3-5, to=3-6]
\end{tikzcd}\]
并且同调的长正合列满足下面的交换图:
\[\begin{tikzcd}
	{H_{n}(B)} & {H_n(\mathrm{cyl}f)} & {H_n(\mathrm{cone}(f))} & {H_{n-1}(B)} \\
	{H_{n+1}(B[-1])} & {H_n(C)} & {H_{n}(\mathrm{cone}(f))} & {H_{n}(B[-1])}
	\arrow[from=1-1, to=1-2]
	\arrow[from=1-2, to=1-3]
	\arrow["{-\partial}", from=1-3, to=1-4]
	\arrow[from=2-2, to=2-3]
	\arrow[from=2-1, to=2-2]
	\arrow["\delta", from=2-3, to=2-4]
	\arrow["{=}", no head, from=1-1, to=2-1]
	\arrow["{=}", no head, from=1-3, to=2-3]
	\arrow["{=}", no head, from=1-4, to=2-4]
	\arrow["{=}", no head, from=1-2, to=2-2]
	\arrow["{f_*}", from=1-1, to=2-2]
\end{tikzcd}\]
为什么交换?我们唯余验证最后一个方块(前面的交换可以直接由第一个图的交换给出)

设$(b,c)$是$\mathrm{cone}(f)$中的圈。因而根据定义有$db=0,f(b)=dc
$.将其提升到$(0,b,c)$,考虑:
\begin{align*}
	d(0,b,c)=(0+b,-db,dc-f(b))=(b,0,0)
\end{align*}
因此$\partial$将$(b,c)$的类映射到$b=-\delta(b,c)$的类。

映射柱和映射锥为我们提供了一个自然的方式于将任何一个链复形映射$f: B \to C$变为一个长正合列。为了说明这里的长正合列是良定的,我们需要说明一般的由$0 \to B \to C \to D \to 0$导引的长正合列与$f$,$g$给出的是一致的。

首先考虑$f$。对于$\mathrm{cone}(f)$而言,存在$\varphi:\mathrm{cone}(f)\to D$,满足$\varphi(b,c)=g(c)$。我们有下面的交换图:
\[\begin{tikzcd}
	& 0 & C & {\mathrm{cone}(f)} & {B[-1]} & 0 \\
	0 & B & {\mathrm{cyl}(f)} & {\mathrm{cone}(f)} & 0 \\
	0 & B & C & D & 0
	\arrow[from=1-2, to=1-3]
	\arrow[from=1-3, to=1-4]
	\arrow[from=1-4, to=1-5]
	\arrow[from=1-5, to=1-6]
	\arrow[from=2-1, to=2-2]
	\arrow[from=3-1, to=3-2]
	\arrow[from=2-2, to=2-3]
	\arrow[from=2-3, to=2-4]
	\arrow[from=2-4, to=2-5]
	\arrow[from=3-2, to=3-3]
	\arrow[from=3-3, to=3-4]
	\arrow[from=3-4, to=3-5]
	\arrow["{=}", no head, from=2-2, to=3-2]
	\arrow["\beta", from=2-3, to=3-3]
	\arrow["\alpha", from=1-3, to=2-3]
	\arrow["{=}", no head, from=1-4, to=2-4]
	\arrow["\varphi", from=2-4, to=3-4]
\end{tikzcd}\]
考虑$\beta$是一个拟同构,因此我们根据5引理可以知道$\varphi$也是一个拟同构。然而$\varphi$并不一定是一个链同伦。

\begin{example}{}
	考虑$B,C$是模,并且给出了一个只在$0$度非$0$的链复形。因此$\mathrm{cone}(f)$在$1$度的模是$B$,在$0$度的模是$C$。根据定义可知$d'(b)=-f(b)$。

	我们断言$\varphi$是链同伦等价,当且仅当$f:B \to C$是一个分裂的单射(换言之,$B$是$C$的直和项)

	实际上,$\varphi$只在$0$度的时候有非零的情况:$\varphi_0=g$。下面的交换图很直接:
	\[\begin{tikzcd}
	0 & B & C & 0 \\
	0 & 0 & D & 0 \\
	0 & B & C & 0
	\arrow[from=1-1, to=1-2]
	\arrow[from=1-3, to=1-4]
	\arrow["{-f}", from=1-2, to=1-3]
	\arrow[from=2-1, to=2-2]
	\arrow[from=2-2, to=2-3]
	\arrow[from=2-3, to=2-4]
	\arrow["g", from=1-3, to=2-3]
	\arrow[from=1-2, to=2-2]
	\arrow["h", from=2-3, to=3-3]
	\arrow[from=2-2, to=3-2]
	\arrow[from=3-1, to=3-2]
	\arrow["{-f}"', from=3-2, to=3-3]
	\arrow[from=3-3, to=3-4]
	\arrow["s"', from=1-3, to=3-2]
\end{tikzcd}\]
注意到$B \to 0\to B$的态射必须和$id$相差无几,所以$s \circ (-f)=-\mathrm{id}_B$,所以$B$内射进入$C$是分裂的。

反过来,如果有这样的分裂,则可以定义$h,s$是另外的投射。不难验证这是一个链同伦。

\end{example}

接下来我们需要验证导引的长正合列。即
% https://q.uiver.app/#q=WzAsMTIsWzEsMCwiSF9uKEIpIl0sWzAsMCwiXFxkb3RzIl0sWzEsMSwiSF9uKEIpIl0sWzAsMSwiXFxkb3RzIl0sWzIsMSwiSF9uKEMpIl0sWzMsMSwiSF9uKEQpIl0sWzQsMSwiSF97bi0xfShCKSJdLFs1LDEsIlxcZG90cyJdLFsyLDAsIkhfbihcXG1hdGhybXtjeWx9KGYpKSJdLFszLDAsIkhfbihcXG1hdGhybXtjb25lfShmKSkiXSxbNCwwLCJIX3tuLTF9KEIpIl0sWzUsMCwiXFxkb3RzIl0sWzEsMCwiXFxwYXJ0aWFsIl0sWzMsMiwiXFxwYXJ0aWFsIl0sWzIsNF0sWzQsNV0sWzYsN10sWzUsNiwiXFxwYXJ0aWFsIl0sWzAsOF0sWzgsOV0sWzksMTAsIlxccGFydGlhbCJdLFsxMCwxMV0sWzAsMiwiIiwxLHsic3R5bGUiOnsiaGVhZCI6eyJuYW1lIjoibm9uZSJ9fX1dLFs4LDQsIlxcY29uZyJdLFs5LDUsIlxcY29uZyJdLFsxMCw2LCJcXHNpbSJdXQ==
\[\begin{tikzcd}
	\dots & {H_n(B)} & {H_n(\mathrm{cyl}(f))} & {H_n(\mathrm{cone}(f))} & {H_{n-1}(B)} & \dots \\
	\dots & {H_n(B)} & {H_n(C)} & {H_n(D)} & {H_{n-1}(B)} & \dots
	\arrow["\partial", from=1-1, to=1-2]
	\arrow["\partial", from=2-1, to=2-2]
	\arrow[from=2-2, to=2-3]
	\arrow[from=2-3, to=2-4]
	\arrow[from=2-5, to=2-6]
	\arrow["\partial", from=2-4, to=2-5]
	\arrow[from=1-2, to=1-3]
	\arrow[from=1-3, to=1-4]
	\arrow["\partial", from=1-4, to=1-5]
	\arrow[from=1-5, to=1-6]
	\arrow[no head, from=1-2, to=2-2]
	\arrow["\cong", from=1-3, to=2-3]
	\arrow["\cong", from=1-4, to=2-4]
	\arrow["\sim", from=1-5, to=2-5]
\end{tikzcd}\]
\begin{proposition}{}
	复合$H_n(D) \cong H_n(\mathrm{cone}f)  \stackrel{-\delta_*}{\rightarrow} H_n(B[-1]) \cong H_{n-1}(B)$给出了$\partial$。
\end{proposition}
\begin{proof}
	取$g(c)$作为$D$中的$n$圈,用$(b,c)$代表其在$\mathrm{cone}(f)$中的像。(这意味着$db=0,dc=f(b)$)。于是$-\delta(b,c)=b$。

	另一方面,仍然考虑$g(c)$。则$dc=f(b)$且$b$是$\partial$的原像。
\end{proof}

同样的,我们也可以了类似的说明$B[-1]$和$\mathrm{cone}(g)$有一个拟同构,其对偶于$\varphi$。

\begin{proposition}{}
	给定$f:B \to C$。设$v$是$C$嵌入到$\mathrm{cone}(f)$的态射。那么存在一个$\mathrm{cone}(v)$到$B[-1]$的链同伦等价。
\end{proposition}
这个结果是拓扑结论:$L \subset Cf$的映射锥同伦于$K$的双角锥的代数版本。
\begin{proposition}{}
	设$B \to C$是链复形态射。自然态射$\ker(f)[-1] \to \mathrm{cone}(f) \to \mathrm{coker}(f)$给出了长正合列。
\end{proposition}
\begin{proposition}{}
	设$C,C'$分别是分裂的复形,其中分裂映射为$s,s'$。设$f:C \to C'$是态射,则$\sigma(c,c')=(-s(c),s'(c')-s'fs(c))$给出了$\mathrm{cone}(f)$的一个分裂当且仅当$f_*$是一个零映射。
\end{proposition}
\section{Abel范畴拓展}
我们不介绍第6节的内容——以后用到再说。

\ifx\allfiles\undefined
	
	% 如果有这一部分的参考文献的话,在这里加上
	% 没有的话不需要
	% 因此各个部分的参考文献可以分开放置
	% 也可以统一放在主文件末尾。
	
	%  bibfile.bib是放置参考文献的文件,可以用zotero导出。
	% \bibliography{bibfile}
	
	end{document}
	\else
	\fi

\chapter{Morse理论的应用——测地线变分}
\section{道路的能量积分}
\section{指标定理}
\section{道路空间的伦型}

\ifx\allfiles\undefined

	% 如果有这一部分另外的package,在这里加上
	% 没有的话不需要
	\newcommand{\id}{\mathrm{id}}
\newcommand{\Hom}{\mathrm{Hom}}
\newcommand{\N}{\mathbb{N}}
\newcommand{\Z}{\mathbb{Z}}
\newcommand{\Q}{\mathbb{Q}}
\newcommand{\R}{\mathbb{R}}
\newcommand{\C}{\mathbb{C}}
\newcommand{\HH}{\mathbb{H}}
\newcommand{\RP}{\mathbb{RP}}
	\begin{document}
\else
\fi
\chapter{Tor函子和Ext函子}
本章的目的是介绍Tor函子和Ext函子的诸多性质。他们是同调代数初等应用中的常客。
\section{Abel群的Tor函子}
我们首先观察一个经典的PID上的模——Abel群的Tor函子。其实,Tor函子的名字就来源于其对Abel群的研究。

\begin{example}{}
    对于Abel群$B$而言,$\mathrm{Tor}_0^{\Z}(\Z/p,B)=B/pB$,$\mathrm{Tor}_1^\Z(\Z/p,B)={}_pB=\{b \in B:pB=0\}$.对于$n\geq 2$,$\mathrm{Tor}_2^\Z(\Z/p,B)=0$.

    上述结果可以这么看。取$\Z/p$的投射解消
    \begin{align}
        0 \to \Z \stackrel{p}{\rightarrow}\Z \to \Z/p \to 0
    \end{align}
    从而我们计算的是:
    \begin{align}
        0 \to B \stackrel{p}{\rightarrow} B \to 0
    \end{align}
    的同调群。
\end{example}
 特殊情况下,Tor函子表现出$1$阶挠子群,高阶为$0$的特点。实际上,我们有下面的命题:
 \begin{proposition}{}
    对于两个Abel群$A$,$B$,我们有:
    
    (a)$\mathrm{Tor}_1^\Z(A,B)$是一个挠群。

    (b)$\mathrm{Tor}_n^\Z(A,B)$在$n \geq 2$的情况下为$0$.
 \end{proposition}
 \begin{proof}
    证明依赖Tor函子与滤过余极限交换性。$A$是其有限生成子群的滤过余极限,所以$\mathrm{Tor}_n(A,B)$是$\mathrm{Tor}_n(A_\alpha,B)$的滤过余极限。

    Abel群的余极限总是他们直和的商子群。所以我们只需要证明对于有限生成子群上述命题成立即可。

    设$A=\Z^m \oplus \Z/p_1 \oplus \Z/p_2 \dots \Z/p_r$。因为$\Z^m$是投射的,所以只用考虑:
    \begin{align}
        \mathrm{Tor}_n(A,B)=\mathrm{Tor}_n(\Z/p_1,B)\oplus \mathrm{Tor}_n(\Z/p_2,B) \oplus \dots \mathrm{Tor}_n(\Z_r,B)
    \end{align}
    于是根据之前的例子我们知道结论成立。
 \end{proof}
 \begin{proposition}{}
    $\mathrm{Tor}_1^\Z(\Q/\Z,B)$是$B$的挠子群。
 \end{proposition}
 \begin{proof}
    可以想见,$\Z/p$提取出$B$中挠性为$p$的元素。$\Q/\Z$是其有限子群的滤过极限,并且每个有限子群都同构于某个$\Z/p$($p$不一定是素数。)
    \begin{align}
        \mathrm{Tor}_*^\Z(\Q/\Z,B)\cong \Colim \mathrm{Tor}_1^\Z(\Z/p,B)\cong \Colim({}_pB)=\cup_p\{b\in B:pb=0\}
    \end{align}

 \end{proof}
 \begin{proposition}{}
    如果$A$是一个无挠交换群,则$\mathrm{Tor}_n(A,B)$对于$n \neq 0$和Abel群$B$总是$0$。
 \end{proposition}
 \begin{proof}
    $A$是有限生成子群的滤过余极限。然而$A$无挠意味着这些有限生成子群都是自由群。用Tor保滤过余极限即可。
 \end{proof}
 如果$R$是交换环,则张量积有典范的同构,因此$\mathrm{Tor}_*(A,B)\cong \mathrm{Tor}_*(B,A)$.

 \begin{corollary}{}
    $\mathrm{Tor}_1^\Z(A,-)=0$等价于$A$无挠等价于$\mathrm{Tor}_1^\Z(-,A)=0$.
 \end{corollary}
 但是Tor函子并非对于所有环都有这么好的性质。比如下面的例子就说明在$R=\Z/m$的情况下可能失败:
 \begin{example}{}
   设$R=\Z/m$,$A=\Z/d$。其中$d|m$。从而$A$是$R$模。

   我们考虑$A$周期性的自由解消:
   \begin{align}
      \dots \to \Z/m \to Z/m \to \Z/m \to \Z/d
   \end{align}
   其中从$\Z/m$到$\Z/d$的映射是商映射,而$\Z/m$各自之间交替出现$d$和$m/d$。所以对于任何一个$\Z/m$模$B$,我们都有:
   \begin{align}
      \mathrm{Tor}_n^{\Z/m}(\Z/d,B)=\begin{cases}
      B/dB,n=0\\ \{b\in B:db=0\}/(m/d)B,n \text{是奇数}\\ \{b \in B:(m/d)b=0\}/dB,n \text{是偶数且}>0
      \end{cases}
   \end{align}
 \end{example}
 然而我们可以尝试对下面特殊的情况进行一些讨论。
 \begin{example}{}
   设$r$是$R$的一个左非零除子。即${}_rR=\{s \in R|rs=0\}$是$0$。对于每个$R$模$B$,记${}_rB=\{b \in B:rb=0\}$。用$R/rR$代替上述$\Z/p\Z$,用相同的计算办法可以算的:
   \begin{align}
      \mathrm{Tor}_0(R/rR,B)=B/rB;\quad \mathrm{Tor}_1^R(R/rR,B)={}_r B; \quad \mathrm{Tor}_n^R(R/rR,B)=0, n\geq 0
   \end{align}
 \end{example}
 \begin{proposition}{}
   若${}_r R\neq 0$,我们只能得到一个并非投射的解消:
   \begin{align}
      0 \to {}_r R \to R \stackrel{r}{\rightarrow} R \to R/rR \to 0
   \end{align}
   然而第二章我们介绍了dimension shelfting办法\ref{dim-Shifting}。所以我们对于$n \geq 3$,存在:
   \begin{align}
      \mathrm{Tor}_n^R(R/rR,B) \cong \mathrm{Tor}_{n-2}^R({}_r R,B)
   \end{align}

   其次,还有正合列:
   \begin{align}
      0 \to \mathrm{Tor}_2^R(R/rR,B) \to {}_rR \otimes B \to {}_rB \to \mathrm{Tor}_1^R(R/rR,B) \to 0
   \end{align}
   因为$\mathrm{Tor}_2^R(R/rR,B)$是$0 \to {}_rR\otimes B \to R\otimes B=B$的核。而该映射的像就在${}_r B$中,所以上述正合列中第一个和第二个已经确实成立。

   考虑$\mathrm{Tor}_1(R/rR,B)$。根据导引长正合列:
   \begin{align}
      0 \to \mathrm{Tor}_1(R/rR,B) \to rR\otimes B \to B \to B/rB
   \end{align}
   为了定义${}_r B \to \mathrm{Tor}_1(R/rR,B)$.我们定义${}_r B \to rR\otimes B$.即$b \mapsto r \otimes b$。则该映射实际上打进$\mathrm{Tor}_1(R/rR,B)$.

   若$\sum (rr_i)\otimes b_i \in \mathrm{Tor}_1(R/rR,B)$且在$B$中像为$\sum r(1\otimes r_ib_i)=0$,则${}_r B$中$\sum r_ib_i$的像是$\sum (rr_i)\otimes b_i$。于是我们定义了满射。

   最后需要说明${}_r B$处的正合。若$r \otimes b=0$,则存在$r_i$和$b_i$使得$rr_i=0$,$b=\sum r_ib_i$.
 \end{proposition}
 \begin{proposition}{}
   设$R$是交换整环,分式域$F$。则$\mathrm{Tor}_1^R(F/R,B)$是$B$的挠子群:$\{b \in B:(\exists r\neq 0)rb=0\}$
 \end{proposition}
 \begin{proposition}{}
   $\mathrm{Tor}_1^R(R/I,R/J) \cong \dfrac{I\cap J}{IJ}$对于任何右理想$I$和左理想$J$都成立。特别的,对于双边理想$I$:
   \begin{align}
      \mathrm{Tor}_1(R/I,R/I)\cong I/I^2
   \end{align}
 \end{proposition}
 \begin{proof}
   % https://q.uiver.app/#q=WzAsMTYsWzAsMSwiMCJdLFsxLDEsIklKIl0sWzIsMSwiSSJdLFszLDEsIklcXG90aW1lcyBSL0oiXSxbNCwxLCIwIl0sWzEsMiwiSiJdLFsyLDIsIlIiXSxbMywyLCJSXFxvdGltZXMgUi9KIl0sWzAsMiwiMCJdLFs0LDIsIjAiXSxbMSwwLCIwIl0sWzEsMywiSi8oSUopIl0sWzIsMCwiMCJdLFsyLDMsIlIvSSJdLFszLDAsIlxca2VyIGkiXSxbMywzLCJSL0kgXFxvdGltZXMgUi9KIl0sWzAsMV0sWzEsMl0sWzIsM10sWzMsNF0sWzEsNV0sWzIsNl0sWzMsNywiaVxcb3RpbWVzXFxtYXRocm17aWR9Il0sWzgsNV0sWzUsNl0sWzYsN10sWzcsOV0sWzEwLDFdLFs1LDExXSxbMTIsMl0sWzYsMTNdLFsxNCwzXSxbNywxNV0sWzEwLDEyXSxbMTIsMTRdLFsxNCwxMSwiIiwxLHsic3R5bGUiOnsiYm9keSI6eyJuYW1lIjoiZGFzaGVkIn19fV0sWzExLDEzXSxbMTMsMTVdXQ==
\[\begin{tikzcd}
	& 0 & 0 & {\ker i} \\
	0 & IJ & I & {I\otimes R/J} & 0 \\
	0 & J & R & {R\otimes R/J} & 0 \\
	& {J/(IJ)} & {R/I} & {R/I \otimes R/J}
	\arrow[from=2-1, to=2-2]
	\arrow[from=2-2, to=2-3]
	\arrow[from=2-3, to=2-4]
	\arrow[from=2-4, to=2-5]
	\arrow[from=2-2, to=3-2]
	\arrow[from=2-3, to=3-3]
	\arrow["{i\otimes\mathrm{id}}", from=2-4, to=3-4]
	\arrow[from=3-1, to=3-2]
	\arrow[from=3-2, to=3-3]
	\arrow[from=3-3, to=3-4]
	\arrow[from=3-4, to=3-5]
	\arrow[from=1-2, to=2-2]
	\arrow[from=3-2, to=4-2]
	\arrow[from=1-3, to=2-3]
	\arrow[from=3-3, to=4-3]
	\arrow[from=1-4, to=2-4]
	\arrow[from=3-4, to=4-4]
	\arrow[from=1-2, to=1-3]
	\arrow[from=1-3, to=1-4]
	\arrow[dashed, from=1-4, to=4-2]
	\arrow[from=4-2, to=4-3]
	\arrow[from=4-3, to=4-4]
\end{tikzcd}\]
上图是蛇形引理\ref{snake}.验证$I/(IJ)$和$I \otimes R/J$有典范同构可以得出第一行正合。第二行则典范正合。

 最右边的列是计算$\mathrm{Tor}_1(R/I,R/J)$的定义式。感觉Dimesion Shifting,$\ker i$是$\mathrm{Tor}_1(R/I,R/J)$。根据snake引理,$\ker i$是$J/(IJ)  \to R/I$的核:$\dfrac{I\cap J}{IJ}$。
 \end{proof}
\section{Tor函子与平坦性}
我们在这一节着重研究Tor函子的ayclic对象——平坦对象。
\begin{definition}[平坦模]{flat-module}
   称一个左$R$模是平坦模,若函子$\otimes_R B$是正合函子。同样,对于右$R$模,也可以定义类似的平坦性。
\end{definition}
如果$A$是投射的,则$\mathrm{Tor}_n(A,B)=0$。不难说明$A$此时是平坦的。因为投射模一定是平坦模。然而平坦模不一定是投射模。例如$\Q$作为交换群而言是平坦的,但不是投射的。(为什么?)
\begin{theorem}{}
   若$S$是$R$中的乘法封闭集,则$S^{-1}R$是一个平坦模。
\end{theorem}
这个定理当然很交换代数,不过影响不大,我们可以尝试证明:
\begin{proof}
   构造一个滤过范畴$I$。对象是$S$中的元素,态射$\Hom_I(s_1,s_2)=\{s \in S:s_1s=s_2\}$。定义函子$F:I \to R$。$F(s)=R$,$F(s_1 \to s_2)$则定义为$R$上该态射自然给出的右乘法。

  
我们断言$F$的余极限$\Colim F(s) \cong S^{-1}R$。从而因为$S^{-1}R$是平坦模的滤过余极限,所以其是平坦的。

   下面计算$\Colim F$。首先定义$F(s) \to S^{-1}R$的映射为$r \mapsto r/s$.这样交换图显然成立:
   % https://q.uiver.app/#q=WzAsMyxbMCwwLCJGKHNfMSk9UjpyIl0sWzEsMCwiRihzXzIpPVI6cnMiXSxbMCwxLCJTXnstMX1SOnIvc18xPXJzLyhzXzFzKT1ycy9zXzIiXSxbMCwxLCJzIl0sWzAsMl0sWzEsMl1d
\[\begin{tikzcd}
	{F(s_1)=R:r} & {F(s_2)=R:rs} \\
	{S^{-1}R:r/s_1=rs/(s_1s)=rs/s_2}
	\arrow["s", from=1-1, to=1-2]
	\arrow[from=1-1, to=2-1]
	\arrow[from=1-2, to=2-1]
\end{tikzcd}\]

如果存在一个新的$B$使得余极限中关系成立,我们直接定义$S^{-1}R$中的元素$r/s$到$B$的态射为$F(s)=R$中$r$在$B$中的像即可。这是唯一的定义方式!
\end{proof}
\begin{proposition}[Tor和平坦]
   下面三个命题等价:

   (1)$B$是平坦模。

   (2)$\mathrm{Tor}_n^R(A,B)=0,\forall n\neq 0$

   (3)$\mathrm{Tor}_1^R(A,B)=0$
\end{proposition}
\begin{corollary}
   若$0 \to A \to B \to C \to 0$是正合列且$B,C$是平坦模,则$A$平坦。
\end{corollary}
\begin{proposition}
   设$R$是主理想整环,则$B$平坦等价于$B$无挠。
\end{proposition}
对于上述命题,我们给出一个反例。首先平坦显然无挠。但是无挠不一定平坦。设$k$是域且$R=k[x,y]$。$R$是经典的非主理想整环。设$I=(x,y)R$。考虑$k=R/I$有投射解消:
\begin{align}
   0 \to R \to R^2 \to R \to k
\end{align}
其中第一个$R$到$R^2$为$[-y,x]$.而$R^2$到$R$为$(x,y)$.从而$\mathrm{Tor}_1^R(I,k)\cong \mathrm{Tor}_2^R(k,k)\cong k$。于是$I$不是平坦模。

我们深入的研究一下平坦模。
\begin{definition}[Pontrjagin对偶]{Pontrjagi}
   左模$B$的Pontrjagin对偶模$B^*$是一个右模:
   \begin{align}
      B^*:=\Hom_{\mathrm{Ab}}(B,\Q/\Z); (fr)(b)=f(rb)
   \end{align}
\end{definition}
\begin{proposition}{}
   下面的命题等价。

   (1)$B$平坦。

   (2)$B^*$内射。

   (3)$I\otimes_R B\cong IB=\{x_1b_1+\dots+x_nb_n\in B:x_i\in I,b_i\in B\}$对于任何右理想$I$都成立。

   (4)$\mathrm{Tor}_I^R(R/I,B)=0$对于任何右理想$I$都成立。
\end{proposition}
\begin{proof}
   (3)和(4)的等价性来源于正合列:
   \begin{align}
      0 \to \mathrm{Tor}_1(R/I,B) \to I\otimes B \to B \to B/IB \to 0
   \end{align}
   现在考虑$A'$是$A$的子模。考虑:
   \begin{align}
      \Hom(A,B^*) \to \Hom(A',B^*)
   \end{align}
   $B^*$等价于说上述映射是满射。根据伴随关系,我们有:
   \begin{align}
      \Hom(A\otimes B,\Q/\Z) \to \Hom(A'\otimes B,\Q/Z)
   \end{align}是满射。即$(A\otimes B)^* \to (A'\otimes B)^*$是满射。

   用下面的\textbf{引理},可以知道此时$A' \otimes B \to A\otimes B$是单射,所以$B$是平坦模。同理也可以反推回去。所以(1)(2)等价。另外带入$A'=I,A=R$,可以推出$I\otimes B \to R\otimes B$是单射。于是$I\otimes B\cong IB$且根据Baer判别法,这是可逆的。所以(1)(3)等价。
\end{proof}
我们描述一个引理。
\begin{lemma}{}
   $f:A' \to A$是单射等价于$f^*:A^* \to A'^*$是满射。
\end{lemma}
\begin{proof}
   因为$\Q/\Z$是内射的$\Z$模,所以保正合。
\end{proof}
\begin{proposition}[Pontrjagin对偶与正合]{}
   $A \to B \to C$是正合的当且仅当对偶$C^* \to B^* \to A^*$是正合的。
\end{proposition}
\begin{proof}
   因为$\Q/\Z$是内射模,所以$\Hom(-,\Q/\Z)$是正合函子,因此$C^* \to B^* \to A^*$是正合的。

   如果$C^* \to B^* \to A^*$正合,则$A \to B \to C$首先复形。若$b \in B$且在$C$中的像为$0$,我们证明$b$在$A$的像中。若不然,则$b+\mathrm{im}A$是$B/\mathrm{im}A$中的非$0$元。我们定义$g:B/\mathrm{im}A \to \Q/\Z$使得$g(b+\mathrm{im}A)\neq 0$。则$g$也给出了$B^*$中的非$0$元且在$A^*$中的像为$0$。

   所以可以给出一个$f \in C^*$。剩下的就是显然了。
\end{proof}
这个证明写的比较模糊。

我们邀请读者回忆有限展示的概念。然后不加证明的给出有限展示与生成元的选取无关.

\begin{proposition}{}
   若$\varphi:F \to M$是满射且$F$是有限生成的,$M$是有限展示的,则$\ker \varphi$是有限生成的。
\end{proposition}
HINT:用蛇形引理。

仍然用$A^*$表示$A$的Pontrjagin对偶,则存在一个自然的映射$\sigma:A^* \otimes_R M \to \Hom_R(M,A)^*$
\begin{align}
   \sigma(f\otimes m)=h \mapsto f(h(m))
\end{align}
其中$f\in A^*,m \in M,h \in \Hom(M,A)$.我们的问题是,什么时候$\sigma$是一个同构?
\begin{theorem}{}
   对于任何有限展示的$M$,$\sigma$都是一个同构。
\end{theorem}
\begin{proof}
   若$M=R$,则自然有$\sigma$是同构。根据可加性,$M=\R^n$的时候也是如此。所以有:
   % https://q.uiver.app/#q=WzAsOCxbMCwwLCJBXipcXG90aW1lcyBSXm0iXSxbMCwxLCJcXEhvbShSXm0sQSleKiJdLFsxLDAsIkFeKlxcb3RpbWVzIFJebiJdLFsyLDAsIkFeKlxcb3RpbWVzIE0iXSxbMywwLCIwIl0sWzMsMSwiMCJdLFsyLDEsIlxcSG9tKE0sQSleKiJdLFsxLDEsIlxcSG9tKFJebixBKV4qIl0sWzAsMV0sWzAsMl0sWzIsM10sWzMsNF0sWzYsNV0sWzcsNl0sWzEsN10sWzIsN10sWzMsNl1d
\[\begin{tikzcd}
	{A^*\otimes R^m} & {A^*\otimes R^n} & {A^*\otimes M} & 0 \\
	{\Hom(R^m,A)^*} & {\Hom(R^n,A)^*} & {\Hom(M,A)^*} & 0
	\arrow[from=1-1, to=2-1]
	\arrow[from=1-1, to=1-2]
	\arrow[from=1-2, to=1-3]
	\arrow[from=1-3, to=1-4]
	\arrow[from=2-3, to=2-4]
	\arrow[from=2-2, to=2-3]
	\arrow[from=2-1, to=2-2]
	\arrow[from=1-2, to=2-2]
	\arrow[from=1-3, to=2-3]
\end{tikzcd}\]
    因为$\otimes$是右正合的,$\Hom$是左正合的,所以图中两个行正合.根据5引理\ref{5lemma}可知$\sigma$是同构。
\end{proof}
\begin{theorem}{}
   每个有限展示的平坦模是投射模。
\end{theorem}
\begin{proof}
   我们证明$\Hom(M,-)$是正合的。设$B\to C$是满射,则$C^* \to B^*$是单射。若$M$是平坦的,则:
   % https://q.uiver.app/#q=WzAsNCxbMCwwLCJDXipcXG90aW1lc19SIE0iXSxbMSwwLCJCXipcXG90aW1lcyBNIl0sWzAsMSwiXFxIb20oTSxDKV4qIl0sWzEsMSwiXFxIb20oTSxCKV4qIl0sWzAsMV0sWzAsMiwiXFxzaWdtYSJdLFsxLDMsIlxcc2lnbWEiLDJdLFsyLDNdXQ==
\[\begin{tikzcd}
	{C^*\otimes_R M} & {B^*\otimes M} \\
	{\Hom(M,C)^*} & {\Hom(M,B)^*}
	\arrow[from=1-1, to=1-2]
	\arrow["\sigma", from=1-1, to=2-1]
	\arrow["\sigma"', from=1-2, to=2-2]
	\arrow[from=2-1, to=2-2]
\end{tikzcd}\]
   给出了$\Hom(M,B)\to Hom(M,C)$的满射。所以$M$是投射模。
  \end{proof}  
   下面的引理来源于dimension shifting.
   \begin{lemma}[平坦解消引理]{}
      群$\mathrm{Tor}_*(A,B)$可以用平坦模进行计算。
   \end{lemma}
\begin{proposition}[Tor的平坦基变换]
    设$R \to T$是环同态,使得$T$成为了$R$模。从而对于所有的$R$模$A$,所有的$T$模$C$和所有的$n$:
    \begin{align}
      \mathrm{Tor}_n^R(A,C)\cong \mathrm{Tor}_n^T(A \otimes_R T,C)
    \end{align}
\end{proposition}
\begin{proof}
   选择$R$模的投射解消$P \to A$,则$\mathrm{Tor}_*^R(A,C)$是$P \otimes_R C$的同调。

   因为$T$是平坦的$R$模,所以$P_n\otimes T$是投射的$T$模且$P\otimes T \to A \otimes T$是$T$模的投射解消。所以$\mathrm{Tor}_n^T(A \otimes T,C)$是复形$(P\otimes_R T)\otimes_T C \cong P\otimes_R C$的同调。
\end{proof}
\begin{corollary}{}
   若$R$是交换环,$T$是平坦的$R$代数,则对于所有的$R$模$A,B$和所有的$n$:
   \begin{align}
      T\otimes_R \mathrm{Tor}_n^R(A,B)\cong \mathrm{Tor}_n^T(A\otimes_R T,T\otimes_R B)
   \end{align}
\end{corollary}
\begin{proof}
   设$C=T\otimes_R B$.根据上面的命题,我们只需要证明$\mathrm{Tor}_*^R(A,T\otimes B)=T\otimes \mathrm{Tor}_*^R(A,B)$.因为$T\otimes_R$是正合函子,所以$T\otimes \mathrm{Tor}_*^R(A,B)$是$T\otimes_R (P\otimes _R B)$的同调,从而为$\mathrm{Tor}_*^R(A,T\otimes B)$.
\end{proof}
为了使得$\mathrm{Tor}$给出模结构,我们必须假设$R$是交换环。原因是下面的引理:

\begin{lemma}{}
   设$\mu:A \to A$是左乘一个中心元$r$。则诱导的$\mu_*:\mathrm{Tor}_n^R(A,B)\to \mathrm{Tor}_n^R(A,B)$也是左乘$r$.
\end{lemma}
\begin{proof}
   选择$A$的投射解消$P \to A$。左乘$r$是一个$R$模的链复形映射$\tilde{\mu}:P \to P$.(因为$r$是一个中心元)。从而$\tilde{mu}\otimes B$是$P\otimes B$的$r$左乘。作为商群$\mathrm{Tor}$也是如此。
\end{proof}
\begin{corollary}{}
   若$A$是一个$R/r$模,则对于每个$R$模$B$,$R$模$\mathrm{Tor}_*^R(A,B)$也是$R/r$模。换句话说,$rR$乘在该模得$0$.
\end{corollary}
\begin{corollary}[Tor的局部化]{}
   若$R$是一个交换环且$A,B$都是$R$模。下面的命题对于所有$n$都成立:
   \begin{enumerate}
      \item $\mathrm{Tor}_n^R(A,B)=0$
      \item 对于$R$的任意素理想$p$,$\mathrm{Tor}_n^{R_p}(A_p,B_p)=0$
      \item 对于$R$的任意极大理想$m$,$\mathrm{Tor}_n^{R_m}(A_m,B_m)=0$.
   \end{enumerate}
\end{corollary}
\begin{proof}
   对于$R$模而言,$M=0$等价于任意素理想$p$,$M_p=0$等价于任意极大理想$m$,$M_m=0=0$.设$M=\mathrm{Tor}(A,B)$:
   \begin{align}
      M_p=R_p \otimes_R M=\mathrm{Tor}_n^{R_p}(A_p,B_p)
   \end{align}
\end{proof}
\section{性质较好的环的Ext函子}
讨论了Ext后,我们讨论Ext函子的性质。首先我们计算一些性质很好的环的Ext函子。

\begin{lemma}{}
   $\mathrm{Ext}_\Z^n(A,B)=0$,$\forall n \geq 2$和所有的交换群$A,B$.
\end{lemma}
\begin{proof}
   把$B$嵌入到一个内射的交换群$I^0$.其商群$I^1$是可除的,因而是内射的,所以我们给出了$B$的内射解消$0 \to B \to I^0 \to I^1 \to 0$.

   所以$\mathrm{Ext}^*(A,B)$可以计算为:
   \begin{align}
      0 \to \Hom(A,I^0) \to \Hom(A,I^1) \to 0
   \end{align}
   的上同调。
\end{proof}
因此我们只需要考虑$n=1$的情况。
\begin{example}{}
   $\mathrm{Ext}_{\Z}^0(\Z/p,B)={}_p B$.$\mathrm{Ext}_\Z^1(\Z/p,B)=B/pB$.

   可以使用$0 \to \Z \to \Z \to \Z/p$作为$\Z/p$的投射解消计算。
\end{example}

因为$\Z$是投射模,所以$\Ext^1(\Z,B)=0$对于任何$B$总是成立。我们可以依据这个结果和上述结果,在$A$是有限生成的Abel群时计算$\Ext(A,B)$:
\begin{align}
   A\cong \Z^m \oplus \Z/p  \Rightarrow \Ext(A,B)=\Ext(\Z/p,B)
\end{align}
然而无限生成的情况因为余极限不交换,要复杂得多。
\begin{example}[$B=\Z$]{}
   设$A$是一个挠群,用$A^*$表示Pontrjagin对偶。$\Z$有经典的内射解消:$0 \to \Z \to \Q \to \Q/\Z \to 0$。用这个解消计算$\Ext^*(A,\Z)$:
   \begin{align}
      0 \to \Hom(A,\Q) \to \Hom(A,\Q/\Z) \to 0 
   \end{align}
   从而$\Ext_\Z^0(A,\Z)=\Hom(A,\Z)=0$,$\Ext_\Z^1(A,\Z)=A^*$。

   为了对这个例子有更深的印象,注意到$\Z_{p^\infty}$是$\Z/p^n$的余极限(并).于是可以计算:
   \begin{align}
      \Ext_\Z^1(\Z_{p^\infty},\Z)=(\Z_{p^\infty})^*
   \end{align}
   这个群是$p$-adic整数的无挠群,$\hat{\Z}_p=\Lim (\Z/p^n)$。

   再考虑一个例子:$A=\Z[1/p],B=\Z$.此时:
   \begin{align}
      0 \to \Q=\Hom(\Z[1/p],\Q) \to \Hom(\Z[1/p],\Q/\Z) \to 0
   \end{align}
   $\Ext^0$比较容易,我们考虑$\Ext^1$.此时给定$f \in \Hom(\Z[1/p],\Q/\Z)$,筛出掉$\Hom(\Z[1/p],\Q)$的元素,本质上留存的是一个$p$-adic数。并且若两个$p$-adic数只差一个整数,与他们给出的$f$是一致的。因此$\Ext^1(\Z[1/p],\Z)=\Z_{p^\infty}$。

   这说明$\Ext$对于平坦模而言也不是vanish的。
\end{example}
\begin{example}[$R=\Z/m$,$B=\Z/p$]{}
   $\Z/p$在这种情况下有无穷的周期内射解消:
   \begin{align}
      0 \to \Z/p \xrightarrow{\iota} \Z/m \xrightarrow{p}  \Z/m \xrightarrow{m/p} \Z/m \xrightarrow{p} \dots 
   \end{align}

   于是$\Ext_{\Z/m}^n(A,\Z/p)$可以计算为:
   \begin{align}
      0 \to \Hom(A,\Z/m) \to \Hom(A,\Z/m) \to \Hom(A,\Z/m) \dots
   \end{align}
   的上同调。

   比如,若$p^2|m$,则$\Ext_{\Z/m}^n(\Z/p,\Z/p)=\Z/p$
\end{example}
\begin{proposition}{}
   对于所有的$n$和$R$:
   \begin{enumerate}
      \item $\Ext_R^n(\bigoplus_\alpha A,B)\cong \prod_{\alpha}\Ext_R^n(A_\alpha,B)$
      \item $\Ext_R^n(A,\prod_\beta B) \cong \prod_\beta \Ext_R^n(A,B_\beta)$
   \end{enumerate}
\end{proposition}
\begin{proof}
   设$P_\alpha$是$A_\alpha$的投射解消。于是$\oplus P_\alpha$是$\oplus A_\alpha$的投射解消。同理,$Q_\beta$是$B_\beta$的内射解消,则$\prod Q_\beta$是$\prod B_\beta$的内射解消。

   根据$\Hom$的性质,再加上:
   \begin{align}
      H^*(\prod C_\gamma)\cong \prod H^*(C_\gamma)
   \end{align}
   可得结果。
\end{proof}
\begin{lemma}{}
   设$R$是交换环,则$\Hom_R(A,B)$和$\Ext^*(A,B)$都是$R$模。若$\mu,\tau$分别是$r$的左乘($A,B$),则诱导的$\mu^*$和$\tau^*$也是左乘。
\end{lemma}
可以看到,这是Tor函子的相似版本,可用于给出Ext与局部化交换的性质。
\begin{proof}
   给$P \to A$投射解消.左乘$r$给出了$\tilde{mu}:P \to P$作为链复形映射。映射$\Hom(\tilde{mu},B)$是$\Hom(P,B)$上链复形,是左乘$r$.

   因此商群$\Ext^n(A,B)$被$\mu^*$作用也是$r$左乘。
\end{proof}
\begin{corollary}{}
   设$R$是交换环,$A$是$R/r$模。则对于$R$模$B$,$\Ext^*_R(A,B)$是$R/r$模。
\end{corollary}
接下来的引理,定理我们不写证明,读者可自查Weibel原书。

考虑$S^{-1}\Hom_R(A,B)$.其到$\Hom_{S^{-1}R}(S^{-1}A,S^{-1}B)$有一个自然的态射$\Phi$。但这个态射一般不是同构。
\begin{lemma}{}
   如果$A$是有限展示的$R$模,则对于每个中心可乘集合$S$,$\Phi$是同构。
\end{lemma}
不难想象证明用到的是5引理\ref{5lemma}。

\begin{proposition}{}
   设$A$是交换Noether环上的有限生成模.则$\Phi$也诱导了Ext的同构:
   \begin{align}
      \Phi:S^{-1}\Ext_R^n(A,B) \cong \Ext_{S^{-1}R}^n(S^{-1}A,S^{-1}B)
   \end{align}
\end{proposition}
不难想到证明的思路是给$A$的投射解消。因为$S^{-1}$是正合函子,所以保$H^*$。因此用$\Hom$的同构性即可给出上述同构。

\begin{corollary}[Ext的局部化]{Ext-loc}
   设$R$是交换Noether环且$A$是有限生成$R$模.则下面的命题之间对于任意$B$和$n$都等价:
   \begin{enumerate}
      \item $\Ext_R^n(A,B)=0$
      \item 对于$R$的任何素理想$p$,$\Ext_{R_p}^n(A_p,B_p)=0$
      \item 对于$R$的任何极大理想$m$,$\Ext_{R_m}^n(A_m,B_m)=0$.
   \end{enumerate}
\end{corollary}
\section{Ext函子与扩张}
我们在这一节探讨Ext到底计算了什么。为此需要介绍扩张的概念。
\begin{definition}{extension}
   一个$A$过$B$的扩张$\xi$是指一个正合列$0 \to B \to X \to A \to 0$.称两个扩张$\xi,\xi'$是等价的,若存在交换图:
   % https://q.uiver.app/#q=WzAsMTAsWzAsMCwiMCJdLFsxLDAsIkEiXSxbMiwwLCJYIl0sWzMsMCwiQiJdLFs0LDAsIjAiXSxbMSwxLCJBIl0sWzIsMSwiWCciXSxbMywxLCJCIl0sWzQsMSwiMCJdLFswLDEsIjAiXSxbMCwxXSxbMSwyXSxbMiwzXSxbMyw0XSxbNSw2XSxbNiw3XSxbNyw4XSxbOSw1XSxbMSw1LCJcXGlkIiwxXSxbMyw3LCJcXGlkIiwxXSxbMiw2LCJcXGNvbmciXV0=
\[\begin{tikzcd}
	0 & A & X & B & 0 \\
	0 & A & {X'} & B & 0
	\arrow[from=1-1, to=1-2]
	\arrow[from=1-2, to=1-3]
	\arrow[from=1-3, to=1-4]
	\arrow[from=1-4, to=1-5]
	\arrow[from=2-2, to=2-3]
	\arrow[from=2-3, to=2-4]
	\arrow[from=2-4, to=2-5]
	\arrow[from=2-1, to=2-2]
	\arrow["\id"{description}, from=1-2, to=2-2]
	\arrow["\id"{description}, from=1-4, to=2-4]
	\arrow["\cong", from=1-3, to=2-3]
\end{tikzcd}\]

   一个扩张是分裂的,若其等价于$0 \to B \to A \oplus B \to 0$(典范的)。
\end{definition}
\begin{example}{}
   若$p$是素数,则仅存在$p$个等价的$\Z/p$过$\Z/p$的扩张。分别是分裂扩张和:
   \begin{align}
      0 \to \Z/p \xrightarrow{p} \Z/p^2 \xrightarrow{i}\Z/p \to 0, i=1,2,\dots,p-1
   \end{align}

   实际上$X$必须是$p^2$阶交换群。若$X$无$p^2$阶元,则根据$X=\Z/p\oplus \Z/p$。若$X$有$p^2$阶元,设该元为$b$。则$pb \in \Z/p=B$。于是有上述$p-1$种投射。
\end{example}
\begin{lemma}{}
   若$\Ext^1(A,B)=0$,则$A$过$B$的扩张总是分裂的。
\end{lemma}
\begin{proof}
   给定一个扩张$\xi$,根据$\Ext^*(A,-)$诱导的长正合列:
   \begin{align}
      \Hom(A,X) \to \Hom(A,A) \xrightarrow{\partial}\Ext^1(A,B)=0
   \end{align}
   所以$\id_A$有原像$\sigma:A \to X$。这就是一个$X \to A$的截面。所以$X=A \oplus B$分裂。
\end{proof}
如果$\Ext^1(A,B)$非$0$,为了给出截面,实际上可以计算$\partial(\id_A)=0$。我们把这个构造记作$\Theta(\xi)$.另外.如果两个扩张等价,那么他们的$\Theta(\xi)$相同.因此这个构造只依赖于$\xi$的等价类。

\begin{theorem}{}
   给定两个模$A,B$,映射$\Theta:\xi \mapsto \partial(\id_A)$给出了一个一一映射:
   \begin{align}
      \{\text{A过B的扩张的等价类}\} \to \Ext^1(A,B)
   \end{align}
\end{theorem}
因此这个定理给出了$\Ext^1(A,B)$的一个初步作用:确定$A$过$B$的扩张个数,并赋予一个群结构。
\begin{proof}
   对于$B$,固定一个正合列$0 \to B \to I \xrightarrow{\pi} N \to 0$.其中$I$内射。作用$\Hom(A,-)$,导出一个正合列:
   \begin{align}
      \Hom(A,I) \to \Hom(A,N) \xrightarrow{\partial} \Ext^1(A,B) \to 0
   \end{align}

   现在给定一个$x \in \Ext^1(A,B)$,选定$\beta \in \Hom(A,N)$使得$\partial(\beta)=x$.根据$\beta:A \to N$和$I \to N$,可以写出拉回$X$:
   % https://q.uiver.app/#q=WzAsMTAsWzAsMCwiMCJdLFsxLDAsIk0iXSxbMiwwLCJQIl0sWzMsMCwiQSJdLFs0LDAsIlxcYnVsbGV0Il0sWzAsMSwiMCJdLFsxLDEsIkIiXSxbMiwxLCJYIl0sWzMsMSwiQSJdLFs0LDEsIlxcYnVsbGV0Il0sWzAsMV0sWzIsM10sWzMsNF0sWzUsNl0sWzYsN10sWzcsOF0sWzgsOV0sWzEsNiwiXFxiZXRhIl0sWzIsN10sWzMsOCwiPSJdLFsxLDIsImoiXV0=
\[\begin{tikzcd}
	0 & B & X & A & 0 \\
	0 & B & I & N & 0
	\arrow[from=1-1, to=1-2]
	\arrow[from=1-3, to=1-4]
	\arrow[from=1-4, to=1-5]
	\arrow[from=2-1, to=2-2]
	\arrow[from=2-2, to=2-3]
	\arrow[from=2-3, to=2-4]
	\arrow[from=2-4, to=2-5]
	\arrow["{=}", from=1-2, to=2-2]
	\arrow[from=1-3, to=2-3]
	\arrow["\beta", from=1-4, to=2-4]
	\arrow[from=1-2, to=1-3]
\end{tikzcd}\]
这不仅是拉回,而且可以验证$0 \to B \to X \to A \to 0$是一个正合列。根据连接同态$\partial$的自然性,可以得到:
% https://q.uiver.app/#q=WzAsNCxbMCwwLCJcXEhvbShBLEEpIl0sWzEsMCwiXFxFeHReMShBLE0pIl0sWzAsMSwiXFxIb20oQSxBKSJdLFsxLDEsIlxcRXh0XjEoQSxCKSJdLFswLDFdLFswLDJdLFsyLDNdLFsxLDNdXQ==
\[\begin{tikzcd}
	{\Hom(A,A)} & {\Ext^1(A,B)} \\
	{\Hom(A,N)} & {\Ext^1(A,B)}
	\arrow[from=1-1, to=1-2]
	\arrow[from=1-1, to=2-1]
	\arrow[from=2-1, to=2-2]
	\arrow[from=1-2, to=2-2]
\end{tikzcd}\]
令上面的扩张是$\xi$,则$\Theta(\xi)=x$。于是我们通过给定$x\in \Ext^1(A,B)$给出一个扩张$\xi$使得$\Theta(\xi)=x$。

为了给出$\Ext^1(A,B)$到等价类的映射,我们还需要说明上述过程$\beta$的选取不改变$\xi$的等价类。实际上选取$\beta'\in \Hom(A,N)$使得$\partial{\beta'}=x$。于是$\beta'-\beta=\pi_*(\alpha),\alpha\in \Hom(A,I)$.于是可以绘制出下面的交换图:

% https://q.uiver.app/#q=WzAsNSxbMCwwLCJYIl0sWzEsMSwiWCciXSxbMiwxLCJBIl0sWzEsMiwiSSJdLFsyLDIsIk4iXSxbMCwxLCIiLDEseyJzdHlsZSI6eyJib2R5Ijp7Im5hbWUiOiJkYXNoZWQifX19XSxbMSwyLCJcXHNpZ21hJyIsMl0sWzAsMiwiXFxzaWdtYSIsMV0sWzEsMywicCciXSxbMyw0LCJcXHBpIiwyXSxbMiw0LCJcXGJldGEnIl0sWzAsMywicCtcXGFscGhhXFxjaXJjXFxzaWdtYSIsMl1d
\[\begin{tikzcd}
	X \\
	& {X'} & A \\
	& I & N
	\arrow[dashed, from=1-1, to=2-2]
	\arrow["{\sigma'}"', from=2-2, to=2-3]
	\arrow["\sigma"{description}, from=1-1, to=2-3]
	\arrow["{p'}", from=2-2, to=3-2]
	\arrow["\pi"', from=3-2, to=3-3]
	\arrow["{\beta'}", from=2-3, to=3-3]
	\arrow["{p+\alpha\circ\sigma}"', from=1-1, to=3-2]
\end{tikzcd}\](交换性已经在草稿纸上验证了)
根据拉回的泛性质,$X$到$X'$有一个态射.

通过具体到集合的验证,可以说明这是一个同构。所以$X$和$X'$是等价的扩张。

另一方面,给定$\xi$作为$A$过$B$的扩张,$I$的延拓性质表明存在一个$\tau:X \to I$满足:
% https://q.uiver.app/#q=WzAsMTAsWzAsMCwiMCJdLFsxLDAsIkIiXSxbMiwwLCJYIl0sWzMsMCwiQSJdLFs0LDAsIjAiXSxbMCwxLCIwIl0sWzEsMSwiQiJdLFsyLDEsIkkiXSxbMywxLCJOIl0sWzQsMSwiMCJdLFswLDFdLFsxLDJdLFsyLDNdLFszLDRdLFs1LDZdLFs2LDddLFs3LDhdLFs4LDldLFsyLDcsIlxcdGF1Il0sWzEsNiwiPSJdLFszLDgsIlxcYmV0YSIsMV1d
\[\begin{tikzcd}
	0 & B & X & A & 0 \\
	0 & B & I & N & 0
	\arrow[from=1-1, to=1-2]
	\arrow[from=1-2, to=1-3]
	\arrow[from=1-3, to=1-4]
	\arrow[from=1-4, to=1-5]
	\arrow[from=2-1, to=2-2]
	\arrow[from=2-2, to=2-3]
	\arrow[from=2-3, to=2-4]
	\arrow[from=2-4, to=2-5]
	\arrow["\tau", from=1-3, to=2-3]
	\arrow["{=}", from=1-2, to=2-2]
	\arrow["\beta"{description}, from=1-4, to=2-4]
\end{tikzcd}\]

其中$\beta$是$\tau$诱导的态射。我们断言$X$是$\beta$和$\pi:I \to N$的拉回。从而$\Psi(\Theta(\xi))=\xi$.
\end{proof}
如果我们可以给出扩张的运算,就能更好的理解上述的对应。
\begin{definition}[Baer和]{Baer-sum}
   设$\xi$和$\xi'$分别是$A$过$B$的两个扩张。设$X''$是$X \to A$和$X'  \to A$的拉回。则$X''$包含了三份$B$:$B \times 0,0 \times B,\{(-b,b):b\in B\}$。

   作$X''$对于对角线$B$的商运算,则$B \times 0$和$0 \times B$被对应为一个子群。而$X''/0\times B\cong X$和$X/B=A$,则我们得到正合列:
   \begin{align}
      \varphi: 0\to B \to Y \to A\to 0
   \end{align}
   $\varphi$的等价类被称为$\xi$和$\xi'$的Baer和。
\end{definition}
\begin{proposition}{Baer-sum-pro}
   扩张等价类的集合在Baer和的意义下生成了一个交换群,分裂扩张是该和的幺元。从而$\Theta$给出了一个群同构。
\end{proposition}
\begin{proof}
   我们说明$\Theta(\varphi)=\Theta(\xi)+\Theta(\xi')$.这说明了Baer和的良定性,也给出了命题成立。

   固定$0\to M \to P \to A\to 0$是一个正合列,且$P$是投射模。因为$P$投射,所以给出$\tau:P \to X$和$\tau':P\to X'$。
   
   接下来设$\tau'': P\to X''$是由$\tau:P \to X$和$\tau': P \to X'$诱导而来的态射。而设$\bar{\tau}:P \to Y$是诱导的态射。

   我们断言$\bar{\tau}$限制在$M$上由映射$\gamma+\gamma':M \to B$诱导。所以下面的交换图:
   % https://q.uiver.app/#q=WzAsMTAsWzAsMCwiMCJdLFsxLDAsIk0iXSxbMiwwLCJQIl0sWzMsMCwiQSJdLFs0LDAsIjAiXSxbMCwxLCIwIl0sWzQsMSwiMCJdLFsyLDEsIlkiXSxbMywxLCJBIl0sWzEsMSwiQiJdLFswLDFdLFszLDgsIj0iXSxbMyw0XSxbOCw2XSxbMiwzXSxbMSwyXSxbMiw3LCJcXGJhcntcXHRhdX0iXSxbNyw4XSxbOSw3XSxbMSw5LCJcXGdhbW1hK1xcZ2FtbWEnIl0sWzUsOV1d
\[\begin{tikzcd}
	0 & M & P & A & 0 \\
	0 & B & Y & A & 0
	\arrow[from=1-1, to=1-2]
	\arrow["{=}", from=1-4, to=2-4]
	\arrow[from=1-4, to=1-5]
	\arrow[from=2-4, to=2-5]
	\arrow[from=1-3, to=1-4]
	\arrow[from=1-2, to=1-3]
	\arrow["{\bar{\tau}}", from=1-3, to=2-3]
	\arrow[from=2-3, to=2-4]
	\arrow[from=2-2, to=2-3]
	\arrow["{\gamma+\gamma'}", from=1-2, to=2-2]
	\arrow[from=2-1, to=2-2]
\end{tikzcd}\]
成立。

因此我们有$\Theta(\varphi)=\partial(\gamma+\gamma')$.然而$\partial(\gamma+\gamma')=\partial(\gamma)+\partial(\gamma')=\Theta(\xi)+\Theta(\xi')$.所以命题成立。
\end{proof}

借助上述的命题,我们实际上可以思考这样的问题:如果一个Abelian范畴没有足够的投射模和内射模,我们也可以借助扩张生成的交换群来定义$\Ext^1$.当然这里的交换群仍需要证明。

相似的,我们也可以思考$\Ext^n$的含义。我们在这里建议大家阅读原书的79页到80页内容。
\section{逆向极限的导出函子}
设$I$是一个小范畴(即对象集和态射集都是集合)。$\mathcal{A}$是一个Abelian范畴。在第二章,我们说明了$\mathcal{A}^I$有足够多的内射对象。(至少是$A$完备且有足够多内射对象的时候)。另外,容易验证逆向极限是左正合函子(保核)。

因此我们可以定义从$\mathcal{A}^I$到$\mathcal{A}$的右导出函子$R^n\Lim_{i\in I}$。

我们在这一节关注$\mathcal{A}$是Ab且$I$是$\dots\to 2\to 1 \to 0$。我们把$\mathrm{Ab}^I$中的元素称作交换群的“塔”。他们的具体形式是:
\begin{align}
   \{A_i\}:\dots \to A_2\to A_1 \to A_0
\end{align}
这一节我们具体给出$\lim^1$的具体构造,并且证明$R^n\Lim=0,n\neq 0,1$。

我们自然想问这样的构造是否可以拓展为其他的Abelian范畴。Grothendieck告诉我们,在满足下面公理的情况下该范畴可以:

(AB$4^*$):$\mathcal{A}$是完备的,且任何集合的满射的乘积都是满射。

满足该公理的范畴大多是有underlying集合的范畴(交换群,模范畴,链复形范畴),但是在层范畴失效。

\begin{definition}{}
   给定Ab中的一个塔$\{A_i\}$。定义映射:
   \begin{align}
      \Delta:\prod_{i=0}^\infty \to \prod_{i=0}^\infty A_i
   \end{align}
   为:
   \begin{align}
      \Delta(\dots,a_i,\dots,a_0)=(\dots,a_i-\bar{a}_{i+1},\dots,a_1-\bar{a}_2,a_0-\bar{a}_1)
   \end{align}
   其中$\bar{a}_{i+1}$代表$a_{i+1}\in A_{i+1}$在$A_i$中的项。
   
   容易看出$\Delta$的$\ker$是$\Lim A_i$.我们定义$\Lim^1 A_i$是$\Delta$的余核,从而$\Lim^1$是从$\mathrm{Ab}^I$到$\mathrm{Ab}$的函子。我们定义$\Lim^0 A_i=\Lim A_i$,$\Lim^n A_i=0,n\geq 2$.
\end{definition}
上述定义给出了具体的构造。当然我们需要说明这是符合要求的函子。
\begin{lemma}{}
   函子$\{\Lim^n\}$给出了一个上同调$\delta$函子。
\end{lemma}
\begin{proof}
   设$0 \to \{A_i\} \to \{B_i\}\to \{C_i\} \to 0$是塔的一个短正合列。用蛇形引理:
   % https://q.uiver.app/#q=WzAsMTAsWzAsMCwiMCJdLFsxLDAsIlxccHJvZCBBX2kiXSxbMiwwLCJcXHByb2QgQl9pIl0sWzMsMCwiXFxwcm9kIENfaSJdLFs0LDAsIjAiXSxbMSwxLCJcXHByb2QgQV9pIl0sWzIsMSwiXFxwcm9kIEJfaSJdLFszLDEsIlxccHJvZCBDX2kiXSxbMCwxLCIwIl0sWzQsMSwiMCJdLFswLDFdLFsxLDJdLFsyLDNdLFszLDRdLFsxLDUsIlxcRGVsdGEiXSxbNSw2XSxbNiw3XSxbMyw3LCJcXERlbHRhIl0sWzIsNiwiXFxEZWx0YSJdLFs4LDVdLFs3LDldXQ==
\[\begin{tikzcd}
	0 & {\prod A_i} & {\prod B_i} & {\prod C_i} & 0 \\
	0 & {\prod A_i} & {\prod B_i} & {\prod C_i} & 0
	\arrow[from=1-1, to=1-2]
	\arrow[from=1-2, to=1-3]
	\arrow[from=1-3, to=1-4]
	\arrow[from=1-4, to=1-5]
	\arrow["\Delta", from=1-2, to=2-2]
	\arrow[from=2-2, to=2-3]
	\arrow[from=2-3, to=2-4]
	\arrow["\Delta", from=1-4, to=2-4]
	\arrow["\Delta", from=1-3, to=2-3]
	\arrow[from=2-1, to=2-2]
	\arrow[from=2-4, to=2-5]
\end{tikzcd}\]

就可以得到我们想要的自然长正合列。
\end{proof}
\begin{lemma}{}
   若所有的$A_{i+1}  \to A_i$都是满射,则$\Lim^1 A_i=0$.更多的,$\Lim A_i\neq 0$(除非每个$A_i$都是$0$),因为每个自然投射$\Lim A_i \to A_j$都是满射。
\end{lemma}
\begin{proof}
   给定$b_i \in A_i(i=0,\dots,n)$,以及任何$a_0\in A_0$。归纳的选择$a_{i+1}\in A_{i+1}$:使得$a_{i+1}$是$a_i-b_i  \in A_i$在$A_{i+1}$中的提升。

   从而$\Delta$将$(\dots,a_1,a_0)$映射到$(\dots,b_1,b_0)$.因此这种情况下$\Delta$是满射,$\Lim^1 A_i=0$。如果$b_i=0$,$(\dots,a_1,a_0)\in \Lim A_i$.
\end{proof}
\begin{corollary}{}
   $\Lim^1 A_i\cong (R^1\Lim)(A_i)$且$R^n \Lim=0,\forall n\neq 0,1$
\end{corollary}
\begin{proof}
   我们说明$\Lim^n$形成了一个泛$\delta$函子,从而根据泛性说明上述成立。我们只需要说明$\Lim^1$在足够多的内射对象(应付内射解消)上vanish。

   我们在第二章给出了足够多的内射对象:
   \begin{align}
      k_*E:\dots=E=E \to 0 \to 0 \dots\to 0
   \end{align}
   其中$E$内射。因此这里面所有的态射都是满射,因此$\Lim^1$在这些内射塔上都vanish。
\end{proof}
上述的证明在AB4*的情况下总是对的。我们给出反例(不满足AB4*)。
\begin{example}{}
   设$A_0=\Z$且$A_i=p^i\Z$是$p^i$生成的子群。对短正合列($p$是素数):
   \begin{align}
      0\to \{p^i\Z\}  \to \{\Z\} \to \{\Z/p^i\Z\}  \to 0
   \end{align}
   使用$\Lim$.

   从而$\Lim^1\{p^i\Z\}\cong\hat{\Z}_p/\Z$.

\end{example}
下面这个命题在原书上是习题。我们仅作记录,证明省略。(可以查找mathstackexchange)。

\begin{proposition}{}
   设$\{A_i\}$是一个塔,$A_{i+1}\to A_i$是包含映射。把$A=A_0$看作拓扑群,其中$a+A_i(a\in A,i\geq 0)$是开集。

   则$\Lim A_i=\cap A_i=0$当且仅当$A$是Hausdorff的.$\Lim^1 A_i=0$当且仅当$A$在下列意义是完备的:每个柯西列都有不一定唯一的极限点.
\end{proposition}
提示:证明$A$是完备的,当且仅当$A\cong \Lim(A/A_i)$
\begin{definition}{}
   我们称一个塔$\{A_i\}$满足Mittag-Leffler条件,若对于每个$k$都存在一个$j\geq k$使得$A_i \to A_k$的像等于$A_j\to A_k$,对于任意$i \geq j$成立。(即$A_i$在$A_k$的像满足降链条件)。

   例如,若$\{A_i\}$都是满射,该塔就满足M-L条件。

   有一种平凡的情况:若对于每个$k$都存在一个$j\geq k$使得$A_i \to A_k$的像是$0$,我们称该塔满足平凡M-L条件。
\end{definition}
\begin{proposition}{}
   若$A_i$满足M-L条件,则:$\Lim^1 A_i=0$
\end{proposition}
\begin{corollary}{}
   设$\{A_i\}$是有限Abel群的塔,或者是有限维向量空间上的塔,我们都有$\Lim^1 A_i=0$
\end{corollary}
下面的定理预示了下一节的泛系数定理。
\begin{theorem}{}
   设$\dots \to C_1 \to C_0$是Ab的链复形的塔链。(每个$C_i$都是链复形),且满足ML条件。设$C=\Colim C_i$。则对于每个$q$都存在一个正合列:
   \begin{align}
      0 \to \textstyle\Lim^1 H_{q+1}(C_i) \to H_q(C) \to \Lim H_q(C_i) \to 0
   \end{align}


若$\dots C_1\to C_0 \to 0$是上链复形的塔链且满足ML条件。则:
\begin{align}
   0 \to \textstyle\Lim^1 H^{q-1}(C_i) \to H^q(C) \to \Lim H^q(C_i) \to 0
\end{align}
正合。
\end{theorem}
在拓扑上,这个定理有一个类似的版本。考虑$X$是CW复形,而$X_i$是$X$的上升子复形链,使得$X=\cup X_i$.则存在一个正合列:
\begin{align}
   0 \to \textstyle\Lim^1 H^{q-1}(X_i) \to H^q(X) \to \Lim H^q(X_i) \to 0
\end{align}
可以一眼看出这个公式的便利之处:可以根据子群的同调群计算最大的群的同调群。
\begin{example}{}
   设$A$是$R$模且是子模$\dots \subset A_i \subset A_{i+1}\subset \dots$的并,则对于任何$R$模$B$和$q$,都存在列:
   \begin{align}
      0 \to \textstyle\Lim^1 \Ext_R^{q-1}(A_i,B) \to \Ext_R^q(A,B) \to \Lim \Ext_R^q(A_i,B)\to 0
   \end{align}
   是正合的。

   对于$\Z_{p^\infty}=\cup \Z/p^i$,上述列化为:
   \begin{align}
      0 \to \textstyle\Lim^1 \Hom(\Z/p^i,B) \to \Ext_R^1(\Z_{p^\infty},B) \to \Lim \Ext_R^1(\Z/p^i,B)=\hat{B}_p \to 0
   \end{align}
   其中$\hat{B}_p=\Lim(B/p^iB)$是$B$的$p$-adic的完备化。
   
   这相当于推广了计算:$\Ext^1_\Z(\Z_{p^\infty},\Z)\cong \hat{\Z}_p$.实际上,设$E$是一个不变的$B$内射解消,考虑上链复形的塔链:
   \begin{align}
      \Hom(A_{i+1},E) \to \Hom(A_i,E) \to \dots \Hom(A_0,E) 
   \end{align}
   因为每个$\Hom(-,E_n)$都是反变正合的,所以塔链中每一个映射都是满射。(单反过来就是满).而$\Hom(A_i,E)$的上同调是$\Ext^*(A_i,B)$,$\Ext^*(A,B)$是:
   \begin{align}
      \Hom(\cup A_i,E)=\Lim \Hom(A_i,E)
   \end{align}
   的上同调。
\end{example}
\begin{corollary}{}
   $Z[1/p]=\cup p^{-1}\Z$,从而$\Ext^1_\Z(\Z[1/p],\Z)\cong \hat{\Z}_p/\Z$.从而对于无挠群$B$,有$\Ext_\Z^1(\Q,B)=(\prod_p \hat{B}_p)/B$.
\end{corollary}

\section{泛系数定理}
这一节我们思考的问题是,在已知$P$的同调下,如何计算$P\otimes M$的同调。由于在拓扑中,有所谓$\Z$系数,$\R$系数,$R$系数的说法,所以我们实际上在思考不同系数情况下一个拓扑空间同调和上同调群的关系。因而这节的名字是泛系数定理。
\begin{theorem}[Kunneth公式]{Kunneth-formula}
   设$P$是由平坦右$R$模给出的链复形,且$d(P_n)$作为$P_{n-1}$的子模总是平坦的。则对于任何$n$和任何左模$M$,都存在正合列:
   \begin{align}
      0 \to H_n(P)\otimes_R M \to H_n(P\otimes_R M) \to \mathrm{Tor}_1^R(H_{n-1}(P),M)\to 0
   \end{align}
\end{theorem}
\begin{proof}
   考虑短正合列:
   \begin{align}
      0 \to Z_n \to P_n \to d(P_n)\to 0
   \end{align}
   对此使用$\Tor$函子,可以得知$Z_n$也是平坦模。考虑到$\Tor_1(d(P_n),M)=0$,则:
   \begin{align}
      0 \to Z_n \otimes M \to P_n\otimes M \to d(P_n)\otimes M \to 0
   \end{align}
   是正合的。从而我们给出了链复形的短正合列:
   \begin{align}
      0 \to Z\otimes M \to P\otimes M \to d(P)\otimes M \to 0
   \end{align}
   注意到$Z$和$d(P)$中的微分算子都是$0$,从而短正合列导引的长正合列为:
   \begin{align}
      H_{n+1}(dP\otimes M) \xrightarrow{\partial} H_n(Z\otimes M) \to H_n(P \otimes M) \to H_n(dP \otimes M) \xrightarrow{\partial}H_{n-1}(Z\otimes M)
   \end{align}
   其中$H_n(dP_n\otimes M)=dP_n \otimes M$,$H_n(Z_n\otimes M)=Z_n\otimes M$.

   设$i:d(P_{n+1}) \to Z_n$是包含映射。我们断言$\partial$实际上是$i\otimes M$。(实际上很容易给出)。另一方面,$0 \to d(P_{n+1}) \to Z_n \to H_n(P) \to 0$是$H_n(P)$的平坦解消,所以$\Tor_1(H_n(P),M)$可以使用:
   \begin{align}
      0 \to d(P_{n+1})\otimes Z_n\otimes M \to 0
   \end{align}
   计算。结合长正合列即可得到结果。
\end{proof}
\begin{theorem}[同调的泛系数定理]{homo-universal}
   设$P$是一个自由Abel群的链复形。则对于任意的$n$和每个交换群$M$而言,定理\ref{Kunneth-formula}中的正合列分裂。但是这个分裂并不典范。
   \begin{align}
      H_n(P\otimes M)\cong H_n(P)\otimes M \oplus \Tor_1^\Z(H_{n-1}(P),M)
   \end{align}
\end{theorem}
\begin{proof}
   众所周知,自由Abel群的子群还是自由的。考虑$d(P_n)$是$P_{n-1}$的子群,则$d(P_n)$是自由Abel群。不典范的,这说明:
   \begin{align}
      P_n=Z_n \oplus d(P_n)
   \end{align}
   从而$Z_n\otimes M$是$P_n\otimes M$的直和项,也是$\ker(d_n\otimes 1)$的直和项。

   商去$d_{n+1}\otimes 1$的像,我们有$H_n(P)\otimes M$是$H_n(P\otimes M)$的直和项。根据Kunneth公式可知另一个项是$\Tor_1^\Z(H_{n-1}(P),M)$.
\end{proof}
\begin{theorem}[复形的Kunneth公式]{Kunneth-formula-complex}
   设$P,Q$是右,左模链复形.如果$P$和$d(P)$都是平坦的,则存在正合列:
   \begin{align}
      0 \to \bigoplus_{p+q=n}H_p(P)\otimes H_q(Q) \to H_n(P\otimes Q) \to \bigoplus_{p+q=n-1}\Tor_1^R(H_p(P),H_q(Q)) \to 0
   \end{align}
\end{theorem}
\begin{proof}
   仿照定理\ref{Kunneth-formula}的证明,把$M$换成$Q$.
\end{proof}
为了节省时间,我们省略拓扑上的泛系数定理。

接下来我们攥写上同调版本的泛系数定理。
\begin{theorem}[上同调的泛系数定理]{cohomo-universal}
   设$P$是投射模给出的链复形,使得$d(P_n)$也是投射模。则对于每个$n$和$R$模$M$,存在一个非典范的分裂正合列:
   \begin{align}
      0 \to \Ext^1_R(H_{n-1}(P),M)\to H^n(\Hom_R(P,M)) \to \Hom_R(H_n(P),M)\to 0
   \end{align}
\end{theorem}
\begin{proof}
   因为$d(P_n)$投射,从而有非典范的分裂:$P_n=d(P_{n+1})\oplus Z_n$.从而$Z_n$也投射,并且有:
   \begin{align}
      0 \to \Hom(dP_{n+1},M) \to \Hom(P_n,M)\to \Hom(Z_n,M) \to 0
   \end{align}
   是正合的。所以$0 \to \Hom(dP,M) \to \Hom(P,M)\to \Hom(Z,M) \to 0$是链复形的正合列。导引的长正合列:
   \begin{align}
      H^{n-1}(\Hom(Z,M)) \xrightarrow{\partial} H^n(\Hom(dP,M)) \to H^n(\Hom(P,M)) \to H^n(\Hom(Z,M)) \xrightarrow{\partial} H^{n+1}(\Hom(dP,M))
   \end{align}
   注意到$dP$和$Z$的微分算子都是$0$,所以$\Hom(dP,M)$的微分也是$0$,因此$H^n(\Hom(dP,M))=\Hom(dP_n,M)$。同理$H^n(\Hom(Z,M))=\Hom(Z_n,M)$。并且这里的$\partial$右$d(P_{n+1})$到$Z_n$的嵌入给出。

   注意到$H_n(P)$有投射解消:
   \begin{align}
      0 \to d(P_{n+1}) \to Z_n \to H^n(P)
   \end{align}
   于是$\Ext^1(H_{n-1}(P),M)$和$\Hom(H_n(P),M)=\Ext^0(H_n(P),M)$都可以用:
   \begin{align}
      0 \to \Hom(Z_{n-1},M) \to \Hom(dP_n,M) \to 0
   \end{align}
   带入上面的长正合列即可得到正合结果。

   而分裂可依照\ref{homo-universal}的结果得出。
\end{proof}
\begin{example}{}
   设$X$是道路连通的,则$H_0(X)=\Z$,且$H^1(X;\Z)\cong \Hom(H_1(X),\Z)$.这是一个无挠的Abel群。($\Z$是投射模。)
\end{example}
\begin{theorem}[上双复形的泛系数定理]{}
   设$P$是一个链复形,$Q$是上链复形.
   
   则可以定义上双复形$\Hom(P,Q)$.用$H^*(\Hom(P,Q))$表示$\mathrm{Tot}(\Hom(P,Q))$的上同调。设$P_n$和$dP_n$总是投射的,则存在正合列:
   \begin{align}
      0 \to \prod_{p+q=n-1}\Ext^1_R(H_p(P),H^q(Q)) \to H^n(\Hom(P,Q)) \to \prod_{p+q=n}\Hom_R(H_p(P),H^q(Q)) \to 0
   \end{align}

\end{theorem}
最后我们给出右继承的概念以结束本节。一个环$R$称作右继承的,如果任何自由(右)模的子模都是投射(右)模。实际上,任何主理想整环都是继承环(他们都是交换的戴德金整环).

继承环这条良好的性质显然可以帮助我们把泛系数定理推广到任何继承环(直接的,主理想整环)。
 \ifx\allfiles\undefined
	
	% 如果有这一部分的参考文献的话,在这里加上
	% 没有的话不需要
	% 因此各个部分的参考文献可以分开放置
	% 也可以统一放在主文件末尾。
	
	%  bibfile.bib是放置参考文献的文件,可以用zotero导出。
	% \bibliography{bibfile}
	
	end{document}
	\else
	\fi
\chapter{Complex Manifold}

\chapter{K\"{a}hler Manifold}
\section{Difinition and K\"{a}hler Identity}
This section we introduce the basic concept of K\"{a}hler manifold and K\"{a}hler Identity.

Let $X$ be a complex manifold.We denote the induced almost complex structure by $I$. The following definition is natural.

\begin{definition}
A Riemann metric $g$ on $X$ is an hermitian structure on X if for any point $x \in X$,the scalar product $g_x$ on $T_x X$ is compatible with the almost complex structure $I_x$. That is, $g_x(I_x w,T_x v)=g_x(w,v)$ for any $x \in X,w,v \in T_x X$.

Then the induced form $\omega:=g(I(),())$ is called the fundamental form.
\end{definition}

\chapter{Vector Bundles}

\chapter{Applications of Cohomology}

\ifx\allfiles\undefined
	
	% 如果有这一部分的参考文献的话,在这里加上
	% 没有的话不需要
	% 因此各个部分的参考文献可以分开放置
	% 也可以统一放在主文件末尾。
	
	%  bibfile.bib是放置参考文献的文件,可以用zotero导出。
	% \bibliography{bibfile}
	
	\end{document}
	\else
	\fi
\ifx\allfiles\undefined

	% 如果有这一部分另外的package,在这里加上
	% 没有的话不需要
	
	\begin{document}
\else
\fi
\part{Morse理论}
设$f$是流形$M$上的光滑函数。我们定义:称一个点$p\in M$是$f$的临界点(critical point),若诱导映射$f_*:T_pM \to T_{f(p)}R$是$0$映射。

在流形上我们最好用各种各样的局部坐标讨论。设$(U;x_i,1\leq i \leq n)$是$p$附近的一个局部坐标系,则临界点的定义可以写为:
\begin{align}
	\pa{f}{x^1}=\pa{f}{x^2}=\dots=\pa{f}{x^n}=0
\end{align}

此时$f(p)$称为$f$的临界值。

在临界点处$f$的性质有着与非临界值完全不同的性质。Morse理论则是研究临界点处,$M$本身拓扑性质的改变的理论。
\ifx\allfiles\undefined

	% 如果有这一部分另外的package,在这里加上
	% 没有的话不需要
	
	\begin{document}
\else
\fi
\chapter{链复形}
\section{$R$-Mod上的链复形}
我们直接给出定义:
\begin{definition}{}
  一个$R$模上的链复形是一族$R$模$\{C_n\}$,与模同态$d_n:C_n \to C_{n-1}$,使得$d_n \circ d_{n-1}=0$.习惯上,我们把这些$d_n$称为微分(来源于微分拓扑),把$d_n$的核$\ker d_n$成为$C$的$n$圈,用$Z_n$表示。$d_{n+1}:C_{n+1}\to C_{n}$的像称为$C$的$n$边界,用$B_n$表示。

  显然$B_n \subset Z_n$($d_{n+1}\circ d_n=0$)。定义$C$的$n$阶同调模为$H_n(C)=Z_n/B_n$。
\end{definition}

实际上,存在范畴$Ch$($R$模下)。其对象为一般的链复形。态射$u:C \to D$定义为一族$R$模同态$u_n:C_n \to D_n$,使得与微分交换:$u_{n-1}d=du_{n}$。在这里,我们混用了$d$的记号,但是其意义并非是容易混淆的。请看交换图:

  \[\begin{tikzcd}
	{C_n} && {C_{n-1}} \\
	\\
	{D_n} && {D_{n-1}}
	\arrow["d", from=1-1, to=1-3]
	\arrow["{u_{n-1}}", from=1-3, to=3-3]
	\arrow["{u_n}"', from=1-1, to=3-1]
	\arrow["d"', from=3-1, to=3-3]
\end{tikzcd}\]
这条交换性质保证了下面的命题:
\begin{proposition}{}
  两个链复形之间的态射将圈映射到圈,将边界映射到边界。因此根据模同态定理,$u$诱导了映射$u:H_n(C) \to H_n(D)$。因此$H_n$是从$Ch$到$R$mod的函子。
\end{proposition}
\begin{proof}
  诱导模同态的证明略。要验证$H_n$是函子,需要说明其保$id$和复合。保id也是显然的,因此仅需要证明复合。对于$u,v$是链复形$C,D,E$之间的态射,我们自然有$(u \circ v)_n=u_n \circ v_n$。所以有诱导的映射满足复合关系。
\end{proof}
备注:这段证明说的比较含糊,但实际上是抽象代数的基本验证。我更建议读者自行验证这个命题。

\begin{example}{}
  考虑链复形$\{\mathrm{Hom}(A,C_n)\}$,其是$\Z$上的链复形。其中$A$是$R$模,$C_n$是已知的链复形的第$n$个模。假设$A=Z_n$($C$的第$n$阶圈),则若$H_n(\mathrm{Hom}(Z_n,C))=0$,则$H_n(C)=0$.

  当然我们需要验证其是一个链复形,以及给出其微分。这里省略。给定$a \in Z_n$,我们需要说明存在$b \in C_{n+1}$使得$db=a$。显然可以定义$f:Z_n \to C_n$使得$f(Ra)=Ra$。并且$d\circ f=0$。于是存在$g:Z_n \to C_{n+1}$满足$d \circ g=f$。于是定义$b=g(a)$,从而$db=f(a)=a$。
\end{example}

\begin{definition}{}
  态射$u:C \to D$被称之为拟同构态射,若其诱导的$H_n(C)\to H_n(D)$都是同构。
\end{definition}

把链复形的定义稍微倒错一下(把$C_n$写为$C^{-n}$),我们可以得到上链复形的概念:
\begin{definition}{}
  一个$R$模上的上链复形是指一族$R$模$\{C^n\}$,与模同态$d^n:C^n \to C^{n+1}$,使得$d^{n+1} \circ d^{n}=0$.习惯上,我们把这些$d^n$称为微分(来源于微分拓扑),把$d^n$的核$\ker d^n$成为$C$的$n$上圈,用$Z^n$表示。$d^{n-1}:C^{n-1}\to C^{n}$的像称为$C$的$n$上边界,用$B^n$表示。

  显然$B^n \subset Z^n$($d^{n+1}\circ d^n=0$)。定义$C$的$n$阶上同调模为$H^n(C)=Z^n/B^n$。

  上链复形的其他定义(态射,拟同构)与链复形一致。
\end{definition}

实际中,我们往往限制链复形和上链复形中不为$0$的模的指数(index)。对于一个链复形而言,若除有限个外其他$C_n$都是$0$,称之为有限链复形。如果对于$n>b(n<a)$有$C_n=0$,则称之为有上边界(下边界)链复形。

显然,有界(上有界,下有界)的链复形构成$Ch$的全子范畴。

上链复形有同样的定义。我们不多赘述。

接下来是一些代数拓扑计算同调群的例子。节省篇幅和时间,就不做记录了。
\section{链复形的运算}
我们希望在范畴论的角度下解释链复形,从而我们介绍Abelian范畴的定义。

\begin{definition}{}
一个范畴$\mathcal{A}$被称为$Ab$范畴,若其每个hom集$\mathrm{Hom}(A,B)$都被赋予了一个abel群的结构,使得态射的复合满足分配:$(f+f')\circ g=f \circ g+f'\circ g$。其中$f,f':B \to C$,$g:A \to B$。同理还有右分配。

显然,$Ch$范畴是一个$Ab$范畴。我们定义加法为模同态的相加$(f+g)_n=f_n+g_n$。

对于$Ab$范畴,我们可以定义加性函子$F:\mathcal{A} \to \mathcal{B}$,假如$F$给出了$\mathrm{Hom}(A,B)$到$\mathrm{Hom}(FA,FB)$的群同态。
\end{definition}
然而$Ab$范畴在范畴论上性质并不强。我们常用的许多结构:积,ker,coker都无法给出。所以我们给出下面的定义:
\begin{definition}{}
  一个可加范畴是指一个$Ab$范畴,外加拥有$0$对象和$A \times B$的积。从而在可加范畴上我们可以定义有限的积。

  显然$Ch$范畴也是一个可加范畴。
\end{definition}
\begin{proposition}
  直和与直积与同调操作交换。即$\oplus H_n(A_\alpha)\cong H_n(\oplus A_\alpha)$和$\otimes H_n(A_\alpha)\cong H_n(\otimes A_\alpha)$
\end{proposition}
\begin{proof}
链复形的直积,直和定义是自然的。对于直和而言,容易有$Z_n(\oplus A_\alpha)=\oplus Z_n(A_\alpha)$和$B_n(\oplus A_\alpha)=\oplus B_n(\oplus A_\alpha)$。因此同调群直接做商即可。
\end{proof}

\begin{definition}{}
  链复形$B$称为$C$的子复形,若$B_n \subset C_n$且$B$的微分算子是$C$微分算子在$B$上的限制。

  子复形的定义自然给出了商复形:
  \begin{align*}
    \dots \to C_{n+1}/B_{n+1} \to C_n/B_n \to C_{n-1}/B_{n-1} \to \dots
  \end{align*}
\end{definition}
\begin{proposition}
  若$f:B \to C$是链复形之间的映射。则可以良定义$\{ker f_n\}$是$B$的子复形。也可以良定义$\{\mathrm{coker} f_n\}$是$C$的商复形。
\end{proposition}
\begin{proof}
  显然。
\end{proof}
下面我们介绍一般可加范畴里面$\ker$和$\mathrm{coker}$的定义。
\begin{definition}{}
  对于态射$f:B \to C$,我们定义$\ker$为$i:A \to B$使得满足$fi=0$且任意$i':A' \to B$满足$fi'=0$,都有存在唯一的$g:A' \to A$使得$ig=i'$。定义$\mathrm{coker}f$为$p:C \to D$满足$pf=0$且使得任意满足$p'f=0$的$p':C \to D'$,存在唯一的$h:D \to D'$使得$p'=hp$。
\end{definition}
我们鼓励读者在这里使用交换图以描述核与余核的区别。

\begin{proposition}{}
  $\ker$是单态,$\mathrm{coker}f$是满态。
\end{proposition}
\begin{proof}
  仅对单态说明。读者可以自己尝试证明满态。回忆单态的定义是,对于$g:B \to C$,若任意$f,f':A \to B$满足$gf=gf'$,则$f=f'$。对于$\ker$而言,这一点由泛性质自然给出。
\end{proof}

在$R$模中,单态,单射,ker的概念是重合的。然而一般的范畴却不一定如此——ker是否能定义都是未决的问题。同样,在$R$模范畴中,满态,满射,coker的概念都是重合的,而一般的范畴则不一定。
\begin{proposition}{}
  对于$R$模上的$Ch$范畴,$f$是$A$到$B$的链映射。则之前定义的$\ker f$和$\mathrm{coker}f$确实是满足一般范畴定义的核和余核。
\end{proposition}
\begin{proof}
  套用定义,然后根据$R$模范畴中核,余核定义即可得到结果。
\end{proof}

定义核和余核是定义阿贝尔范畴的关键。
\begin{definition}{}
  称一个加性范畴是一个Abelian范畴,若其满足三个条件:

  1.每个态射都有核和余核。

  2.任何一个单态都是其余核态射的核。

  3.任何一个满态都是其核态射的余核。
\end{definition}
显然$R$模范畴是一个Abelian范畴。可以证明以下事实:
\begin{proposition}{}
  给定$\mathcal{A}$作为Abelian范畴,可以良定义态射的像$im(f)$。对于$f:B \to C$,我们有$\ker(\mathrm{coker}f)=\cong \mathrm{coker}(\ker f)$。因而定义$im(f)=\ker(\mathrm{coker}f)$
\end{proposition}
用交换图可以更加准确的描述命题里面的同构。
\[
  \begin{tikzcd}
	{\ker f} & B & C & {\mathrm{coker}f} \\
	& {\mathrm{coker}i} & {\ker p}
	\arrow["f", from=1-2, to=1-3]
	\arrow["i", from=1-1, to=1-2]
	\arrow["p", from=1-3, to=1-4]
	\arrow[from=1-2, to=2-2]
	\arrow[from=2-3, to=1-3]
	\arrow[dashed, from=2-2, to=2-3]
\end{tikzcd}\]
图中虚线的存在是因为泛性质。不妨考虑$fi=0$,所以根据$\mathrm{coker}$泛性质,存在唯一的$\mathrm{coker}i \to C$.由于$\mathrm{coker}i$是满态,所以构造的态射与$p$复合后是$0$。因此根据$\ker$泛性质,有虚线态射的存在。若此态射是同构,我们称这是一个严格的态射$f$。从而Abelian范畴有一个性质为:
\begin{proposition}{}
  Abelian范畴的态射都严格。
\end{proposition}
\begin{definition}{}

  1.称一个$\mathcal{A}$的列是正合列,若每个对象处都有$\ker=im$。

  2.考虑$\mathcal{B}$是$\mathcal{A}$的子Abelian范畴,若其本身是Abelian的,并且任何一个在$\mathcal{B}$的正合列,在$\mathcal{A}$中都正合。
\end{definition}

在Abelian范畴下可以讨论一般通过链复形的结构。因此我们给出了可加范畴$Ch(\mathcal{A})$。从而同调成为了从这个范畴到$\mathcal{A}$的函子。

\begin{theorem}{}
  $Ch(\mathcal{A})$是一个Abelian范畴。
\end{theorem}
\begin{proof}
  首先我们验证该范畴有核和余核。构造与$R$模范畴中的构造类似,因此留给读者做验证。(ker和coker的映射需要用泛性质给出)。

  如果$f:B \to C$是单态链映射,我们断言,$f$是单的,当且仅当$f_n$对于每个$n$都是单的。(可以构造一个非常简单的复形)。对于$f_n$,自然的$B_n$是$\ker (\mathrm{coker}f_n)$。我们断言$B\cong \ker (\{\mathrm{coker}f_n\})$。显然在$B_n$上类似。而对于微分算子:
\[
    \begin{tikzcd}
	{B_n} && {B_{n-1}} \\
	\\
	{\ker (\mathrm{coker} f_n)} && {\ker(\mathrm{coker}f_{n-1})}
	\arrow[from=1-1, to=1-3]
	\arrow[from=1-1, to=3-1]
	\arrow[from=1-3, to=3-3]
	\arrow[from=3-1, to=3-3]
\end{tikzcd}
\]
  这张图本身是交换的。因为同构$B_n\cong \ker (\mathrm{coker}f_n)$本身来自于$\ker$的泛性质(不妨自己研究一下)。

  满态的情况不予验证。
\end{proof}
\begin{proposition}{}
  链复形的正合(范畴论的定义)等价于每个列$0 \to A_n \to B_n \to C_n \to 0$都正和。
\end{proposition}
\begin{proof}
  考察链复形中的像。不妨考虑像定义为$\ker$。对于$0 \to A \to B \to C \to 0$,其中$i:A \to B$的像定义为$\ker (\mathrm{coker}i)$。同时,$\ker p$存在,并且根据泛性质(自行验证),存在$\ker (\mathrm{coker}i) \to \ker p$的态射。

  正合意味着这个态射是一个同构。不难发现这个态射本身为$\{\ker (\mathrm{coker}i_n) \to \ker p_n\}$。同构于是等价于每一个小的态射都是同构。
\end{proof}

接下来我们讨论双链复形。这一节会更多出现在谱序列的章节中,就之后记录。

链复形有着丰富的构造,这一节只做了最基本的描述。我们将在第五节见到更多构造。
\section{长正合列}
\begin{theorem}{}
  设$0 \to A \to B \to C \to 0$是链复形的正合列。那么存在一个自然的映射$\partial: H_n(C)\to H_{n-1}(A)$,称为连接同态,使得:
  \begin{align*}
    \dots \to H_{n+1}(C) \to H_n(A) \to H_n(B) \to H_n(C) \to H_{n-1}(A) \to \dots
  \end{align*}
  是一个正合列。

  同样,对于上链复形的正合列:$0 \to A \to B \to C  \to 0$,存在一个自然的$\partial: H^n(C)\to H^{n+1}(A)$和一个长正合列:
  \begin{align*}
    \dots \to H^{n-1}(C) \to H^n(A)\to H^n(B) \to H^n(C) \to H^{n+1}(A)
  \end{align*}
\end{theorem}
如果使用追图的技巧,这个定理的证明是相当容易的。(显得繁琐,但是没有思维难度)为此,我们试图介绍一些可能看起来不那么初等的证明。

\begin{proposition}{}
  对于链复形的正合列:$0 \to A \to B \to C \to 0$,如果其中有两个链复形是正合的,则第三个链复形也是正合的。
\end{proposition}
\begin{proof}
  在长正合列中考虑两个同调群为$0$。由正合性可以得到另外一个也是$0$。
\end{proof}

\begin{proposition}{}[33引理]
  给定交换图
  \[
    \centering
    \begin{tikzcd}
	& 0 & 0 & 0 \\
	0 & {A'} & {B'} & {C'} & 0 \\
	0 & A & B & C & 0 \\
	0 & {A''} & {B''} & {C''} & 0 \\
	& 0 & 0 & 0
	\arrow[from=1-2, to=2-2]
	\arrow[from=2-1, to=2-2]
	\arrow[from=3-1, to=3-2]
	\arrow[from=2-2, to=2-3]
	\arrow[from=3-2, to=3-3]
	\arrow[from=2-3, to=2-4]  
	\arrow[from=3-3, to=3-4]
	\arrow[from=4-2, to=4-3]
	\arrow[from=4-3, to=4-4]
	\arrow[from=4-1, to=4-2]
	\arrow[from=4-4, to=4-5]
	\arrow[from=3-4, to=3-5]
	\arrow[from=2-4, to=2-5]
	\arrow[from=1-3, to=2-3]
	\arrow[from=1-4, to=2-4]
	\arrow[from=2-2, to=3-2]
	\arrow[from=3-2, to=4-2]
	\arrow[from=4-2, to=5-2]
	\arrow[from=2-3, to=3-3]
	\arrow[from=3-3, to=4-3]
	\arrow[from=4-3, to=5-3]
	\arrow[from=2-4, to=3-4]
	\arrow[from=3-4, to=4-4]
	\arrow[from=4-4, to=5-4]
\end{tikzcd}
  \]
  这是一个在某个阿贝尔范畴的交换图,使得每一列都是正合的。则:

  1.若底部两行正合,则第一行也正合。

  2.若顶部两行正合,则第三行也正合。

  3.若顶部和底部两行正合,且复合$A \to C$是$0$,则中间一行也正合。
\end{proposition}
\begin{proof}
  追图即可。读者自证不难。或者也可以考虑上述定理,只需要说明这是链复形之间的映射。
\end{proof}

我们用蛇形引理证明上述的长正合列定理。然而我们不准备给出蛇形引理的证明。
\begin{lemma}[Snake]{snake}
  考虑交换图:
  \[
    \begin{tikzcd}
	& {\ker f} & {\ker g} & {\ker h} \\
	& {A'} & {B'} & {C'} & 0 \\
	0 & A & B & C \\
	& {\mathrm{coker} f} & {\mathrm{coker} g} & {\mathrm{coker} h}
	\arrow[from=2-2, to=2-3]
	\arrow[from=2-3, to=2-4]
	\arrow[from=2-4, to=2-5]
	\arrow["f", from=2-2, to=3-2]
	\arrow[from=3-1, to=3-2]
	\arrow["g", from=2-3, to=3-3]
	\arrow[from=3-3, to=3-4]
	\arrow["h", from=2-4, to=3-4]
	\arrow[from=3-2, to=3-3]
	\arrow[from=1-2, to=2-2]
	\arrow[from=1-3, to=2-3]
	\arrow[from=1-4, to=2-4]
	\arrow[from=3-2, to=4-2]
	\arrow[from=3-3, to=4-3]
	\arrow[from=3-4, to=4-4]
	\arrow[dashed, from=1-2, to=1-3]
	\arrow[dashed, from=1-3, to=1-4]
	\arrow[dashed, from=4-2, to=4-3]
	\arrow[dashed, from=4-3, to=4-4]
	\arrow[curve={height=24pt}, squiggly, from=1-4, to=4-2]
\end{tikzcd}
  \]
   如果图中实两行都是正合的,那么存在一个正合列:
   \begin{align*}
    \ker f \to \ker g \to \ker h \to \mathrm{coker}f \to \mathrm{coker} g \to \mathrm{coker} h
   \end{align*}
   其中$\ker h \to \mathrm{coker}f$需要构造。可以简单描述为:
   \begin{align*}
    \partial(c')=i^{-1}g p^{-1}(c') ,c' \in \ker h
   \end{align*}
\end{lemma}

蛇形引理在一般的abelian范畴里面也是成立的。原因是我们可以把一个小abelian范畴$\mathcal{A}$嵌入到$R$-mod范畴中。对于非小范畴$\mathcal{C}$,对于任何交换图,我们都可以找到小范畴包含这张交换图。所以在abelian范畴里面这也是成立的。

\begin{lemma}[5引理]{5lemma}
  考虑交换图
  \[
    \begin{tikzcd}
	{A'} & {B'} & {C'} & {D'} & {E'} \\
	A & B & C & D & E
	\arrow["a"', from=1-1, to=2-1]
	\arrow["b"', from=1-2, to=2-2]
	\arrow["c"', from=1-3, to=2-3]
	\arrow["d"', from=1-4, to=2-4]
	\arrow["e"', from=1-5, to=2-5]
	\arrow[from=1-1, to=1-2]
	\arrow[from=1-2, to=1-3]
	\arrow[from=1-3, to=1-4]
	\arrow[from=1-4, to=1-5]
	\arrow[from=2-1, to=2-2]
	\arrow[from=2-2, to=2-3]
	\arrow[from=2-3, to=2-4]
	\arrow[from=2-4, to=2-5]
\end{tikzcd}
  \]
  若$a,b,d,e$是同构,且行正合,则$c$也是同构。
\end{lemma}

现在我们讨论导引长正合列的办法。对于$0 \to A \to B \to C \to 0$是正合的链复形链。我们给出:
\[ \begin{tikzcd}
	& 0 & 0 & 0 \\
	0 & {Z_nA} & {Z_n B} & {Z_nC} \\
	0 & {A_n} & {B_n} & {C_n} & 0 \\
	0 & {A_{n-1}} & {B_{n-1}} & {C_{n-1}} & 0 \\
	& {A_{n-1}/dA_n} & {B_{n-1}/dB_n} & {C_{n-1}/dC_n} & 0 \\
	& 0 & 0 & 0
	\arrow[from=3-2, to=3-3]
	\arrow[from=3-3, to=3-4]
	\arrow[from=3-4, to=3-5]
	\arrow["d", from=3-2, to=4-2]
	\arrow[from=4-1, to=4-2]
	\arrow["d", from=3-3, to=4-3]
	\arrow[from=4-3, to=4-4]
	\arrow["d", from=3-4, to=4-4]
	\arrow[from=4-2, to=4-3]
	\arrow[from=2-2, to=3-2]
	\arrow[from=2-3, to=3-3]
	\arrow[from=2-4, to=3-4]
	\arrow[from=4-2, to=5-2]
	\arrow[from=4-3, to=5-3]
	\arrow[from=4-4, to=5-4]
	\arrow[from=5-2, to=6-2]
	\arrow[from=5-3, to=6-3]
	\arrow[from=5-4, to=6-4]
	\arrow[from=5-2, to=5-3]
	\arrow[from=5-3, to=5-4]
	\arrow[from=2-2, to=2-3]
	\arrow[from=2-3, to=2-4]
	\arrow[from=2-1, to=2-2]
	\arrow[from=1-3, to=2-3]
	\arrow[from=1-4, to=2-4]
	\arrow[from=1-2, to=2-2]
	\arrow[from=3-1, to=3-2]
	\arrow[from=5-4, to=5-5]
	\arrow[from=4-4, to=4-5]
\end{tikzcd}\]

以及:
\[
  \begin{tikzcd}
	& {A_n/dA_{n+1}} & {B _n/dB_{n+1}} & {C_n/dC_{n+1}} & 0 \\
	0 & {Z_{n-1}B} & {Z_{n-1}B} & {Z_{n-1}C}
	\arrow[from=1-2, to=1-3]
	\arrow[from=1-3, to=1-4]
	\arrow[from=1-4, to=1-5]
	\arrow[from=2-1, to=2-2]
	\arrow[from=2-2, to=2-3]
	\arrow[from=2-3, to=2-4]
	\arrow[from=1-2, to=2-2]
	\arrow[from=1-3, to=2-3]
	\arrow[from=1-4, to=2-4]
\end{tikzcd}\]


根据第一幅图我们知道第二幅图的第一列和第二列都是正合的。根据snake引理可知第二幅图给出了我们想要的长正合列。

当然也可以直接追图得到长正合列。这没有什么本质困难的东西。

最后我们说明长正合列中$\partial$是自然的。即对于两个正合链复形列之间的态射,我们能给出一个交换图。
\[
  % https://q.uiver.app/#q=WzAsMTAsWzAsMCwiMCJdLFsxLDAsIkEiXSxbMiwwLCJCIl0sWzMsMCwiQyJdLFs0LDAsIjAiXSxbNCwxLCIwIl0sWzMsMSwiQyciXSxbMiwxLCJCJyJdLFsxLDEsIkEnIl0sWzAsMSwiMCJdLFswLDFdLFsxLDJdLFsyLDNdLFszLDRdLFs5LDhdLFs4LDddLFs3LDZdLFs2LDVdLFsxLDhdLFsyLDddLFszLDZdXQ==
\begin{tikzcd}
	0 & A & B & C & 0 \\
	0 & {A'} & {B'} & {C'} & 0
	\arrow[from=1-1, to=1-2]
	\arrow[from=1-2, to=1-3]
	\arrow[from=1-3, to=1-4]
	\arrow[from=1-4, to=1-5]
	\arrow[from=2-1, to=2-2]
	\arrow[from=2-2, to=2-3]
	\arrow[from=2-3, to=2-4]
	\arrow[from=2-4, to=2-5]
	\arrow[from=1-2, to=2-2]
	\arrow[from=1-3, to=2-3]
	\arrow[from=1-4, to=2-4]
\end{tikzcd}\]
\begin{proposition}{}
  长正合列是一个从$\mathcal{S}$到$\mathcal{T}$的函子。其中$\mathcal{S}$是短链复形正合列范畴,$\mathcal{T}$是长正合列范畴。
\end{proposition}
\begin{proof}
  由于$H_n$是函子,所以我们只需要给出$\partial$的自然性。用嵌入定理可以只考虑$R$模范畴,从而追图。给定$z \in H_n(C)$,用$c$表示$z$在$Z_nC$的代表元。显然$z'$作为$H_n(C')$中$z$的像拥有$c'$作为代表元。

   用$b$表示$c$的一个原像,则$b'$作为$b$在$B'$中像是$c'$的一个原像。从而根据$\partial$的构造可以得到上图交换。
\end{proof}

\begin{proposition}{}
  正合列$0 \to Z \to C \to B[-1] \to 0$给出一个可以分裂为短正合列的长正合列。
\end{proposition}
\begin{proof}
  考虑$Z$的同调群为$Z_n$.而$B[-1]$的同调群是$B_{n-1}$。长正合列为$Z_n \to H_n(C) \to B_{n-1} \to Z_{n-1}$。其中$H_n(C) \to B_{n-1}$是$0$态射。
\end{proof}
\begin{proposition}{}
  作为链复形之间的态射$f$,若$\ker f$和$\mathrm{coker}f$零调,则$f$是一个拟同构。
\end{proposition}
\begin{proof}
  若ker和coker其中有一个平凡,则诱导的长正合列即可得到结果。

  现在考虑两个都不平凡。不妨考虑$\ker f \to A \to im(f) \to B \to \mathrm{coker} f$。其中前三个是短正合的,后三个也是短正合的。$\ker f$零调意味着$H_n(A)=H_n(im(f))$。$\mathrm{coker}f$零调意味着$H_n(im(f))=H_n(B)$

  余下要验证的是$f$确实可以如此分解。但是根据abelian范畴,上述两个同构的复合确实是$f_*$。
\end{proof}
\section{链同伦}
链同伦来自于代数拓扑。这是一个很好的概念(用于证明拟同构)。

\begin{definition}{}
  一个复形$C$称为分裂的,如果存在$s_n:C_n \to C_{n+1}$使得$dsd=s$.如果$C$还是零调的,则称$C$是分裂正合的。
\end{definition}
下面的例子说明零调的复形也可能不是分裂的。
\begin{example}{}
  $\to \Z/4 \to \Z/4 \to \Z/4 \to \dots$

  其中每个映射都是把元素乘$2$。这是一个零调的列(不管是作为$\Z$还是$\Z_4$模)。然而不是正合的,原因是不存在直和分解:$\Z_4 \cong \Z_2\oplus \Z_2$。
\end{example}
下面的命题涉及一些投射模的性质。不懂这个的读者可以先不看。
\begin{proposition}{}
  零调有下界,且均为自由模的链复形是分裂正合的。零调且均为自由生成abel群的链复形是分裂正合的。
\end{proposition}
\begin{proof}
  对于第一句话,考虑$R^k \to R^m \to R^n \to 0$。因为$R^n$是投射模,所以是$R^m$的直和项,于是有自然嵌入到$R^m$的映射。并且这个映射满足$dsd=d$。接着考虑$R^m$的直和项(分解$R^n$后剩下的),其也是投射模,并且$R^k$到其为满射,则其为$R^k$的直和项。定义$R^m$到$R^k$的映射为该直和打回去的映射。

  对于第二句话。因为$\Z$是主理想整环,所以有限生成自由模的子模是有限生成自由模。因此大家都是投射模。
\end{proof}

\begin{definition}{}
  对于链映射$f:C \to D$,称其为0伦的,若存在一族映射$s_n:C_n \to D_{n+1}$使得$f=ds+sd$。$\{s_n\}$称作$f$的链收缩。

  称$f,g:C \to D$是同伦的链映射,若$f-g$是零伦的。同伦等价的定义不再赘述。
\end{definition}
\begin{proposition}{}
  同伦的链映射诱导一样的同调群同态,因此同伦等价的链复形同调群同构。
\end{proposition}
\section{映射锥和映射柱}
设$f: B \to C$是链复形之间的映射。我们可以定义$f$的映射锥$\mathrm{cone}(f)$为一个新的链复形.

其$n$阶元为$B_{n-1}\oplus C_n$,微分定义为:
\begin{align*}
	d(b,c)=(-d(b),d(c)-f(b))
\end{align*}
我们省略验证$d \circ d$是一个复形的计算。同理,对于上链复形之间的映射$f: B \to C$,也可以定义$\mathrm{cone}(f)$.其第$n$阶元为$B^{n+1}\oplus C^n$,而微分:
\begin{align*}
	d(b,c)=(-db,dc-f(b))
\end{align*}
\begin{proposition}{}
	设$\mathrm{cone}(C)$定义为$C \to C$的恒同映射给出的映射柱。则$\mathrm{Cone}(C)$是分裂正合的。并且$s(b,c)=(-c,0)$是分裂映射。
\end{proposition}
\begin{proof}
	我们先考虑$0$调。实际上$d(c,c')=(-dc,dc'-c)$。因此$\ker d$满足$dc=0$且$dc'=c$.(自然可以只写为$dc'=c$)。所以自然有$d(-c',0)=(c,c')$,于是$\ker d =im(d)$所以零调。

	再考虑分裂正合。实际上$dsd(c,c')=ds(-dc,dc'-c)=d(c-dc',0)=(-dc,dc'-c)$。所以分裂正合成立。

\end{proof}

\begin{proposition}{}
	设$f$是$C$和$D$之间的链映射。$f$零伦当且仅当$f$可以延拓为$(-s,f):\mathrm{Cone}(C)\to D$的链映射。
\end{proposition}
\begin{proof}
	设$f$零伦,则$f=sd+ds$,$s:C_n \to D_{n+1}$。从而$(-s,f)(c,c')=-s(c)+f(c'),c \in C_n,c' \in C_{n+1}$.显然我们有$d(-s,f)(c,c')=-ds(c)+df(c')$,$(-s,f)d(c,c')=(-s,f)(-dc,dc'-c)=sdc+fdc'-c$.

	不难验证两个结果是相同的。

	另一方面,如果可以延拓为上述映射,则不难发现$f=ds+sd$。
\end{proof}

现在我们说明任何$f_*$都可以用下面的方式描述(导引长正合列)。让我们考虑短正合列:
\begin{align*}
	0 \to C \to \mathrm{cone}(f) \to B[-1] \to 0
\end{align*}
其中$c \mapsto (0,c)$,$(b,c) \mapsto -b$。这是正合列,所以导引长正合:
\begin{align*}
	\dots \to H_{n+1}(\mathrm{cone}(f)) \to H_n(B) \to H_n(C) \to H_n(\mathrm{cone} f) \to H_{n-1}(B) \to \dots
\end{align*}
其中连接同态$\partial$正是$H_n(B) \to H_n(C)$。因此下面命题就是自然的了。

\begin{proposition}{}
	$\partial=f_*$
\end{proposition}
\begin{proof}
	考虑$b \in B_n$是一个圈,那么$(-b,0)$在映射锥复形中提升了$b$。求一次微分,有$(db,fb)=(0,fb)$。于是:
	\begin{align*}
		\partial [b]=[fb]=f_*[b]
	\end{align*}
\end{proof}
\begin{corollary}{}
	$f:B \to C$是拟同构,当且仅当$\mathrm{cone}(f)$是正合的。因此我们把拟同构的问题化成了分裂的正合列的问题。
\end{corollary}

一个类似的构造是映射柱。我们用$\mathrm{cyl}(f)$表示。

对于$f:B \to C$,定义$\mathrm{cyl}(f)$的$n$阶元为$B_n \oplus B_{n-1}\oplus C_n$。定义微分:
\begin{align*}
	d(b,b',c)=(d(b)+b',-d(b'),d(c)-f(b'))
\end{align*}
我们最好用矩阵来描述:
\begin{align*}
	\begin{pmatrix}
	d_B & \mathrm{id}_B & 0\\
	0 & -d_B & 0\\
	0  & -f & d_C 
\end{pmatrix}
\end{align*}
通过计算该矩阵的平方,我们知道该微分满足$d^2=0$。

\begin{proposition}{}
	映射柱$\mathrm{cyl}(C)$表示恒同映射$\mathrm{id}_C$的映射柱。则$f,g$是$C \to D$的同伦的映射当且仅当存在$(f,s,g):\mathrm{cyl}(C) \to D$的链映射。
\end{proposition}
\begin{proof}
	这一点从拓扑上其实很好想。当然我们也可以计算一下交换性:
	\begin{align*}
		d(f,s,g)(c_1,c,c_2)=d(f(c_1)+s(c)+g(c_2))=df(c_1)+ds(c)+dg(c_2) \\
        (f,s,g)d(c_1,c,c_2)=(f,s,g)(dc_1+c,-dc,dc_2-c)=fd(c_1)+f(c)-sd(c)+gd(c_2)-g(c)
	\end{align*}
	相减得到:$ds(c)-sd(c)-f(c)+g(c)$。所以$f,g$同伦等价于说该式子等于$0$,也就意味着交换。
\end{proof}
\begin{lemma}{}
	由$(0,0,c)$生成的子复形同构于$C$。并且$\alpha:C \to \mathrm{cyl}(f)$是一个拟同构。
\end{lemma}
\begin{proof}
	这一条也是非常符合拓扑感受的(映射柱和底盘是同伦的)。首先,子复形是显然的。我们只需要说明存在这样一个正合
	\begin{align*}
		0 \to C \to \mathrm{cyl}(f) \to \mathrm{cone}(-\mathrm{id}_B) \to 0
	\end{align*}
	这个正合也是显然的。最后根据$\mathrm{cone}(-id_B)$的零调性,根据蛇形引理可得结果。
\end{proof}
事实上,这是一个链同伦等价。原因是我们定义$\beta(b,b',c)=f(b)+c$.则有:
\begin{align*}
	\alpha \beta(b,b',c)=(0,0,f(b)+c), \quad \beta(\alpha(c))=c
\end{align*}
只用说明第一个复合同伦于$\mathrm{id_{\mathrm{cyl(f)}}}$。 事实上,相减后得到$(0,0,f):\mathrm{cyl}(f) \to \mathrm{cyl}(f)$。定义$s(b,b',c)=(0,b,0)$.则容易得到$(0,0,f)=ds+sd$。

我们也可以用映射柱考量$f_*$。显然$(b,0,0)$生成的子复形同构于$B$,并且$\mathrm{cyl}(f)/B$同构于$\mathrm{cone}(f)$。

定义$B \to \mathrm{cyl}(f) \to C$,第二个映射是$\beta$。这个复合正好是$f$。所以$f_*$也分解开。我们可以构造下面的交换图:
\[\begin{tikzcd}
	&& C \\
	0 & B & {\mathrm{cyl}(f)} & {\mathrm{cone}(f)} & 0 \\
	& 0 & C & {\mathrm{cone}(f)} & {B[-1]} & 0
	\arrow[from=2-1, to=2-2]
	\arrow[from=2-2, to=2-3]
	\arrow[from=2-3, to=2-4]
	\arrow[from=2-4, to=2-5]
	\arrow["f", from=2-2, to=1-3]
	\arrow["\beta", from=2-3, to=1-3]
	\arrow["\alpha", from=3-3, to=2-3]
	\arrow[from=3-2, to=3-3]
	\arrow[from=3-3, to=3-4]
	\arrow["{=}", no head, from=2-4, to=3-4]
	\arrow["\delta", from=3-4, to=3-5]
	\arrow[from=3-5, to=3-6]
\end{tikzcd}\]
并且同调的长正合列满足下面的交换图:
\[\begin{tikzcd}
	{H_{n}(B)} & {H_n(\mathrm{cyl}f)} & {H_n(\mathrm{cone}(f))} & {H_{n-1}(B)} \\
	{H_{n+1}(B[-1])} & {H_n(C)} & {H_{n}(\mathrm{cone}(f))} & {H_{n}(B[-1])}
	\arrow[from=1-1, to=1-2]
	\arrow[from=1-2, to=1-3]
	\arrow["{-\partial}", from=1-3, to=1-4]
	\arrow[from=2-2, to=2-3]
	\arrow[from=2-1, to=2-2]
	\arrow["\delta", from=2-3, to=2-4]
	\arrow["{=}", no head, from=1-1, to=2-1]
	\arrow["{=}", no head, from=1-3, to=2-3]
	\arrow["{=}", no head, from=1-4, to=2-4]
	\arrow["{=}", no head, from=1-2, to=2-2]
	\arrow["{f_*}", from=1-1, to=2-2]
\end{tikzcd}\]
为什么交换?我们唯余验证最后一个方块(前面的交换可以直接由第一个图的交换给出)

设$(b,c)$是$\mathrm{cone}(f)$中的圈。因而根据定义有$db=0,f(b)=dc
$.将其提升到$(0,b,c)$,考虑:
\begin{align*}
	d(0,b,c)=(0+b,-db,dc-f(b))=(b,0,0)
\end{align*}
因此$\partial$将$(b,c)$的类映射到$b=-\delta(b,c)$的类。

映射柱和映射锥为我们提供了一个自然的方式于将任何一个链复形映射$f: B \to C$变为一个长正合列。为了说明这里的长正合列是良定的,我们需要说明一般的由$0 \to B \to C \to D \to 0$导引的长正合列与$f$,$g$给出的是一致的。

首先考虑$f$。对于$\mathrm{cone}(f)$而言,存在$\varphi:\mathrm{cone}(f)\to D$,满足$\varphi(b,c)=g(c)$。我们有下面的交换图:
\[\begin{tikzcd}
	& 0 & C & {\mathrm{cone}(f)} & {B[-1]} & 0 \\
	0 & B & {\mathrm{cyl}(f)} & {\mathrm{cone}(f)} & 0 \\
	0 & B & C & D & 0
	\arrow[from=1-2, to=1-3]
	\arrow[from=1-3, to=1-4]
	\arrow[from=1-4, to=1-5]
	\arrow[from=1-5, to=1-6]
	\arrow[from=2-1, to=2-2]
	\arrow[from=3-1, to=3-2]
	\arrow[from=2-2, to=2-3]
	\arrow[from=2-3, to=2-4]
	\arrow[from=2-4, to=2-5]
	\arrow[from=3-2, to=3-3]
	\arrow[from=3-3, to=3-4]
	\arrow[from=3-4, to=3-5]
	\arrow["{=}", no head, from=2-2, to=3-2]
	\arrow["\beta", from=2-3, to=3-3]
	\arrow["\alpha", from=1-3, to=2-3]
	\arrow["{=}", no head, from=1-4, to=2-4]
	\arrow["\varphi", from=2-4, to=3-4]
\end{tikzcd}\]
考虑$\beta$是一个拟同构,因此我们根据5引理可以知道$\varphi$也是一个拟同构。然而$\varphi$并不一定是一个链同伦。

\begin{example}{}
	考虑$B,C$是模,并且给出了一个只在$0$度非$0$的链复形。因此$\mathrm{cone}(f)$在$1$度的模是$B$,在$0$度的模是$C$。根据定义可知$d'(b)=-f(b)$。

	我们断言$\varphi$是链同伦等价,当且仅当$f:B \to C$是一个分裂的单射(换言之,$B$是$C$的直和项)

	实际上,$\varphi$只在$0$度的时候有非零的情况:$\varphi_0=g$。下面的交换图很直接:
	\[\begin{tikzcd}
	0 & B & C & 0 \\
	0 & 0 & D & 0 \\
	0 & B & C & 0
	\arrow[from=1-1, to=1-2]
	\arrow[from=1-3, to=1-4]
	\arrow["{-f}", from=1-2, to=1-3]
	\arrow[from=2-1, to=2-2]
	\arrow[from=2-2, to=2-3]
	\arrow[from=2-3, to=2-4]
	\arrow["g", from=1-3, to=2-3]
	\arrow[from=1-2, to=2-2]
	\arrow["h", from=2-3, to=3-3]
	\arrow[from=2-2, to=3-2]
	\arrow[from=3-1, to=3-2]
	\arrow["{-f}"', from=3-2, to=3-3]
	\arrow[from=3-3, to=3-4]
	\arrow["s"', from=1-3, to=3-2]
\end{tikzcd}\]
注意到$B \to 0\to B$的态射必须和$id$相差无几,所以$s \circ (-f)=-\mathrm{id}_B$,所以$B$内射进入$C$是分裂的。

反过来,如果有这样的分裂,则可以定义$h,s$是另外的投射。不难验证这是一个链同伦。

\end{example}

接下来我们需要验证导引的长正合列。即
% https://q.uiver.app/#q=WzAsMTIsWzEsMCwiSF9uKEIpIl0sWzAsMCwiXFxkb3RzIl0sWzEsMSwiSF9uKEIpIl0sWzAsMSwiXFxkb3RzIl0sWzIsMSwiSF9uKEMpIl0sWzMsMSwiSF9uKEQpIl0sWzQsMSwiSF97bi0xfShCKSJdLFs1LDEsIlxcZG90cyJdLFsyLDAsIkhfbihcXG1hdGhybXtjeWx9KGYpKSJdLFszLDAsIkhfbihcXG1hdGhybXtjb25lfShmKSkiXSxbNCwwLCJIX3tuLTF9KEIpIl0sWzUsMCwiXFxkb3RzIl0sWzEsMCwiXFxwYXJ0aWFsIl0sWzMsMiwiXFxwYXJ0aWFsIl0sWzIsNF0sWzQsNV0sWzYsN10sWzUsNiwiXFxwYXJ0aWFsIl0sWzAsOF0sWzgsOV0sWzksMTAsIlxccGFydGlhbCJdLFsxMCwxMV0sWzAsMiwiIiwxLHsic3R5bGUiOnsiaGVhZCI6eyJuYW1lIjoibm9uZSJ9fX1dLFs4LDQsIlxcY29uZyJdLFs5LDUsIlxcY29uZyJdLFsxMCw2LCJcXHNpbSJdXQ==
\[\begin{tikzcd}
	\dots & {H_n(B)} & {H_n(\mathrm{cyl}(f))} & {H_n(\mathrm{cone}(f))} & {H_{n-1}(B)} & \dots \\
	\dots & {H_n(B)} & {H_n(C)} & {H_n(D)} & {H_{n-1}(B)} & \dots
	\arrow["\partial", from=1-1, to=1-2]
	\arrow["\partial", from=2-1, to=2-2]
	\arrow[from=2-2, to=2-3]
	\arrow[from=2-3, to=2-4]
	\arrow[from=2-5, to=2-6]
	\arrow["\partial", from=2-4, to=2-5]
	\arrow[from=1-2, to=1-3]
	\arrow[from=1-3, to=1-4]
	\arrow["\partial", from=1-4, to=1-5]
	\arrow[from=1-5, to=1-6]
	\arrow[no head, from=1-2, to=2-2]
	\arrow["\cong", from=1-3, to=2-3]
	\arrow["\cong", from=1-4, to=2-4]
	\arrow["\sim", from=1-5, to=2-5]
\end{tikzcd}\]
\begin{proposition}{}
	复合$H_n(D) \cong H_n(\mathrm{cone}f)  \stackrel{-\delta_*}{\rightarrow} H_n(B[-1]) \cong H_{n-1}(B)$给出了$\partial$。
\end{proposition}
\begin{proof}
	取$g(c)$作为$D$中的$n$圈,用$(b,c)$代表其在$\mathrm{cone}(f)$中的像。(这意味着$db=0,dc=f(b)$)。于是$-\delta(b,c)=b$。

	另一方面,仍然考虑$g(c)$。则$dc=f(b)$且$b$是$\partial$的原像。
\end{proof}

同样的,我们也可以了类似的说明$B[-1]$和$\mathrm{cone}(g)$有一个拟同构,其对偶于$\varphi$。

\begin{proposition}{}
	给定$f:B \to C$。设$v$是$C$嵌入到$\mathrm{cone}(f)$的态射。那么存在一个$\mathrm{cone}(v)$到$B[-1]$的链同伦等价。
\end{proposition}
这个结果是拓扑结论:$L \subset Cf$的映射锥同伦于$K$的双角锥的代数版本。
\begin{proposition}{}
	设$B \to C$是链复形态射。自然态射$\ker(f)[-1] \to \mathrm{cone}(f) \to \mathrm{coker}(f)$给出了长正合列。
\end{proposition}
\begin{proposition}{}
	设$C,C'$分别是分裂的复形,其中分裂映射为$s,s'$。设$f:C \to C'$是态射,则$\sigma(c,c')=(-s(c),s'(c')-s'fs(c))$给出了$\mathrm{cone}(f)$的一个分裂当且仅当$f_*$是一个零映射。
\end{proposition}
\section{Abel范畴拓展}
我们不介绍第6节的内容——以后用到再说。

\ifx\allfiles\undefined
	
	% 如果有这一部分的参考文献的话,在这里加上
	% 没有的话不需要
	% 因此各个部分的参考文献可以分开放置
	% 也可以统一放在主文件末尾。
	
	%  bibfile.bib是放置参考文献的文件,可以用zotero导出。
	% \bibliography{bibfile}
	
	end{document}
	\else
	\fi

\chapter{Morse理论的应用——测地线变分}
\section{道路的能量积分}
\section{指标定理}
\section{道路空间的伦型}

\ifx\allfiles\undefined

	% 如果有这一部分另外的package,在这里加上
	% 没有的话不需要
	\newcommand{\id}{\mathrm{id}}
\newcommand{\Hom}{\mathrm{Hom}}
\newcommand{\N}{\mathbb{N}}
\newcommand{\Z}{\mathbb{Z}}
\newcommand{\Q}{\mathbb{Q}}
\newcommand{\R}{\mathbb{R}}
\newcommand{\C}{\mathbb{C}}
\newcommand{\HH}{\mathbb{H}}
\newcommand{\RP}{\mathbb{RP}}
	\begin{document}
\else
\fi
\chapter{Tor函子和Ext函子}
本章的目的是介绍Tor函子和Ext函子的诸多性质。他们是同调代数初等应用中的常客。
\section{Abel群的Tor函子}
我们首先观察一个经典的PID上的模——Abel群的Tor函子。其实,Tor函子的名字就来源于其对Abel群的研究。

\begin{example}{}
    对于Abel群$B$而言,$\mathrm{Tor}_0^{\Z}(\Z/p,B)=B/pB$,$\mathrm{Tor}_1^\Z(\Z/p,B)={}_pB=\{b \in B:pB=0\}$.对于$n\geq 2$,$\mathrm{Tor}_2^\Z(\Z/p,B)=0$.

    上述结果可以这么看。取$\Z/p$的投射解消
    \begin{align}
        0 \to \Z \stackrel{p}{\rightarrow}\Z \to \Z/p \to 0
    \end{align}
    从而我们计算的是:
    \begin{align}
        0 \to B \stackrel{p}{\rightarrow} B \to 0
    \end{align}
    的同调群。
\end{example}
 特殊情况下,Tor函子表现出$1$阶挠子群,高阶为$0$的特点。实际上,我们有下面的命题:
 \begin{proposition}{}
    对于两个Abel群$A$,$B$,我们有:
    
    (a)$\mathrm{Tor}_1^\Z(A,B)$是一个挠群。

    (b)$\mathrm{Tor}_n^\Z(A,B)$在$n \geq 2$的情况下为$0$.
 \end{proposition}
 \begin{proof}
    证明依赖Tor函子与滤过余极限交换性。$A$是其有限生成子群的滤过余极限,所以$\mathrm{Tor}_n(A,B)$是$\mathrm{Tor}_n(A_\alpha,B)$的滤过余极限。

    Abel群的余极限总是他们直和的商子群。所以我们只需要证明对于有限生成子群上述命题成立即可。

    设$A=\Z^m \oplus \Z/p_1 \oplus \Z/p_2 \dots \Z/p_r$。因为$\Z^m$是投射的,所以只用考虑:
    \begin{align}
        \mathrm{Tor}_n(A,B)=\mathrm{Tor}_n(\Z/p_1,B)\oplus \mathrm{Tor}_n(\Z/p_2,B) \oplus \dots \mathrm{Tor}_n(\Z_r,B)
    \end{align}
    于是根据之前的例子我们知道结论成立。
 \end{proof}
 \begin{proposition}{}
    $\mathrm{Tor}_1^\Z(\Q/\Z,B)$是$B$的挠子群。
 \end{proposition}
 \begin{proof}
    可以想见,$\Z/p$提取出$B$中挠性为$p$的元素。$\Q/\Z$是其有限子群的滤过极限,并且每个有限子群都同构于某个$\Z/p$($p$不一定是素数。)
    \begin{align}
        \mathrm{Tor}_*^\Z(\Q/\Z,B)\cong \Colim \mathrm{Tor}_1^\Z(\Z/p,B)\cong \Colim({}_pB)=\cup_p\{b\in B:pb=0\}
    \end{align}

 \end{proof}
 \begin{proposition}{}
    如果$A$是一个无挠交换群,则$\mathrm{Tor}_n(A,B)$对于$n \neq 0$和Abel群$B$总是$0$。
 \end{proposition}
 \begin{proof}
    $A$是有限生成子群的滤过余极限。然而$A$无挠意味着这些有限生成子群都是自由群。用Tor保滤过余极限即可。
 \end{proof}
 如果$R$是交换环,则张量积有典范的同构,因此$\mathrm{Tor}_*(A,B)\cong \mathrm{Tor}_*(B,A)$.

 \begin{corollary}{}
    $\mathrm{Tor}_1^\Z(A,-)=0$等价于$A$无挠等价于$\mathrm{Tor}_1^\Z(-,A)=0$.
 \end{corollary}
 但是Tor函子并非对于所有环都有这么好的性质。比如下面的例子就说明在$R=\Z/m$的情况下可能失败:
 \begin{example}{}
   设$R=\Z/m$,$A=\Z/d$。其中$d|m$。从而$A$是$R$模。

   我们考虑$A$周期性的自由解消:
   \begin{align}
      \dots \to \Z/m \to Z/m \to \Z/m \to \Z/d
   \end{align}
   其中从$\Z/m$到$\Z/d$的映射是商映射,而$\Z/m$各自之间交替出现$d$和$m/d$。所以对于任何一个$\Z/m$模$B$,我们都有:
   \begin{align}
      \mathrm{Tor}_n^{\Z/m}(\Z/d,B)=\begin{cases}
      B/dB,n=0\\ \{b\in B:db=0\}/(m/d)B,n \text{是奇数}\\ \{b \in B:(m/d)b=0\}/dB,n \text{是偶数且}>0
      \end{cases}
   \end{align}
 \end{example}
 然而我们可以尝试对下面特殊的情况进行一些讨论。
 \begin{example}{}
   设$r$是$R$的一个左非零除子。即${}_rR=\{s \in R|rs=0\}$是$0$。对于每个$R$模$B$,记${}_rB=\{b \in B:rb=0\}$。用$R/rR$代替上述$\Z/p\Z$,用相同的计算办法可以算的:
   \begin{align}
      \mathrm{Tor}_0(R/rR,B)=B/rB;\quad \mathrm{Tor}_1^R(R/rR,B)={}_r B; \quad \mathrm{Tor}_n^R(R/rR,B)=0, n\geq 0
   \end{align}
 \end{example}
 \begin{proposition}{}
   若${}_r R\neq 0$,我们只能得到一个并非投射的解消:
   \begin{align}
      0 \to {}_r R \to R \stackrel{r}{\rightarrow} R \to R/rR \to 0
   \end{align}
   然而第二章我们介绍了dimension shelfting办法\ref{dim-Shifting}。所以我们对于$n \geq 3$,存在:
   \begin{align}
      \mathrm{Tor}_n^R(R/rR,B) \cong \mathrm{Tor}_{n-2}^R({}_r R,B)
   \end{align}

   其次,还有正合列:
   \begin{align}
      0 \to \mathrm{Tor}_2^R(R/rR,B) \to {}_rR \otimes B \to {}_rB \to \mathrm{Tor}_1^R(R/rR,B) \to 0
   \end{align}
   因为$\mathrm{Tor}_2^R(R/rR,B)$是$0 \to {}_rR\otimes B \to R\otimes B=B$的核。而该映射的像就在${}_r B$中,所以上述正合列中第一个和第二个已经确实成立。

   考虑$\mathrm{Tor}_1(R/rR,B)$。根据导引长正合列:
   \begin{align}
      0 \to \mathrm{Tor}_1(R/rR,B) \to rR\otimes B \to B \to B/rB
   \end{align}
   为了定义${}_r B \to \mathrm{Tor}_1(R/rR,B)$.我们定义${}_r B \to rR\otimes B$.即$b \mapsto r \otimes b$。则该映射实际上打进$\mathrm{Tor}_1(R/rR,B)$.

   若$\sum (rr_i)\otimes b_i \in \mathrm{Tor}_1(R/rR,B)$且在$B$中像为$\sum r(1\otimes r_ib_i)=0$,则${}_r B$中$\sum r_ib_i$的像是$\sum (rr_i)\otimes b_i$。于是我们定义了满射。

   最后需要说明${}_r B$处的正合。若$r \otimes b=0$,则存在$r_i$和$b_i$使得$rr_i=0$,$b=\sum r_ib_i$.
 \end{proposition}
 \begin{proposition}{}
   设$R$是交换整环,分式域$F$。则$\mathrm{Tor}_1^R(F/R,B)$是$B$的挠子群:$\{b \in B:(\exists r\neq 0)rb=0\}$
 \end{proposition}
 \begin{proposition}{}
   $\mathrm{Tor}_1^R(R/I,R/J) \cong \dfrac{I\cap J}{IJ}$对于任何右理想$I$和左理想$J$都成立。特别的,对于双边理想$I$:
   \begin{align}
      \mathrm{Tor}_1(R/I,R/I)\cong I/I^2
   \end{align}
 \end{proposition}
 \begin{proof}
   % https://q.uiver.app/#q=WzAsMTYsWzAsMSwiMCJdLFsxLDEsIklKIl0sWzIsMSwiSSJdLFszLDEsIklcXG90aW1lcyBSL0oiXSxbNCwxLCIwIl0sWzEsMiwiSiJdLFsyLDIsIlIiXSxbMywyLCJSXFxvdGltZXMgUi9KIl0sWzAsMiwiMCJdLFs0LDIsIjAiXSxbMSwwLCIwIl0sWzEsMywiSi8oSUopIl0sWzIsMCwiMCJdLFsyLDMsIlIvSSJdLFszLDAsIlxca2VyIGkiXSxbMywzLCJSL0kgXFxvdGltZXMgUi9KIl0sWzAsMV0sWzEsMl0sWzIsM10sWzMsNF0sWzEsNV0sWzIsNl0sWzMsNywiaVxcb3RpbWVzXFxtYXRocm17aWR9Il0sWzgsNV0sWzUsNl0sWzYsN10sWzcsOV0sWzEwLDFdLFs1LDExXSxbMTIsMl0sWzYsMTNdLFsxNCwzXSxbNywxNV0sWzEwLDEyXSxbMTIsMTRdLFsxNCwxMSwiIiwxLHsic3R5bGUiOnsiYm9keSI6eyJuYW1lIjoiZGFzaGVkIn19fV0sWzExLDEzXSxbMTMsMTVdXQ==
\[\begin{tikzcd}
	& 0 & 0 & {\ker i} \\
	0 & IJ & I & {I\otimes R/J} & 0 \\
	0 & J & R & {R\otimes R/J} & 0 \\
	& {J/(IJ)} & {R/I} & {R/I \otimes R/J}
	\arrow[from=2-1, to=2-2]
	\arrow[from=2-2, to=2-3]
	\arrow[from=2-3, to=2-4]
	\arrow[from=2-4, to=2-5]
	\arrow[from=2-2, to=3-2]
	\arrow[from=2-3, to=3-3]
	\arrow["{i\otimes\mathrm{id}}", from=2-4, to=3-4]
	\arrow[from=3-1, to=3-2]
	\arrow[from=3-2, to=3-3]
	\arrow[from=3-3, to=3-4]
	\arrow[from=3-4, to=3-5]
	\arrow[from=1-2, to=2-2]
	\arrow[from=3-2, to=4-2]
	\arrow[from=1-3, to=2-3]
	\arrow[from=3-3, to=4-3]
	\arrow[from=1-4, to=2-4]
	\arrow[from=3-4, to=4-4]
	\arrow[from=1-2, to=1-3]
	\arrow[from=1-3, to=1-4]
	\arrow[dashed, from=1-4, to=4-2]
	\arrow[from=4-2, to=4-3]
	\arrow[from=4-3, to=4-4]
\end{tikzcd}\]
上图是蛇形引理\ref{snake}.验证$I/(IJ)$和$I \otimes R/J$有典范同构可以得出第一行正合。第二行则典范正合。

 最右边的列是计算$\mathrm{Tor}_1(R/I,R/J)$的定义式。感觉Dimesion Shifting,$\ker i$是$\mathrm{Tor}_1(R/I,R/J)$。根据snake引理,$\ker i$是$J/(IJ)  \to R/I$的核:$\dfrac{I\cap J}{IJ}$。
 \end{proof}
\section{Tor函子与平坦性}
我们在这一节着重研究Tor函子的ayclic对象——平坦对象。
\begin{definition}[平坦模]{flat-module}
   称一个左$R$模是平坦模,若函子$\otimes_R B$是正合函子。同样,对于右$R$模,也可以定义类似的平坦性。
\end{definition}
如果$A$是投射的,则$\mathrm{Tor}_n(A,B)=0$。不难说明$A$此时是平坦的。因为投射模一定是平坦模。然而平坦模不一定是投射模。例如$\Q$作为交换群而言是平坦的,但不是投射的。(为什么?)
\begin{theorem}{}
   若$S$是$R$中的乘法封闭集,则$S^{-1}R$是一个平坦模。
\end{theorem}
这个定理当然很交换代数,不过影响不大,我们可以尝试证明:
\begin{proof}
   构造一个滤过范畴$I$。对象是$S$中的元素,态射$\Hom_I(s_1,s_2)=\{s \in S:s_1s=s_2\}$。定义函子$F:I \to R$。$F(s)=R$,$F(s_1 \to s_2)$则定义为$R$上该态射自然给出的右乘法。

  
我们断言$F$的余极限$\Colim F(s) \cong S^{-1}R$。从而因为$S^{-1}R$是平坦模的滤过余极限,所以其是平坦的。

   下面计算$\Colim F$。首先定义$F(s) \to S^{-1}R$的映射为$r \mapsto r/s$.这样交换图显然成立:
   % https://q.uiver.app/#q=WzAsMyxbMCwwLCJGKHNfMSk9UjpyIl0sWzEsMCwiRihzXzIpPVI6cnMiXSxbMCwxLCJTXnstMX1SOnIvc18xPXJzLyhzXzFzKT1ycy9zXzIiXSxbMCwxLCJzIl0sWzAsMl0sWzEsMl1d
\[\begin{tikzcd}
	{F(s_1)=R:r} & {F(s_2)=R:rs} \\
	{S^{-1}R:r/s_1=rs/(s_1s)=rs/s_2}
	\arrow["s", from=1-1, to=1-2]
	\arrow[from=1-1, to=2-1]
	\arrow[from=1-2, to=2-1]
\end{tikzcd}\]

如果存在一个新的$B$使得余极限中关系成立,我们直接定义$S^{-1}R$中的元素$r/s$到$B$的态射为$F(s)=R$中$r$在$B$中的像即可。这是唯一的定义方式!
\end{proof}
\begin{proposition}[Tor和平坦]
   下面三个命题等价:

   (1)$B$是平坦模。

   (2)$\mathrm{Tor}_n^R(A,B)=0,\forall n\neq 0$

   (3)$\mathrm{Tor}_1^R(A,B)=0$
\end{proposition}
\begin{corollary}
   若$0 \to A \to B \to C \to 0$是正合列且$B,C$是平坦模,则$A$平坦。
\end{corollary}
\begin{proposition}
   设$R$是主理想整环,则$B$平坦等价于$B$无挠。
\end{proposition}
对于上述命题,我们给出一个反例。首先平坦显然无挠。但是无挠不一定平坦。设$k$是域且$R=k[x,y]$。$R$是经典的非主理想整环。设$I=(x,y)R$。考虑$k=R/I$有投射解消:
\begin{align}
   0 \to R \to R^2 \to R \to k
\end{align}
其中第一个$R$到$R^2$为$[-y,x]$.而$R^2$到$R$为$(x,y)$.从而$\mathrm{Tor}_1^R(I,k)\cong \mathrm{Tor}_2^R(k,k)\cong k$。于是$I$不是平坦模。

我们深入的研究一下平坦模。
\begin{definition}[Pontrjagin对偶]{Pontrjagi}
   左模$B$的Pontrjagin对偶模$B^*$是一个右模:
   \begin{align}
      B^*:=\Hom_{\mathrm{Ab}}(B,\Q/\Z); (fr)(b)=f(rb)
   \end{align}
\end{definition}
\begin{proposition}{}
   下面的命题等价。

   (1)$B$平坦。

   (2)$B^*$内射。

   (3)$I\otimes_R B\cong IB=\{x_1b_1+\dots+x_nb_n\in B:x_i\in I,b_i\in B\}$对于任何右理想$I$都成立。

   (4)$\mathrm{Tor}_I^R(R/I,B)=0$对于任何右理想$I$都成立。
\end{proposition}
\begin{proof}
   (3)和(4)的等价性来源于正合列:
   \begin{align}
      0 \to \mathrm{Tor}_1(R/I,B) \to I\otimes B \to B \to B/IB \to 0
   \end{align}
   现在考虑$A'$是$A$的子模。考虑:
   \begin{align}
      \Hom(A,B^*) \to \Hom(A',B^*)
   \end{align}
   $B^*$等价于说上述映射是满射。根据伴随关系,我们有:
   \begin{align}
      \Hom(A\otimes B,\Q/\Z) \to \Hom(A'\otimes B,\Q/Z)
   \end{align}是满射。即$(A\otimes B)^* \to (A'\otimes B)^*$是满射。

   用下面的\textbf{引理},可以知道此时$A' \otimes B \to A\otimes B$是单射,所以$B$是平坦模。同理也可以反推回去。所以(1)(2)等价。另外带入$A'=I,A=R$,可以推出$I\otimes B \to R\otimes B$是单射。于是$I\otimes B\cong IB$且根据Baer判别法,这是可逆的。所以(1)(3)等价。
\end{proof}
我们描述一个引理。
\begin{lemma}{}
   $f:A' \to A$是单射等价于$f^*:A^* \to A'^*$是满射。
\end{lemma}
\begin{proof}
   因为$\Q/\Z$是内射的$\Z$模,所以保正合。
\end{proof}
\begin{proposition}[Pontrjagin对偶与正合]{}
   $A \to B \to C$是正合的当且仅当对偶$C^* \to B^* \to A^*$是正合的。
\end{proposition}
\begin{proof}
   因为$\Q/\Z$是内射模,所以$\Hom(-,\Q/\Z)$是正合函子,因此$C^* \to B^* \to A^*$是正合的。

   如果$C^* \to B^* \to A^*$正合,则$A \to B \to C$首先复形。若$b \in B$且在$C$中的像为$0$,我们证明$b$在$A$的像中。若不然,则$b+\mathrm{im}A$是$B/\mathrm{im}A$中的非$0$元。我们定义$g:B/\mathrm{im}A \to \Q/\Z$使得$g(b+\mathrm{im}A)\neq 0$。则$g$也给出了$B^*$中的非$0$元且在$A^*$中的像为$0$。

   所以可以给出一个$f \in C^*$。剩下的就是显然了。
\end{proof}
这个证明写的比较模糊。

我们邀请读者回忆有限展示的概念。然后不加证明的给出有限展示与生成元的选取无关.

\begin{proposition}{}
   若$\varphi:F \to M$是满射且$F$是有限生成的,$M$是有限展示的,则$\ker \varphi$是有限生成的。
\end{proposition}
HINT:用蛇形引理。

仍然用$A^*$表示$A$的Pontrjagin对偶,则存在一个自然的映射$\sigma:A^* \otimes_R M \to \Hom_R(M,A)^*$
\begin{align}
   \sigma(f\otimes m)=h \mapsto f(h(m))
\end{align}
其中$f\in A^*,m \in M,h \in \Hom(M,A)$.我们的问题是,什么时候$\sigma$是一个同构?
\begin{theorem}{}
   对于任何有限展示的$M$,$\sigma$都是一个同构。
\end{theorem}
\begin{proof}
   若$M=R$,则自然有$\sigma$是同构。根据可加性,$M=\R^n$的时候也是如此。所以有:
   % https://q.uiver.app/#q=WzAsOCxbMCwwLCJBXipcXG90aW1lcyBSXm0iXSxbMCwxLCJcXEhvbShSXm0sQSleKiJdLFsxLDAsIkFeKlxcb3RpbWVzIFJebiJdLFsyLDAsIkFeKlxcb3RpbWVzIE0iXSxbMywwLCIwIl0sWzMsMSwiMCJdLFsyLDEsIlxcSG9tKE0sQSleKiJdLFsxLDEsIlxcSG9tKFJebixBKV4qIl0sWzAsMV0sWzAsMl0sWzIsM10sWzMsNF0sWzYsNV0sWzcsNl0sWzEsN10sWzIsN10sWzMsNl1d
\[\begin{tikzcd}
	{A^*\otimes R^m} & {A^*\otimes R^n} & {A^*\otimes M} & 0 \\
	{\Hom(R^m,A)^*} & {\Hom(R^n,A)^*} & {\Hom(M,A)^*} & 0
	\arrow[from=1-1, to=2-1]
	\arrow[from=1-1, to=1-2]
	\arrow[from=1-2, to=1-3]
	\arrow[from=1-3, to=1-4]
	\arrow[from=2-3, to=2-4]
	\arrow[from=2-2, to=2-3]
	\arrow[from=2-1, to=2-2]
	\arrow[from=1-2, to=2-2]
	\arrow[from=1-3, to=2-3]
\end{tikzcd}\]
    因为$\otimes$是右正合的,$\Hom$是左正合的,所以图中两个行正合.根据5引理\ref{5lemma}可知$\sigma$是同构。
\end{proof}
\begin{theorem}{}
   每个有限展示的平坦模是投射模。
\end{theorem}
\begin{proof}
   我们证明$\Hom(M,-)$是正合的。设$B\to C$是满射,则$C^* \to B^*$是单射。若$M$是平坦的,则:
   % https://q.uiver.app/#q=WzAsNCxbMCwwLCJDXipcXG90aW1lc19SIE0iXSxbMSwwLCJCXipcXG90aW1lcyBNIl0sWzAsMSwiXFxIb20oTSxDKV4qIl0sWzEsMSwiXFxIb20oTSxCKV4qIl0sWzAsMV0sWzAsMiwiXFxzaWdtYSJdLFsxLDMsIlxcc2lnbWEiLDJdLFsyLDNdXQ==
\[\begin{tikzcd}
	{C^*\otimes_R M} & {B^*\otimes M} \\
	{\Hom(M,C)^*} & {\Hom(M,B)^*}
	\arrow[from=1-1, to=1-2]
	\arrow["\sigma", from=1-1, to=2-1]
	\arrow["\sigma"', from=1-2, to=2-2]
	\arrow[from=2-1, to=2-2]
\end{tikzcd}\]
   给出了$\Hom(M,B)\to Hom(M,C)$的满射。所以$M$是投射模。
  \end{proof}  
   下面的引理来源于dimension shifting.
   \begin{lemma}[平坦解消引理]{}
      群$\mathrm{Tor}_*(A,B)$可以用平坦模进行计算。
   \end{lemma}
\begin{proposition}[Tor的平坦基变换]
    设$R \to T$是环同态,使得$T$成为了$R$模。从而对于所有的$R$模$A$,所有的$T$模$C$和所有的$n$:
    \begin{align}
      \mathrm{Tor}_n^R(A,C)\cong \mathrm{Tor}_n^T(A \otimes_R T,C)
    \end{align}
\end{proposition}
\begin{proof}
   选择$R$模的投射解消$P \to A$,则$\mathrm{Tor}_*^R(A,C)$是$P \otimes_R C$的同调。

   因为$T$是平坦的$R$模,所以$P_n\otimes T$是投射的$T$模且$P\otimes T \to A \otimes T$是$T$模的投射解消。所以$\mathrm{Tor}_n^T(A \otimes T,C)$是复形$(P\otimes_R T)\otimes_T C \cong P\otimes_R C$的同调。
\end{proof}
\begin{corollary}{}
   若$R$是交换环,$T$是平坦的$R$代数,则对于所有的$R$模$A,B$和所有的$n$:
   \begin{align}
      T\otimes_R \mathrm{Tor}_n^R(A,B)\cong \mathrm{Tor}_n^T(A\otimes_R T,T\otimes_R B)
   \end{align}
\end{corollary}
\begin{proof}
   设$C=T\otimes_R B$.根据上面的命题,我们只需要证明$\mathrm{Tor}_*^R(A,T\otimes B)=T\otimes \mathrm{Tor}_*^R(A,B)$.因为$T\otimes_R$是正合函子,所以$T\otimes \mathrm{Tor}_*^R(A,B)$是$T\otimes_R (P\otimes _R B)$的同调,从而为$\mathrm{Tor}_*^R(A,T\otimes B)$.
\end{proof}
为了使得$\mathrm{Tor}$给出模结构,我们必须假设$R$是交换环。原因是下面的引理:

\begin{lemma}{}
   设$\mu:A \to A$是左乘一个中心元$r$。则诱导的$\mu_*:\mathrm{Tor}_n^R(A,B)\to \mathrm{Tor}_n^R(A,B)$也是左乘$r$.
\end{lemma}
\begin{proof}
   选择$A$的投射解消$P \to A$。左乘$r$是一个$R$模的链复形映射$\tilde{\mu}:P \to P$.(因为$r$是一个中心元)。从而$\tilde{mu}\otimes B$是$P\otimes B$的$r$左乘。作为商群$\mathrm{Tor}$也是如此。
\end{proof}
\begin{corollary}{}
   若$A$是一个$R/r$模,则对于每个$R$模$B$,$R$模$\mathrm{Tor}_*^R(A,B)$也是$R/r$模。换句话说,$rR$乘在该模得$0$.
\end{corollary}
\begin{corollary}[Tor的局部化]{}
   若$R$是一个交换环且$A,B$都是$R$模。下面的命题对于所有$n$都成立:
   \begin{enumerate}
      \item $\mathrm{Tor}_n^R(A,B)=0$
      \item 对于$R$的任意素理想$p$,$\mathrm{Tor}_n^{R_p}(A_p,B_p)=0$
      \item 对于$R$的任意极大理想$m$,$\mathrm{Tor}_n^{R_m}(A_m,B_m)=0$.
   \end{enumerate}
\end{corollary}
\begin{proof}
   对于$R$模而言,$M=0$等价于任意素理想$p$,$M_p=0$等价于任意极大理想$m$,$M_m=0=0$.设$M=\mathrm{Tor}(A,B)$:
   \begin{align}
      M_p=R_p \otimes_R M=\mathrm{Tor}_n^{R_p}(A_p,B_p)
   \end{align}
\end{proof}
\section{性质较好的环的Ext函子}
讨论了Ext后,我们讨论Ext函子的性质。首先我们计算一些性质很好的环的Ext函子。

\begin{lemma}{}
   $\mathrm{Ext}_\Z^n(A,B)=0$,$\forall n \geq 2$和所有的交换群$A,B$.
\end{lemma}
\begin{proof}
   把$B$嵌入到一个内射的交换群$I^0$.其商群$I^1$是可除的,因而是内射的,所以我们给出了$B$的内射解消$0 \to B \to I^0 \to I^1 \to 0$.

   所以$\mathrm{Ext}^*(A,B)$可以计算为:
   \begin{align}
      0 \to \Hom(A,I^0) \to \Hom(A,I^1) \to 0
   \end{align}
   的上同调。
\end{proof}
因此我们只需要考虑$n=1$的情况。
\begin{example}{}
   $\mathrm{Ext}_{\Z}^0(\Z/p,B)={}_p B$.$\mathrm{Ext}_\Z^1(\Z/p,B)=B/pB$.

   可以使用$0 \to \Z \to \Z \to \Z/p$作为$\Z/p$的投射解消计算。
\end{example}

因为$\Z$是投射模,所以$\Ext^1(\Z,B)=0$对于任何$B$总是成立。我们可以依据这个结果和上述结果,在$A$是有限生成的Abel群时计算$\Ext(A,B)$:
\begin{align}
   A\cong \Z^m \oplus \Z/p  \Rightarrow \Ext(A,B)=\Ext(\Z/p,B)
\end{align}
然而无限生成的情况因为余极限不交换,要复杂得多。
\begin{example}[$B=\Z$]{}
   设$A$是一个挠群,用$A^*$表示Pontrjagin对偶。$\Z$有经典的内射解消:$0 \to \Z \to \Q \to \Q/\Z \to 0$。用这个解消计算$\Ext^*(A,\Z)$:
   \begin{align}
      0 \to \Hom(A,\Q) \to \Hom(A,\Q/\Z) \to 0 
   \end{align}
   从而$\Ext_\Z^0(A,\Z)=\Hom(A,\Z)=0$,$\Ext_\Z^1(A,\Z)=A^*$。

   为了对这个例子有更深的印象,注意到$\Z_{p^\infty}$是$\Z/p^n$的余极限(并).于是可以计算:
   \begin{align}
      \Ext_\Z^1(\Z_{p^\infty},\Z)=(\Z_{p^\infty})^*
   \end{align}
   这个群是$p$-adic整数的无挠群,$\hat{\Z}_p=\Lim (\Z/p^n)$。

   再考虑一个例子:$A=\Z[1/p],B=\Z$.此时:
   \begin{align}
      0 \to \Q=\Hom(\Z[1/p],\Q) \to \Hom(\Z[1/p],\Q/\Z) \to 0
   \end{align}
   $\Ext^0$比较容易,我们考虑$\Ext^1$.此时给定$f \in \Hom(\Z[1/p],\Q/\Z)$,筛出掉$\Hom(\Z[1/p],\Q)$的元素,本质上留存的是一个$p$-adic数。并且若两个$p$-adic数只差一个整数,与他们给出的$f$是一致的。因此$\Ext^1(\Z[1/p],\Z)=\Z_{p^\infty}$。

   这说明$\Ext$对于平坦模而言也不是vanish的。
\end{example}
\begin{example}[$R=\Z/m$,$B=\Z/p$]{}
   $\Z/p$在这种情况下有无穷的周期内射解消:
   \begin{align}
      0 \to \Z/p \xrightarrow{\iota} \Z/m \xrightarrow{p}  \Z/m \xrightarrow{m/p} \Z/m \xrightarrow{p} \dots 
   \end{align}

   于是$\Ext_{\Z/m}^n(A,\Z/p)$可以计算为:
   \begin{align}
      0 \to \Hom(A,\Z/m) \to \Hom(A,\Z/m) \to \Hom(A,\Z/m) \dots
   \end{align}
   的上同调。

   比如,若$p^2|m$,则$\Ext_{\Z/m}^n(\Z/p,\Z/p)=\Z/p$
\end{example}
\begin{proposition}{}
   对于所有的$n$和$R$:
   \begin{enumerate}
      \item $\Ext_R^n(\bigoplus_\alpha A,B)\cong \prod_{\alpha}\Ext_R^n(A_\alpha,B)$
      \item $\Ext_R^n(A,\prod_\beta B) \cong \prod_\beta \Ext_R^n(A,B_\beta)$
   \end{enumerate}
\end{proposition}
\begin{proof}
   设$P_\alpha$是$A_\alpha$的投射解消。于是$\oplus P_\alpha$是$\oplus A_\alpha$的投射解消。同理,$Q_\beta$是$B_\beta$的内射解消,则$\prod Q_\beta$是$\prod B_\beta$的内射解消。

   根据$\Hom$的性质,再加上:
   \begin{align}
      H^*(\prod C_\gamma)\cong \prod H^*(C_\gamma)
   \end{align}
   可得结果。
\end{proof}
\begin{lemma}{}
   设$R$是交换环,则$\Hom_R(A,B)$和$\Ext^*(A,B)$都是$R$模。若$\mu,\tau$分别是$r$的左乘($A,B$),则诱导的$\mu^*$和$\tau^*$也是左乘。
\end{lemma}
可以看到,这是Tor函子的相似版本,可用于给出Ext与局部化交换的性质。
\begin{proof}
   给$P \to A$投射解消.左乘$r$给出了$\tilde{mu}:P \to P$作为链复形映射。映射$\Hom(\tilde{mu},B)$是$\Hom(P,B)$上链复形,是左乘$r$.

   因此商群$\Ext^n(A,B)$被$\mu^*$作用也是$r$左乘。
\end{proof}
\begin{corollary}{}
   设$R$是交换环,$A$是$R/r$模。则对于$R$模$B$,$\Ext^*_R(A,B)$是$R/r$模。
\end{corollary}
接下来的引理,定理我们不写证明,读者可自查Weibel原书。

考虑$S^{-1}\Hom_R(A,B)$.其到$\Hom_{S^{-1}R}(S^{-1}A,S^{-1}B)$有一个自然的态射$\Phi$。但这个态射一般不是同构。
\begin{lemma}{}
   如果$A$是有限展示的$R$模,则对于每个中心可乘集合$S$,$\Phi$是同构。
\end{lemma}
不难想象证明用到的是5引理\ref{5lemma}。

\begin{proposition}{}
   设$A$是交换Noether环上的有限生成模.则$\Phi$也诱导了Ext的同构:
   \begin{align}
      \Phi:S^{-1}\Ext_R^n(A,B) \cong \Ext_{S^{-1}R}^n(S^{-1}A,S^{-1}B)
   \end{align}
\end{proposition}
不难想到证明的思路是给$A$的投射解消。因为$S^{-1}$是正合函子,所以保$H^*$。因此用$\Hom$的同构性即可给出上述同构。

\begin{corollary}[Ext的局部化]{Ext-loc}
   设$R$是交换Noether环且$A$是有限生成$R$模.则下面的命题之间对于任意$B$和$n$都等价:
   \begin{enumerate}
      \item $\Ext_R^n(A,B)=0$
      \item 对于$R$的任何素理想$p$,$\Ext_{R_p}^n(A_p,B_p)=0$
      \item 对于$R$的任何极大理想$m$,$\Ext_{R_m}^n(A_m,B_m)=0$.
   \end{enumerate}
\end{corollary}
\section{Ext函子与扩张}
我们在这一节探讨Ext到底计算了什么。为此需要介绍扩张的概念。
\begin{definition}{extension}
   一个$A$过$B$的扩张$\xi$是指一个正合列$0 \to B \to X \to A \to 0$.称两个扩张$\xi,\xi'$是等价的,若存在交换图:
   % https://q.uiver.app/#q=WzAsMTAsWzAsMCwiMCJdLFsxLDAsIkEiXSxbMiwwLCJYIl0sWzMsMCwiQiJdLFs0LDAsIjAiXSxbMSwxLCJBIl0sWzIsMSwiWCciXSxbMywxLCJCIl0sWzQsMSwiMCJdLFswLDEsIjAiXSxbMCwxXSxbMSwyXSxbMiwzXSxbMyw0XSxbNSw2XSxbNiw3XSxbNyw4XSxbOSw1XSxbMSw1LCJcXGlkIiwxXSxbMyw3LCJcXGlkIiwxXSxbMiw2LCJcXGNvbmciXV0=
\[\begin{tikzcd}
	0 & A & X & B & 0 \\
	0 & A & {X'} & B & 0
	\arrow[from=1-1, to=1-2]
	\arrow[from=1-2, to=1-3]
	\arrow[from=1-3, to=1-4]
	\arrow[from=1-4, to=1-5]
	\arrow[from=2-2, to=2-3]
	\arrow[from=2-3, to=2-4]
	\arrow[from=2-4, to=2-5]
	\arrow[from=2-1, to=2-2]
	\arrow["\id"{description}, from=1-2, to=2-2]
	\arrow["\id"{description}, from=1-4, to=2-4]
	\arrow["\cong", from=1-3, to=2-3]
\end{tikzcd}\]

   一个扩张是分裂的,若其等价于$0 \to B \to A \oplus B \to 0$(典范的)。
\end{definition}
\begin{example}{}
   若$p$是素数,则仅存在$p$个等价的$\Z/p$过$\Z/p$的扩张。分别是分裂扩张和:
   \begin{align}
      0 \to \Z/p \xrightarrow{p} \Z/p^2 \xrightarrow{i}\Z/p \to 0, i=1,2,\dots,p-1
   \end{align}

   实际上$X$必须是$p^2$阶交换群。若$X$无$p^2$阶元,则根据$X=\Z/p\oplus \Z/p$。若$X$有$p^2$阶元,设该元为$b$。则$pb \in \Z/p=B$。于是有上述$p-1$种投射。
\end{example}
\begin{lemma}{}
   若$\Ext^1(A,B)=0$,则$A$过$B$的扩张总是分裂的。
\end{lemma}
\begin{proof}
   给定一个扩张$\xi$,根据$\Ext^*(A,-)$诱导的长正合列:
   \begin{align}
      \Hom(A,X) \to \Hom(A,A) \xrightarrow{\partial}\Ext^1(A,B)=0
   \end{align}
   所以$\id_A$有原像$\sigma:A \to X$。这就是一个$X \to A$的截面。所以$X=A \oplus B$分裂。
\end{proof}
如果$\Ext^1(A,B)$非$0$,为了给出截面,实际上可以计算$\partial(\id_A)=0$。我们把这个构造记作$\Theta(\xi)$.另外.如果两个扩张等价,那么他们的$\Theta(\xi)$相同.因此这个构造只依赖于$\xi$的等价类。

\begin{theorem}{}
   给定两个模$A,B$,映射$\Theta:\xi \mapsto \partial(\id_A)$给出了一个一一映射:
   \begin{align}
      \{\text{A过B的扩张的等价类}\} \to \Ext^1(A,B)
   \end{align}
\end{theorem}
因此这个定理给出了$\Ext^1(A,B)$的一个初步作用:确定$A$过$B$的扩张个数,并赋予一个群结构。
\begin{proof}
   对于$B$,固定一个正合列$0 \to B \to I \xrightarrow{\pi} N \to 0$.其中$I$内射。作用$\Hom(A,-)$,导出一个正合列:
   \begin{align}
      \Hom(A,I) \to \Hom(A,N) \xrightarrow{\partial} \Ext^1(A,B) \to 0
   \end{align}

   现在给定一个$x \in \Ext^1(A,B)$,选定$\beta \in \Hom(A,N)$使得$\partial(\beta)=x$.根据$\beta:A \to N$和$I \to N$,可以写出拉回$X$:
   % https://q.uiver.app/#q=WzAsMTAsWzAsMCwiMCJdLFsxLDAsIk0iXSxbMiwwLCJQIl0sWzMsMCwiQSJdLFs0LDAsIlxcYnVsbGV0Il0sWzAsMSwiMCJdLFsxLDEsIkIiXSxbMiwxLCJYIl0sWzMsMSwiQSJdLFs0LDEsIlxcYnVsbGV0Il0sWzAsMV0sWzIsM10sWzMsNF0sWzUsNl0sWzYsN10sWzcsOF0sWzgsOV0sWzEsNiwiXFxiZXRhIl0sWzIsN10sWzMsOCwiPSJdLFsxLDIsImoiXV0=
\[\begin{tikzcd}
	0 & B & X & A & 0 \\
	0 & B & I & N & 0
	\arrow[from=1-1, to=1-2]
	\arrow[from=1-3, to=1-4]
	\arrow[from=1-4, to=1-5]
	\arrow[from=2-1, to=2-2]
	\arrow[from=2-2, to=2-3]
	\arrow[from=2-3, to=2-4]
	\arrow[from=2-4, to=2-5]
	\arrow["{=}", from=1-2, to=2-2]
	\arrow[from=1-3, to=2-3]
	\arrow["\beta", from=1-4, to=2-4]
	\arrow[from=1-2, to=1-3]
\end{tikzcd}\]
这不仅是拉回,而且可以验证$0 \to B \to X \to A \to 0$是一个正合列。根据连接同态$\partial$的自然性,可以得到:
% https://q.uiver.app/#q=WzAsNCxbMCwwLCJcXEhvbShBLEEpIl0sWzEsMCwiXFxFeHReMShBLE0pIl0sWzAsMSwiXFxIb20oQSxBKSJdLFsxLDEsIlxcRXh0XjEoQSxCKSJdLFswLDFdLFswLDJdLFsyLDNdLFsxLDNdXQ==
\[\begin{tikzcd}
	{\Hom(A,A)} & {\Ext^1(A,B)} \\
	{\Hom(A,N)} & {\Ext^1(A,B)}
	\arrow[from=1-1, to=1-2]
	\arrow[from=1-1, to=2-1]
	\arrow[from=2-1, to=2-2]
	\arrow[from=1-2, to=2-2]
\end{tikzcd}\]
令上面的扩张是$\xi$,则$\Theta(\xi)=x$。于是我们通过给定$x\in \Ext^1(A,B)$给出一个扩张$\xi$使得$\Theta(\xi)=x$。

为了给出$\Ext^1(A,B)$到等价类的映射,我们还需要说明上述过程$\beta$的选取不改变$\xi$的等价类。实际上选取$\beta'\in \Hom(A,N)$使得$\partial{\beta'}=x$。于是$\beta'-\beta=\pi_*(\alpha),\alpha\in \Hom(A,I)$.于是可以绘制出下面的交换图:

% https://q.uiver.app/#q=WzAsNSxbMCwwLCJYIl0sWzEsMSwiWCciXSxbMiwxLCJBIl0sWzEsMiwiSSJdLFsyLDIsIk4iXSxbMCwxLCIiLDEseyJzdHlsZSI6eyJib2R5Ijp7Im5hbWUiOiJkYXNoZWQifX19XSxbMSwyLCJcXHNpZ21hJyIsMl0sWzAsMiwiXFxzaWdtYSIsMV0sWzEsMywicCciXSxbMyw0LCJcXHBpIiwyXSxbMiw0LCJcXGJldGEnIl0sWzAsMywicCtcXGFscGhhXFxjaXJjXFxzaWdtYSIsMl1d
\[\begin{tikzcd}
	X \\
	& {X'} & A \\
	& I & N
	\arrow[dashed, from=1-1, to=2-2]
	\arrow["{\sigma'}"', from=2-2, to=2-3]
	\arrow["\sigma"{description}, from=1-1, to=2-3]
	\arrow["{p'}", from=2-2, to=3-2]
	\arrow["\pi"', from=3-2, to=3-3]
	\arrow["{\beta'}", from=2-3, to=3-3]
	\arrow["{p+\alpha\circ\sigma}"', from=1-1, to=3-2]
\end{tikzcd}\](交换性已经在草稿纸上验证了)
根据拉回的泛性质,$X$到$X'$有一个态射.

通过具体到集合的验证,可以说明这是一个同构。所以$X$和$X'$是等价的扩张。

另一方面,给定$\xi$作为$A$过$B$的扩张,$I$的延拓性质表明存在一个$\tau:X \to I$满足:
% https://q.uiver.app/#q=WzAsMTAsWzAsMCwiMCJdLFsxLDAsIkIiXSxbMiwwLCJYIl0sWzMsMCwiQSJdLFs0LDAsIjAiXSxbMCwxLCIwIl0sWzEsMSwiQiJdLFsyLDEsIkkiXSxbMywxLCJOIl0sWzQsMSwiMCJdLFswLDFdLFsxLDJdLFsyLDNdLFszLDRdLFs1LDZdLFs2LDddLFs3LDhdLFs4LDldLFsyLDcsIlxcdGF1Il0sWzEsNiwiPSJdLFszLDgsIlxcYmV0YSIsMV1d
\[\begin{tikzcd}
	0 & B & X & A & 0 \\
	0 & B & I & N & 0
	\arrow[from=1-1, to=1-2]
	\arrow[from=1-2, to=1-3]
	\arrow[from=1-3, to=1-4]
	\arrow[from=1-4, to=1-5]
	\arrow[from=2-1, to=2-2]
	\arrow[from=2-2, to=2-3]
	\arrow[from=2-3, to=2-4]
	\arrow[from=2-4, to=2-5]
	\arrow["\tau", from=1-3, to=2-3]
	\arrow["{=}", from=1-2, to=2-2]
	\arrow["\beta"{description}, from=1-4, to=2-4]
\end{tikzcd}\]

其中$\beta$是$\tau$诱导的态射。我们断言$X$是$\beta$和$\pi:I \to N$的拉回。从而$\Psi(\Theta(\xi))=\xi$.
\end{proof}
如果我们可以给出扩张的运算,就能更好的理解上述的对应。
\begin{definition}[Baer和]{Baer-sum}
   设$\xi$和$\xi'$分别是$A$过$B$的两个扩张。设$X''$是$X \to A$和$X'  \to A$的拉回。则$X''$包含了三份$B$:$B \times 0,0 \times B,\{(-b,b):b\in B\}$。

   作$X''$对于对角线$B$的商运算,则$B \times 0$和$0 \times B$被对应为一个子群。而$X''/0\times B\cong X$和$X/B=A$,则我们得到正合列:
   \begin{align}
      \varphi: 0\to B \to Y \to A\to 0
   \end{align}
   $\varphi$的等价类被称为$\xi$和$\xi'$的Baer和。
\end{definition}
\begin{proposition}{Baer-sum-pro}
   扩张等价类的集合在Baer和的意义下生成了一个交换群,分裂扩张是该和的幺元。从而$\Theta$给出了一个群同构。
\end{proposition}
\begin{proof}
   我们说明$\Theta(\varphi)=\Theta(\xi)+\Theta(\xi')$.这说明了Baer和的良定性,也给出了命题成立。

   固定$0\to M \to P \to A\to 0$是一个正合列,且$P$是投射模。因为$P$投射,所以给出$\tau:P \to X$和$\tau':P\to X'$。
   
   接下来设$\tau'': P\to X''$是由$\tau:P \to X$和$\tau': P \to X'$诱导而来的态射。而设$\bar{\tau}:P \to Y$是诱导的态射。

   我们断言$\bar{\tau}$限制在$M$上由映射$\gamma+\gamma':M \to B$诱导。所以下面的交换图:
   % https://q.uiver.app/#q=WzAsMTAsWzAsMCwiMCJdLFsxLDAsIk0iXSxbMiwwLCJQIl0sWzMsMCwiQSJdLFs0LDAsIjAiXSxbMCwxLCIwIl0sWzQsMSwiMCJdLFsyLDEsIlkiXSxbMywxLCJBIl0sWzEsMSwiQiJdLFswLDFdLFszLDgsIj0iXSxbMyw0XSxbOCw2XSxbMiwzXSxbMSwyXSxbMiw3LCJcXGJhcntcXHRhdX0iXSxbNyw4XSxbOSw3XSxbMSw5LCJcXGdhbW1hK1xcZ2FtbWEnIl0sWzUsOV1d
\[\begin{tikzcd}
	0 & M & P & A & 0 \\
	0 & B & Y & A & 0
	\arrow[from=1-1, to=1-2]
	\arrow["{=}", from=1-4, to=2-4]
	\arrow[from=1-4, to=1-5]
	\arrow[from=2-4, to=2-5]
	\arrow[from=1-3, to=1-4]
	\arrow[from=1-2, to=1-3]
	\arrow["{\bar{\tau}}", from=1-3, to=2-3]
	\arrow[from=2-3, to=2-4]
	\arrow[from=2-2, to=2-3]
	\arrow["{\gamma+\gamma'}", from=1-2, to=2-2]
	\arrow[from=2-1, to=2-2]
\end{tikzcd}\]
成立。

因此我们有$\Theta(\varphi)=\partial(\gamma+\gamma')$.然而$\partial(\gamma+\gamma')=\partial(\gamma)+\partial(\gamma')=\Theta(\xi)+\Theta(\xi')$.所以命题成立。
\end{proof}

借助上述的命题,我们实际上可以思考这样的问题:如果一个Abelian范畴没有足够的投射模和内射模,我们也可以借助扩张生成的交换群来定义$\Ext^1$.当然这里的交换群仍需要证明。

相似的,我们也可以思考$\Ext^n$的含义。我们在这里建议大家阅读原书的79页到80页内容。
\section{逆向极限的导出函子}
设$I$是一个小范畴(即对象集和态射集都是集合)。$\mathcal{A}$是一个Abelian范畴。在第二章,我们说明了$\mathcal{A}^I$有足够多的内射对象。(至少是$A$完备且有足够多内射对象的时候)。另外,容易验证逆向极限是左正合函子(保核)。

因此我们可以定义从$\mathcal{A}^I$到$\mathcal{A}$的右导出函子$R^n\Lim_{i\in I}$。

我们在这一节关注$\mathcal{A}$是Ab且$I$是$\dots\to 2\to 1 \to 0$。我们把$\mathrm{Ab}^I$中的元素称作交换群的“塔”。他们的具体形式是:
\begin{align}
   \{A_i\}:\dots \to A_2\to A_1 \to A_0
\end{align}
这一节我们具体给出$\lim^1$的具体构造,并且证明$R^n\Lim=0,n\neq 0,1$。

我们自然想问这样的构造是否可以拓展为其他的Abelian范畴。Grothendieck告诉我们,在满足下面公理的情况下该范畴可以:

(AB$4^*$):$\mathcal{A}$是完备的,且任何集合的满射的乘积都是满射。

满足该公理的范畴大多是有underlying集合的范畴(交换群,模范畴,链复形范畴),但是在层范畴失效。

\begin{definition}{}
   给定Ab中的一个塔$\{A_i\}$。定义映射:
   \begin{align}
      \Delta:\prod_{i=0}^\infty \to \prod_{i=0}^\infty A_i
   \end{align}
   为:
   \begin{align}
      \Delta(\dots,a_i,\dots,a_0)=(\dots,a_i-\bar{a}_{i+1},\dots,a_1-\bar{a}_2,a_0-\bar{a}_1)
   \end{align}
   其中$\bar{a}_{i+1}$代表$a_{i+1}\in A_{i+1}$在$A_i$中的项。
   
   容易看出$\Delta$的$\ker$是$\Lim A_i$.我们定义$\Lim^1 A_i$是$\Delta$的余核,从而$\Lim^1$是从$\mathrm{Ab}^I$到$\mathrm{Ab}$的函子。我们定义$\Lim^0 A_i=\Lim A_i$,$\Lim^n A_i=0,n\geq 2$.
\end{definition}
上述定义给出了具体的构造。当然我们需要说明这是符合要求的函子。
\begin{lemma}{}
   函子$\{\Lim^n\}$给出了一个上同调$\delta$函子。
\end{lemma}
\begin{proof}
   设$0 \to \{A_i\} \to \{B_i\}\to \{C_i\} \to 0$是塔的一个短正合列。用蛇形引理:
   % https://q.uiver.app/#q=WzAsMTAsWzAsMCwiMCJdLFsxLDAsIlxccHJvZCBBX2kiXSxbMiwwLCJcXHByb2QgQl9pIl0sWzMsMCwiXFxwcm9kIENfaSJdLFs0LDAsIjAiXSxbMSwxLCJcXHByb2QgQV9pIl0sWzIsMSwiXFxwcm9kIEJfaSJdLFszLDEsIlxccHJvZCBDX2kiXSxbMCwxLCIwIl0sWzQsMSwiMCJdLFswLDFdLFsxLDJdLFsyLDNdLFszLDRdLFsxLDUsIlxcRGVsdGEiXSxbNSw2XSxbNiw3XSxbMyw3LCJcXERlbHRhIl0sWzIsNiwiXFxEZWx0YSJdLFs4LDVdLFs3LDldXQ==
\[\begin{tikzcd}
	0 & {\prod A_i} & {\prod B_i} & {\prod C_i} & 0 \\
	0 & {\prod A_i} & {\prod B_i} & {\prod C_i} & 0
	\arrow[from=1-1, to=1-2]
	\arrow[from=1-2, to=1-3]
	\arrow[from=1-3, to=1-4]
	\arrow[from=1-4, to=1-5]
	\arrow["\Delta", from=1-2, to=2-2]
	\arrow[from=2-2, to=2-3]
	\arrow[from=2-3, to=2-4]
	\arrow["\Delta", from=1-4, to=2-4]
	\arrow["\Delta", from=1-3, to=2-3]
	\arrow[from=2-1, to=2-2]
	\arrow[from=2-4, to=2-5]
\end{tikzcd}\]

就可以得到我们想要的自然长正合列。
\end{proof}
\begin{lemma}{}
   若所有的$A_{i+1}  \to A_i$都是满射,则$\Lim^1 A_i=0$.更多的,$\Lim A_i\neq 0$(除非每个$A_i$都是$0$),因为每个自然投射$\Lim A_i \to A_j$都是满射。
\end{lemma}
\begin{proof}
   给定$b_i \in A_i(i=0,\dots,n)$,以及任何$a_0\in A_0$。归纳的选择$a_{i+1}\in A_{i+1}$:使得$a_{i+1}$是$a_i-b_i  \in A_i$在$A_{i+1}$中的提升。

   从而$\Delta$将$(\dots,a_1,a_0)$映射到$(\dots,b_1,b_0)$.因此这种情况下$\Delta$是满射,$\Lim^1 A_i=0$。如果$b_i=0$,$(\dots,a_1,a_0)\in \Lim A_i$.
\end{proof}
\begin{corollary}{}
   $\Lim^1 A_i\cong (R^1\Lim)(A_i)$且$R^n \Lim=0,\forall n\neq 0,1$
\end{corollary}
\begin{proof}
   我们说明$\Lim^n$形成了一个泛$\delta$函子,从而根据泛性说明上述成立。我们只需要说明$\Lim^1$在足够多的内射对象(应付内射解消)上vanish。

   我们在第二章给出了足够多的内射对象:
   \begin{align}
      k_*E:\dots=E=E \to 0 \to 0 \dots\to 0
   \end{align}
   其中$E$内射。因此这里面所有的态射都是满射,因此$\Lim^1$在这些内射塔上都vanish。
\end{proof}
上述的证明在AB4*的情况下总是对的。我们给出反例(不满足AB4*)。
\begin{example}{}
   设$A_0=\Z$且$A_i=p^i\Z$是$p^i$生成的子群。对短正合列($p$是素数):
   \begin{align}
      0\to \{p^i\Z\}  \to \{\Z\} \to \{\Z/p^i\Z\}  \to 0
   \end{align}
   使用$\Lim$.

   从而$\Lim^1\{p^i\Z\}\cong\hat{\Z}_p/\Z$.

\end{example}
下面这个命题在原书上是习题。我们仅作记录,证明省略。(可以查找mathstackexchange)。

\begin{proposition}{}
   设$\{A_i\}$是一个塔,$A_{i+1}\to A_i$是包含映射。把$A=A_0$看作拓扑群,其中$a+A_i(a\in A,i\geq 0)$是开集。

   则$\Lim A_i=\cap A_i=0$当且仅当$A$是Hausdorff的.$\Lim^1 A_i=0$当且仅当$A$在下列意义是完备的:每个柯西列都有不一定唯一的极限点.
\end{proposition}
提示:证明$A$是完备的,当且仅当$A\cong \Lim(A/A_i)$
\begin{definition}{}
   我们称一个塔$\{A_i\}$满足Mittag-Leffler条件,若对于每个$k$都存在一个$j\geq k$使得$A_i \to A_k$的像等于$A_j\to A_k$,对于任意$i \geq j$成立。(即$A_i$在$A_k$的像满足降链条件)。

   例如,若$\{A_i\}$都是满射,该塔就满足M-L条件。

   有一种平凡的情况:若对于每个$k$都存在一个$j\geq k$使得$A_i \to A_k$的像是$0$,我们称该塔满足平凡M-L条件。
\end{definition}
\begin{proposition}{}
   若$A_i$满足M-L条件,则:$\Lim^1 A_i=0$
\end{proposition}
\begin{corollary}{}
   设$\{A_i\}$是有限Abel群的塔,或者是有限维向量空间上的塔,我们都有$\Lim^1 A_i=0$
\end{corollary}
下面的定理预示了下一节的泛系数定理。
\begin{theorem}{}
   设$\dots \to C_1 \to C_0$是Ab的链复形的塔链。(每个$C_i$都是链复形),且满足ML条件。设$C=\Colim C_i$。则对于每个$q$都存在一个正合列:
   \begin{align}
      0 \to \textstyle\Lim^1 H_{q+1}(C_i) \to H_q(C) \to \Lim H_q(C_i) \to 0
   \end{align}


若$\dots C_1\to C_0 \to 0$是上链复形的塔链且满足ML条件。则:
\begin{align}
   0 \to \textstyle\Lim^1 H^{q-1}(C_i) \to H^q(C) \to \Lim H^q(C_i) \to 0
\end{align}
正合。
\end{theorem}
在拓扑上,这个定理有一个类似的版本。考虑$X$是CW复形,而$X_i$是$X$的上升子复形链,使得$X=\cup X_i$.则存在一个正合列:
\begin{align}
   0 \to \textstyle\Lim^1 H^{q-1}(X_i) \to H^q(X) \to \Lim H^q(X_i) \to 0
\end{align}
可以一眼看出这个公式的便利之处:可以根据子群的同调群计算最大的群的同调群。
\begin{example}{}
   设$A$是$R$模且是子模$\dots \subset A_i \subset A_{i+1}\subset \dots$的并,则对于任何$R$模$B$和$q$,都存在列:
   \begin{align}
      0 \to \textstyle\Lim^1 \Ext_R^{q-1}(A_i,B) \to \Ext_R^q(A,B) \to \Lim \Ext_R^q(A_i,B)\to 0
   \end{align}
   是正合的。

   对于$\Z_{p^\infty}=\cup \Z/p^i$,上述列化为:
   \begin{align}
      0 \to \textstyle\Lim^1 \Hom(\Z/p^i,B) \to \Ext_R^1(\Z_{p^\infty},B) \to \Lim \Ext_R^1(\Z/p^i,B)=\hat{B}_p \to 0
   \end{align}
   其中$\hat{B}_p=\Lim(B/p^iB)$是$B$的$p$-adic的完备化。
   
   这相当于推广了计算:$\Ext^1_\Z(\Z_{p^\infty},\Z)\cong \hat{\Z}_p$.实际上,设$E$是一个不变的$B$内射解消,考虑上链复形的塔链:
   \begin{align}
      \Hom(A_{i+1},E) \to \Hom(A_i,E) \to \dots \Hom(A_0,E) 
   \end{align}
   因为每个$\Hom(-,E_n)$都是反变正合的,所以塔链中每一个映射都是满射。(单反过来就是满).而$\Hom(A_i,E)$的上同调是$\Ext^*(A_i,B)$,$\Ext^*(A,B)$是:
   \begin{align}
      \Hom(\cup A_i,E)=\Lim \Hom(A_i,E)
   \end{align}
   的上同调。
\end{example}
\begin{corollary}{}
   $Z[1/p]=\cup p^{-1}\Z$,从而$\Ext^1_\Z(\Z[1/p],\Z)\cong \hat{\Z}_p/\Z$.从而对于无挠群$B$,有$\Ext_\Z^1(\Q,B)=(\prod_p \hat{B}_p)/B$.
\end{corollary}

\section{泛系数定理}
这一节我们思考的问题是,在已知$P$的同调下,如何计算$P\otimes M$的同调。由于在拓扑中,有所谓$\Z$系数,$\R$系数,$R$系数的说法,所以我们实际上在思考不同系数情况下一个拓扑空间同调和上同调群的关系。因而这节的名字是泛系数定理。
\begin{theorem}[Kunneth公式]{Kunneth-formula}
   设$P$是由平坦右$R$模给出的链复形,且$d(P_n)$作为$P_{n-1}$的子模总是平坦的。则对于任何$n$和任何左模$M$,都存在正合列:
   \begin{align}
      0 \to H_n(P)\otimes_R M \to H_n(P\otimes_R M) \to \mathrm{Tor}_1^R(H_{n-1}(P),M)\to 0
   \end{align}
\end{theorem}
\begin{proof}
   考虑短正合列:
   \begin{align}
      0 \to Z_n \to P_n \to d(P_n)\to 0
   \end{align}
   对此使用$\Tor$函子,可以得知$Z_n$也是平坦模。考虑到$\Tor_1(d(P_n),M)=0$,则:
   \begin{align}
      0 \to Z_n \otimes M \to P_n\otimes M \to d(P_n)\otimes M \to 0
   \end{align}
   是正合的。从而我们给出了链复形的短正合列:
   \begin{align}
      0 \to Z\otimes M \to P\otimes M \to d(P)\otimes M \to 0
   \end{align}
   注意到$Z$和$d(P)$中的微分算子都是$0$,从而短正合列导引的长正合列为:
   \begin{align}
      H_{n+1}(dP\otimes M) \xrightarrow{\partial} H_n(Z\otimes M) \to H_n(P \otimes M) \to H_n(dP \otimes M) \xrightarrow{\partial}H_{n-1}(Z\otimes M)
   \end{align}
   其中$H_n(dP_n\otimes M)=dP_n \otimes M$,$H_n(Z_n\otimes M)=Z_n\otimes M$.

   设$i:d(P_{n+1}) \to Z_n$是包含映射。我们断言$\partial$实际上是$i\otimes M$。(实际上很容易给出)。另一方面,$0 \to d(P_{n+1}) \to Z_n \to H_n(P) \to 0$是$H_n(P)$的平坦解消,所以$\Tor_1(H_n(P),M)$可以使用:
   \begin{align}
      0 \to d(P_{n+1})\otimes Z_n\otimes M \to 0
   \end{align}
   计算。结合长正合列即可得到结果。
\end{proof}
\begin{theorem}[同调的泛系数定理]{homo-universal}
   设$P$是一个自由Abel群的链复形。则对于任意的$n$和每个交换群$M$而言,定理\ref{Kunneth-formula}中的正合列分裂。但是这个分裂并不典范。
   \begin{align}
      H_n(P\otimes M)\cong H_n(P)\otimes M \oplus \Tor_1^\Z(H_{n-1}(P),M)
   \end{align}
\end{theorem}
\begin{proof}
   众所周知,自由Abel群的子群还是自由的。考虑$d(P_n)$是$P_{n-1}$的子群,则$d(P_n)$是自由Abel群。不典范的,这说明:
   \begin{align}
      P_n=Z_n \oplus d(P_n)
   \end{align}
   从而$Z_n\otimes M$是$P_n\otimes M$的直和项,也是$\ker(d_n\otimes 1)$的直和项。

   商去$d_{n+1}\otimes 1$的像,我们有$H_n(P)\otimes M$是$H_n(P\otimes M)$的直和项。根据Kunneth公式可知另一个项是$\Tor_1^\Z(H_{n-1}(P),M)$.
\end{proof}
\begin{theorem}[复形的Kunneth公式]{Kunneth-formula-complex}
   设$P,Q$是右,左模链复形.如果$P$和$d(P)$都是平坦的,则存在正合列:
   \begin{align}
      0 \to \bigoplus_{p+q=n}H_p(P)\otimes H_q(Q) \to H_n(P\otimes Q) \to \bigoplus_{p+q=n-1}\Tor_1^R(H_p(P),H_q(Q)) \to 0
   \end{align}
\end{theorem}
\begin{proof}
   仿照定理\ref{Kunneth-formula}的证明,把$M$换成$Q$.
\end{proof}
为了节省时间,我们省略拓扑上的泛系数定理。

接下来我们攥写上同调版本的泛系数定理。
\begin{theorem}[上同调的泛系数定理]{cohomo-universal}
   设$P$是投射模给出的链复形,使得$d(P_n)$也是投射模。则对于每个$n$和$R$模$M$,存在一个非典范的分裂正合列:
   \begin{align}
      0 \to \Ext^1_R(H_{n-1}(P),M)\to H^n(\Hom_R(P,M)) \to \Hom_R(H_n(P),M)\to 0
   \end{align}
\end{theorem}
\begin{proof}
   因为$d(P_n)$投射,从而有非典范的分裂:$P_n=d(P_{n+1})\oplus Z_n$.从而$Z_n$也投射,并且有:
   \begin{align}
      0 \to \Hom(dP_{n+1},M) \to \Hom(P_n,M)\to \Hom(Z_n,M) \to 0
   \end{align}
   是正合的。所以$0 \to \Hom(dP,M) \to \Hom(P,M)\to \Hom(Z,M) \to 0$是链复形的正合列。导引的长正合列:
   \begin{align}
      H^{n-1}(\Hom(Z,M)) \xrightarrow{\partial} H^n(\Hom(dP,M)) \to H^n(\Hom(P,M)) \to H^n(\Hom(Z,M)) \xrightarrow{\partial} H^{n+1}(\Hom(dP,M))
   \end{align}
   注意到$dP$和$Z$的微分算子都是$0$,所以$\Hom(dP,M)$的微分也是$0$,因此$H^n(\Hom(dP,M))=\Hom(dP_n,M)$。同理$H^n(\Hom(Z,M))=\Hom(Z_n,M)$。并且这里的$\partial$右$d(P_{n+1})$到$Z_n$的嵌入给出。

   注意到$H_n(P)$有投射解消:
   \begin{align}
      0 \to d(P_{n+1}) \to Z_n \to H^n(P)
   \end{align}
   于是$\Ext^1(H_{n-1}(P),M)$和$\Hom(H_n(P),M)=\Ext^0(H_n(P),M)$都可以用:
   \begin{align}
      0 \to \Hom(Z_{n-1},M) \to \Hom(dP_n,M) \to 0
   \end{align}
   带入上面的长正合列即可得到正合结果。

   而分裂可依照\ref{homo-universal}的结果得出。
\end{proof}
\begin{example}{}
   设$X$是道路连通的,则$H_0(X)=\Z$,且$H^1(X;\Z)\cong \Hom(H_1(X),\Z)$.这是一个无挠的Abel群。($\Z$是投射模。)
\end{example}
\begin{theorem}[上双复形的泛系数定理]{}
   设$P$是一个链复形,$Q$是上链复形.
   
   则可以定义上双复形$\Hom(P,Q)$.用$H^*(\Hom(P,Q))$表示$\mathrm{Tot}(\Hom(P,Q))$的上同调。设$P_n$和$dP_n$总是投射的,则存在正合列:
   \begin{align}
      0 \to \prod_{p+q=n-1}\Ext^1_R(H_p(P),H^q(Q)) \to H^n(\Hom(P,Q)) \to \prod_{p+q=n}\Hom_R(H_p(P),H^q(Q)) \to 0
   \end{align}

\end{theorem}
最后我们给出右继承的概念以结束本节。一个环$R$称作右继承的,如果任何自由(右)模的子模都是投射(右)模。实际上,任何主理想整环都是继承环(他们都是交换的戴德金整环).

继承环这条良好的性质显然可以帮助我们把泛系数定理推广到任何继承环(直接的,主理想整环)。
 \ifx\allfiles\undefined
	
	% 如果有这一部分的参考文献的话,在这里加上
	% 没有的话不需要
	% 因此各个部分的参考文献可以分开放置
	% 也可以统一放在主文件末尾。
	
	%  bibfile.bib是放置参考文献的文件,可以用zotero导出。
	% \bibliography{bibfile}
	
	end{document}
	\else
	\fi
\ifx\allfiles\undefined
	
	% 如果有这一部分的参考文献的话,在这里加上
	% 没有的话不需要
	% 因此各个部分的参考文献可以分开放置
	% 也可以统一放在主文件末尾。
	
	%  bibfile.bib是放置参考文献的文件,可以用zotero导出。
	% \bibliography{bibfile}
	
	\end{document}
	\else
	\fi
\include{part/AT/Algebra Topoplogy}
\ifx\allfiles\undefined

	% 如果有这一部分另外的package,在这里加上
	% 没有的话不需要
	
	\begin{document}
\else
\fi

\part{流形的代数拓扑}

\chapter{De Rham理论}
\section{}
\begin{proposition}{}
   
\end{proposition}
\chapter{\u{C}ech-de Rham复形}

\chapter{代数拓扑与谱序列}

\chapter{示性类}
\section{}
\begin{theorem}[Borsuk-Ulam定理]{Borsuk-Ulam定理1}
    设$f:S^n \to \mathbb{R}^n$是连续函数。则一定存在点$p \in S^n$使得$f(p)=f(-p)$。
\end{theorem}
\ifx\allfiles\undefined
	
	% 如果有这一部分的参考文献的话,在这里加上
	% 没有的话不需要
	% 因此各个部分的参考文献可以分开放置
	% 也可以统一放在主文件末尾。
	
	%  bibfile.bib是放置参考文献的文件,可以用zotero导出。
	% \bibliography{bibfile}
	
	\end{document}
	\else
	\fi
\ifx\allfiles\undefined

	% 如果有这一部分另外的package,在这里加上
	% 没有的话不需要
	
	\begin{document}
\else
\fi
\part{指标理论}
\chapter{Clifford代数及其表示}
\section{Clifford表示}
 Most of the important applications of Clifford algebras come through a detailed understanding of their representations.We can deduce them with a deep understanding of the classification in Section 4.

 Now we begin with a general definition.
    
  \begin{definition}\label{def:reps}
  Let $K \supset k$ be a field containing $k$.Then a \textbf{K-representation} of Clifford algebra $\mathrm{Cl}(V,q)$ is a $k$-algebra homomorphism
    \begin{align*}
        \rho:\mathrm{Cl}(V,q) \to \mathrm{Hom}_{K}(W,W)
    \end{align*}

    where $\mathrm{Hom}_{K}(W,W)$ is the algebra of linear transformations of a finite dimensional vector space $W$ of $K$.

    We call $W$ a $\mathrm{Cl}(V,q)$-module over $K$ and write $\rho(\varphi)(w)=\varphi \cdot w$ to simplify the notation.
  \end{definition}



We shall be interested in the cases that $K=\R,\C,\HH$.Note that a complex vector space is just a real space $W$ together with a linear map $J:W \to W$ such that $J^2=-\mathrm{Id}$.So a complex reps. is just a real resp. $\rho:\mathrm{Cl}(V,q)$ such that:
\begin{align*}
    \rho(\varphi)\circ J=J \circ \rho(\varphi)
\end{align*}

Thus the image of $\rho$ commmutes with the subalgebra $\mathrm{span}{\mathrm{Id},J}\cong $. This algebra is called a \textbf{commmuting subalgebra} of $\rho$.

Similar method applies to quaternionic resp. of $\mathrm{Cl}(V,q)$.


    Any complex resp. can be extended to a reps. of $\mathrm{Cl}_{r,s}\otimes $.This is also similar for quaternionic resp.


\begin{definition}
   Let $V,q,k \subset K$ be as in definition \ref{def:reps}.A resp. of $\mathrm{Cl}(V,q)$ will be said to be reducible if the vector space $W$ can be written as a non-trivial direct sum( over $K$):
   \begin{align*}
    W=W_1 \oplus W_2
   \end{align*}

   such that $\rho(\varphi)(W_i)\subset W_i, \forall \varphi \in \mathrm{Cl}(V,q)$.

   A reps. is called irreducible if it is not reducible.
\end{definition}
  


This definition is not conventional as "irreducible" is often regarded as a property that there are no proper invariant subspaces. However.

\chapter{Spin几何和Dirac算子}



\section{向量丛上的Spin结构}
\section{Spin流形}
\section{Clifford丛,旋量丛}
In this section, we suppose that readers have known some basic propertys of principal bundle and the associated bundle construction.We will briefly introduce these conceptions in appendix B.




\section{联络}
\section{Dirac算子}
\section{基本椭圆算子}
\section{$Cl_k$线性Dirac算子}
\section{消灭定理}
\chapter{指标定理}

\chapter{Chern-Weil理论}
\chapter{热核}



%\problemset
\begin{problemset}
  \item Solve the Riemann Conjecture.
\end{problemset}


\chapter{复几何上的指标定理}
\ifx\allfiles\undefined
	
	% 如果有这一部分的参考文献的话,在这里加上
	% 没有的话不需要
	% 因此各个部分的参考文献可以分开放置
	% 也可以统一放在主文件末尾。
	
	%  bibfile.bib是放置参考文献的文件,可以用zotero导出。
	% \bibliography{bibfile}
	
	\end{document}
	\else
	\fi


\nocite{en2,en3}

\printbibliography[heading=bibintoc, title=\ebibname]
\appendix


\chapter{示性类与拓扑K理论}

This appendix covers some of the basic property of characteristic classes used in this note.We discuss these conceptions in a pure topological viewpoint.

\section{Grassmannians流形的上同调环}

\textbf{Summation Operator} is an abbreviation used to express the summation of numbers, it plays an important role in statistics and econometrics analysis. If $\{x_i: i=1, 2, \ldots, n\}$ is a sequence of $n$ numbers, the summation of the $n$ numbers is:

\begin{equation}
\sum_{i=1}^n x_i \equiv x_1 + x_2 +\cdots + x_n
\end{equation}
\chapter{Principal G-bundle}

\end{document}

\end{document}
