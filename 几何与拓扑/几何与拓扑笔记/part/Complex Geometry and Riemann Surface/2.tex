
\chapter{向量丛与上同调}
\section{向量丛}
向量丛是上章中切丛与全纯切丛的自然推广.其核心想法是,在$M$的每一个点处附加一个线性空间(依据情况而定复,实),并且这样的附加与$M$的局部坐标有着强烈的关联.

向量丛是研究黎曼曲面的重要工具,其伴随的概念如示性类,上同调,联络等是现代几何学的基础.
\subsection*{光滑(复)向量丛}
\begin{definition}
	设$M$是一个黎曼曲面.一个$M$上的光滑(复)向量丛是指一个拓扑空间$E$和一个连续映射$\pi:E \to M$,满足如下性质:
	\begin{enumerate}
		\item $\forall x\in M$,$p$的原像集$\pi^{-1}(x)$是一个线性空间$E_x$,称为$x$处$E$的纤维(fiber)
		\item 存在一个$M$的开覆盖$\{U_i\}$满足:对于每个$U_i$,都存在一个微分同胚$\varphi_U:\pi^{-1}(U)\cong U\times \C^k$,并且对于$x \in U_i$,$\varphi_U(E_x)=\{x\}\times \C^k$.$\varphi_U$称为$E$的局部平凡化.
		\item 考虑$U_i\cap U_j\neq \emptyset$.此时转移映射(transition map)$\varphi_i \circ \varphi_j^{-1}$是从$U_i\cap U_j$到$\mathrm{GL}(k,\C)$的光滑映射.
	\end{enumerate}
	其中,$k$称为$E$的秩.
\end{definition}
上述定义中的第三个条件说明,对于$\pi^{-1}(x)$中的同一个向量$v_x$,尽管在不同的平凡化下会有不同的$\C^k$坐标表示,但是他们只相差一个$\mathrm{GL}(k,\C)$的矩阵.这个矩阵只与$x$有关,是$x$的光滑函数.

直观上,一个光滑(复)向量丛即是在黎曼曲面$M$上每个点都“长”出一个复向量空间,并且在局部上就是$U\times \C^k$.在整体上,向量丛却不一定是平凡的.

我们看几个向量丛的例子.
\begin{example}
	对于黎曼曲面$M$,$E=M\times \C^k$.此时开覆盖即$M$本身,因而可以不用考虑转移函数.
\end{example}
\begin{example}
	黎曼曲面的复切丛$T_{\C}M$是$M$上的光滑复向量丛.$\pi:T_{\C}M \to M$取典范的映射,开覆盖取$M$的地图册即可.此时$\pi^{-1}(U_\alpha)=U_\alpha \times \C^2$,基为$\pa{}{x}$和$\pa{}{y}$.

	其中,转移函数的定义为:对于$x\in U_\alpha\cap U_\beta$,
	\begin{align*}
		g_{\alpha\beta}(x)=\begin{pmatrix}
			\pa{x_\alpha}{x_\beta}&\pa{y_\alpha}{x_\beta}\\
			\pa{x_\alpha}{y_\beta}&\pa{y_\alpha}{y_\beta}\end{pmatrix}
	\end{align*}

   接下来把这个例子考虑的更“复”一些.设:
   \begin{align*}
	\pa{}{z_\alpha}=\frac{1}{2}(\pa{}{x_\alpha}-\ii\pa{}{y_\alpha})\\
	\pa{}{\bar{z}_\alpha}=\frac{1}{2}(\pa{}{x_\alpha}+\ii\pa{}{y_\alpha})
   \end{align*}
   则$\pa{}{z_\alpha}$和$\pa{}{\bar{z}_\alpha}$都是$T_{\C}M$中的向量.同时,也是$T_{\C}M$的一组基.在这种基下,我们考虑转移函数的表达式.

   做计算:
   \begin{align*}
	\pa{}{z_\beta}&=\frac{1}{2}(\pa{}{x_\beta}-\ii \pa{}{y_\beta})\\&=\frac{1}{2}(\pa{x_\alpha}{x_\beta}\pa{}{x_\alpha}+\pa{y_\alpha}{x_\beta}\pa{}{y_\alpha}-\ii \pa{x_\alpha}{y_\beta}\pa{}{x_\alpha}-\ii\pa{y_\alpha}{y_\beta}\pa{}{y_\alpha})
   \end{align*}

   把$x_\alpha$和$y_\alpha$换成$z_\alpha$和$\bar{z}_\alpha$,我们最后能得到:
   \begin{align*}
	\pa{}{z_\beta}=\frac{1}{2}(\pa{(x_\alpha+\ii y_\alpha)}{x_\beta}-\ii\pa{(x_\alpha+\ii y_\alpha)}{y_\beta})\pa{}{z_\alpha}+\pa{\bar{z}_\alpha}{z_\beta}\pa{}{\bar{z}_\alpha}=\pa{z_\alpha}{z_\beta}\pa{}{z_\alpha}
   \end{align*}
   最后一个等号来源于全纯性.即$z_\alpha\circ z_\beta^{-1}$是全纯函数,因而对$\bar{z}_\beta$求导是$0$.同样我们有:
   \begin{align*}
	\pa{}{\bar{z}_\beta}=\pa{\bar{z}_\alpha}{\bar{z}_\beta}\pa{}{\bar{z}_\alpha}
   \end{align*}

   因此在这种坐标写法下,转移函数的表达式为:
   \begin{align*}
	x \mapsto \begin{pmatrix}
		\pa{z_\beta}{z_\alpha}&0\\
		0&\pa{\bar{z}_\beta}{\bar{z}_\alpha}
	\end{pmatrix}
   \end{align*}

   这样的表达式启发我们,或许$T_{\C}M$本身可以分解为两个本质上互不相关的部分.我们将在全纯向量丛继续讨论这个问题.
\end{example}
细心的读者可能已经意识到,我们对向量丛的直观理解里面蕴含了一个有意思的事实——向量丛本身的结构其实是由转移函数决定的.即局部平凡化加上重叠处的转移函数就给出了一个向量丛.

为此,首先我们需要给出向量丛同构的定义.
\begin{definition}
	给定$E,E'$作为$M$上的两个向量丛,若存在光滑映射$f:E \to E'$满足:$\pi'(f(x))=\pi(x)$,即$f$把$x$的纤维映射到$x$的纤维,且$f|_{E_x}$是一个线性映射,则称$f$是$E$到$E'$的线性映射.若$f$限制在每个纤维上都是线性同构,则称$E$和$E'$是同构的向量丛.不区分同构的向量丛.
\end{definition}
\begin{proposition}
	设$M$是一个黎曼曲面,$\{U_i\}$是一个坐标覆盖.对于每个$U_i\cap U_j\neq \emptyset$的情况,定义$g_{ij}:U_i\cap U_j \to \mathrm{GL}(k,\C)$.若$\{g_{ij}\}$作为函数族满足:
	\begin{enumerate}
		\item $g_{ij}g_{ji}=1$
		\item $g_{ij}g_{jk}g_{ki}=1$
	\end{enumerate}
	则存在唯一的复光滑向量丛$E$,其局部平凡化为$\{U_i\}$,且转移函数为$\{g_{ij}\}$.
\end{proposition}
\begin{proof}
	定义$E$为如下空间:
	\begin{align*}
		E:=\bigcup_i U_i\times \C^k/\sim, \text{其中}(x,u_i)\sim (y,v_j)\Leftrightarrow x=y,u_i=g_{ij}(x)v_j,(x,u_i)\in U_i,(y,v_j)\in U_j
	\end{align*}

	用$[x,u]$表示$(x,u)$所在的等价类,定义$\pi:E\to M$
	\begin{align*}
		\pi:[x,u]\mapsto x
	\end{align*}

	验证$E$是一个向量丛的工作留给读者.关键是如何使用到命题中$g_{ij}$的限制条件.

	接下来说明唯一性.我们已经说明给定转移函数的情况下,可以构造一个向量丛,并且向量丛的转移函数就是给定的.现在只需要说明给定向量丛,用该向量丛的转移映射构造的向量丛与原来的向量丛同构.实际上这也是很容易的,我们同样留给读者证明.
\end{proof}


上面的命题说明,转移映射实际上是向量丛的另一种等价定义.由于转移映射本身是从$M$到矩阵的函数,因此这种定义方式更“本质”.我们会在之后的上同调讨论中更充分的意识到这一点.

最后我们介绍一些基础的概念.
\begin{definition}[子丛,商丛,张量积与直和]
	设$E$是$M$的向量丛.称$F\subset E$是$E$的一个子丛,若$F$满足:
	\begin{enumerate}
		\item $\pi_F$是光滑映射.且$\pi|_F$本身是光滑向量丛.
		\item $\forall x \in M$,$F_x \subset E_x$是一个线性子空间.
		\item $F$是$E$的子流形.
	\end{enumerate}

	不难验证,若$F$拥有转移函数$a_{\alpha\beta}$,则$E$的转移函数可以写为:
	\begin{align*}
		\begin{pmatrix}
			a_{\alpha\beta}&*\\
			0&c_{\alpha\beta}
		\end{pmatrix}
	\end{align*}

	设$F$是$E$的子丛,则可以定义商丛$E/F$.其限制在每个$x$上的纤维都是商空间$E_x/F_x$.转移函数为:$c_{\alpha\beta}$.

	设$E,F$是两个$M$的向量丛.可以定义$E,F$的张量积$E\otimes F$.其限制在每个纤维上$E_x\otimes F_x$,转移函数为$g_{\alpha\beta}\otimes g'_{\alpha\beta}$.

	设$E,F$是两个$M$的向量丛.可以定义$E,F$的张量积$E\oplus F$.其限制在每个纤维上$E_x\oplus F_x$,转移函数为$\begin{pmatrix}g_{\alpha\beta}&0\\0&g'_{\alpha\beta}\end{pmatrix}$.
\end{definition}
\subsection*{全纯向量丛}
全纯向量丛是光滑向量丛的深化.这里我们要求向量丛本身带上复的结构(成为一个复流形),并且与$M$的交互中时刻保证全纯.
\begin{definition}[复流形]
	设$M$是具有可数拓扑基的Hausdorff空间.若$M$上存在开覆盖$\{U_\alpha\}_{\alpha \in \Gamma}$以及定义在每个开集$U_\alpha$上的连续映射$\varphi_\alpha:U_\alpha \to \C^n$满足:
	\begin{enumerate}
		\item $\varphi_\alpha(U_\alpha)$是$\C^n$的开集,且$\varphi_\alpha$是给出$U_\alpha$与$\varphi_\alpha(U_\alpha)$的同胚.
		\item 若$U_\alpha \cap U_\beta \neq \emptyset$,则转移映射$\varphi_\alpha \circ \varphi_{\beta}^{-1}:\varphi_\beta(U_\alpha \cap U_\beta) \to \varphi_\alpha(U_\alpha \cap U_\beta)$是全纯映射.
	\end{enumerate}
则称$M$是一个\textbf{复流形(complex manifold)}.同时,称$\{U_\alpha,\varphi_\alpha\}$是$M$上的\textbf{地图册(alts)}.
\end{definition}
\begin{definition}[全纯向量丛]
	称$\pi:E \to M$是一个全纯向量丛,若$E$是一个$k+1$维复流形,且满足:
	\begin{enumerate}
		\item $\pi:E \to M$是全纯映射.
		\item $\forall x \in M$,存在邻域$U$和双全纯映射:
		\begin{align*}
			\varphi_U:\pi^{-1}(U)\to U\times \C^k, E_x \mapsto \{x\} \times \C^k
		\end{align*}
		\item 转移映射是从$U_{i}\cap U_j$到$\mathrm{GL}(k,\C)$的全纯映射.
	\end{enumerate}
\end{definition}

同样,我们可以只使用转移映射定义全纯向量丛.这里我们对转移映射提的要求比光滑的时候提的要求只多了一条——必须是全纯的映射.

全纯向量丛之间的映射,全纯向量丛的子丛,直和,张量积与直和都与光滑时刻一致.因为这些运算都不会影响映射的全纯性.

\begin{example}[全纯切丛]
	延续上一节的第二个例子.我们定义:
	\begin{align*}
		T^{(1,0)}M:=\la \pa{}{z_\alpha}\ra
	\end{align*}
	显然$T^{(1,0)}M$是$T_{\C}M$的子丛,并且转移函数为$\pa{x_\alpha}{x_\beta}$.由于全纯性,转移函数是全纯的,因而$T^{(1,0)}M$是全纯的向量丛.

	我们称这个向量丛是$M$的全纯切丛.它的秩为$1$.
\end{example}


\begin{example}[余切空间与余切丛]
	我们知道,黎曼曲面$M$是一个二维的实微分流形.因此$M$具有实的余切空间与余切丛.这里不再赘述他们的详细定义.

	现在把余切丛复化$T_{\C}^*M:=T^*M \otimes \C$.并且定义:
	\begin{align*}
	dz_\alpha:=dx_\alpha+\ii dy_\alpha\\
	d\bar{z}_\alpha:=dx_\alpha-\ii dy_\alpha 
	\end{align*}
	不难验证:
	\begin{align*}
		dz_\alpha(\pa{}{z_\alpha})=d\bar{z}_\alpha(\pa{}{\bar{z}_\alpha})=1\\
		dz_\alpha(\pa{}{\bar{z}_\alpha})=d\bar{z}_\alpha(\pa{}{z_\alpha})=1
	\end{align*}
	任意$f \in \C^\infty(M,\C)$,有:
	\begin{align*}
		df=\pa{f}{z_\alpha}dz_\alpha+\pa{f}{\bar{z}_\alpha}d\bar{z}_\alpha
	\end{align*}

	从上面的讨论不难看出,若定义$T^{*(1,0)}M=\la dz_\alpha\ra$,则这个丛恰好是全纯切丛的对偶丛(对偶丛的概念是容易想到的).通过直接计算,也可以得到$T^{*(1,0)}M$的转移映射为$g^*_{\alpha\beta}=\pa{z_\beta}{z_\alpha}^{-1}$.

   因此$T^{*(1,0)}M$也是一个全纯向量丛.
\end{example}

最后我们给出向量丛的一个重要定义以结束本节.
\begin{definition}
	称$s:M \to E$为光滑(全纯)向量丛$\pi:E \to M$的光滑(全纯)截面(section),若$s$是一个光滑(全纯)的映射,并且$\pi \circ s =\mathrm{id}_M$.
\end{definition}
对于全纯向量丛,我们也考虑其光滑截面.实际上,不加说明的情况下,我们的截面总是指光滑的截面.

截面的定义是简单的,但是其存在性是很不平凡的问题.我们这里没有办法过多阐述这个问题,仅仅只能阐述这个概念.
\begin{example}[$\PP^1$的全纯线丛]
	我们考虑一个具体的例子.对于黎曼曲面$\PP^1$,其拥有一个秩为$1$的全纯切丛$T^{(1,0)}\PP^1$.

	考虑$\PP^1$的地图册,我们只需要给出一个$U\cap V \to \C$的全纯函数,即可表达出该全纯线丛.设$x \in U\cap V$,则$\varphi_{UV}(z)=1/z$.对该函数求导:
	\begin{align*}
		g_{UV}=-\frac{1}{z^2}
	\end{align*}
	这确实是一个全纯函数($z\neq 0$).因此$\PP^1$的全纯线丛由如上转移函数表示.

	接下来我们考虑这个丛有没有全纯截面.假设存在这个截面$s$.则$s$限制在$U$和$V$上分别为两个开集上的全纯函数$s_U$和$s_V$.并且根据转移映射,在$V$上坐标为$z$的点(在$U$上坐标为$1/z$),满足:
	\begin{align*}
		s_U(1/z)=-\frac{1}{z^2}s_V(z)
	\end{align*}

	实际上,令$s_U=z$,$s_V=-z$,上述关系即满足.因此$T^{(1,0)}M$存在全纯截面.
\end{example}
\subsection*{线丛}
对于一个黎曼曲面$M$,线丛是指那些秩为$1$的向量丛.一般用$L$表示线丛.

线丛在黎曼曲面中的研究占着非常重要的地位.我们首先看一个定义.
\begin{definition}
	记$\mathrm{Pic}(M):=\{\pi:L \to M|\mathrm{rk}L=1\}$,即$\mathrm{Pic}(M)$是所有全纯线丛的集合.关于向量丛的张量积,该集合构成一个群,称为该黎曼曲面的Picard群.

	该群的单位元是平凡丛,该群的逆元是对偶丛.
\end{definition}

Picard群是黎曼曲面重要的一个概念.例如:
\begin{example}
	$\mathrm{Pic}(\PP^1)=\Z$
\end{example}
这个结论目前还无法证明.但是如此简洁的结论至少揭示了$\mathrm{Pic}(M)$的重要性.

考虑一个全纯线丛$\pi:L \to M$.记$\Gamma_{\OO}(L)$表示$L$所有的全纯截面.显然该集合是$\C$上的线性空间.于此同时,我们有:
\begin{align*}
	s_1 \in \Gamma_{\OO}(L_1),s_2 \in \Gamma_{\OO}(L_2) \Rightarrow s_1\otimes s_2 \in \Gamma_{\OO}(L_1\otimes L_2)
\end{align*}


\section{de Rham上同调和Dolboult上同调}
在微分流形中,我们曾经接触过外微分算子$d$.其满足:
\begin{enumerate}
	\item $d \circ d=0$
	\item $df=\pa{f}{x^i}dx^i$
	\item $d(\omega \wedge \eta)=d\omega \wedge \eta+(-1)^{\mathrm{deg}\omega}\omega \wedge d\eta$
\end{enumerate}
三条性质.

由第一条性质,我们可以定义微分流形$M$的de Rham上同调群:
\begin{align*}
	\HdR{n}(M;\R)=\frac{\ker d_{n+1}}{\mathrm{im}d_n}
\end{align*}

由代数拓扑的一般性理论,可以证明,$\HdR{n}(M;\R)$与流形的微分结构无关,只与流形的拓扑结构有关,因此是拓扑不变量.

现在我们把问题转到一个黎曼曲面$M$上.因为黎曼曲面带有自然的复结构,因此我们的上同调可以考虑复数的版本.
\subsection*{外微分与外代数的复化,外代数的分次}
称余切丛$T_{\C}^*M$的截面为$1$阶微分形式.类似的,我们可以构造出$M$上$(0,1)$阶和$(1,0)$阶微分形式,其分别为$T^{*(1,0)}M$和$T^{*(0,1)}M$的截面.类似的,可以定义$2$阶的微分形式,以及$(0,2)$,$(1,1)$,$(2,0)$阶微分形式.

注:在黎曼曲面中,实际上只存在$(1,1)$阶的微分形式.读者自证不难.

由于上述定义本质上只是把$T^*M$做了复化,因此我们仍然可以定义外微分算子和外积.其本质是实数情况的复线性延拓。

因此可以定义复值的de Rham上同调:
\begin{align*}
	\HdR{p}(M,\C):=\frac{Z^p(M,\C)}{d\bigwedge^{p-1}(M,\C)}
\end{align*}
显然我们有:
\begin{align*}
	\HdR{p}(M,\C)=\HdR{p}(M,\R)\otimes \C
\end{align*}

到目前为止都是简单的复线性延拓.但接下来的事情会复杂一些.

回忆$\bigwedge^p(M,\C)$的定义,我们有如下结果:
\begin{align*}
	\bigwedge^p(M,\C)=\bigoplus_{i=0}^p(\wedge^i(T^{*(1,0)}M)\otimes \wedge^{p-i}(T^{*0,1}M))=\bigoplus_{i=0}^p \wedge^{i,p-i}
\end{align*}

因此我们尤为想要关注将$$\bigwedge^p(M,\C)$$分次后,外微分算子的变化.
\begin{definition}
	定义两个算子$\partial$和$\bar{\partial}$:
	\begin{align*}
		\partial:=\pi^{p+1,q}\circ d:\wedge^{p,q}\to \wedge^{p+1,q}\\
		\bar{\partial}:=\pi^{p,q+1}\circ d:\wedge^{p,q}\to \wedge^{p,q+1}
	\end{align*}
\end{definition}
换句话说,$\partial$关注的是全纯分量的次数增加,$\bar{\partial}$关注的是反全纯分量的增加。

我们看一个实际的例子.对于函数$f \in \wedge^0(M;\C)$.这是一个光滑的复值函数.因此:
\begin{align*}
	df=\pa{f}{z_\alpha}dz_\alpha+\pa{f}{\bar{z}_\alpha}d\bar{z}_\alpha
\end{align*}
不难发现,第一个量是全纯的,第二个量是反全纯的.所以$\pa{f}{z_\alpha}dz_\alpha=\partial f$,$\pa{f}{\bar{z}_\alpha}d\bar{z}_\alpha=\bar{\partial}f$.也就是说$df=\partial f+\bar{\partial}f$.


我们想要知道上述结果对一般的光滑截面$s \in \wedge^{p,q}(M,\C)$是否还对.实际上,对于黎曼曲面而言,这是正确的.通过分析$M$的实际维数,读者自证不难.
\begin{proposition}
	$d=\partial+\bar{\partial}$
\end{proposition}
\begin{proposition}
	$\partial$和$\bpa$与拉回可交换.
\end{proposition}
\begin{proof}
	直接计算即可.验证0,1阶,然后验证同时满足莱布尼兹律.
\end{proof}
\begin{proposition}
	$\partial^2=\bar{\partial}^2=0$.因而可以建立其对应的上同调.我们用$H^{p,q}_{\bar{\partial}}(M)$表示上空间:
	\begin{align*}
		H^{p,q}_{\bar{\partial}}(M)=\frac{Z^{p,q}}{\bar{\partial}\wedge^{p,q-1}}
	\end{align*}
\end{proposition}

下面这个定理的重要性等同于Poincar\'{e}定理($d$-Poincar\'{e}引理).
\begin{lemma}[$\bar{\partial}$-Poincar\'{e}引理]
	对于可缩的区域$\Delta$,$H^{p,q}_{\bar{\partial}}(\Delta)=0$对于$q\geq 1$恒成立.这个命题对于复流形都是成立的.
\end{lemma}
\begin{proof}
	设$\varphi=\sum_{|I|=p,|J|=q}\varphi_{IJ}dz^I\wedge d\bar{z}^J$

	设$\varphi_I:=\sum_{|J|=q}\varphi_{IJ}dz^I d\bar{z}^J$.根据$\bar{\partial}$的定义不难看出$\bar{\partial}\varphi_I=0$也成立.

	若上述命题对于$(0,q)$阶上同调成立,即$\varphi_I=\bar{\partial}\eta_I$,则:
	\begin{align*}
		\varphi=\bar{\partial}(\sum_I dz^I\wedge \eta_I)
	\end{align*}

	因而我们的问题转为证明$(0,q)$阶的上同调消灭.

	先考虑$H^{0,1}$的情况.此时选取$[\omega]\in H^{(0,1)}(\Delta)$,不妨设$\omega=f(z)d\bar{z}$.其中$f$是光滑复值函数.考虑$\bpa{fd\bar{z}}$

	令:
	\begin{align*}
		g(z):=\frac{1}{2\pi \ii}\int_{\Delta}\frac{f(w)}{w-z}dw\wedge d\bar{w}
	\end{align*}
	注意到这个积分奇异的地方在于$w=z$处.下面这个技巧处理了这个问题

	令$\rho$是光滑的函数,且满足在$z$处的小邻域$B_{\epsilon/2}(z)$内恒为$1$,在$B_{\epsilon}(z)$外恒为$0$.这样的函数是存在的.令$f_1=\rho f$,$f_2=(1-\rho)f$.则:
	\begin{align*}
		g(z):=\frac{1}{2\pi \ii}\int_{\Delta}\frac{f_1(w)}{w-z}dw\wedge d\bar{w}+\frac{1}{2\pi \ii}\int_{\Delta}\frac{f_2(w)}{w-z}dw\wedge d\bar{w}
	\end{align*}
	第二项积分失去了奇异性,因此:
	\begin{align*}
		\pa{g}{\bar{z}}&=\pa{}{\bar{z}}[\frac{1}{2\pi \ii}\int_{\Delta}\frac{f_1(w)}{w-z}dw\wedge d\bar{w}]\\=&\pa{}{\bar{z}}[\frac{1}{2\pi}\int_0^1\int_0^{\pi}f_1(z+re^{\ii \theta})e^{-\ii\theta}dr\wedge d\theta](\text{换元}w-z=re^{\ii \theta})\\&=\frac{1}{\pi}\int_{\C}\pa{f_1(z+re^{\ii \theta})}{\bar{z}}e^{-\ii\theta}d\theta \wedge dr\\&=\frac{1}{\pi}\int_{\C}\pa{f_1}{\bar{w}}(z+re^{\ii\theta})e^{-\ii\theta}d\theta \wedge dr\\&=\frac{1}{2\pi \ii}\int_{B}\pa{f_1}{\bar{w}}(w)\frac{dw\wedge d\bar{w}}{w-z}
	\end{align*}
	根据Stokes定理,我们有:
	\begin{align*}
		\frac{1}{2\pi \ii}\int_{B}\pa{f_1}{\bar{w}}(w)\frac{dw\wedge d\bar{w}}{w-z}&=\lim_{\delta \to 0 }\frac{1}{2\pi \ii}\int_{B \setminus B_\delta(z)}\pa{f_1}{\bar{w}}(w)\frac{dw\wedge d\bar{w}}{w-z}\\&=\frac{1}{2\pi \ii}\lim_{\delta \to 0}\int_{\partial B_\delta(z)}\frac{f_1(w)}{w-z}dw\\&=\frac{1}{2\pi}\int_0^{2\pi}f_1(z+\delta e^{\ii\theta})d\theta=f_1(z)
	\end{align*}

	因此$\pa{g}{\bar{z}}=f_1(z)=f(z)$,即$\bpa g=\omega$.

	对于$q>1$的情况,我们考虑$\alpha=\sum_{I}f_Id\bar{z}_I$.设$k$是所有$I$中最大的整数,从而对于$i>k$,$d\bar{z}_i$不出现在$\alpha$中.于是把$\alpha$写为:
	\begin{align*}
		\alpha=\alpha_1 \wedge d\bar{z}_k+\alpha_2
	\end{align*}
	$\bpa\alpha=\bpa \alpha_1 \wedge d\bar{z}_k+\bpa \alpha_2$.

	对于含有$k$的$I$,定义$g_I$
	\begin{align*}
		g_I(z_1,\dots,z_n)=\frac{1}{2\pi \ii}\int_{\Delta}\frac{f_I(z_1,\dots,z_{k-1},w,z_{k+1},\dots,z_n)}{w-z_k}dw\wedge d\bar{w}
	\end{align*}
	同样的,我们有:$\pa{g_I}{\bar{z_k}}=f_I$.

	定义$\gamma=(-1)^I\sum_{k\in I}g_I d\bar{z}_{I\setminus k}$从而$\bpa\gamma(z)=-\alpha_1$.注意到$\alpha+\bpa{\gamma}$仍然是$\bpa$闭的,并且已经减少了一个可能的$d\bar{z}_k$.从而归纳下去,即可得证.
\end{proof}

\section{除子与线丛}
本节我们论述除子的相关内容.
\section{层}
\subsection*{Motivation}
Question1:Mittag-Leffler问题

令$M$是一个黎曼面.$P_1,\dots,P_N$是$N$个点.设:
\begin{align*}
	f_j:=\sum_{k=-1}^{-m_j}a_{kj}(z-z_j)^k
\end{align*}
为$P_j$附近的一个Laurent技术的主项.

问:是否存在整体的$f\in \mu(M)$,使得$f$限制在$B_{\epsilon}(P_j)$为某个全纯函数加$f_j$?或者说,是否存在$f\in \mu(M)$,使得$(f)+D\geq 0$?


Question2:对于每个除子$D\geq 0$,局部的,$D|_{U_\alpha}$均为某个全纯函数的零点.问:是否存在一个线丛$L$使得$s\in \Gamma(L)$,$(s)=D$.即$D|_{U_\alpha}=f_{\alpha}$.是否存在$(g_{\alpha\beta},U_{\alpha\beta})$使得$f_\alpha/f_\beta=g_{\beta\alpha}\in \OO^*(U_{\beta\alpha})$.

Question3:Cousin问题

对于$\C^2$中的一条全纯曲线,问是否存在$f \in \OO(\C^2)$使得$(f)$就是曲线.

\subsection*{预层}
\begin{definition}
	一个预层$\mathcal{F}$是指一个映射$\mathcal{F}$:
	\begin{align*}
		\mathcal{F}:\mathrm{Open}(M) \to \mathrm{Abel}
	\end{align*}
    称$F(U)$的元素为截面(section)

	并且对于开集之间的含入映射$i_{UV}:U \to V$,均诱导一个同态:
	\begin{align*}
		\rho_{VU}:F(V)\to F(U)
	\end{align*}
   称$\rho_{VU}$为限制映射(restriction)

   且满足:
   \begin{enumerate}
	\item $\rho_{UU}=\mathrm{id}$
	\item $\rho_{UV}\cdot \rho_{VW}=\rho_{UW}$
   \end{enumerate}
\end{definition}
学过范畴论的读者会注意到,预层实际上就是一个从$M$的开集范畴到交换群范畴的一个反变函子.

\begin{example}[函数层]
	$\OO:\mathrm{Open}(M) \to \mathrm{Abel}$定义为$U\mapsto \OO(U)$.$\rho_{UV}$即函数的限制.

	$\OO^*:\mathrm{Open}(M) \to \mathrm{Abel}$定义为$\OO^*(U)$即$U$上的非零全纯函数(处处不为$0$).该群用乘法作为运算.限制映射同样为函数的限制映射.

	$\mu^*:\mathrm{Open}(M) \to \mathrm{Abel}$定义为$\mu^*(U)$即$U$上的非零亚纯函数(不恒为$0$).限制映射同样为函数的限制映射.
\end{example}

\begin{example}
	对于线丛$\pi:L \to M$,定义$\OO(L)$是$L$对应的预层.
$\OO(L)$将$U$映射为$U$上的全纯截面.限制映射则为$s$作为映射的限制.
\end{example}
\begin{example}
	$\mu^*/\OO^*:\mathrm{Open}(M) \to \mathrm{Abel}$定义为$\mu^*/\OO^*(U)$即商群$\mu^*(U)/\OO^*(U)$.限制映射同样为函数的限制映射.
\end{example}

\subsection*{层}
\begin{definition}
	称预层$\mathcal{F}$为一个层,若$\mathcal{F}$满足:对于任意开集$U \subset M$,且$U$有一个开覆盖$\{U_i\}$
	\begin{enumerate}
		\item 若$s \in F(U)$满足对于任意$U_i$,都有$s|_{U_i}=0$,则$s=0$.
		\item 若存在$\{s_i\}$满足$s_i\in \mathcal{F}(U_i)$且$s_i|_{U_i\cap U_j}=s_j|_{U_i\cap U_j}$对于任意$i\neq j$都成立,则存在唯一的$s\in \mathcal{U}$使得$s|_{U_i}=s_i$.
	\end{enumerate}
\end{definition}
上述两个条件被称为层公理.是区分预层与层的重要条件.之前讲的预层的例子都是层.读者可以自己尝试验证.

现在我们考虑层之间的映射.这一概念是层论的基础.熟悉范畴论的读者可以意识到,层之间的映射实际上是函子的自然变换.
\begin{definition}
	$\alpha:\mathcal{F}\to \mathcal{G}$作为层之间的映射满足:对于每个$U$,都存在群同态$\mathcal{F}(U)\to \mathcal{G}(U)$.并且该映射对于限制同态是交换的.
\end{definition}
\begin{example}
    \quad

	$k:\Z \to \OO$,对于开集$U$,$k(U)$将整数$m$映射为$\OO(U)$上的函数$2\pi \ii m$.

	$\exp:\OO \to \OO^*$.对于开集$U$,$\exp(U)$将函数$f$映射为$\exp (f)$. 

	$Quotient:\mu^* \to \mu^*/\OO^*$.映射办法就是把函数映射为对应的等价类.
\end{example}
\begin{definition}[层映射的ker和Im]
	对于层映射$\alpha$,可以逐开集定义ker:
	\begin{align*}
		\ker(\alpha):=\ker(\alpha_U)
	\end{align*}

	然而Im不能逐点定义.(逐开集定义的并不能构成一个层,不能拼接)

	实际上我们定义为:
	\begin{align*}
		\mathrm{Im}(\alpha)(U):=\{s \in \mathcal{G}(U)|\forall p \in U,\exists U(p)\subset U,s.t.s|_{U(p)}\in \mathrm{Im}\alpha_{U(p)}\}
	\end{align*}

	也就是说,整体上Im$(U)$中的元素不一定是$\alpha(U)$像.
\end{definition}

考虑$\exp$映射,我们用这个例子表明逐开集定义的预层不一定是层.

例如考虑开集$\C\setminus \{0\}$,$z$是$\OO^*(\C\setminus \{0\})$上的函数.对于每个点$p$而言,都存在小开集$U(p)$使得$z$限制在$U(p)$上时,指数映射有原像.但是整体而言,并不存在这个原像.

因此直接逐开集定义,会导致$z$无法在$\C\setminus \{0\}$拼出来.

定义了Im和ker,自然就有正合列的定义(Im=ker).我们看两个例子:
\begin{example}
    下面三个列都是正合列.验证的工作留给读者.

	1.$ 0\to \Z \to \OO \to \OO^* \to 0$.

	2.$0 \to \OO^* \to \mu^* \to \mu^*/\OO^*$.

	3.$0 \to \R \to C^\infty \to \wedge^1 \dots$.
\end{example}

现在我们回到最开始的问题,看一看层论能给我们提供什么思路.

对于Mittag-Leffler问题.我们做如下的分析:

	设$\{U_\alpha\}$是$M$的一个开覆盖,且$U_j=B_j$(即$P_j$被$U_j$包裹).现在考虑关于这个开覆盖的单位分解$\rho_\alpha$.同时,对于每个$U_\alpha$,定义函数:
	\begin{align*}
		f_\alpha \in \OO(U_\alpha),P_j \notin U_\alpha\\
		f_\alpha=f_j,P_j \in U_\alpha
	\end{align*}

	因此$\sum_{\alpha \in \varLambda}\rho_\alpha f_\alpha$是$M\setminus \{P_1,\dots,P_N\}$上的光滑函数.我们考虑:
	\begin{align*}
		\varphi:=\bpa(\sum \rho_\alpha f_\alpha) \in \Lambda^{0,1}(M)\text{, 因为奇异点附近都是全纯函数}
	\end{align*}

	若$H^{0,1}_{\bpa}(M)=0$,或者$\varphi=[0]\in H^{0,1}_{\bpa}(M)$,则我们有:$\varphi=\bpa h$,$h \in \C^\infty(M)$.并且$f:=\sum \rho_\alpha f_\alpha-h$是$M$上的一个亚纯函数.

	因此我们的问题转变成了对上同调群$H^{0,1}_{\bpa}(M)$的研究.这导引了我们对层上同调的研究.上述的方法被称为Dolboult方法,即使用Dolboult上同调的办法解决ML问题.

现在我们换一个方法.任取$M$的开覆盖$\mathcal{U}=\{U_\alpha\}_{\alpha \in \varLambda}$.如上言,对每个$\alpha$指定一个$f_\alpha$.

记$f_{\alpha\beta}:=f_\beta-f_\alpha \in \OO(U_{\alpha\beta})$.如果我们能找到:
\begin{align*}
	\{(g_\alpha,U_\alpha)|g_\alpha \in \OO(U_\alpha)\}
\end{align*}
使得$f_{\alpha\beta}=g_\alpha-g_\beta$,则令$h_\alpha:=f_\alpha+g_\alpha$,则$h_\alpha$可以拼凑出一个亚纯函数$f$.

上述方法称为C\v{e}ch方法.该方法的背景是层的C\v{e}ch上同调.
\subsection*{C\v{e}ch上同调}
现在我们清空一下大脑,然后考虑一个具有开覆盖$\mathcal{U}$的黎曼曲面$M$.定义如下两个集合:
\begin{align*}
C^1(\mathcal{U},\OO)&:=\{(U_{\alpha\beta},f_{\alpha\beta})|f_{\alpha\beta}\in \OO(U_{\alpha\beta})\}\\
	Z^1(\mathcal{U},\OO)&:=\{(U_{\alpha\beta},f_{\alpha\beta})|f_{\alpha\beta}+f_{\beta\gamma}+f_{\gamma\alpha}=0\}\\
    B^1(\mathcal{U},\OO)&:=\{(U_{\alpha\beta},f_{\alpha\beta})|\exists g_\alpha\in \OO(U_\alpha),f_{\alpha\beta}=g_{\alpha}-g_{\beta}\}
\end{align*}
显然后两个是第一个集合的子集,第三个集合是第二个集合的子集.定义:
\begin{definition}
	开覆盖$\mathcal{U}$的一阶C\v{e}ch上同调定义为:
	\begin{align*}
		H^1(\mathcal{U},\OO):=Z^1(\mathcal{U},\OO)/B^1(\mathcal{U},\OO)
	\end{align*}
\end{definition}

显而易见.如果上述商群为$0$,则所有满足$f_{\alpha\beta}+f_{\beta\gamma}+f_{\gamma\alpha}=0$的$(U_{\alpha\beta},f_{\alpha\beta})$都拥有$(g_\alpha)$的形式.这就给出了我们想要的$g_\alpha$构造.

一般的,我们也可以定义高阶的上同调群.正式的定义如下:

为了定义层的C\v{e}ch上同调,我们要做如下操作.

	1.定义局部有限的“Good Cover”.
   \begin{definition}
	设$M$是一个黎曼面.称$\mathcal{U}:=\{U_\alpha\}$是一个局部有限的“Good Cover”,若满足:
	\begin{enumerate}[(1)]
		\item 任意$U \in \mathcal{U}$,都存在$N>0$使得$U\cap U_{\alpha_0}\cap \dots \cap U_{\alpha_N}=\emptyset$.
		\item 从开覆盖中任意取有限个开集,他们的交集都是可缩的.
	\end{enumerate}
   \end{definition}
	
\begin{example}
	对于$S^1$而言,两个略微大于180°的弧即可.对于$\PP^1$而言,则需要6个半圆.
\end{example}
考虑局部有限的Good Cover的理由:层的上同调理论实际上很丰富.C\v{e}ch上同调是一种抵达层上同调理论的方法,但是缺点是与开覆盖有关.如果是局部有限的Good Cover,我们有一个较好的结果:
\begin{theorem}[Leray]
	若$\forall q\geq 1$,$\forall i_1,\dots,i_p,p\geq 9$均有:$H^q(U_{i_0 i_1\dots i_q},\mathcal{F})=0$成立,则$H^*(\mathcal{U},\mathcal{F})\cong H^*(M,\mathcal{F})$.
\end{theorem}

我们暂时不需要理解上述定理中较多的含义.我们只需要知道,在$U$可缩的时候,上述定理成立,从而我们抵达了层的上同调.

2.定义高阶的C\v{e}ch上同调.
\begin{definition}[$C^p$上链]
	对于给定的开覆盖$\mathcal{U}$和给定的层$\mathcal{F}$,定义$C^p(\mathcal{U},\mathcal{F}):=\bigoplus_{\alpha_0\neq \dots \neq \alpha_p}\mathcal{F}(U_{\alpha_0,\dots,\alpha_p})$
\end{definition}
例如,$p=0$时,$C^0(\mathcal{U},\mathcal{F})=\bigoplus_{\alpha}F(U_\alpha)$.


\begin{definition}
	对于给定的开覆盖$\mathcal{U}$和给定的层$\mathcal{F}$,定义$\delta$:
	\begin{align*}
		\delta:&C^p(\mathcal{U},\mathcal{F}) \to C^{p+1}(\mathcal{U},\mathcal{F}) \\
		&\sigma \mapsto (\delta \sigma)_{\alpha_0\dots\alpha_{p+1}}:=\sum_{j=0}^{p+1}(-1)^j \sigma_{\alpha_0\dots \hat{\alpha_j}\dots\alpha_{p+1}}
	\end{align*}
\end{definition}
\begin{example}[$\PP^1$]
	在球面$S^2$上,取good covering(共六个半球面).同时,考虑层$\OO^*$.我们把$\delta$,$C^0,C^1$的表达式的结果留给读者完成.
\end{example}


3.层的同态诱导上链群之间的映射.

对于$\alpha:\mathcal{F} \to \mathcal{G}$,$\alpha$自然诱导$\alpha^*:C^p(\mathcal{U},\mathcal{F}) \to C^p(\mathcal{U},\mathcal{G})$,使得交换图:

\[\begin{tikzcd}
	{C^0(\mathcal{U},\mathcal{F})} & {C^1(\mathcal{U},\mathcal{F})} & {C^2(\mathcal{U},\mathcal{F})} & \dots \\
	{C^0(\mathcal{U},\mathcal{G})} & {C^1(\mathcal{U},\mathcal{G})} & {C^2(\mathcal{U},\mathcal{G})} & \dots
	\arrow[from=1-1, to=1-2]
	\arrow["\alpha^*"', from=1-1, to=2-1]
	\arrow[from=1-2, to=1-3]
	\arrow["\alpha^*"', from=1-2, to=2-2]
	\arrow[from=1-3, to=1-4]
	\arrow["\alpha^*"', from=1-3, to=2-3]
	\arrow[from=2-1, to=2-2]
	\arrow[from=2-2, to=2-3]
	\arrow[from=2-3, to=2-4]
\end{tikzcd}\]

这个事实的成立是显然的.因为$\alpha^*$实际上继承于$\mathcal{F}(U)$到$\mathcal{G}(U)$的$\alpha(U)$.根据限制映射与$\alpha$交换可知上述图标成立.

4.定义cocycle和coboundary
\begin{definition}
	称$p$-cochain$\sigma$为cocyle,若$\delta \sigma=0$.称$p$-cochain$\sigma$为coboundary,若$\sigma=\delta \tau$.
\end{definition}

\begin{proposition}
	cocyle是反对称的.
\end{proposition}
\begin{proof}
	$(\delta\sigma)_{123}=\sigma_{12}-\sigma_{13}+\sigma_{23}=0$,$(\delta\sigma)_{213}=\sigma_{21}-\sigma_{23}+\sigma_{13}=0$.

	于是$\sigma_{12}+\sigma_{21}=0$.
\end{proof}

\begin{proposition}
	$\delta^2=0$.
\end{proposition}
\begin{proof}
	计算比较tedious.留给读者.
\end{proof}
设$Z^p$是$p$-cocyle的全体,$B^p$是$p$-coboundary全体.
\begin{definition}[C\v{e}ch上同调]
	$H^p(\mathcal{U},\mathcal{F}):=Z^p/B^p$.
\end{definition}

到此,我们完成了C\v{e}ch上同调的定义.

回顾商群的定义,在商这个过程中,以一阶为例,$\{(g_{\alpha\beta},U_{\alpha\beta})\}\sim\{(g_{\alpha\beta}',U_{\alpha\beta})\}$当且仅当:
\begin{align*}
	g_{\alpha\beta}=\frac{f_\alpha}{f_\beta} g_{\alpha\beta}'
	\text{for}\{f_\alpha\in \OO^*(U_\alpha)\}
\end{align*}


回忆层间的同态诱导链群之间的同态,并且满足交换图.从而根据同调代数知识,我们有:
\begin{proposition}
	层间的同态诱导C\v{e}ch上同调之间的群同态.
\end{proposition}
\subsection*{Leray定理与Leray条件的验证}
上述定义的上同调存在一个问题:结果与开覆盖$\mathcal{U}$有关.我们希望定义出来的层上同调是一个与开覆盖无关的结果.

为了解决这个问题,我们首先考虑,如果$\mathcal{W}$是$\mathcal{U}$的一个加细.此时,实际上由限制映射可知,对于每一种$\tau:W_i \subset U_\alpha$,
\begin{align*}
	H^p(\mathcal{U},\mathcal{F}) \to H^p(\mathcal{W},\mathcal{F})
\end{align*}
存在一个自然的群同态,由限制映射诱导.

因此可以定义:
\begin{align*}
	\rho_{\tau}^{\mathcal{U},\mathcal{W}}:H^p(\mathcal{U},\mathcal{F}) \to H^p(\mathcal{W},\mathcal{F})
\end{align*}
用范畴论的角度来看,如果我们把所有的开覆盖作为一个范畴,用加细的方式表示两个开覆盖之间的映射,则$\rho$给出了从上述范畴到C\v{e}ch上同调群的一个函子.

对这个函子取正向极限.
\begin{definition}
	定义层$\mathcal{F}$的上同调为:$H^p(M,\mathcal{F}):=\Colim H^p(\mathcal{U},\mathcal{F})$.极限的方式如上叙所示.

	如果不使用范畴的语言:
	\begin{align*}
		H^p(M,\mathcal{F}):=\bigcup_{\mathcal{U}}H^p(\mathcal{U},\mathcal{F})/\sim,   \alpha \sim \beta \Leftrightarrow \exists B\supset \mathcal{U},B \supset \mathcal{W},\rho(\alpha)=\rho(\beta)
	\end{align*}
\end{definition}

很明显,上述结果只是形式的给出了不依赖于开覆盖的定义.但是实际上我们根本不可能用这个定义算出具体的上同调来.因此下面的Leray定理是重要的.
\begin{theorem}
	若$\forall q \geq 1$,$\forall \alpha_0,\dots,\alpha_p$,$p\geq 0$均有:
	\begin{align*}
		H^q(U_{\alpha_0\dots \alpha_p},\mathcal{F})=0
	\end{align*}
	则对于这样的开覆盖:
	\begin{align*}
		H^*(\mathcal{U},\mathcal{F})=H^*(M,\mathcal{F})
	\end{align*}
\end{theorem}

我们不打算证明这个定理.但这个定理告诉我们,如果我们能够找到合适的开覆盖,那么就能通过计算这个开覆盖的办法,算出层自身的上同调.

下面我们讨论的都是具体的层.
\begin{lemma}
	任意$\C$中的连通开集$\Omega$,都有$H^q(\Omega,\OO)=0,\forall q\geq 1$.
\end{lemma}
\begin{proof}
	可以使用定义来证明.任取$\Omega$的局部有限Good Cover,任取$\{(f_{ij},U_{ij})\} \in Z^1(\mathcal{U},\OO)$.

	任取单位分解$\rho_i$,以及紧集$K_i <<U_i$,使得:
	\begin{align*}
		\begin{cases}
			\rho_i|_{K_i}\equiv 1,\rho_i|_{\C\setminus \bar{U_i}}\equiv 0  \\
			\sum_{i=1}^\infty \rho_i(x)\equiv 1
		\end{cases}
	\end{align*}

	则令$h_j:=\sum_{U_{kj}\neq \emptyset}\rho_kf_{kj}$.则$h_j$是$U_j$上的光滑函数.且:
	\begin{align*}
		h_j-h_i=\sum_{kij\neq \emptyset}\rho_{k}f_{kj}-\sum_{lij\neq \emptyset}\rho_l f_{li}=(\sum_{lij\neq \emptyset}\rho_k)f_{ij}=f_{ij}
	\end{align*}
	因而在$U_{ij}$上$\bpa h_j=\bpa h_i$.
	
	因而存在$\omega \in \Lambda^{0,1} \Omega$使得$w|_{U_i}=\bpa h_i$.不妨设$\omega=h_0 d\bar{z}$.

	我们断言,存在光滑函数$u$使得$\bpa u=\omega$.从而设$f_i=h_i-u$.则$\bpa{f_i}=0$.于是$f_i \in \OO(U_i)$且$f_i-f_j=f_{ij}$.

	下面证明这个断言.注意到,如果$\Omega$单连通,则根据$\bpa$-Poincar\'{e}引理,断言自然成立.这实际上已经说明了$\OO$层是满足leray条件的层.

	为了证明这个断言,我们需要计算$H^{0,1}_{\bpa}(\Omega)$.我们需要一个分析学工具——Runge逼近定理.
\end{proof}
\begin{lemma}[Runge逼近定理]
	设$K$是紧集,$U$是开集,且$K \subset U \subset \C$.则下面两个命题等价.
	\begin{enumerate}[(1)]
		\item 设$W$是$U \setminus K$的任一连通分支,$\bar{W}\cap U$非紧.则$W$触及到$U$的边界.
		\item 任意$K$上的全纯函数$f$,都存在$\{f_k\} \subset \OO(U)$,使得$f_n$一致收敛于$f$.
	\end{enumerate}
\end{lemma}

现在构造$u$.取$K_1\subset K_2 \subset \dots \subset K_n\subset \Omega$.且满足:
\begin{enumerate}
	\item $\bigcup_{i=1}^\infty K_i =\Omega$.
	\item $\Omega\setminus K_i$任一连通分支$W$均有$\bar{W}\cap \Omega$非紧.
\end{enumerate}

令$\rho_i$是$\Omega$的紧支光滑函数,使得$\rho_i|_{K_i}\equiv 1$.再令$\varphi_1=\rho_1$,$\varphi_j=\rho_j-\rho_{j-1},j\geq 2$.则$\varphi_j|_{K_{j-1}}=0$,$\sum \varphi_i=1$.

注意到此时$\varphi_i h_0 \in C_0^\infty(\C)$,$\varphi_i f_0|_{K_{i-1}}=0$.

令$u_i$为$\pa{u_i}{\bar{z}}=\varphi_i h_0$的解.存在性来源于$\bpa$-Poincar\'{e}引理.

根据Runge逼近定理,以及$u_i \in \OO(K_{i-1})$可知,存在$v_i \in \OO(\Omega)$使得$|v_i-u_i|<2^{-i}$.令$u:=\displaystyle\sum_{i=1}^\infty (u_i-v_i)$.

根据我们的假设,$u$是一致收敛的,因此可以逐项求导:
\begin{align*}
	\pa{u}{\bar{z}}=\sum_{i=1}^\infty \varphi_i h_0=h_0
\end{align*}

因而对于微分形式$h_0d\bar{z}$,存在$u$:$\bpa u=h_0 d\bar{z}$.从而$H^1(\mathcal{U},\OO)=0$.

考虑$H^1(\Omega,\OO)$.对于任何一个开覆盖,总存在一个局部有限的Good cover.(流形总是仿紧的).所以:
\begin{align*}
	H^1(\Omega,\OO)=0
\end{align*}

这里我们不要求$\Omega$单连通,且$\Omega$非紧.由证明我们有:
\begin{align*}
	H^1(\Omega,\OO)=H^{0,1}_{\bpa}(\Omega)=0
\end{align*}
这是难能可贵的结论.因为对于非紧的黎曼曲面,等式的前两项不一定相等.

\begin{corollary}
	$\OO$满足Leray定理的要求.从而$\Omega$上Mittag-Leffler问题有解.
\end{corollary}

\begin{lemma}
	任意全纯线丛$\pi:L \to M$,层$\OO(L)$配合上$M$上的局部有限Good cover满足Leray条件.
\end{lemma}
\begin{proof}
	以$q=1$为例.我们仍然选取一个局部有限的Good Cover$\mathcal{U}$.并且$U_j \subset \mathcal{U}$为开圆盘.令$\psi_j$是$\pi^{-1}(U_j)$的局部平凡化.则:
	\begin{align*}
		s_j^0:=\psi_j^{-1}(z,1)
	\end{align*}
	是$\OO(L)(U_j)$中的元素.但是$s_j^0$不能构成整体的截面,因为可能不满足相容性转移映射.

	因而对于$U_j$的全纯截面$s_j$,我们总有$s_j=f \cdot s_j^0$.于是诱导了一个同构:$H^1(U_j,\OO(L))\cong H^1(U_j,\OO)$.
\end{proof}
\begin{lemma}
	$\mu^*$,$\mu^*/\OO^*$和$\OO^*$也满足Leray条件.
\end{lemma}
\begin{proof}
    只说明第一个.选取开覆盖$\mathcal{A}$.取$\sigma=\{(f_{ij},A_{ij})|f_{ij}\in \mu^*(A_{ij})\}$且$\delta \sigma=0$.

	在每个开集$A_j\in \mathcal{A}$上,$A_{ij}$部分的零点与极点都是孤立点.记录全体$f_{ij}$的零点极点为$N$.

	对$\mathcal{A}$取加细得$\mathcal{U}$,使得$\mathcal{U}$分别为两部分.一部分$\mathcal{U}_1$为$U\setminus N$的开覆盖,且元素均为圆盘.另一部分$\mathcal{U}_2$为$\mathcal{U}$中与$N$有交集的开集.

	在$\mathcal{U}_1$上$f_{ij}$不为$0$也不奇异,因此可取对数$h_{ij}:=\log f_{ij}$.则$\{(h_{ij}),U_{ij}|h_{ij}\in \OO(U_{ij})\} \in Z^1(\mathcal{U}_1,\OO)$.

	如法炮制,可以得到$h_i-h_j=h_{ij}$,于是$f_{ij}=e^{h_i}/e^{h_j}$,也即$H^1(\mathcal{U}_1,\mu^*)=0$.(存疑?此处好像没有证明完)
\end{proof}

\subsection*{进一步讨论层上同调}
1.讨论$H^1(M,\OO^*)$.

对于$M$取局部有限Good Cover.则:
\begin{align*}
	H^1(M,\OO^*)\cong H^1(\mathcal{U},\OO^*)=\{[g_{\alpha\beta},U_{\alpha\beta}]|g_{\alpha\beta}g_{\beta\gamma}g_{\gamma\alpha}=1\}
\end{align*}
回忆我们对线丛的讨论,上述条件说明,每个元素$\sigma \in Z^1(\mathcal{U},\OO^*)$可以定义一个线丛,以$\sigma$为转移函数.

如果$\sigma_1 \sim \sigma_2$,即两个元素同属$H^1(\mathcal{U},\OO^*)$中的元素,也即:
\begin{align*}
	g_{\alpha\beta}=\frac{f_\alpha}{f_\beta}g_{\alpha\beta}'
\end{align*}

则$\sigma_1$和$\sigma_2$对应的线丛是同构的.实际上,我们只需要定义:
\begin{align*}
	\varphi: L_1 \to L_2, [z_\alpha,v]\mapsto [z_\alpha,v/f_\alpha]
\end{align*}
即可.

因此,$H^1(M,\OO^*)$中的任何一个元素都对应了一个线丛$L$.另一方面,对于线丛$L$,自然可以用转移函数定义$H^1(M,\OO^*)$中的元素.不难验证这是一个一一对应.于是:
\begin{proposition}
	黎曼曲面$M$的Picard群$\mathrm{Pic}(M)\cong H^1(M,\OO^*)$.
\end{proposition}

2.讨论$H^0(M,\OO(L))$.

设$\pi:L \to M$是全纯线丛,转移函数为$\{(g_{\alpha\beta},U_{\alpha\beta})\}$.不妨这是局部有限的Good Cover.于是根据同构有:
\begin{align*}
	H^0(M,\OO(L))\cong H^0(\mathcal{U},\OO(L))
\end{align*}
也即,其中的元素为:
\begin{align*}
	\{(s_\alpha,U_\alpha)|s_\alpha=g_{\beta\alpha}s_\beta\}
\end{align*}
因而上述元素给出了一个整体截面$s:M \to L$.从而$H^0$实际上是$M$的全体截面.

\begin{corollary}
	$H^0(M,\Lambda^{1,0}):=H^0(M,\OO(K))$是全体的全纯1-形式.
\end{corollary}

3.讨论常数层$\Z$的层上同调和奇异上同调的关系.

事实上,我们有:
\begin{align*}
	H^p(M,\Z)\cong H^p_{\mathrm{sing}}(M,\Z)
\end{align*}
\begin{proof}
	首先使用一个基本结论——任何黎曼曲面都是三角剖分$\Gamma$.

	设$\alpha$是$\Gamma$的一个顶点,记录$\mathrm{St}(\alpha)$为这样一个开集:包含$\alpha$的所有三角形的并的内部.

   则$\mathrm{St}(\alpha)\cap \mathrm{St}(\beta)$要么是空集,要么是(此时$\alpha$$\beta$在一条边$E$上)除掉$\alpha$和$\beta$的两个三角形的并.而三个这样的开集的交要么是空集,要么只是一个三角形.因此,$\{\mathrm{St}(\alpha)\}$是$M$的一个局部有限Good Cover$\mathcal{U}$.

   定义$\Phi:C^p(\mathcal{U},\Z) \to C^p_{\mathrm{sing}}(M)$.其中$\Phi(\sigma)$定义为:
   \begin{align*}
	\Phi(\sigma)(\Delta_{\alpha_0\dots\alpha_p})=\sigma_{\alpha_0\dots\alpha_p}
   \end{align*}

   因而我们定义了一个从$C_p(M)$到$\Z$的同态.

   断言,$\Phi$是一个同构,并且与余边缘算子交换.即:
   \[\begin{tikzcd}
	{C^p(\mathcal{U},\Z)} & {C^p_{\mathrm{sing}}(M)} \\
	{C^{p+1}(\mathcal{U},\Z)} & {C^{p+1}_{\mathrm{sing}}(M)}
	\arrow["\Phi"', from=1-1, to=1-2]
	\arrow["\delta", from=1-1, to=2-1]
	\arrow["\partial"', from=1-2, to=2-2]
	\arrow["\Phi", from=2-1, to=2-2]
\end{tikzcd}\]

则根据基本的抽象代数可知有上同调群的同构.我们把断言的验证交给读者.

但是对于一般的情况,我们还要考虑任意的开覆盖.(我们没有验证Leray条件成立).因此对于开覆盖$\mathcal{W}$,利用上述的三角剖分,取重心重分,可以给出一个更细致的,且为$\mathcal{W}$的三角剖分$\Gamma'$.构造$\mathcal{U}'$为类似的开覆盖,从而也有上同调群的同构.

因此取极限后$H^p(M,\Z)=H^p(M)$.
\end{proof}
\subsection*{层短正合列导引上同调群长正合列}

熟悉同调代数的同学对于本小节的标题应该不陌生.事实上,我们要说明存在如下的结论.
\begin{theorem}
	考虑黎曼曲面$M$.若存在层的短正合列:
	\begin{align*}
		0 \to \mathcal{E} \stackrel{\alpha}{\rightarrow} \mathcal{F} \stackrel{\beta}{\rightarrow} \mathcal{G} \to 0
	\end{align*}
	则这个短正合列导引了一个群的长正合列:
	\begin{align*}
		0 \to H^0(M,\mathcal{E}) \stackrel{\alpha^*}{\rightarrow} H^0(M,\mathcal{F}) \stackrel{\beta^*}{\rightarrow} H^0(M,\mathcal{G}) \stackrel{\delta^*}{\rightarrow} H^1(M,\mathcal{E}) \stackrel{\alpha^*}{\rightarrow} H^1(M,\mathcal{F}) \stackrel{\beta^*}
{\rightarrow} H^1(M,\mathcal{G}) \stackrel{\delta^*}{\rightarrow} \dots
	\end{align*}

	其中$\alpha^*$和$\beta^*$是诱导的映射.$\delta^*$的定义则在证明中给出.
\end{theorem}
\begin{proof}
	碍于篇幅限制,我们不会详细阐述证明.首先我们定义$\delta^*$.称为余边缘算子.$\delta^*:H^p(M,\mathcal{G}) \to H^{p+1}(M,\mathcal{E})$.
    \[\begin{tikzcd}
	0 & {C^p(\mathcal{U},\mathcal{E})} & {C^p(\mathcal{U},\mathcal{F})} & {C^p(\mathcal{U},\mathcal{G})} & 0 \\
	0 & {C^{p+1}(\mathcal{U},\mathcal{E})} & {C^{p+1}(\mathcal{U},\mathcal{F})} & {C^{p+1}(\mathcal{U},\mathcal{G})} & 0
	\arrow[from=1-1, to=1-2]
	\arrow[from=1-2, to=1-3]
	\arrow["\delta", from=1-2, to=2-2]
	\arrow[from=1-3, to=1-4]
	\arrow["\delta"', from=1-3, to=2-3]
	\arrow[from=1-4, to=1-5]
	\arrow["\delta"', from=1-4, to=2-4]
	\arrow[from=2-1, to=2-2]
	\arrow[from=2-2, to=2-3]
	\arrow[from=2-3, to=2-4]
	\arrow[from=2-4, to=2-5]
\end{tikzcd}\]

   观察交换图.我们选取$\sigma \in Z^p(\mathcal{U},\mathcal{G})$.由追图可以得到,存在$\mu \in C^p(\mathcal{U},\mathcal{E})$使得$\alpha(\mu)=\delta \tau$,而$\tau$满足$\beta(\tau)=\sigma$.

   上述的$\mu$的等价类唯一确定于$\sigma$在$H^p$中的等价类.因此定义$\delta^*([\sigma])=[\mu]$.

   接下来要做的事情是说明上述列正合.我们留作感兴趣的读者作为练习.

\end{proof}

用这个定理可以分析出许多事情.我们考虑下面的正合列:
\begin{align*}
	0 \to \OO^* \to \mu^* \to \mu^*/\OO^* \to 0
\end{align*}

根据上面的定理,余边缘算子写为:
\begin{align*}
	\delta^*: H^0(M,\mu^*/\OO^*) \to H^1(M,\OO^*)
\end{align*}

实际上这个映射给出了$M$的除子群到Picard群的自然映射.我们需要说明三件事:1.$H^0$确实是除子群. 2.除子群到Picard群有自然的映射. 3.这个映射就是$\delta^*$.

对于$H^0(M,\mu^*/\OO^*)$中的元素,容易发现其唯一决定了$M$上的若干个零点与极点.因此这就是除子群.

考虑除子群中的元素$\sum n_z [z]$.我们需要构造一个线丛.为此,用$([f_\alpha],U_\alpha)$表示这个除子.则不难验证$(f_\alpha/f_\beta,U_{\alpha\beta})$给出了一个转移映射.并且得到的线丛在同构意义下唯一取决于$f_\alpha$的等价类.

最后,不难发现,我们上述的操作过程正好契合于$\delta^*$的一般构造.因此$\delta^*$是这个自然的映射.

在本节的最后,我们回答开头提出的Cousin问题.
\begin{theorem}[Cousin]
	$\C^2$中的全纯曲线都是全纯函数的零点.
\end{theorem}
\begin{proof}
	设$\mathcal{U}$是$\C^2$的局部有限Good Cover,且任意$U_j \in \mathcal{U}$,若$U_j \cap \C$不是空集.则$U_j=\C=(f_j)$,$f_j \in \OO(U_j)$.

	因此问题转化为,是否存在整体的$f \in \OO(\C^2)$,使得$f|_{U_j}/f_j \in \OO(U_j)$.

	记$f_j/f_i:=g_{ij}\in \OO^*(U_{ji})$.显然$(g_{ij},U_{ij})$是$Z^1(\mathcal{U},\OO^*)$中的元素,从而给出了$H^1(\mathcal{U},\OO^*)$中的元素.

	另一方面,由长正合列:
	\begin{align*}
		0 \to \Z \to \OO \to \OO^* \to 0
	\end{align*}
	可知$0=H^1(\mathcal{U},\OO) \to H^1(\mathcal{U},\OO^*) \stackrel{\delta^*}{\rightarrow}H^2(\mathcal{U},\Z)=0$是一个正合列. 于是:
	\begin{align*}
		H^1(\mathcal{U},\OO^*)=0
	\end{align*}
	由此存在$(g_i,U_i)$使得$g_j/g_i=g_{ij}$.

	因而$f_i/f_j=g_j/g_i$,于是$f:=f_i \cdot g_i$是$\C^2$上的全纯函数.
\end{proof}
\subsection*{De Rham定理与Dolbeaut定理}
\begin{theorem}[de Rham定理]
	令$M$为黎曼面(紧或非紧),则:
	\begin{align*}
		\HdR{p}(M)\cong H^p(M,\C)(\cong H^p_{\mathrm{sing}}(M)\otimes \C)
	\end{align*}
\end{theorem}
\begin{proof}
	考虑正合列:
	\begin{align*}
		0 \to \C \to C^\infty \stackrel{d}{\rightarrow} Z^1 \to 0\\
		0 \to Z^1 \stackrel{i}{\rightarrow} \Lambda^1 \stackrel{d}{\rightarrow} Z^2 \to 0 
	\end{align*}
	其中$\C$是局部常数层.$Z^1$是$1$阶闭形式层,$Z^2$是$2$阶闭形式层.

	由Poincar\'{e}引理可知,上述两个列都是正合列.

	导引长正合列:
	\begin{align*}
		H^{p-1}(M,C^\infty) \to H^{p-1}(M,Z^1) \to H^p(M,\C)\\
		H^{p-2}(M,\Lambda^1) \to H^{p-2}(M,Z^2) \to H^{p-1}(M,Z^1)\\
	\end{align*}
	
	因为存在单位分解,从而$H^p(M,C^\infty)$为$0$.同样的情况还有$\Lambda^p$.

	因此我们有:
	\begin{align*}
		H^p(M,\C)\cong H^{p-1}(M,Z^1) \cong \dots H^1(M,Z^{p-1})
	\end{align*}

	考虑$H^1(M,Z^{p-1})$.即:
	\begin{align*}
		H^0(M,\Lambda^{p-1}) \to H^0(M,Z^p) \to H^1(M,Z^{p-1}) \to 0
	\end{align*}
	而$H^0(M,\Lambda^{p-1})$即所有的$p-1$形式,$H^0(M,Z^p)$即所有的$p$阶闭形式.结合正合列:
	\begin{align*}
		H^1(M,Z^{p-1}) \cong \mathrm{coker}(H^0(M,\Lambda^{p-1}) \to H^0(M,Z^p))=H^p(M,\C)
	\end{align*}
\end{proof}
